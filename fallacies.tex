
\documentclass[a4paper,12pt,single,pdftex]{scrbook}
%\usepackage[ngerman]{babel}
\usepackage{color}
\usepackage{amsmath} 
\usepackage{times}
\usepackage{graphicx}
\usepackage{fancyheadings}
\usepackage{hyperref}
\setlength{\parindent}{0.6pt}
\setlength{\parskip}{0.6pt}
\title{}
 

\begin{document} 
\maketitle
%\newpage
\tableofcontents
\newpage

\chapter{
    
      {\bf Questionable cause}
    \\

  
    
      (also known as: butterfly logic, ignoring a common cause, neglecting a common cause, confusing correlation and causation, confusing cause and effect, false cause, third cause, third-cause fallacy, juxtaposition [form of], reversing causality/wrong direction [form of])
    \\

  
    Description: Concluding that one thing caused another, simply because they are regularly associated.

    
      Logical Form:
    \\

    
      A is regularly associated with B; therefore, A causes B.
    \\

    
      Example \#1:
    \\

    
      Every time I go to sleep, the sun goes down.  Therefore, my going to sleep causes the sun to set.
    \\

    
      Explanation: I hope the fallacious reasoning here is very clear and needs no explanation. 
    \\

    
      Example \#2:
    \\

    
      Many homosexuals have AIDS. Therefore, homosexuality causes AIDS.
    \\

    
      Explanation: While AIDS is found in a much larger percentage of the homosexual population than in the heterosexual population, we cannot conclude that homosexuality is the cause of AIDS, any more than we can conclude that heterosexuality is the cause of pregnancy.
    \\

    
      Exception: When strong evidence is provided for causation, it is not a fallacy.
    \\

    
      Variation: The {\it juxtaposition fallacy} is putting two items/ideas together, implying a causal connection, but never actually stating that one exists.
    \\

    
      It’s funny how whenever you are around, the room smells bad.
    \\

    
      {\it Reversing causality} or {\it wrong direction }is just what is sounds like -- it is still a false cause, but the specific case where one claims something like the sun sets because night time is coming.
    \\

    
      Fun Fact: To establish causality you need to show three things: 1) that X came before Y, 2) that the observed relationship between X and Y didn't happen by chance alone, and 3) that there is nothing else that accounts for the X then Y relationship.
    \\

    References:

    
      
        
      \\

      
        
          Johnson, R. H., \& Blair, J. A. (2006). {\it Logical Self-defense}. IDEA.
        
      
    
  }
\section{Correlation does not imply causation
    
      (also known as: cum hoc ergo propter hoc, Correlation Fallacy)
    \\

  
    
      Description: The cum hoc fallacy assumes that because two events occur together, they must be causally connected. It disregards the possibility of coincidence or an external factor influencing both events. This fallacy erroneously concludes causation based merely on simultaneous occurrence.
    \\

    
      Logical Form:
    \\

    
        1. Event A occurs at the same time as Event B.
    \\

    
        2. It is assumed that Event A causes Event B or vice versa.
    \\

    
        3. The possibility of coincidence or an external factor is ignored.
    \\

    
      Example \#1:
    \\

    
        - Scenario: A tourist gives a Spanish peasant and his wife bananas for the first time. As the farmer bites into his, the train enters a tunnel. He shouts, "Don’t eat it, Carmen! They make you blind."
    \\

    
        - Explanation: The farmer assumes that the act of eating the banana caused the blindness (darkness), ignoring the coincidental timing of the train entering the tunnel.
    \\

    
      Example \#2:
    \\

    
        - Scenario: A statistician finds that neatness of handwriting in 7-12-year-old children correlates with shoe size, concluding that neat handwriting causes bigger feet.
    \\

    
        - Explanation: The correlation is due to older children, who naturally have bigger feet and more developed handwriting skills, not because neat handwriting causes larger feet.
    \\

    
      Tip: Always question whether a relationship between two events is genuinely causal or simply coincidental. Look for evidence of causation and consider external factors that might influence both events. Correlation does not imply causation.
    \\

  }


post hoc ergo propter hoc
    
      (also known as: after this, therefore because of this, post hoc rationalization)
    \\

  
    
      Description: Claiming that because event Y followed event X, event Y must have been caused by event X, without properly establishing causality.
    \\

    
      Logical Form:
    \\

    
      Y occurred, then X occurred.
    \\

    
      Therefore, Y caused X.
    \\

    
       \newline

      

      
        Example \#1: Doug is convinced he has lucky underwear. One time when he forgot to put on his lucky underwear, he got a parking ticket. Doug concludes that because he forgot to wear his lucky underwear, he got the ticket. Doug doesn't wash his lucky underwear often. This part isn't relevant to the example, but it is disturbing nonetheless.
      \\

      
        Explanation: Doug is not making any attempts to determine the most probabilistic causal factors for getting the parking ticket. Instead, largely because of the {\em confirmation bias}, he is making a claim of causality based on the order of events (first he forgot his lucky underwear, then he got the ticket.)
      \\

      
        Example \#2: 
      \\

      
        {\em Tony: I bought a book on the law of attraction and two days later I won \$30k in a lottery. I wasn’t a believer of the law of attraction before, but now I am!}
      \\

      
        Explanation: Although Tony bought the lottery ticket prior to buying the book, he believes that the act of wanting to win badly enough caused him to win the lottery, because first came the desire, then the win. The fact is, there is no evidence that his desire had any effect on the win whatsoever. To suggest this is the case, in addition to {\em ad hoc reasoning}, would be {\em magical thinking}  since there is no naturalistic mechanism that could account for his desire resulting in his win.
      \\

      
        Exception: If one claims that Y caused X by adding additional details that properly establish causality, this fallacy would not apply.
      \\

      
        Fun Fact: There is some non-mystical truth to the law of attraction. That which you obsess over you are more likely to adjust your behaviors in a way that will help make your obsession a reality. For example, if you want to graduate college more than anything, you will most likely study harder, ask for extra help, and blow off fewer classes. No magic required.
      \\

    
    References:

    
      Post Hoc, Ergo Propter Hoc on JSTOR. (n.d.). Retrieved July 16, 2020, from https://www.jstor.org/stable/20126985?seq=1
    
  

Circular cause and consequence

B causes A (reverse causation or reverse causality)
    
      (also known as: Reverse causation, reverse causality, wrong direction)
    \\

  

Third factor C (the common-causal variable) causes both A and B
    
      (also known as: third-cause fallacy)
    \\

  

Bidirectional causation: A causes B, and B causes A

The relationship between A and B is coincidental\section{Gambler's fallacy (inverse)
    Description: Reasoning that, in a situation that is pure random chance, the outcome can be affected by previous outcomes.

    
      Logical Form:
    \\

    
      {\em Situation X is purely random.} \newline
{\em Situation X resulted in Y.} \newline
{\em Y is less likely to be the result next time.}
    \\

    
      Example \#1:
    \\

    
      {\em I have flipped heads five times in a row.  As a result, the next flip will probably be tails.}
    \\

    
      Explanation: The odds for each and every flip are calculated independently from other flips.  The chance for each flip is 50/50, no matter how many times heads came up before.
    \\

    
      Example \#2:
    \\

    
      {\em Eric: For my lottery numbers, I chose 6, 14, 22, 35, 38, 40.  What did you choose?}
    \\

    
      {\em Steve: I chose 1, 2, 3, 4, 5, 6.}
    \\

    
      {\em Eric: You idiot!  Those numbers will never come up!}
    \\

    
      Explanation: “Common sense” is contrary to logic and probability, when people think that any possible lottery number is more probable than any other.  This is because we see meaning in patterns -- but probability doesn’t.  Because of what is called the {\it clustering illusion}, we give the numbers 1, 2, 3, 4, 5, and 6 special meaning when arranged in that order, random chance is just as likely to produce a 1 as the first number as it is a 6.  Now the second number produced is only affected by the first selection in that the first number is no longer a possible choice, but still, the number 2 has the same odds of being selected as 14, and so on.
    \\

    
      Example \#3:
    \\

    
      {\em Maury: Please put all my chips on red 21.}
    \\

    
      {\em Dealer: Are you sure you want to do that?  Red 21 just came up in the last spin.}
    \\

    
      {\em Maury:  I didn’t know that!  Thank you!  Put it on black 15 instead.  I can’t believe I almost made that mistake!}
    \\

    
      Explanation: The dealer (or whatever you call the person spinning the roulette wheel) really should know better -- the fact that red 21 just came up is completely irrelevant to the chances that it will come up again for the next spin.  If it did, to us, that would seem “weird,” but it is simply the inevitable result of probability.
    \\

    
      Exception: If you think something is random, but it really isn’t -- like a loaded die, then previous outcomes can be used as an indicator of future outcomes.
    \\

    
      Tip: Gamble for fun, not for the money, and don’t wager more than you wouldn’t mind losing.  Remember, at least as far as casinos go, the odds are against you.
    \\

    References:

    
      
        
      \\

      
        
          Tversky, A., \& Kahneman, D. (1974). Judgment under Uncertainty: Heuristics and Biases. {\it Science}, {\it 185}(4157), 1124–1131. https://doi.org/10.1126/science.185.4157.1124
        
      
    
  }


Hot hand fallacy
    
      (also known as: hot hand phenomenon)
    \\

  
    
      Description: The hot hand fallacy is the irrational belief that if you win or lose several chance games in a row, you are either “hot” or “cold,” respectively, meaning that the streak is likely to continue and has to do with something other than pure probability.  Because we are generally stupid when it comes to realizing this, and pigheaded when it comes to accepting this fact, casinos around the world make a lot of money. This is similar to the{\it  gambler’s fallacy}.
    \\

    
      Logical Form:
    \\

    
      Person 1 has won a probability game X times in a row.
    \\

    
      Therefore, person 1 is "hot" and likely to win again the next game.
    \\

    
      Example \#1:
    \\

    
      Marta: (shooting craps) Let's just cash in now. We did great, but let's quit while we're ahead.
    \\

    
      Carlos: Forget it! We are hot! Let's see how long this streak will last!
    \\

    
      Explanation: Statistically, Carlos and Marta are more likely to lose the next game since a) their "streak" is probability-based, not talent-based and b) the odds are against them in the game of craps. Carlos is incorrectly viewing the string of wins as a streak that is more likely to continue than not.
    \\

    
      Example \#2:
    \\

    
      There are examples based on events that are not purely probabilistic, such as shooting baskets in the game of basketball. These examples are controversial because although some of the success, some of the time can be attributed to probability alone, there is no question that talent and belief can affect the outcome.
    \\

    
      Fun Fact: The belief that you are "on fire" (i.e., performing exceptionally well) can lead to better performance, via the {\it self-fulfilling prophecy}. There are times when thinking too critically can negatively affect performance.
    \\

    
      Exception: As mentioned, a streak of favorable talent-based actions can be the result of superior physical and mental performance. 
    \\

    References:

    
      
        
      \\

      
        
          The Hot Hand Fallacy: Cognitive Mistakes or Equilibrium Adjustments? Evidence from Baseball. (n.d.). Retrieved from https://www.gsb.stanford.edu/faculty-research/working-papers/hot-hand-fallacy-cognitive-mistakes-or-equilibrium-adjustments
        
      
    
  

Ludic fallacy
    
      (also known as: ludus)
    \\

  
    Description: Assuming flawless statistical models apply to situations where they actually don’t.  This can result in the over-confidence in probability theory or simply not knowing exactly where it applies as opposed to chaotic situations or situations with external influences too subtle or numerous to predict.

    
      Logical Form:
    \\

    
      {\em Claim is made.} \newline
{\em Statistics are referenced, reason is ignored.} \newline
{\em Therefore, the statistical answer is used to support or reject the claim.}
    \\

    
      Example \#1: The best example of this fallacy is presented by the person who coined this term, Nassim Nicholas Taleb in his 2007 book, {\it The Black Swan}.  There are two people:
    \\

    
      {\em Dr. John, who is regarded as a man of science and logical thinking.}
    \\

    
      {\em Fat Tony, who is regarded as a man who lives by his wits.}
    \\

    
      {\em A third party asks them, "assume a fair coin is flipped 99 times, and each time it comes up heads. What are the odds that the 100th flip would also come up heads?"  Dr. John says that the odds are not affected by the previous outcomes so the odds must still be 50/50.  Fat Tony says that the odds of the coin coming up heads 99 times in a row are so low (less than 1 in 6.33 × 1029) that the initial assumption that the coin had a 50/50 chance of coming up heads is most likely incorrect.}
    \\

    
      Explanation: You can imagine yourself watching a coin flip.  Knowing all about the {\it gambler’s fallacy}, you would hold out much longer than someone like Fat Tony when you get to the point where you say, “All right, something’s going on here with the coin”.  At what point does it become fallacious reasoning to continue to insist that you are just witnessing the “inevitable result of probability”?  There is no definite answer -- your decision will need to be argued and supported by solid reasons.
    \\

    
      Example \#2:
    \\

    
      {\em Lolita: Since about half the people in the world are female, the chances of the next person to walk out that door being female is about 50/50.}
    \\

    
      {\em Celina: Do you realize that is the door to Dr. Vulvastein, the gynecologist?}
    \\

    
      Explanation: Lolita is focusing on pure statistics while ignoring actual reason.
    \\

    
      Exception: See the explanation for example \#1.
    \\

    
      Fun Fact:  Chaos theory plays a huge role in our universe, and it is way beyond the scope of this book.  As for this fallacy, many things that appear to be random are actually chaotic systems, or unpredictable, deterministic systems.  Attempting to apply the rules of random probability in those cases will result in all kinds of errors.
    \\

    References:

    
      
        
      \\

      
        
          Taleb, N. N. (2010). {\it The Black Swan: Second Edition: The Impact of the Highly Improbable Fragility"}. Random House Publishing Group.
        
      
    
  

Drought fallacy

Regression fallacy
    
      (also known as: Regressive fallacy)
    \\

  
    Description: Ascribing a cause where none exists in situations where natural fluctuations exist while failing to account for these natural fluctuations.

    
      Logical Form:
    \\

    
      B occurred after A (although B naturally fluctuates).
    \\

    
      Therefore, A caused B.
    \\

    
      Example \#1:
    \\

    
      I had a real bad headache, then saw my doctor.  Just by talking with him, my headache started to subside, and I was all better the next day.  It was well worth the \$200 visit fee.
    \\

    
      Explanation: Headaches are a part of life, and naturally come and go on their own with varying degrees of pain. They regress to the mean on their own, the “mean” being a normal state of no pain, with or without medical or chemical intervention. Had the person seen a gynecologist instead, the headache would have still subsided, and it would have been a much more interesting visit—especially if he were a man.
    \\

    
      Example \#2:
    \\

    
      After surgery, my wife was not doing too well -- she was in a lot of pain.  I bought these magnetic wristbands that align with the body's natural vibrations to reduce the pain, and sure enough, a few days later the pain subsided!  Thank you magic wristbands!
    \\

    
      Explanation: It is normal to be in pain after any significant surgery.  It is also normal for the pain to subside as the body heals -- this is the body {\it regressing to the mean}.  Assuming the magic wristbands caused the pain relief and ignoring the regression back to the mean, is fallacious.
    \\

    
      Exception: Of course, if the “cause” is explained as the natural regression to the mean, then in a way it is not fallacious.
    \\

    
      My headache went away because that’s what headaches eventually do -- they are a temporary disruption in the normal function of a brain.
    \\

    
      Fun Fact: Seeing a doctor can have a real effect on pain relief, even if the doctor does nothing but provide a sympathetic ear. This is known as the {\em psychosocial context of the therapeutic intervention}  and is often considered part of the {\em placebo effect}.
    \\

    References:

    
      
        
      \\

      
        
          Poulton, E. C. (1994). {\it Behavioral Decision Theory: A New Approach}. Cambridge University Press.
        
      
    
  \section{Causal Reductionism
    
      (also known as: complex cause, fallacy of the single cause, causal oversimplification, reduction fallacy)
    \\

  
    
      Description: Assuming a single cause or reason when there were actually multiple causes or reasons.
    \\

    
      Logical Form:
    \\

    
      X occurred after Y.
    \\

    
      Therefore, Y caused X (although X was also a result of A,B,C... etc.)
    \\

    
      Example \#1:
    \\

    
      Hank: I ran my car off the side of the road because that damn squirrel ran in front of my car.
    \\

    
      Officer Sam: You don’t think it had anything to do with the fact that you were trying to text your girlfriend, and driving drunk?
    \\

    
      Explanation: While if it were not for the squirrel, perhaps Hank wouldn’t have totaled his car.  However, if it weren’t for him texting while driving drunk, he could have almost certainly prevented taking his unauthorized shortcut through the woods and into a tree.
    \\

    
      Example \#2:
    \\

    
      The reason more and more people are giving up belief in ghosts is because of Bo’s books.
    \\

    
      Explanation: Thank you, but that would be fallacious reasoning.  While my books {\it may have played a role} in {\it some} people giving up belief in ghosts, I doubt it was the only cause, and am pretty darn sure that overall, my books have very little effect on the population at large.
    \\

    
      Exception: Causes and reasons can be debatable, so if you can adequately defend the fact that you believe there was only a single reason, it won’t be fallacious.
    \\

    
      Tip: Use “contributing factors” more and “the reason” or “the cause” a lot less.
    \\

  }


Insignificant Cause
    
      (also known as: fallacy of insignificant, genuine but insignificant cause, insufficient cause)
    \\

  
    Description: An explanation that posits one minor factor, out of several that contributed, as its sole cause. This fallacy also occurs when an explanation is requested, and the one that is given is not sufficient to entirely explain the incident yet it is passed off as if it is sufficient.

    
      Logical Form:
    \\

    
      Factors A, B, and C caused X.
    \\

    
      Factor A, the least significant factor, is said to have caused X.
    \\

    
      Example \#1:
    \\

    
      Billy murdered all those people because I spanked him when he was a child.
    \\

    
      Explanation: Assuming that spanking did contribute to Billy's murderous behavior as an adult (which is a very weak assumption), to sell that as the cause is extremely fallacious.
    \\

    
      Example \#2:
    \\

    
      The reason Donald Trump got elected was because liberals took political correctness too far.
    \\

    
      Explanation: Assuming liberals did take political correctness too far, and this did have some effect on voters in favor of Donald Trump, it is unreasonable to claim that this was "the reason" for his win.
    \\

    
      Exception: Very often causes can be “insignificant” in that they don’t seem meaningful enough considering the meaning of the cause. This is one of the prime drivers of the{\em  {\it conspiracy theory}  fallacy}. For example, a lone gunman seems like an insignificant  cause in the death of John F. Kennedy. This is our bias at work where we want significant causes for significant effects. “Insignificant,” in the context of this fallacy, refers to “insignificant to adequately account for the cause” rather than “insignificant in meaning.”
    \\

    
      Tip: Establishing causality is very difficult. Be very weary of claims of causality in casual conversation.
    \\

  

Slippery slope
    
      (also known as: absurd extrapolation, thin edge of the wedge, camel's nose, domino fallacy, Overextension Fallacy, Runaway train)
    \\

  
    Description: When a relatively insignificant first event is suggested to lead to a more significant event, which in turn leads to a more significant event, and so on, until some ultimate, significant event is reached, where the connection of each event is not only unwarranted but with each step it becomes more and more improbable.  Many events are usually present in this fallacy, but only two are actually required -- usually connected by “the next thing you know...”

    
      Logical Form:
    \\

    
      If A, then B, then C, ... then ultimately Z!
    \\

    
      Example \#1:
    \\

    
      We cannot unlock our child from the closet because if we do, she will want to roam the house.  If we let her roam the house, she will want to roam the neighborhood.  If she roams the neighborhood, she will get picked up by a stranger in a van, who will sell her in a sex slavery ring in some other country.  Therefore, we should keep her locked up in the closet.
    \\

    
      Explanation: In this example, it starts out with reasonable effects to the causes.  For example, yes, if the child is allowed to go free in her room, she would most likely want to roam the house -- 95\% probability estimate[1].  Sure, if she roams the house, she will probably want the freedom of going outside, but not necessarily “roaming the neighborhood”, but let’s give that a probability of say 10\%.  Now we start to get very improbable.  The chances of her getting picked up by a stranger (.05\%) in a van (35\%) to sell her into sex slavery (.07\%) in another country (40\%) is next to nothing when you do all the math:
    \\

    
      .95 x .10 x .0005 x .35 x .0007 x .4 = about 1 in 25,000,000.
    \\

    
      Morality and legality aside, is it really worth it to keep a child locked in a closet based on those odds?
    \\

    
      Example \#2:
    \\

    
      If you accept that the story of Adam and Eve was figurative, then you will do the same for most of the Old Testament stories of similar literary styles.  Once you are there, the New Testament and the story of Jesus does not make sense, which will lead you to believe that the resurrection of Jesus was a “spiritual” one.  Once you accept that, you won’t be a Christian anymore; you will be a dirty atheist, then you will have no morals and start having sex with animals of a barnyard nature.  So you better take the story of Adam and Eve literally, before the phrase, “that chicken looks delicious”, takes on a whole new meaning.
    \\

    
      Explanation: Accepting the story of Adam and Eve as figurative rarely (it is sad that I cannot confidently say “never”) leads to bestiality.
    \\

    
      Exception: When a chain of events has an inevitable cause and effect relationship, as in a mathematical, logical, or physical certainty, it is not a fallacy.
    \\

    
      Tip: The concept of a “bad day” is part of this fallacy.  You wake up in the morning, and you discover that you are out of coffee.  From there, you fallaciously reason that this means you will be grumpy, late for work, then behind all day in work, then have to stay late, then miss dinner with the family, then cause more friction at home, etc.  This is only true if you act it out as if it is true.  Of course, with an already bad attitude, you look back on the day, block out the good and wallow in the bad, just so you can tell yourself, that you were right all along about having a “bad day”.
    \\

    
      Don’t let that happen.
    \\

    References:

    
      
        
      \\

      
        
          Walton, D. N. (1992). {\it Slippery Slope Arguments}. Clarendon Press.
        
      
      
        [1] I am basing these estimates on my best guess... this is not meant to be an accurate study on child abduction, just an illustration of how odds work in the fallacy.
      \\

    
  

Texas Sharpshooter Fallacy
    
      (also known as: clustering illusion, clustering fallacy)
    \\

  
    
      Description: Ignoring the difference while focusing on the similarities, thus coming to an inaccurate conclusion.  Similar to the {\it gambler’s fallacy}, this is an example of inserting meaning into randomness.  This is also similar to the {\it post-designation} fallacy, but with the {\em Texas sharpshooter fallacy} the focus is generally a result of deliberate misleading.
    \\

    
      Logical Form:
    \\

    
      {\em X and Y are compared by several criteria.} \newline
{\em A conclusion is made based on only the criteria that produce the desired outcome.}
    \\

    
      Example \#1:
    \\

    
      {\em The “prophet” Nostradamus wrote about 500 years ago:}
    \\

    
      {\em Beasts wild with hunger will cross the rivers, \newline
The greater part of the battle will be against Hister. \newline
He will cause great men to be dragged in a cage of iron, \newline
When the son of Germany obeys no law.}
    \\

    
      {\em Surely he must have had some vision of Hitler!}
    \\

    
      Explanation: When you focus on just that prediction, then it might seem that way, but realize that Nostradamus made over 1000 “predictions”, most (all?) of which are vague nonsense.  Given that many predictions, it is statistically impossible NOT to match at least one with an actual event.  Again, if you ignore the noise (the predictions that do not make any sense), it looks amazing.  By the way, “Hister” is the Latin name for the Danube River.
    \\

    
      Example \#2:
    \\

    
      {\em SuperCyberDate.con determined that Sally and Billy are a great match because they both like pizza, movies, junk food, Janet Jackson, and vote republican.}
    \\

    
      Explanation: What SuperCyberDate.con did not take into consideration were the 245 other likes and dislikes that were very different for both Sally and Billy—such as the fact that Billy likes men.
    \\

    
      Exception: It's never a good idea to ignore the differences in the data while only focusing on the similarities.
    \\

    
      Fun Fact: The name “{\em Texas sharpshooter fallacy}” comes from the idea that someone could shoot randomly at a barn, then draw a bullseye around the largest cluster, making it appear as if they were a sharpshooter.
    \\

    References:

    
      
        
      \\

      
        
          Forshaw, M. (2012). {\it Critical Thinking For Psychology: A Student Guide}. John Wiley \& Sons.
        
      
    
  \section{Magical thinking
    
      (also known as: superstitious thinking)
    \\

  
    Description: Making causal connections or correlations between two events not based on logic or evidence, but primarily based on superstition.  Magical thinking often causes one to experience irrational  fear of performing certain acts or having certain thoughts because they assume a correlation with their acts and threatening calamities.

    
      Logical Form:
    \\

    
      {\em Event A occurs.} \newline
{\em Event B occurs.} \newline
{\em Because of superstition or magic, event A is causally connected to or correlated with event B.}
    \\

    
      Example \#1:
    \\

    
      {\em Mr. Governor issues a proclamation for the people of his state to pray for rain.  Several months later, it rains.  Praise the gods!}
    \\

    
      Explanation: Suggesting that appealing to the gods for rain via prayer or dance is just the kind of thing crazy enough to get you elected president of the United States, but there is absolutely no logical reason or evidence to support the claim that appealing to the gods will make it rain.
    \\

    
      Example \#2:
    \\

    
      {\em I refuse to stay on the 13th floor of any hotel because it is bad luck.  However, I don’t mind staying on the same floor as long as we call it the 14th floor.}
    \\

    
      Explanation: This demonstrates the kind of {\it magical thinking}  that so many people engage in, that, according to Dilip Rangnekar of Otis Elevators, an estimated 85\% of buildings with elevators did not have a floor numbered “13”.  There is zero evidence that the number 13 has any property that causes bad luck -- of course, it is the superstitious mind that connects that number with bad luck.
    \\

    
      Example \#3:
    \\

    
      {\em I knew I should have helped that old lady across the road.  Because I didn’t, I have been having bad Karma all day.}
    \\

    
      Explanation: This describes how one who believes that they deserve bad fortune, will most likely experience it due to the {\it confirmation bias} and other self-fulfilling prophecy-like behavior.  Yet there is no logical or rational basis behind the concept of Karma.
    \\

    
      Exception: If you can empirically prove your magic, then you can use your magic to reason.
    \\

    
      Tip: {\it Magical thinking} may be comforting at times, but reality is always what’s true.
    \\

    References:

    
      
        
      \\

      
        
          Hutson, M. (2012). {\it The 7 Laws of Magical Thinking: How Irrational Beliefs Keep Us Happy, Healthy, and Sane}. Penguin.
        
      
    
  }


Folk religion

Illusion of control

Law of attraction (New Thought)

Law of contagion

Obsessive–compulsive disorder

Performativity

Placebo button

Psychological theories of magic

Magic and religion

Segen

Wish fulfillment

Argument of the Beard
    
      (also known as: fallacy of the beard, heap fallacy, heap paradox fallacy, bald man fallacy, continuum fallacy, line drawing fallacy, sorites fallacy, Loki's Wager)
    \\

  
    Description: When one argues that no useful distinction can be made between two extremes, just because there is no definable moment or point on the spectrum where the two extremes meet.  The name comes from the heap paradox in philosophy, using a man’s beard as an example.  At what point does a man go from clean-shaven to having a beard?

    
      Logical Form:
    \\

    
      X is one extreme, and Y is another extreme.
    \\

    
      There is no definable point where X becomes Y.
    \\

    
      Therefore, there is no difference between X and Y.
    \\

    
      Example \#1:
    \\

    
      Why does the law state that you have to be 21 years old to drink?  Does it really make any difference if you are 20 years and 364 days old?  That is absurd.  Therefore, if a single day makes no difference, then a collection of 1095 single days won’t make any difference. Therefore, changing the drinking age to 18 will not make any difference.
    \\

    
      Explanation: Although this does appear to be typical 18-year-old thinking (sorry 18 year-olds), it is quite a common fallacy.  Just because any single step makes no {\it apparent} difference, there is a difference that becomes more noticeable as the number of those steps increase.
    \\

    
      Example \#2:
    \\

    
      Willard: I just realized that I will probably never go bald!
    \\

    
      Fanny: Why is that?
    \\

    
      Willard:  Well, if I lose just one hair, I will not be bald, correct?
    \\

    
      Fanny: Of course.
    \\

    
      Willard: If I lose two hairs?
    \\

    
      Fanny: No.
    \\

    
      Willard: Every time I lose a hair, the loss of that one hair will not make me bald; therefore, I will never go bald.
    \\

    
      Fanny: Congratulations, you found the cure to baldness -- stupidity!
    \\

    
      Explanation: What Willard did not take into consideration is “baldness” is a term used to define a state along a continuum, and although there is no clear point between bald and not bald, the extremes are both clearly recognizable and achievable.
    \\

    
      Exception: The larger the spread, the more fallacious the argument, the smaller the spread, the less fallacious.
    \\

    
      Fun Fact: There are very few clear lines we can draw between categories in any area of life.  Categories are human constructs that we create to help us make sense of things, yet they often end up creating more confusion by tricking us into thinking abstract concepts actually exist.
    \\

  

Counterfactual fallacy
    
      (also known as: argumentum ad speculum,
    \\

    hypothesis contrary to fact "what if" ,wouldchuck)
  
    Description: Offering a poorly supported claim about what might have happened in the past or future, if (the hypothetical part) circumstances or conditions were different.  The fallacy also entails treating future hypothetical situations as if they are fact.

    
      Logical Form:
    \\

    
      {\em If event X did happen, then event Y would have happened (based only on speculation).}
    \\

    
      Example \#1:
    \\

    
      {\em If you took that course on CD player repair right out of high school, you would be doing well and gainfully employed right now.}
    \\

    
      Explanation: This is speculation at best, not founded on evidence, and is {\it unfalsifiable}.  There are many people with far more useful talents who are unemployed, and many who are “gainfully” employed who are not doing well at all. Besides, perhaps those with certificates in compact disc repair are gainfully employed... at McDonald’s.
    \\

    
      Example \#2:
    \\

    
      {\em John, if you would have taken a shower more often, you would still be dating Tina.}
    \\

    
      Explanation: Past hypotheticals that are stated as fact are most often nothing more than one possible outcome of many.  One cannot ignore probabilities when making these kinds of statements.  Perhaps Tina likes the smell of man sweat.  Perhaps Tina would have still preferred Renaldo over John despite John's personal hygiene because of Renaldo's enormous intellect.
    \\

    
      Exception: In either/or situations, general predictions can obviously be made without fallacy:
    \\

    
      {\em If you didn’t flip heads on the coin, it would have been tails.}
    \\

    
      Fun Fact: Right out of college, with a degree in marketing, I worked at the Olive Garden (an Italian-like semi-fast food chain here in the States). Perhaps I should have opted for the CD repair right out of high school and saved \$120,000. I am pretty sure the Olive Garden would have still hired me.
    \\

    References:

    
      
        
      \\

      
        
          Moore, W. E. (1967). {\it Creative and Critical Thinking}. Houghton Mifflin.
        
      
    
  

Sociologist's fallacy\chapter{Fallacies of relevance}
\section{
    
      {\bf Appeals to emotion}
    \\

  
    
      (also known as: Argumentum ad passiones, Emotional appeals, appeal to pathos, argument by vehemence, playing on emotions, emotional appeal, for the children)
    \\

  
    Description: This is the general category of many fallacies that use emotion in place of reason in order to attempt to win the argument.  It is a type of manipulation used in place of valid logic.

    
      There are several specifically emotional fallacies that I list separately in this book, because of their widespread use.  However, keep in mind that you can take any emotion, precede it with, “appeal to”, and you have created a new fallacy, but by definition, the emotion must be used in place of a valid reason for supporting the conclusion.
    \\

    
      Logical Form:
    \\

    
      Claim X is made without evidence.
    \\

    
      In place of evidence, emotion is used to convince the interlocutor that X is true.
    \\

    
      Example \#1:
    \\

    
      Power lines cause cancer.  I met a little boy with cancer who lived just 20 miles from a power line who looked into my eyes and said, in his weak voice, “Please do whatever you can so that other kids won’t have to go through what I am going through.”  I urge you to vote for this bill to tear down all power lines and replace them with monkeys on treadmills.
    \\

    
      Explanation: Notice the form of the example: assertion, emotional appeal, request for action (conclusion) -- nowhere is there any evidence presented.  We can all tear up over the image of a little boy with cancer who is expressing concern for others rather than taking pity on himself, but that has nothing to do with the assertion or the conclusion.
    \\

    
      Example \#2:
    \\

    
      There must be objective rights and wrongs in the universe.  If not, how can you possibly say that torturing babies for fun could ever be right?
    \\

    
      Explanation: The thought of people torturing babies for fun immediately brings up unpleasant images (in sane people).  The actual argument (implied) is that there are objective (universal) rights and wrongs (morality).  The argument is worded in such a way to connect the argument's conclusions (that there is objective morality) with the idea that torturing babies for fun is wrong (this is also a {\it non sequitur  fallacy}).  No matter how we personally feel about a horrible act, our feelings are not a valid substitution for an objective reason behind {\it why}  the act is horrible.
    \\

    
      Exceptions: Appealing to emotions is a very powerful and necessary technique in persuasion.  We are emotional creatures; therefore, we often make decisions and form beliefs erroneously based on emotions, when reason and logic tell us otherwise.  However, using appeals to emotion as a backup to rational and logical arguments is not only valid, but a skill possessed by virtually every great communicator. 
    \\

    
      Tip: By appealing to both the brain and the heart, you will persuade the greatest number of people.
    \\

  }
\subsection{argumentum in terrorem
    (also known as: argumentum ad metum, argument from adverse consequences, scare tactics, appeal to fear)
  
    Description:  When fear, not based on evidence or reason, is being used as the primary motivator to get others to accept an idea, proposition, or conclusion.

    
      Logical Form:
    \\

    
      If you don’t accept X as true, something terrible will happen to you.
    \\

    
      Therefore, X must be true.
    \\

    
      Example \#1:
    \\

    
      If we don’t bail out the big automakers, the US economy will collapse.  Therefore, we need to bail out the automakers.
    \\

    
      Explanation: The idea of a collapsed economy is frightening enough for many people to overlook the fact that this is a premise without justification, resulting in them just accepting the conclusion. There is no evidence or reason provided for the claim that if we don’t bail out the big automakers, the US economy will collapse.
    \\

    
      Example \#2:
    \\

    
      Timmy: Mom, what if I don’t believe in God?
    \\

    
      Mom: Then you burn in Hell forever.  Why do you ask?
    \\

    
      Timmy: No reason.
    \\

    
      Explanation: Timmy’s faith is waning, but Mom, like most moms, is very good at scaring the Hell, in this case, into, Timmy.  This is a fallacy because Mom provided no evidence that disbelief in God will lead to an eternity of suffering in Hell, but because the possibility is terrifying to Timmy, he “accepts” the proposition (to believe in God), despite the lack of actual evidence.
    \\

    
      Exception: When fear is not the primary motivator, but a supporting one and the probabilities of the fearful event happening are honestly disclosed, it would not be fallacious.
    \\

    
      Timmy: Mom, what if I don’t believe in God?
    \\

    
      Mom: Then I would hope that you don’t believe in God for the right reasons, and not because your father and I didn’t do a good enough job telling you why you should believe in him, including the possibility of what some believe is eternal suffering in Hell.
    \\

    
      Timmy: That’s a great answer mom.  I love you.  You are so much better than my mom in the other example.
    \\

    
      Tip: Think in terms of probabilities, not possibilities.  Many things are possible, including a lion busting into your home at night and mauling you to death -- but it is very, very improbable.  People who use fear to manipulate you, count on you to be irrational and emotional rather than reasonable and calculating.  Prove them wrong.
    \\

  }


 Fear, uncertainty, and doubt (FUD)
    
      - **Name**: Fear, Uncertainty, and Doubt (FUD)
    \\

    
      - **Also known as**: FUD
    \\

    
      - **Description**: FUD is a strategic tactic used to influence perception by spreading negative, misleading, or uncertain information. It aims to create fear, uncertainty, and doubt about a competitor or product, thereby affecting decisions and behaviors.
    \\

    
      - **Logical Form**: Presenting dubious or negative information with the intention of undermining confidence in a product, idea, or competitor.
    \\

    
      - **Example \#1**: A company spreads rumors about the security vulnerabilities of a competitor's software to deter potential customers from choosing it.
    \\

    
      - **Explanation**: By emphasizing the perceived risks associated with the competitor’s product, the company creates fear and uncertainty, making customers more likely to choose their own, supposedly safer, product.
    \\

    
      - **Example \#2**: An organization claims that a new technological innovation might lead to job losses and economic instability without solid evidence to support these claims.
    \\

    
      - **Explanation**: The organization uses these unfounded fears to create doubt about the benefits of the new technology, thereby discouraging its adoption and protecting their own interests.
    \\

    
      - **Variation**: Fear, Uncertainty, Doubt, and Deception (FUDD)
    \\

    
      - **Tip**: To counteract FUD, focus on providing clear, factual information and evidence to address and dispel the concerns raised.
    \\

    
      - **Exception**: FUD can be legitimate if it is based on credible information and genuine concerns. However, it is often used unethically to manipulate perceptions.
    \\

    
      - **Fun Fact**: The term FUD was originally coined in the 1970s by IBM employees to describe tactics used by IBM to discredit competitors and maintain its market dominance.
    \\

  

Appeal to prejudice
    
      (also known as: argumentum ad invidiam)
    \\

  
    
      - **Name**: Argumentum Ad Invidiam
    \\

    
      - **Also known as**: Appeal to Envy
    \\

    
      - **Description**: Argumentum ad invidiam is a logical fallacy where an argument is made by appealing to the audience's envy or resentment towards the target. Rather than addressing the merits of the argument itself, it seeks to undermine the target by exploiting negative emotions.
    \\

    
      - **Logical Form**: The argument suggests that the target should be discredited or dismissed because of the audience's envy or negative feelings towards them.
    \\

    
      - **Example \#1**: "You shouldn't trust his opinion on investment strategies—he's just a wealthy elitist who doesn’t understand the struggles of ordinary people."
    \\

    
      - **Explanation**: This statement dismisses the person's credibility based on envy towards their wealth rather than addressing the actual validity of their investment advice.
    \\

    
      - **Example \#2**: "Why should we listen to her arguments about environmental conservation? She's just a successful celebrity who lives in luxury and doesn't face the same issues we do."
    \\

    
      - **Explanation**: This example undermines the individual's arguments by appealing to the audience's resentment towards the celebrity’s wealth and lifestyle, rather than engaging with the content of her arguments.
    \\

    
      - **Variation**: Ad Hominem Envy
    \\

    
      - **Tip**: To avoid falling into the trap of Argumentum ad Invidiam, focus on evaluating the argument based on its own merits and evidence, rather than the personal attributes or status of the person presenting it.
    \\

    
      - **Exception**: It can be relevant to address the credibility of an argument if the personal attributes of the speaker directly impact the validity of their claims, but this should be done based on evidence, not envy.
    \\

    
      - **Fun Fact**: The term "ad invidiam" comes from Latin, where "invidia" means "envy," highlighting the role of negative emotions in this fallacy.
    \\

  

Just In Case Fallacy
    
      (also known as: worst case scenario fallacy)
    \\

  
    Description: Making an argument based on the worst-case scenario rather than the most probable scenario, allowing fear to prevail over reason.

    
      Logical Form:
    \\

    
      It would be a good idea to accept claim X since it is possible for event Y.
    \\

    
      Example \#1:
    \\

    
      Maury, you should really wear a helmet when playing chess.  You can easily get excited, fall off your chair, and crack your head open.
    \\

    
      Explanation: Every decision you make has both costs and benefits.  Fallacious arguments, like the one above, will attempt to get you to make a decision out of fear rather than reason, thus increasing the {\it perceived}  cost of choosing not to wear a helmet.  Of course, the cost of wearing a helmet while playing chess is peer ridicule of historic proportions.
    \\

    
      Example \#2:
    \\

    
      If Hell is real, then you would be wise to accept Christianity as true.
    \\

    
      Explanation: The attempt is to get you to make a decision out of fear rather than reason, thus increasing the {\it perceived} cost of not accepting Christianity as true.  There are many Christians who reject the idea of Hell and eternal torment by a perfectly loving God.  Plus, there are over a billion people who subscribe to the religion that believes worshiping anyone besides Allah will buy you a one-way ticket into the fiery pits of Hell.  Through reason, you can evaluate these choices and make a decision on reason, not on fear.
    \\

    
      Exception: When you can come to a reasonable conclusion that preparing for the worst-case scenario is the most economically sound course of action (as in cost-benefit—not necessarily financial), then the fallacy is not committed.
    \\

    
      Tip: Buying insurance or a warranty is not always a good idea—mitigating risk comes with costs, that are often not obvious.
    \\

  

Flag-waving

Appeal to flattery
    
      (also known as: apple polishing, wheel greasing, brown nosing, appeal to pride / argumentum ad superbiam, appeal to vanity)
    \\

  
    Description: When an attempt is made to win support for an argument, not by the strength of the argument, but by using flattery on those whom you want to accept your argument.  This fallacy is often the cause of people getting tricked into doing something they don’t really want to do.

    
      Logical Form:
    \\

    
      X is true.
    \\

    
      (flattery goes here)
    \\

    
      Therefore, X is true.
    \\

    
      Example \#1:
    \\

    
      You should certainly be the one who washes the dishes -- you are just so good at it!
    \\

    
      Explanation: You may be great at washing dishes, but that fact in itself is not a sufficient reason for you being the one to wash the dishes.  Is it necessary for someone as skilled at dish-washing as you to do the job, or is it a mindless job that anyone can do just fine?
    \\

    
      Example \#2:
    \\

    
      Salesguy: You should definitely buy this car.  You look so good in it -- you look at least ten years younger behind that wheel.
    \\

    
      Tamera: I’ll take it!
    \\

    
      Explanation: The comment about looking ten years younger just because of the car is obvious flattery and not a fact.  This would not qualify as a valid reason for making such a purchase.
    \\

    
      Exception: Sincere praise is not flattery and is universally appreciated[1].  However, even sincere praise in place of reason in an argument is a fallacy, unless the argument is directly related to the sincere praise.
    \\

    
      You are a stunningly beautiful girl -- you should be a model.
    \\

    
      Fun Fact: Flattery might get you somewhere, but it’s usually a place you don’t want to be.
    \\

  \subsection{Appeal to novelty
    
      (also known as: argumentum ad novitatem, appeal to the new, ad novitam [sometimes spelled as])
    \\

  
    Description: Claiming that something that is new or modern is superior to the status quo, based exclusively on its newness.

    
      Logical Form:
    \\

    
      X has been around for years now.
    \\

    
      Y is new.
    \\

    
      Therefore, Y is better than X.
    \\

    
      Example \#1: Two words: New Coke.
    \\

    
      Explanation: Those who lived through the Coca-Cola identity crises of the mid-eighties know what a mess it was for the company.  In fact, the “New Coke Disaster”, as it is commonly referred to, is literally a textbook example of attempting to fix what isn’t broken.  Coke’s main marketing ploy was appealing to the novelty, and it failed miserably -- even though more people (55\%) actually preferred the taste of New Coke, the old was “better”.
    \\

    
      Example \#2:
    \\

    
      Bill: Hey, did you hear we have a new operating system out now?  It is better than anything else out there because we just released it!
    \\

    
      Steve: What’s it called?
    \\

    
      Bill: Windows Vista!
    \\

    
      Steve: Sounds wonderful!  I can’t wait until all of your users install it on all their computers!
    \\

    
      Explanation: For anyone who went through the experience of Vista, this fallacy should hit very close to home.  You were most likely assuming that you were getting a superior product to your old operating system -- you were thinking “upgrade” when, in fact, those who stuck with the status quo (Windows XP) were much better off.
    \\

    
      Exception: There are obvious exceptions, like in claiming that your fresh milk is better than your month old milk that is now growing legs in your refrigerator.
    \\

    
      Tip: Diets and exercise programs/gadgets are notorious for preying on our desire for novelty.  Don’t be swayed by the “latest research” or latest fads.  Just remember this: burn more calories than you take in, {\it and you will lose weight}.
    \\

  }


Chronological snobbery

Appeal to pity
    
      (also known as: ad misericordiam, appeal to sympathy, appeal to victimhood [form of], argumentum ad misericordiam, the sob story, the Galileo argument)
    \\

  
    
      Description: The attempt to distract from the truth of the conclusion by the use of pity.
    \\

    
      Logical Forms:
    \\

    
      Person 1 is accused of Y, but person 1 is pathetic.
    \\

    
      Therefore, person 1 is innocent.
    \\

    
       
    \\

    
      X is true because person 1 worked really hard at making X true.
    \\

    
      Example \#1:
    \\

    
      I really deserve an “A” on this paper, professor.  Not only did I study during my grandmother’s funeral, but I also passed up the heart transplant surgery, even though that was the first matching donor in 3 years.
    \\

    
      Explanation: The student deserves an “A” for effort and dedication but, unfortunately, papers are not graded that way.  The fact that we should pity her has nothing to do with the quality of the paper written, and if we were to adjust the grade because of the sob stories, we would have fallen victim to the {\it appeal to pity}.
    \\

    
      Example \#2:
    \\

    
      Ginger: Your dog just ran into our house and ransacked our kitchen!
    \\

    
      Mary: He would never do that, look at how adorable he is with those puppy eyes!
    \\

    
      Explanation: Being pathetic does not absolve one from his or her crimes, even when he or she is a ridiculously-adorable puppy.
    \\

    
      Exception: Like any argument, if it is agreed that logic and reason should take a backseat to emotion, and there is no objective truth claim being made, but rather an opinion of something that should or should not be done, then it could escape the fallacy.
    \\

    
      Let's not smack Spot for ransacking the neighbor's kitchen—he's just too damn cute!
    \\

    
      Variation: The {\em appeal to victimhood} uses a form of pity to either establish the innocence of the victim or suggest the victim has the truth on their side. This is an application of the {\em halo effect}, where victims tend to be seen positively, therefore that which is associated with the victim (e.g., their innocence, or their claims) is also seen as positive (unreasonably and without evidence).
    \\

    
      Tip: Avoid pity in argumentation.  It is a clear indicator that you have weak evidence for your argument.
    \\

  

Appeal to hate\subsection{reductio ad ridiculum
    
      (also known as: argumentum ad absurdum, apagogical arguments, reduce to absurdity)
    \\

  
    
      
        Description: Presenting the argument in such a way that makes the argument look ridiculous, usually by misrepresenting the argument or the use of exaggeration.
      \\

      
        Logical Form:
      \\

      
        Person 1 claims that X is true.
      \\

      
        Person 2 makes X look ridiculous by misrepresenting X.
      \\

      
        Therefore, X is false.
      \\

      
        Example \#1:
      \\

      
        It takes faith to believe in God just like it takes faith to believe in the Easter Bunny -- but at least the Easter Bunny is based on a creature that actually exists!
      \\

      
        Explanation: Comparing the belief in God to belief in the Easter Bunny is an attempt at ridicule and not a good argument.  In fact, this type of fallacy usually shows desperation in the one committing the fallacy.
      \\

      
        Example \#2:
      \\

      
        Evolution is the idea that humans come from pond scum.
      \\

      
        Explanation: It is ridiculous to think that we come from pond scum, and it is not true.  It is more accurate to say that we come from exploding stars as every atom in our bodies was once in a star.  By creating a ridiculous and misleading image, the truth claim of the argument is overlooked.
      \\

      
        Exception: It is legitimate to use ridicule when a position is worthy of ridicule.  This is a risky proposition, however, because of the subjectiveness of what kind of argument is actually ridicule worthy.  As we have seen, misplaced ridicule can appear as a sign of desperation, but carefully placed ridicule can be a witty move that can work logically and win over an audience emotionally, as well.
      \\

      
        Matt: You close-minded fool!  Seeing isn’t believing, believing is seeing!
      \\

      
        Cindy: Does that go for the Easter Bunny as well, or just the imaginary beings of your choice?
      \\

      
        Tip: Do your best to maintain your composure when someone commits this fallacy at your expense.  Remember, they are the ones who have committed the error in reasoning.  Tactfully point it out to them.
      \\

    
    \chapter{
      Reductio ad Absurdum
    }
  
    
      
        Description: A mode of argumentation or a form of argument in which a proposition is disproven by following its implications logically to an absurd conclusion.  Arguments that use universals such as, “always”, “never”, “everyone”, “nobody”, etc., are prone to being reduced to absurd conclusions.  The fallacy is in the argument that could be reduced to absurdity -- so in essence, {\it reductio ad absurdum} is a technique to expose the fallacy.
      \\

      
        Logical Form:
      \\

      
        Assume P is true.
      \\

      
        From this assumption, deduce that Q is true.
      \\

      
        Also, deduce that Q is false.
      \\

      
        Thus, P implies both Q and not Q (a contradiction, which is necessarily false).
      \\

      
        Therefore, P itself must be false.
      \\

      
        Example \#1:
      \\

      
        I am going into surgery tomorrow so please pray for me.  If enough people pray for me, God will protect me from harm and see to it that I have a successful surgery and speedy recovery.
      \\

      
        Explanation: We first assume the premise is true: if “enough” people prayed to God for the patient's successful surgery and speedy recovery, then God would make it so.  From this, we can deduce that God responds to popular opinion.  However, if God simply granted prayers based on popularity contests, that would be both unjust and absurd.  Since God cannot be unjust, then he cannot both respond to popularity and not respond to popularity, the claim is absurd, and thus false.
      \\

      
        Example \#2:
      \\

      
        If everyone lived his or her life exactly like Jesus lived his life, the world would be a beautiful place!
      \\

      
        Explanation: We first assume the premise is true: if everyone lived his or her life {\it exactly} like Jesus lived his, the world would be a beautiful place.  If this were true, we would have 7 billion people on this earth roaming from town to town, living off the charity of others, preaching about God (with nobody listening). Without anyone creating wealth, there would be nobody to get charity from -- there would just be 7 billion people all trying to tell each other about God.  After a few weeks, everyone would eventually starve and die.  This world might be a beautiful place for the vultures and maggots feeding on all the Jesus wannabes, but far from a beautiful world from a human perspective.  Since the world cannot be both a beautiful place and a horrible place, the proposition is false.
      \\

      
        Exception: Be sure to see the {\it appeal to extremes}  fallacy.
      \\

    
    
      References:
    \\

    
      
        
      \\

      
        
          Eemeren, F. H. van, Garssen, B., \& Meuffels, B. (2009). {\it Fallacies and Judgments of Reasonableness: Empirical Research Concerning the Pragma-Dialectical Discussion Rules}. Springer Science \& Business Media.
        
      
    
  }


Proving too much
    
      - **Description:** Proving too much is a logical fallacy that occurs when an argument reaches a desired conclusion in such a way that this conclusion only becomes a special case or a corollary of a larger, absurd conclusion. The fallacy arises because, if the reasoning were valid, it would also apply to the absurd conclusion.
    \\

    
      - **Logical Form:** If an argument leads to an absurd or obviously false conclusion, then the reasoning must be flawed. The argument "proves too much" if it extends its reasoning to an unacceptable or absurd conclusion.
    \\

    
      - **Example \#1:** Gaunilo’s critique of Anselm’s ontological argument for the existence of God. Gaunilo argued that if Anselm's reasoning were valid, it would imply the existence of a perfect island, which is absurd.
    \\

    
      - **Explanation:** Gaunilo's argument showed that if the same logic used to prove the existence of God were applied to the concept of a perfect island, it would lead to an obviously ridiculous conclusion. This highlighted a flaw in Anselm's argument.
    \\

    
      - **Example \#2:** Henry Coppée’s argument against slavery. He argued that if one claims slavery is evil because it leads to situations where masters can violently abuse slaves, then by similar logic, marriage and parenthood are also evil due to the existence of domestic violence.
    \\

    
      - **Explanation:** Coppée used the same reasoning to draw an absurd conclusion, showing that the original argument against slavery was flawed. By demonstrating that the argument could lead to an equally unacceptable conclusion about marriage and parenthood, he highlighted the fallacy.
    \\

    
      - **Variation:** Reductio ad Absurdum, where the argument is shown to lead to an absurd or contradictory outcome to disprove the initial reasoning.
    \\

    
      - **Tip:** To avoid the fallacy of proving too much, ensure that your argument does not lead to implausible or ridiculous conclusions. Test your argument's validity by considering its broader implications.
    \\

    
      - **Exception:** In some cases, proving too much can be a valid strategy to demonstrate the absurdity of an opponent's argument, as long as the larger conclusion is indeed absurd and not just normatively opposed.
    \\

    
      - **Fun Fact:** The term "proving too much" is often used in philosophical debates and can be traced back to medieval and early modern discussions of logic and argumentation.
    \\

  \par \textbf{Pooh-pooh
    
      (Also Known As Dismissing an Argument, Ridiculing an Argument)
    \\

  
    
      - **Name:** Pooh-Poohing an Argument
    \\

    
      - **Description:** Pooh-poohing an argument is a rhetorical fallacy in which a speaker dismisses an argument as unworthy of serious consideration, often by ridiculing it rather than addressing its substance. It involves undermining the argument through derision rather than rational critique.
    \\

    
      - **Logical Form:** If an argument is dismissed as unworthy of consideration through ridicule or contempt, without engaging with its actual content, then the dismissal is a fallacy.
    \\

    
      - **Example \#1:** A politician dismisses a policy proposal by saying, "Only someone who’s out of touch with reality would propose something so silly," without addressing the actual merits or details of the proposal.
    \\

    
      - **Explanation:** The politician ridicules the proposal instead of evaluating its substance, thus avoiding a serious discussion of its merits or flaws.
    \\

    
      - **Example \#2:** An academic scoffs at a new theory by calling it "a laughable fantasy" and "utter nonsense," without providing a reasoned critique of the theory's claims or evidence.
    \\

    
      - **Explanation:** The academic uses derision to dismiss the theory rather than engaging with its arguments or evidence, thus avoiding a substantive critique.
    \\

    
      - **Variation:** Straw Man Fallacy, where an argument is misrepresented and then dismissed. Pooh-poohing often involves ridicule without engagement, while the straw man involves creating a distorted version of the argument.
    \\

    
      - **Tip:** To avoid pooh-poohing, address the argument’s substance with reasoned responses rather than dismissing it with ridicule or contempt. Engage with the actual points made rather than resorting to derision.
    \\

    
      - **Exception:** In some cases, dismissing an argument with ridicule can be a valid rhetorical strategy if the argument is genuinely trivial or absurd. However, this should be done with caution and clear reasoning.
    \\

    
      - **Fun Fact:** The term "pooh-pooh" is an onomatopoeic expression derived from the sound of contemptuous dismissal, historically representing the act of spitting or snorting in scorn. It reflects the act of dismissing something with a dismissive attitude.
    \\

  }


Big lie
    
      - **Description:** The "big lie" is a rhetorical technique or fallacy where a falsehood is presented as truth with such boldness and repetition that people begin to accept it as true. It relies on the scale of the lie rather than its veracity, exploiting the tendency of people to believe something if it is repeated often enough.
    \\

    
      - **Logical Form:** If a falsehood is stated with enough confidence and repetition, it may be accepted as truth by the public. The lie is so large that it appears too implausible to be false, and its repetition helps solidify its acceptance.
    \\

    
      - **Example \#1:** A political leader repeatedly claims that their country has the highest economic growth rate in the world, despite evidence showing otherwise. The boldness and frequent repetition of this claim eventually lead many to accept it as true.
    \\

    
      - **Explanation:** The leader’s repeated assertion, despite being false, is taken seriously by the public due to its sheer magnitude and frequency. The “big lie” becomes accepted as reality by those who hear it often enough.
    \\

    
      - **Example \#2:** A corporation advertises a product as having "unmatched effectiveness" and "revolutionary results" in curing a common ailment, despite scientific evidence showing it is no more effective than competing products. The repeated, grandiose claims cause many consumers to believe in its superiority.
    \\

    
      - **Explanation:** The corporation’s exaggerated and repeated claims make the product seem far superior to others, leading to widespread belief in its effectiveness despite evidence to the contrary.
    \\

    
      - **Variation:** Smaller Scale Lie, where the falsehood is not as grand or repeated. The big lie involves more significant deception and repetition, while smaller lies may not have the same level of impact.
    \\

    
      - **Tip:** To guard against falling for the big lie, critically evaluate the evidence and sources of information. Repetition and bold claims should not replace factual verification.
    \\

    
      - **Exception:** In some cases, a "big lie" can be a deliberate tactic to mislead or manipulate public opinion. However, not every repeated claim is a "big lie"; some may be honest errors or misjudgments.
    \\

    
      - **Fun Fact:** The term "big lie" is often associated with political propaganda and was notably used by Adolf Hitler in his book *Mein Kampf* to describe the use of large-scale lies to manipulate public perception.
    \\

  

Think of the children
    
      - **Description:** "Think of the children" is a rhetorical strategy where an argument is made by appealing to the audience’s emotions, particularly their concern for children. It is used to elicit a strong emotional reaction, often to persuade or dissuade people from a particular viewpoint or action by invoking a sense of moral responsibility or urgency regarding the welfare of children.
    \\

    
      - **Logical Form:** If an argument appeals to the emotional concern for children, it may be used to persuade others by evoking feelings of guilt, compassion, or moral duty, rather than by presenting rational evidence or logical reasoning.
    \\

    
      - **Example \#1:** A campaign against a proposed policy argues, "If this policy is implemented, it will harm countless children and their futures. We must reject it to protect them."
    \\

    
      - **Explanation:** The argument leverages emotional appeal to concern for children to persuade the audience against the policy, rather than addressing the policy’s actual merits or drawbacks.
    \\

    
      - **Example \#2:** A political figure opposes a new regulation by stating, "This regulation will hurt families and children, leading to increased hardship for the youngest members of our society."
    \\

    
      - **Explanation:** The political figure uses the emotional appeal of protecting children to argue against the regulation, focusing on the perceived negative impact on families and children rather than providing a detailed critique of the regulation itself.
    \\

    
      - **Variation:** Appeal to Emotion, where emotions are used more broadly to influence opinions. "Think of the children" specifically targets the emotional concern for children.
    \\

    
      - **Tip:** To critically evaluate arguments that appeal to emotions, such as "think of the children," assess whether the argument is based on sound evidence and logical reasoning or if it is solely designed to elicit an emotional response.
    \\

    
      - **Exception:** In some cases, appealing to the welfare of children can be a valid consideration if it is part of a well-reasoned argument supported by evidence. It becomes fallacious when it is used to distract from the lack of substantive evidence or reasoning.
    \\

    
      - **Fun Fact:** The phrase "think of the children" has become a popular catchphrase in discussions about emotional manipulation in arguments and is often used ironically to highlight when an argument relies excessively on emotional appeals rather than rational analysis.
    \\

  

In-group favoritism
    
      - **Description:** In-group favoritism refers to the tendency for individuals to preferentially support, favor, or give benefits to members of their own group (the in-group) over those from other groups (the out-group). This bias often leads to unfair treatment or unequal opportunities based on group membership rather than individual merit.
    \\

    
      - **Logical Form:** If individuals show preferential treatment or support for members of their own group over members of other groups, despite comparable qualifications or merits, then this behavior exemplifies in-group favoritism.
    \\

    
      - **Example \#1:** A manager consistently promotes employees from their own social circle or shared background, regardless of the qualifications of other candidates.
    \\

    
      - **Explanation:** The manager’s bias towards promoting people with whom they share a common background demonstrates in-group favoritism, potentially leading to unfair advantages for those within their social circle.
    \\

    
      - **Example \#2:** A sports team’s fans exhibit more positive attitudes and support towards players from their own country, while showing less enthusiasm or even negative attitudes towards players from other countries.
    \\

    
      - **Explanation:** The fans’ preference for players from their own country, despite the players’ comparable skills and performance, illustrates in-group favoritism based on national affiliation.
    \\

    
      - **Variation:** Out-Group Discrimination, where individuals show bias against those who are not part of their in-group. In-group favoritism involves positive bias towards the in-group, while out-group discrimination involves negative bias towards the out-group.
    \\

    
      - **Tip:** To mitigate in-group favoritism, consciously evaluate individuals based on their merits and qualifications rather than group membership. Encourage diverse perspectives and equitable treatment in decision-making processes.
    \\

    
      - **Exception:** In-group favoritism can sometimes be a natural human tendency, but it becomes problematic when it leads to systematic discrimination or unfair treatment. Awareness and deliberate efforts are required to ensure fair practices.
    \\

    
      - **Fun Fact:** In-group favoritism is a well-documented phenomenon in psychology and sociology and has been observed in various contexts, from workplace dynamics to national and ethnic identities. Studies have shown that even minimal group distinctions, such as arbitrary labeling, can trigger this bias.
    \\

  
    
      (Also Known As In-Group Bias, In-Group Preference)
    \\

  

Invented here
    
      (Also Known As Local Bias, Home Bias)
    \\

  
    
      - **Description:** The "invented here" fallacy is a cognitive bias where people favor or give preferential treatment to ideas, products, or innovations that originated locally or within their own country, often dismissing or undervaluing equally effective or superior solutions from outside sources.
    \\

    
      - **Logical Form:** If individuals or groups preferentially support or accept ideas and innovations based solely on their local origin rather than on their objective merits, this behavior exemplifies the "invented here" bias.
    \\

    
      - **Example \#1:** A company opts to use software developed locally despite superior and more cost-effective options available from international vendors, simply because the local option was developed "here."
    \\

    
      - **Explanation:** The company’s preference for the local software, regardless of its objective quality or cost-effectiveness compared to international options, demonstrates "invented here" bias by valuing origin over merit.
    \\

    
      - **Example \#2:** A country’s government funds a domestic research project at a higher rate than a similar international project, even though the international project has more advanced technology and greater potential impact.
    \\

    
      - **Explanation:** The government’s preference for funding the domestic project over the more promising international one illustrates "invented here" bias by prioritizing local origin over the project’s potential benefits and quality.
    \\

    
      - **Variation:** Out-of-Sight Bias, where ideas or innovations from distant or unfamiliar sources are undervalued. "Invented here" focuses on favoring local innovations, while out-of-sight bias concerns a general undervaluation of unfamiliar ideas.
    \\

    
      - **Tip:** To avoid falling into the "invented here" fallacy, evaluate ideas, products, and innovations based on their actual merits and effectiveness rather than their origin. Consider a broad range of options and be open to external contributions.
    \\

    
      - **Exception:** Local bias can be justified in cases where local products or solutions are designed to meet specific regional needs, support local economies, or adhere to local regulations. In such cases, local preference may be beneficial or necessary.
    \\

    
      - **Fun Fact:** The "invented here" bias is a common phenomenon in various sectors, including technology, politics, and consumer preferences. It reflects a natural tendency for people to favor what is familiar or close to home, even when external alternatives may offer greater advantages.
    \\

  

Island mentality

Appeal to loyalty
    
      (also known as: appeal to patriotism [form of])
    \\

  
    
      
        Description: When one is either implicitly or explicitly encouraged to consider loyalty when evaluating the argument when the truth of the argument is independent of loyalty. Alternatively, one considers loyalty in concluding that the argument is true, false, or not worth investigating.
      \\

      
        Logical Form:
      \\

      
        X is loyal to Y. \newline
Y makes false claim Z. \newline
Therefore, X accepts Z as true due to X’s loyalty to Y.
      \\

      
        Y makes false claim Z. \newline
It is implied that disagreeing with Y is disloyal. \newline
Therefore, X does not question claim Z out of loyalty.
      \\

      
        Example \#1: Cult leaders appear to have a magical level of influence over their followers. They can do no wrong, and anything they say must be true. This mindset is enforced by rewards and punishments related to loyalty. When Jim Jones claimed that hostile forces would convert captured children of the cult to fascism, no fact-checking was involved. Out of loyalty to the leader, Jones’ followers reasoned that suicide was a better alternative and “drank the Kool-Aid” (Flavor Aid). Nine hundred and nine inhabitants of Jonestown died of apparent cyanide poisoning. Three hundred and four of them were children.
      \\

      
        Example \#2: 
      \\

      
        Liberal Friend: Posts fake quote allegedly by a conservative. \newline
Me: Asks for a source because the quote is unbelievable. \newline
Liberal Friend: Refuses to seek the source because it “sounds like something [this conservative] would say.”
      \\

      
        Explanation: While my liberal friend did not insist the quote was true, he did refuse to investigate it further out of loyalty to his ideological position. In both politics and religion, people on social media uncritically accept or reject information based on loyalty to their ideology. These are implied arguments rather than explicit arguments. The implication is that because the information is in-line with/goes against the person’s ideology, it must be true/false. This is a form of confirmation bias that is applied to a specific argument.
      \\

      
        Exception: There is no fallacy when one claims to follow someone or support them out of loyalty; the fallacy is committed when loyalty is considered in their evaluation of a truth claim.
      \\

      
        Tip: Be loyal to truth and reason, even if it is seen as disloyalty to an ideology.
      \\

    
  

Not invented (t)here

Parade of horribles
    
      (Also Known As: Insular Thinking, Island Syndrome)
    \\

  
    
       **Description:** Island mentality refers to a mindset where individuals or groups adopt an inward-looking perspective, often characterized by isolationist attitudes and a focus on their own immediate concerns or interests, while neglecting broader or external perspectives and realities.
    \\

    
      - **Logical Form:** If individuals or groups focus exclusively on their own internal concerns or local context, disregarding broader or external viewpoints and influences, this behavior exemplifies island mentality.
    \\

    
      - **Example \#1:** A local community refuses to participate in regional or national initiatives, believing that only their local issues matter and ignoring the benefits of broader collaboration.
    \\

    
      - **Explanation:** The community’s focus on local concerns to the exclusion of regional or national perspectives reflects island mentality, as they prioritize their own immediate issues over potential wider benefits.
    \\

    
      - **Example \#2:** A company operates with a narrow focus on domestic markets while neglecting global trends and opportunities, resulting in missed chances for growth and innovation.
    \\

    
      - **Explanation:** The company’s inward-looking approach, where it fails to consider global market trends and opportunities, demonstrates island mentality by concentrating solely on domestic issues and ignoring the broader international landscape.
    \\

    
      - **Variation:** Global Thinking, where individuals or groups consider and incorporate broader or external perspectives. Island mentality involves narrow, insular thinking, while global thinking emphasizes a more expansive and inclusive approach.
    \\

    
      - **Tip:** To overcome island mentality, actively seek out and consider external perspectives and opportunities. Engage with broader networks, stay informed about global trends, and be open to collaboration beyond immediate circles.
    \\

    
      - **Exception:** In some cases, focusing on local issues or internal concerns may be necessary and beneficial, especially if immediate or specialized needs require attention. However, balance this with awareness of broader contexts to avoid complete isolation.
    \\

    
      - **Fun Fact:** Island mentality can often be observed in various contexts, from local communities and businesses to entire nations. The term evokes the imagery of being cut off from the wider world, similar to an isolated island, and highlights the limitations of an insular perspective.
    \\

  

Appeal to spite
    
      also known as: argumentum ad adium
    \\

  
    Description: Substituting spite (petty ill will or hatred with the disposition to irritate, annoy, or thwart) for evidence in an argument, or as a reason to support or reject a claim.

    
      Logical Form:
    \\

    
      Claim X is made.
    \\

    
      Claim X is associated with thing Y that people feel spite towards.
    \\

    
      Therefore, X is true / false.
    \\

    
      Example \#1:
    \\

    
      Aren't you tired of the political divisiveness in this country? Republicans know what they are talking about when it comes to immigration. Don't you agree?
    \\

    
      Explanation: This is a slick way of having someone agree with your claim. The arguer began by introducing a common idea that many people despise—political divisiveness (thing Y). Then, made a claim (claim X) in which the person would have to show political divisiveness to reject, in effect, causing the person to substitute spite in the idea of political divisiveness for reason.
    \\

    
      Example \#2:
    \\

    
      Jon: Why should I bother exercising while my spouse is on vacation stuffing her face with food.
    \\

    
      Explanation: The reasons for exercising are independent of the Jon's wife's actions. The claim here is that Jon should not bother exercising. The claim is associated with the idea that his wife is "stuffing her face with food" (something Jon feels spite towards). Jon concludes that he shouldn't exercise. If Jon were using reason rather than the emotion of spite, he would find another reason not to exercise—like the fact that he is too far behind on{\it  The Golden Girls}  reruns.
    \\

    
      Exception: This doesn't apply to emotional, relatively insignificant arguments.
    \\

    
      Sib: Dude, can you give me a ride to the mall?
    \\

    
      Eddie: You mean in my car about which you said it was "just slightly better than getting around on a drunk donkey"?
    \\

    
      Sib: Yea.
    \\

    
      Eddie. No. You are not worthy of a ride in my fine automobile.
    \\

    
      The claim is that Sib is not worthy of a ride in Eddie's car (an emotional/subjective claim). Although Eddie appeals to spite in his reason, he has the right to in this case.
    \\

    
      Tip: Be happy. Avoid spite in all of its forms.
    \\

  

Stirring symbols
    
      - **Name:** Stirring Symbols
    \\

    
      - **Description:** Stirring symbols refer to the use of powerful, emotionally charged symbols or images to evoke strong feelings and persuade an audience, often bypassing rational analysis. These symbols are designed to generate an emotional response rather than a logical or reasoned argument.
    \\

    
      - **Logical Form:** If an argument or message uses emotionally charged symbols or images to elicit strong feelings and influence opinion, rather than presenting rational evidence or reasoning, it employs stirring symbols.
    \\

    
      - **Example \#1:** A political campaign uses images of national flags and stirring music in ads to evoke patriotism and rally support, without discussing specific policies or issues.
    \\

    
      - **Explanation:** The campaign relies on emotional symbols such as the national flag and patriotic music to stir up feelings of national pride, aiming to persuade voters without addressing substantive policy details.
    \\

    
      - **Example \#2:** A charity uses heartbreaking images of suffering children and dramatic music in its advertisements to encourage donations, focusing on the emotional impact rather than providing detailed information about how the donations will be used.
    \\

    
      - **Explanation:** The charity's use of emotionally charged images and music aims to generate sympathy and urge action through emotional appeal, rather than presenting a detailed and rational case for why donations are needed.
    \\

    
      - **Variation:** Rational Appeal, where arguments are based on logical reasoning and evidence rather than emotional symbols. Stirring symbols focus on emotional impact, while rational appeal emphasizes reasoned argumentation.
    \\

    
      - **Tip:** When evaluating messages that use stirring symbols, look for substantive content and evidence behind the emotional appeal. Be cautious of decisions or opinions swayed primarily by emotional imagery rather than rational arguments.
    \\

    
      - **Exception:** Stirring symbols can be effective in raising awareness or mobilizing action in cases where immediate emotional engagement is necessary, such as in emergency appeals or advocacy campaigns. However, they should be complemented with substantive information to ensure informed decision-making.
    \\

    
      - **Fun Fact:** The use of stirring symbols has a long history in propaganda and marketing, dating back to ancient times. Iconic symbols and emotionally evocative imagery have been powerful tools for influencing public opinion and rallying support throughout history.
    \\

  
    
      (Also Known As Emotional Symbols, Symbolic Appeal)
    \\

  

Judgmental language
    
      (Also Known As: Evaluative Language, Pejorative Language)
    \\

  
    
      - **Description:** Judgmental language involves the use of words or phrases that express a personal judgment, evaluation, or criticism, often with a negative or biased tone. It is used to convey disapproval or to influence opinions by framing a subject in a particular, usually unfavorable, light.
    \\

    
      - **Logical Form:** If language is used to express personal judgments or criticisms, often with a negative or biased tone, rather than providing objective information, it constitutes judgmental language.
    \\

    
      - **Example \#1:** Referring to someone as "lazy" or "incompetent" without providing specific evidence or context for the evaluation.
    \\

    
      - **Explanation:** The terms "lazy" and "incompetent" are judgmental because they imply negative personal traits without objective support or detailed explanation, focusing on disparagement rather than constructive criticism.
    \\

    
      - **Example \#2:** Describing a policy as "flawed and misguided" rather than providing a detailed analysis of its shortcomings and offering constructive alternatives.
    \\

    
      - **Explanation:** Labeling the policy as "flawed and misguided" uses judgmental language by offering a negative evaluation without substantive critique or discussion of specific issues, thereby influencing opinions through evaluative rather than objective language.
    \\

    
      - **Variation:** Neutral Language, which avoids personal judgments and focuses on objective, descriptive information. Judgmental language involves evaluative terms and personal opinions, while neutral language aims for impartiality.
    \\

    
      - **Tip:** To avoid judgmental language, focus on providing objective evidence and constructive feedback. Use descriptive language and avoid terms that imply personal judgments or criticisms.
    \\

    
      - **Exception:** In some contexts, judgmental language can be appropriate when clearly stated as personal opinion or when providing subjective evaluations is necessary for the discussion. However, even in such cases, it should be balanced with objective evidence and constructive criticism.
    \\

    
      - **Fun Fact:** Judgmental language can significantly influence how arguments and opinions are perceived. Its use in media, politics, and everyday conversation often shapes public attitudes and can impact the effectiveness of communication by focusing on emotional responses rather than rational analysis.
    \\

  

Argumentum ad captandum vulgus
    
      - **Description:** Argumentum ad captandum vulgus is a rhetorical strategy that involves appealing to popular opinion or the desires of the general public to persuade or gain support. It leverages the emotional or popular appeal rather than logical reasoning or evidence to convince an audience.
    \\

    
      - **Logical Form:** If an argument is made by appealing to popular opinion or the emotions of the masses, rather than relying on logical reasoning or evidence, it is an example of argumentum ad captandum vulgus.
    \\

    
      - **Example \#1:** A politician promises significant benefits and perks to voters to gain their support in an election, focusing on what is popular or emotionally appealing rather than presenting detailed policy plans.
    \\

    
      - **Explanation:** The politician uses appealing promises to attract voter support by capitalizing on popular desires and emotions, rather than engaging with substantive policy issues or rational arguments.
    \\

    
      - **Example \#2:** An advertisement for a product highlights how many people use it and how well-loved it is, rather than providing factual information about its quality or effectiveness.
    \\

    
      - **Explanation:** The ad appeals to the product's popularity to persuade consumers, leveraging the idea that "everyone else" is using it, instead of presenting objective evidence or rational reasons for its effectiveness.
    \\

    
      - **Variation:** Argumentum ad Populum, where the appeal is to the general public's emotions or desires, but it can involve more general appeals to shared beliefs rather than specifically focusing on popular trends. Argumentum ad captandum vulgus specifically targets popular opinion.
    \\

    
      - **Tip:** To critically evaluate appeals to the masses, assess the argument based on logical reasoning and evidence rather than its popularity or emotional appeal. Consider whether the appeal addresses substantive issues or simply seeks to sway opinion through popularity.
    \\

    
      - **Exception:** In some cases, appealing to popular sentiment can be a valid strategy, especially when it aligns with genuine values or concerns of the audience. However, it should be balanced with factual information and rational arguments.
    \\

    
      - **Fun Fact:** The term "argumentum ad captandum vulgus" is Latin for "argument designed to capture the crowd," reflecting its origins in rhetorical strategies used in ancient Rome. It highlights how effective appeals to popular sentiment can be in persuasion.
    \\

  
    
      (Also Known As Appeal to the Masses, Popular Appeal)
    \\

  

Appeal to gravity
    
      - **Name:** Appeal to Gravity
    \\

    
      - **Description:** The appeal to gravity, also known as appeal to seriousness, is a rhetorical fallacy where an argument is made based on the gravity or seriousness of the situation rather than on its actual merits. It involves emphasizing the importance or severity of an issue to persuade others, often bypassing rational analysis or evidence.
    \\

    
      - **Logical Form:**
    \\

    
      P1: X asserts Y, but not in a serious way.
    \\

    
      P2: (unstated) All statements not asserted seriously are false.
    \\

    
      C: Y is false.
    \\

    
      - **Example \#1:** A leader argues that a particular policy must be implemented immediately because the situation is "critical" and "requires urgent action," without presenting detailed evidence or a rationale for why the policy is the best solution.
    \\

    
      - **Explanation:** The leader's emphasis on the urgency and critical nature of the situation aims to pressure others into supporting the policy without critically examining its merits or evidence.
    \\

    
      - **Example \#2:** A company insists on making a controversial change by claiming it is essential for "the future of the industry" and "the well-being of employees," while providing little information on the specific benefits or drawbacks of the change.
    \\

    
      - **Explanation:** The company uses the gravity of the situation to justify the change, focusing on its importance and impact rather than presenting a thorough analysis or justification.
    \\

    
      - **Variation:** Appeal to Emotion, where persuasion is based on eliciting emotional responses rather than on the gravity of the situation. Appeal to gravity specifically emphasizes seriousness, while appeal to emotion targets emotional reactions.
    \\

    
      - **Tip:** To evaluate arguments that use appeal to gravity, look for substantive evidence and logical reasoning. Ensure that the seriousness of the situation is supported by clear, rational arguments rather than being used to distract from a lack of evidence.
    \\

    
      - **Exception:** Emphasizing the gravity of a situation can be appropriate when it genuinely reflects the seriousness of the issue and is supported by evidence. However, it should be accompanied by detailed reasoning and evidence to avoid manipulation through emotional appeal.
    \\

    
      - **Fun Fact:** The appeal to gravity, or seriousness, is often used in high-stakes scenarios, such as crisis management or urgent policy decisions. Its effectiveness depends on balancing the perceived seriousness with solid evidence and rational arguments.
    \\

  
    
      (Also Known As: Appeal to Seriousness, Appeal to Authority)
    \\

  \subsection{Appeal to shame
    
      (also known as: appeal to mockery, the horse laugh, appeal to guilt, appeal to pride, argumentum ad superbium, ad hominem ridicule, appeal to humor, appeal to mockery, appeal to ridicule, horse laugh, refutation by caricature)
    \\

  
    
      - **Description:** The appeal to shame is a logical fallacy that occurs when an argument or statement is deemed "shameful" or "guilt-inducing" to discredit the position of the person making it. It relies on emotional pressure rather than rational argument to persuade by invoking a sense of shame or guilt.
    \\

    
      - **Logical Form:**
    \\

    
        - **Shameful Opinion:**
    \\

    
          - P1: X asserts Y.
    \\

    
          - P2: Y is shameful.
    \\

    
          - C: X is shameful.
    \\

    
        - **Shameful Person:**
    \\

    
          - P1: X asserts Y.
    \\

    
          - P2: X is shameful.
    \\

    
          - C: Y is false.
    \\

    
      - **Example \#1:** "Aren't you ashamed for supporting that policy? Only morally depraved individuals would endorse such things."
    \\

    
      - **Explanation:** This argument uses shame to discredit the person's support for the policy by implying that only those with poor moral character would support it, rather than engaging with the policy’s actual merits or drawbacks.
    \\

    
      - **Example \#2:** "Oh, you're only 13. You can't have a valid opinion on economics. You should be ashamed of even trying to discuss this topic."
    \\

    
      - **Explanation:** This argument dismisses the person's opinion on economics by attacking their age and implying that their youth is inherently shameful or disqualifying, rather than addressing the validity of their arguments.
    \\

    
      - **Variation:**
    \\

    
        - **Appeal to Ridicule:** Uses mockery or laughter to undermine an argument, focusing on making the subject a target of humor rather than engaging with the argument itself.
    \\

    
        - **Appeal to Pride:** The inverse of appeal to shame, where an argument appeals to one's sense of pride or honor, suggesting that holding a particular position is something to be proud of.
    \\

    
      - **Tip:** When encountering appeals to shame, focus on the substance of the argument rather than the emotional pressure. Evaluate the claims based on evidence and reasoning rather than being influenced by attempts to induce guilt or shame.
    \\

    
      - **Exception:** In some cases, pointing out moral or ethical failings can be relevant, but it should be done with substantive arguments and evidence rather than solely relying on inducing shame or guilt.
    \\

    
      - **Fun Fact:** The appeal to shame is often used in political discourse and social media, where quick and emotionally charged arguments can sometimes overshadow more reasoned and evidence-based discussions.
    \\

  }


Every Schoolboy Knows
    
      Description: This fallacy involves asserting that a particular point is so well-known and obvious that even a schoolboy would know it. The intent is to shame the audience into accepting the claim without question, as they would not want to appear ignorant of something supposedly known by even the least educated. This tactic bypasses logical argument and evidence, instead appealing to the audience's fear of seeming uninformed.
    \\

    
      Logical Form:
    \\

    
        1. Speaker makes a claim.
    \\

    
        2. Speaker asserts that "every schoolboy knows" the claim is true.
    \\

    
        3. Audience is shamed into accepting the claim without questioning it.
    \\

    
      Example \#1:
    \\

    
        - Scenario: "Every schoolboy knows that the earth is flat."
    \\

    
        - Explanation: The speaker implies that the flatness of the earth is such common knowledge that even children are aware of it. This shames the audience into not questioning the claim, despite the overwhelming evidence to the contrary.
    \\

    
      Example \#2:
    \\

    
        - Scenario: "Every schoolboy knows that vaccines cause autism."
    \\

    
        - Explanation: This claim leverages the fallacy by suggesting that the link between vaccines and autism is universally acknowledged, even by schoolchildren. The aim is to pressure the audience into accepting the claim without demanding scientific evidence, which is crucial given that this assertion is widely debunked.
    \\

    
      Tip: Be wary of arguments that appeal to what is supposedly "common knowledge," especially when such claims lack supporting evidence. Always seek out reliable information and verify facts independently, regardless of how obvious or well-known a claim is presented to be.
    \\

  
    
      (also known as: Argumentum ad verecundiam (Appeal to Shame), Appeal to Common Knowledge)
    \\

  

Argumentum ad fastidium
    
      (also known as: appeal to disgust, "argument from disgust. "wisdom of repugnance", "yuck factor")
    \\

  
    
      - **Description:** The argumentum ad fastidium is a logical fallacy that occurs when something is argued to be morally wrong or unacceptable based on its perceived grossness or disgusting nature. It also involves the reverse, where something is argued to be good simply because it is beautiful or pleasant.
    \\

    
      - **Logical Form:**
    \\

    
        - **Grossness Determines Wrongness:**
    \\

    
          - P1: X is gross.
    \\

    
          - P2: Gross things are wrong.
    \\

    
          - C: X is wrong.
    \\

    
        - **Beauty Determines Goodness:**
    \\

    
          - P1: X is beautiful.
    \\

    
          - P2: Beautiful things are good.
    \\

    
          - C: X is good.
    \\

    
      - **Example \#1:** "Homosexuality is wrong because it's disgusting," where disgust is used to argue against the moral acceptability of homosexuality.
    \\

    
      - **Explanation:** This example uses the feeling of disgust as a basis for moral judgment, implying that because homosexuality might be considered gross by some, it must be morally wrong, rather than providing rational arguments or evidence.
    \\

    
      - **Example \#2:** "Eating insects is gross, so it must be an unacceptable practice," where the disgust at eating insects is used to argue against the practice, ignoring cultural differences and potential benefits.
    \\

    
      - **Explanation:** The argument dismisses the practice of eating insects solely based on personal or cultural disgust, rather than considering the nutritional value or cultural context in which it is practiced.
    \\

    
      - **Variation:**
    \\

    
        - **Appeal to Ridicule:** Uses mockery or humor to discredit an argument by making it seem laughable or contemptible.
    \\

    
        - **Appeal to Beauty:** Argues that something is good or desirable simply because it is beautiful or pleasant, reversing the logic of disgust.
    \\

    
      - **Tip:** When encountering arguments based on disgust or beauty, focus on the substance of the argument rather than emotional responses. Evaluate claims based on rational reasoning and evidence rather than visceral reactions.
    \\

    
      - **Exception:** Disgust or aesthetic appreciation can be relevant in discussions of cultural practices or ethical issues, but these feelings should be supported by rational arguments and evidence, rather than being the sole basis for judgment.
    \\

    
      - **Fun Fact:** The term "wisdom of repugnance" was popularized by bioethicist Leon Kass, who argued that feelings of disgust could sometimes reflect deeper moral insights. However, this approach is controversial and often criticized for relying on emotional responses rather than reasoned analysis.
    \\

  

Style over substance
    
      (also known as: argument by slogan [form of], cliché thinking - or thought-terminating cliché, argument by rhyme [form of], argument by poetic language [form of])
    \\

  
    
      Description: When the arguer embellishes the argument with compelling language or rhetoric, and/or visual aesthetics. This comes in many forms as described below. “If it sounds good or looks good, it must be right!”
    \\

    
      Logical Form:
    \\

    
      Person 1 makes claim Y.
    \\

    
      Claim Y sounds catchy.
    \\

    
      Therefore, claim Y is true.
    \\

    
      Example \#1:
    \\

    
      A chain is only as strong as its weakest link.
    \\

    
      Explanation: Most applications of language, like the example above, are not taken literally, but figuratively.  However, even figurative language is a way to make an argument.  In this case, it might be used to imply that a team is no better than the least productive member of that team which is just not true.  Very often the “weakest links” fade away into the background and the strong players lead the team.
    \\

    
      Example \#2:
    \\

    
      It’s not a religion; it is a relationship.
    \\

    
      Explanation: “Yeah... wow, I can see that!” is the common response to a cliché that diverts critical thought by substitution of poetry, rhyme, or other rhetoric.  In fact, these are not arguments, but assertions absent of any evidence or reasons that rely on one's confusion of their emotional connection to language with the truth of the assertion.  Tell me {\it why} it’s not a religion.  Tell me what a relationship is exactly. 
    \\

    
      Do not accept information as truth because it sounds nice.
    \\

    
      Exception: Compelling language or rhetoric can be useful when used in addition to evidence or strong claims.
    \\

    
      Tip: Keep in mind that for every poetic saying there is another one with an opposite meaning.  They rarely ever make good arguments.
    \\

    
      Variations: The {\it argument by slogan} fallacy is when a slogan (catchy phrase) is taken as truth because it sounds good and we might be used to hearing it, e.g. “Coke is the real thing!”  Bumper stickers are great examples of {\it argument by slogan}: “Born Again? Excuse me for getting it right the first time.”
    \\

    
      {\it Cliché thinking} is the fallacy when sayings like, “leave no stone unturned”, are accepted as truth, regardless of the situation -- especially if taken literally.
    \\

    
      When poetic language is used in an argument as reason or evidence for the truth of the conclusion, the {\it argument by poetic language fallacy}  is committed.
    \\

    
      The {\it argument by rhyme} uses words that rhyme to make the proposition more attractive.  It works... don’t ask me how, but it does (“if it doesn’t fit, you must acquit”).  Rhymes tend to have quite a bit of persuasive power, no matter how false they might be.  The best defense against this kind of fallacious rhetoric is a good counter attack using the same fallacy.
    \\

    
      Whoever smelled it, dealt it!
    \\

    
      Whoever denied it, supplied it!
    \\

  

argumentum ad fidentia
    
      (also known as: against self-confidence)
    \\

  
    Description: Attacking the person’s self-confidence in place of the argument or the evidence.

    
      Logical Form:
    \\

    
      Person 1 claims that Y is true, but is person 1 really sure about that?
    \\

    
      Therefore, Y is false.
    \\

    
      Example \#1:
    \\

    
      Rick: I had a dream last night that I won the lottery!  I have \$1000 saved up, so I am buying 1000 tickets!
    \\

    
      Vici: You know, dreams are not accurate ways to predict the future; they are simply the result of random neurons firing.
    \\

    
      Rick: The last time I checked, you are no neurologist or psychologist, so how sure are you that I am not seeing the future?
    \\

    
      Vici: It’s possible you can be seeing the future, I guess.
    \\

    
      Explanation: Although Vici is trying to reason with his friend, Rick attempts to weaken Vici’s argument by making Vici more unsure of his position.  This is a fallacious tactic by Rick, and if Vici falls for it, fallacious reasoning on his part.
    \\

    
      Example \#2:
    \\

    
      Chris: You claim that you don’t believe in the spirit world that is all around us, with spirits coming in and out of us all the time.  How can you be sure this is not the case?  Are you 100\% certain?
    \\

    
      Joe: Of course not, how can I be?
    \\

    
      Chris: Exactly! One point for me!
    \\

    
      Joe: What?
    \\

    
      Explanation: This is a common fallacy among those who argue for the supernatural or anything else not falsifiable.  If Joe was not that reasonable of a thinker, then he might start to question the validity of his position, not based on any new counter evidence presented, but a direct attack on his self-confidence.  Fortunately for Joe, he holds no dogmatic beliefs and is perfectly aware of the difference between possibilities and probabilities (see also {\it appeal to possibility}).
    \\

    
      Exception: When one claims certainty for something where certainty is unknowable, it is your duty to point it out.
    \\

    
      Tip: Have confidence that you are probably or even very probably right, but avoid dogmatic certainty at all costs in areas where certainty is unknowable.
    \\

  

Appeal to Desperation
    Description: Arguing that your conclusion, solution, or proposition is right based on the fact that something must be done, and your solution is "something."

    
      Logical Form:
    \\

    
      Something must be done.
    \\

    
      X is something.
    \\

    
      Therefore, X must be done.
    \\

    
      Example \#1:
    \\

    
      These are desperate times, and desperate times call for desperate measures.  Therefore, I propose we exterminate all baby seals.  It is obvious that something must be done, and this is something.
    \\

    
      Explanation: No reason is given for why we should exterminate all baby seals.  Perhaps the reason is that they all have a virus that will spread to the human race and kill us all, perhaps exterminating all baby seals will leave more fish for humans, or perhaps exterminating all baby seals will be a way to put an end to the clubbing of baby seals—but without these or any other reasons given, we have nothing to go on except the desperation that something must be done.
    \\

    
      Example \#2:
    \\

    
      Chairman: We are out of money come Monday.  Any suggestions?
    \\

    
      Felix: I suggest we take what money we do have, and go to Disney World.
    \\

    
      Chairman: Any other suggestions?
    \\

    
      (silence)
    \\

    
      Chairman: Since there are no other suggestions, Disney World it is.
    \\

    
      Explanation: Desperate times don’t necessarily call for any measure over no measure.  Many times, no action is better than a bad action.  Blowing what money is left on over-priced soft drinks and what appears to be rotisserie ostrich legs, may not be a wise choice -- especially when investors are involved.
    \\

    
      Exception: At times, especially in situations where time is limited, taking some action will be better than taking no action, and in the absence of better reasoning, the best available reason might have to do.  However, a reason, no matter how poor, should still be given -- not simply a conclusion.
    \\

    
      Tip: Do your best to avoid situations of desperation where emotion very often takes the lead over reason.  Although not all desperate situations can be avoided, many can, by proper planning and foresight.
    \\

  

Appeal to Intuition
    
      (also known as: appeal to the gut)
    \\

  
    
      Description: Evaluating an argument based on "intuition" or "gut feeling" that is unable to be articulated, rather than evaluating the argument using reason.
    \\

    
      Logical Forms:
    \\

    
      Evidence is given for argument X.
    \\

    
      X doesn't match person 1's intuition or gut feeling.
    \\

    
      Therefore, argument X is rejected.
    \\

    
       
    \\

    
      Person 1 has a gut feeling about claim X.
    \\

    
      Therefore, claim X is true.
    \\

    
      Example \#1:
    \\

    
      Nick: Did you know that if the sun were just a few miles closer to Earth, we would burn up, or if it were just a few miles further away we would all freeze? It is like someone put the sun there just for us!
    \\

    
      Suzy: Actually, the distance of the sun from Earth varies from about 91 million miles to 94.5 million miles, depending on the time of year.
    \\

    
      Nick: That can't be right. The sun never appears a few million miles further away!
    \\

    
      Explanation: Besides Nick's flat out rejection of a fact, Nick is evaluating Suzy's refutation based on what feels wrong to him. Nick is abandoning the reasoning process.
    \\

    
      Example \#2:
    \\

    
      Maura: Stop wasting your money on those scratch-off lottery tickets. You know the odds are seriously stacked against you, don't you?
    \\

    
      Philip: I do, but I have a really good feeling about this next batch!
    \\

    
      Explanation: Maura makes a good argument as to why Phillip should not buy any more tickets, but Philip abandons the reasoning process and makes an appeal to his intuition.
    \\

    
      Exception: This doesn't include arguments where subjective feeling plays a significant role.
    \\

    
      Maureen: So who are you going to marry?
    \\

    
      Joanne: Martin. I have a much better feeling about my future with him than with Tony.
    \\

    
      Tip: Intuition can be defined as the sum of our experiences reflected in the feeling of knowledge that cannot be articulated. For example, if one has 30 years of experience as a firefighter, they may “know” when not to open a door in a burning building but not be able to explain why rationally. The problem is, in the moment, intuition is indistinguishable from imagination. When we get these feelings, if time permits, we should do what we can to back them up rationally.
    \\

  

Prejudicial Language
    
      (also known as: variant imagization)
    \\

  
    Description: Loaded or emotive terms used to attach value or moral goodness to believing the proposition.

    
      Logical Form:
    \\

    
      Claim A is made using loaded or emotive terms.
    \\

    
      Therefore, claim A is true.
    \\

    
      Example \#1:
    \\

    
      All good Catholics know that impure thoughts are the work of the devil, and should be resisted at all costs.
    \\

    
      Explanation: The phrase “all good Catholics” is the loaded or prejudicial language being used.  The implication is that Catholics who {\it don’t}  resist impure thoughts are “bad Catholics”, which is not fair -- they may just not be as strong willed, or perhaps they don’t agree with the Church's views on sex.
    \\

    
      Example \#2:
    \\

    
      Students who want to succeed in life will do their homework each and every night.
    \\

    
      Explanation: The assertion is that students who{\it  don’t}  do their homework every night {\it don’t} want to succeed in life, which is bad reasoning.  Perhaps the student is sick one night, tired, doesn’t understand the work, or was busy making out with his father’s secretary in the office supply closet next to the memo pads.  The point is, dad, you cannot assume that just because I skipped homework a few nights that it means I didn’t want to succeed in life!
    \\

    
      Exception: This is often used for motivation, but even if the intent is honorable, it is still fallacious.
    \\

    
      Tip: {\em Prejudicial language} can be a powerful and effective persuasion tool. Use it in addition to a well-reasoned argument, not in place of one.
    \\

    References:

    
      
        
      \\

      
        
          Damer, T. E. (2008). {\it Attacking Faulty Reasoning: A Practical Guide to Fallacy-Free Arguments}. Cengage Learning.
        
      
    
  \subsection{Love bombing
    
      - **Description:** Love bombing is a tactic used to influence or manipulate a person by overwhelming them with excessive attention, affection, and flattery. While it can be employed positively or negatively, it is often associated with abusive behavior or psychological manipulation. It involves showering someone with praise and emotional support to create a sense of dependency or to gain control over them.
    \\

    
      - **Logical Form:**
    \\

    
        - **Manipulative Use:**
    \\

    
          - P1: X provides excessive attention and affection to Y.
    \\

    
          - P2: Excessive attention and affection are used to influence or control Y.
    \\

    
          - C: X is using love bombing to manipulate Y.
    \\

    
        - **Benign Use:**
    \\

    
          - P1: X provides excessive attention and affection to Y.
    \\

    
          - P2: There is no pattern of abuse or manipulation following the affection.
    \\

    
          - C: X is not engaging in love bombing but is simply showing care.
    \\

    
      - **Example \#1:** A new romantic partner showers someone with constant attention, extravagant gifts, and declarations of love within a short period, pressuring them to commit quickly to the relationship.
    \\

    
      - **Explanation:** This example illustrates how love bombing can be used to accelerate relationship commitments and create a sense of dependency, often masking underlying manipulative intentions.
    \\

    
      - **Example \#2:** A parent uses love bombing techniques by dedicating extensive one-on-one time, indulging every wish of their troubled child, and providing lavish attention to improve the child's behavior and emotional state.
    \\

    
      - **Explanation:** Here, love bombing is applied in a positive context as a technique to enhance the child's emotional well-being, without the intention of manipulation or control.
    \\

    
      - **Variation:**
    \\

    
        - **Manipulative Affection:** Uses affection and attention to gain control or manipulate, often followed by a cycle of idealization, devaluation, and discard.
    \\

    
        - **Romantic Courtship:** Involves genuine affection and attention without subsequent abusive patterns or manipulation.
    \\

    
      - **Tip:** To distinguish between love bombing and genuine affection, observe the long-term behavior and intentions. Love bombing typically involves excessive attention followed by manipulative or controlling behaviors, whereas genuine affection is consistent and free of ulterior motives.
    \\

    
      - **Exception:** Love bombing is not necessarily harmful if it is a sincere expression of care without any underlying manipulative intent or patterns of abuse. The context and subsequent actions are key to determining its nature.
    \\

    
      - **Fun Fact:** The term "love bombing" originated in the 1970s among members of the Unification Church and has since been used to describe various forms of excessive and manipulative affection.
    \\

  
    
      (Also Known As: Attention Bombing, Affection Bombing)
    \\

  }


Milieu control
    
      - **Description:** Milieu control refers to tactics used to regulate and manipulate communication and environment within a group. This includes the use of social pressure, group-specific language, dogma, protocols, and other means to foster group identity, limit external influence, and promote cognitive changes in members. Originally used to describe brainwashing and mind control, the concept has been applied more broadly to various contexts of social and psychological manipulation.
    \\

    
      - **Logical Form:**
    \\

    
        - **Group-Controlled Communication:**
    \\

    
          - P1: X uses specific language, protocols, and social pressure within a group.
    \\

    
          - P2: These tactics restrict members' communication and exposure to outside viewpoints.
    \\

    
          - C: X controls the cognitive environment and communication of the group.
    \\

    
      - **Example \#1:** A religious cult enforces unique jargon, rituals, and dress codes, while discouraging contact with outsiders and labeling them as morally inferior.
    \\

    
      - **Explanation:** This example demonstrates how milieu control is used to create a strong group identity and isolate members from external influences, reinforcing the group’s doctrines and limiting exposure to alternative perspectives.
    \\

    
      - **Example \#2:** A totalitarian regime imposes strict media censorship, controls educational content, and uses state-sanctioned language to influence public opinion and suppress dissent.
    \\

    
      - **Explanation:** The regime’s control over information and communication channels exemplifies milieu control by restricting the public’s access to diverse viewpoints and promoting state-approved ideologies.
    \\

    
      - **Variation:**
    \\

    
        - **Social Isolation:** Restricting members' contact with outsiders to reinforce group cohesion and control.
    \\

    
        - **Language Manipulation:** Using specific jargon and terminology to distinguish between insiders and outsiders.
    \\

    
      - **Tip:** To identify milieu control, look for signs of restricted communication, group-specific language, and systematic isolation from external viewpoints. These tactics are often used to reinforce group ideology and limit critical thinking.
    \\

    
      - **Exception:** Milieu control is not necessarily harmful if it’s used to enhance positive group cohesion without restricting members' ability to engage with external perspectives or make independent judgments.
    \\

    
      - **Fun Fact:** The concept of milieu control was popularized by psychiatrist Robert Jay Lifton in the context of brainwashing and mind control techniques, but it has since been applied to various forms of social influence and organizational behavior.
    \\

  
    
      (Also Known As: Environment Control, Group Control, Cognitive Control)
    \\

  

 victim playing
    
      (also known as: victim card, playing the victim, self-victimization, professional victim, victim mentality, Emotional Manipulation)
    \\

  
    
      - **Description:** Victim playing is a behavior where an individual presents themselves as a victim to gain sympathy, avoid responsibility, or manipulate others. It involves exaggerating or fabricating distress or harm to elicit a specific response or outcome from others, often to deflect blame or secure special treatment.
    \\

    
      - **Logical Form:**
    \\

    
        - **Manipulative Use:**
    \\

    
          - P1: X claims to be a victim of circumstance or mistreatment.
    \\

    
          - P2: X's claims are exaggerated or untrue.
    \\

    
          - C: X is using victim playing to manipulate others or avoid responsibility.
    \\

    
      - **Example \#1:** An employee consistently blames their underperformance on being unfairly targeted by their supervisor, despite evidence of their own lack of effort.
    \\

    
      - **Explanation:** In this case, the employee uses victim playing to shift focus from their own shortcomings to an external supposed injustice, aiming to garner sympathy and avoid accountability for their performance.
    \\

    
      - **Example \#2:** A person frequently claims that everyone in their social circle is out to get them or is conspiring against them, in order to manipulate friends into giving them special favors or support.
    \\

    
      - **Explanation:** Here, the individual uses victim playing to create a sense of solidarity or obligation among their friends, leveraging perceived victimhood to gain advantages or benefits.
    \\

    
      - **Variation:**
    \\

    
        - **False Victimization:** Fabricating stories of abuse or neglect to gain sympathy or attention.
    \\

    
        - **Exaggerated Victimhood:** Overemphasizing real but minor grievances to elicit a stronger emotional response from others.
    \\

    
      - **Tip:** To identify victim playing, observe if the individual frequently portrays themselves as a victim in situations where evidence suggests their claims are overstated or self-serving. Look for patterns of using victimhood to avoid responsibility or manipulate others.
    \\

    
      - **Exception:** Genuine victimhood involves real suffering and should be treated with empathy and support. Distinguishing between genuine cases and victim playing requires careful consideration of the context and the individual's behavior.
    \\

    
      - **Fun Fact:** The concept of victim playing is often used in discussions of psychological manipulation and interpersonal dynamics, highlighting how people may use emotional strategies to influence others' perceptions and behaviors.
    \\

  

What's the harm\section{
    
      {\bf Genetic fallacies}
    \\

  
    
      (also known as: fallacy of origins, fallacy of virtue, Genetic Fallacy)
    \\

  
    Description: Basing the truth claim of an argument on the origin of its claims or premises.

    
      Logical Form:
    \\

    
      The origin of the claim is presented.
    \\

    
      Therefore, the claim is true/false.
    \\

    
      Example \#1:
    \\

    
      Lisa was brainwashed as a child into thinking that people are generally good.  Therefore, people are not generally good.
    \\

    
      Explanation: That fact that Lisa may have been brainwashed as a child, is irrelevant to the claim that people are generally good.
    \\

    
      Example \#2:
    \\

    
      He was born to Catholic parents and raised as a Catholic until his confirmation in 8th grade.  Therefore, he is bound to want to defend some Catholic traditions and, therefore, cannot be taken seriously.
    \\

    
      Explanation: I am referring to myself here.  While my upbringing was Catholic, and I have long since considered myself a Catholic, that is irrelevant to any defenses I make of Catholicism -- like the fact that many local churches do focus on helping the community through charity work.  If I make an argument defending anything Catholic, the argument should be evaluated on the argument itself, not on the history of the one making the argument or how I came to hold the claims as true or false.
    \\

    
      Exception: At times, the origin of the claim is relevant to the truth of the claim. 
    \\

    
      I believe in closet monsters because my big sister told me unless I do whatever she tells me, the closet monsters will eat me.
    \\

    
      Tip: Remember that considering the source is often a useful heuristic in quickly assessing if the claim is probably true or not, but dismissing the claim or accepting it as true based on the source is fallacious.
    \\

    References:

    
      
        
      \\

      
        
          Engel, S. M., Soldan, A., \& Durand, K. (2007). {\it The Study of Philosophy}. Rowman \& Littlefield.
        
      
    
  }
\subsection{argumentum ad hominem
    
      (also known as: The Fallacy of Personal Attack, Argumentum ex concessis)
    \\

  }
\par \textbf{Association fallacy
    
      (also known as: guilt by association, Bad Company Fallacy, The Company that You Keep Fallacy)
    \\

  
    Description: When the source is viewed negatively because of its association with another person or group who is already viewed negatively.

    
      Logical Form:
    \\

    
      Person 1 states that Y is true.
    \\

    
      Person 2 also states that Y is true, and person 2 is a moron.
    \\

    
      Therefore, person 1 must be a moron too.
    \\

    
      Example \#1:
    \\

    
      Delores is a big supporter for equal pay for equal work.  This is the same policy that all those extreme feminist groups support.  Extremists like Delores should not be taken seriously -- at least politically.
    \\

    
      Explanation: Making the assumption that Delores is an extreme feminist simply because she supports a policy that virtually every man and woman also supports, is fallacious.
    \\

    
      Example \#2:
    \\

    
      Pol Pot, the Cambodian Maoist revolutionary, was against religion, and he was a very bad man.  Frankie is against religion; therefore, Frankie also must be a very bad man.
    \\

    
      Explanation: The fact that Pol Pot and Frankie share one particular view does not mean they are identical in other ways unrelated, specifically, being a very bad man.  Pol Pot was not a bad man {\it because}  he was against religion, he was a bad man for his genocidal actions.
    \\

    
      Example \#3:
    \\

    
      {\em Callie: Did you know that Jake Tooten was a racist?} \newline
{\em Chris: I know Jake well. Why do you say he’s a racist?} \newline
{\em Callie: He was on a podcast the other day...} \newline
{\em Chris: Did he say something racist?} \newline
{\em Callie: No, but the podcast host did an interview two years ago with a woman who said she supported an organization that had a history of racism back in the 1960s. Jake clearly supports racism!}
    \\

    
      Explanation: In this case, Jake Tooten is the “source” who is viewed negatively (as a racist) because of his association with the organization referenced. Note that his “association” with this group says nothing about his beliefs, which makes this fallacious. It is even more fallacious due to 1) Jake being several steps removed from this organization and 2) the organization’s history of racism rather than the organization’s current position on racism.
    \\

    
      Callie actually was right, but for the wrong reason. Jake runs a Nazi youth group.
    \\

    
      Exception: If one can demonstrate that the connection between the two characteristics that were inherited by association is causally linked, or the probability of taking on a characteristic would be high, then it would be valid. In example \#1, if Delores supported the “all men should be castrated” position, we can call her an “extremist.” In example \#2, if we used “murdered children” instead of “against religion,” the claim of being a “very bad man” would be justified. In example \#3, if Jake appeared on the podcast titled “I am racist, and you should be too,” Callie’s claim of Jake being a racist would be justified.
    \\

    
      Tip: People change. Be forgiving of one’s questionable past associations, especially if they realize and admit those associations were wrong.
    \\

  }


Godwin law
    
      (also known as: Reductio ad Hitlerum, reductio ad racism, Red-baiting, reductio ad Stalinum, Gore's Law, playing the Nazi card, Hitler card, Argumentum ad Nazium, Godwin's Rule, Godwin's Law of Nazi Analogies)
    \\

  
    
      - **Description:** Godwin's Law is an internet adage that asserts that as an online discussion grows longer, the probability of a comparison involving Nazis or Hitler approaches 1. In other words, as discussions progress, someone is likely to make a comparison to the Nazis or Hitler, often regardless of the topic or relevance.
    \\

    
      - **Logical Form:**
    \\

    
        - **P1:** In any extended online discussion, the likelihood of a Nazi or Hitler comparison increases.
    \\

    
        - **P2:** Once such a comparison is made, the discussion is often derailed or becomes less constructive.
    \\

    
        - **C:** Over time, online discussions are prone to devolving into comparisons with Nazis or Hitler.
    \\

    
      - **Example \#1:** In a debate about healthcare policy, a participant compares their opponent's viewpoint to Nazi Germany's medical ethics.
    \\

    
      - **Explanation:** This example illustrates Godwin's Law by showing how an unrelated discussion about healthcare policy ends up with a comparison to Nazis, which can derail the conversation and reduce its constructive value.
    \\

    
      - **Example \#2:** During a discussion about internet privacy, someone asserts that any restriction on online freedom is akin to the censorship seen in totalitarian regimes like Nazi Germany.
    \\

    
      - **Explanation:** Here, the reference to Nazi censorship in a discussion about internet privacy exemplifies Godwin's Law by introducing an extreme and historical comparison that shifts focus away from the original topic.
    \\

    
      - **Variation:**
    \\

    
        - **Hitler's Rule:** Similar to Godwin's Law, it focuses specifically on comparisons to Hitler rather than Nazis in general.
    \\

    
        - **Godwin’s Law of Analogies:** A broader interpretation that applies to any extreme or irrelevant analogy in discussions.
    \\

    
      - **Tip:** To avoid invoking Godwin's Law, focus on relevant, specific arguments and evidence rather than making extreme historical comparisons. This helps maintain the discussion's relevance and constructive nature.
    \\

    
      - **Exception:** In discussions where the topic genuinely involves totalitarian regimes or historical comparisons with Nazi Germany, such references may be relevant and not fall under Godwin's Law.
    \\

    
      - **Fun Fact:** Godwin's Law was coined by Mike Godwin in 1990 as an observation of online debate patterns, and it has since become a well-known principle in discussions about internet discourse and logical fallacies
    \\

  

Galileo Fallacy
    
      (also known as: Galileo argument, Galileo defense, Galileo gambit, Galileo wannabe)
    \\

  
    Description: The claim that because an idea is forbidden, prosecuted, detested, or otherwise mocked, it must be true, or should be given more credibility. This originates from Galileo Galilei's famous persecution by the Roman Catholic Church for his defense of heliocentrism when the commonly accepted belief at the time was an earth-centered universe.

    
      Logical Form:
    \\

    
      Claim X is made.
    \\

    
      Claim X is ridiculous.
    \\

    
      Person A argues that claim Y was seen as ridiculous at the time, and it turned out to be right.
    \\

    
      Therefore, claim X is true (or should be given more credibility).
    \\

    
      Example \#1:
    \\

    
      Lindi and Jonah claim that Elvis is still alive and living on the planet Hounddogian, in the constellation Bluesuede. When questioned about their odd beliefs, Lindi and Jonah confidently reply, "You know, people thought Galileo was nuts, too."
    \\

    
      Explanation: Lindi and Jonah are making an extraordinary claim and offering no evidence to support their claim. They are using Galileo in an attempt to get the audience to doubt their skepticism about the claim.
    \\

    
      Example \#2:
    \\

    
      Sidney: I am mere weeks away from getting my time machine to work, at which time, I will go back to 1626 and buy Manhattan from the Native Americans before the Dutch West India Company gets their greedy hands on it. I'll be much more generous and give the Native Americans 70 guilders, not a measly 60.
    \\

    
      Pete: Is this the time travel kit you bought online for \$99.99?
    \\

    
      Sidney: Go ahead and mock me. People mocked the Wright brothers too for wanting to fly like birds!
    \\

    
      Explanation: Although Sidney did not use the exact example of Galileo, the fallacy is the same. Any reference to a similar story counts.
    \\

    
      Exception: Using Galileo or similar success stories to serve as effective inspirational anecdotes to encourage people to reach outside their comfort zone is not fallacious. It does not mean, however, that because they succeeded, that everyone else will or even can.
    \\

    
      Tip: Remember that for every Galileo, there are millions of cranks, quacks, and wackos, and statistically speaking, those who use the Galileo defense are one of the latter.
    \\

  

Poisoning the well
    
      (also known as: discrediting, smear tactics, appeal to ethos [form of])
    \\

  
    Description: To commit a preemptive {\it ad hominem (abusive)}  attack against an opponent.  That is, to prime the audience with adverse information about the opponent from the start, in an attempt to make your claim more acceptable or discount the credibility of your opponent’s claim.

    
      Logical Form:
    \\

    
      Adverse information (be it true or false) about person 1 is presented.
    \\

    
      Therefore, the claim(s) of person 1 will be false.
    \\

    
      Example \#1:
    \\

    
      Tim: Boss, you heard my side of the story why I think Bill should be fired and not me.  Now, I am sure Bill is going to come to you with some pathetic attempt to weasel out of this lie that he has created.
    \\

    
      Explanation: Tim is {\it poisoning the well} by priming his boss by attacking Bill’s character, and setting up any defense Bill might present as “pathetic”.  Tim is using this fallacious tactic here, but if the boss were to accept Tim’s advice about Bill, she would be committing the fallacy.
    \\

    
      Example \#2:
    \\

    
      I hope I presented my argument clearly.  Now, my opponent will attempt to refute my argument by his own fallacious, incoherent, illogical version of history.
    \\

    
      Explanation: Not a very nice setup for the opponent.  As an audience member, if you allow any of this “poison” to affect how you evaluate the opponent’s argument, you are guilty of fallacious reasoning.
    \\

    
      Exception: Remember that if a person states facts relevant to the argument, it is not an {\it ad hominem (abusive)} attack.  In the first example, if the other “poison” were left out, no fallacy would be committed.
    \\

    
      Tim: Boss, you heard my side of the story why I think Bill should be fired and not me.  Now, I am sure Bill is going to come to you with his side of the story, but please keep in mind that we have two witnesses to the event who both agree that Bill was the one who told the client that she had ugly children.
    \\

    
      Variation: The {\em appeal to ethos} involves rejection of an argument based on a character attack of the person making the argument.
    \\

    
      {\em Gertie: Tony says that the movie starts at 8:00 tonight.} \newline
{\em Jane: Well, Tony is misogynist pig!} \newline
{\em Gertie: Hmm, we better double check that time then. }
    \\

    
      Fun Fact: To understand how powerful priming the audience with adverse information can be, consider the Rosenhan experiment where eight mentally healthy students and researchers briefly feigned auditory hallucinations in order to get admitted to psychiatric hospitals. After admission, they said they were no longer having hallucinations and acted normally. One of the patients, who was also a student, was taking notes for the experiment which was interpreted as pathological “writing behavior” by one of the hospital staff. 
    \\

    References:

    
      
        
      \\

      
        
          Walton, D. (1998). {\it Ad Hominem Arguments}. University of Alabama Press.
        
      
    
  \par \textbf{Tone policing
    
      (also known as: Tone argument)
    \\

  }


Political Correctness Fallacy
  
    Description: This is a common one in recent history.  It is the assumption or admission that two or more groups, individuals, or ideas of groups or individuals, are equal, of equal value, or both true, based on the recent phenomenon of political correctness, which is defined as, {\it a term which denotes language, ideas, policies, and behavior seen as seeking to minimize social and institutional offense in occupational, gender, racial, cultural, sexual orientation, certain other religions, beliefs or ideologies, disability, and age-related contexts, and, as purported by the term, doing so to an excessive extent.}

    
      This can be seen as an over-correction of{\it  stereotyping (the fallacy). \newline
}
    \\

    
      Logical Form:
    \\

    
      Claim A is politically incorrect.
    \\

    
      Therefore, claim A is false.
    \\

    
      Example \#1:
    \\

    
      Racial/cultural profiling at airports is wrong.  An adult, middle-eastern male is just as likely to be a terrorist as a four-year-old American girl.
    \\

    
      Explanation: While many things are possible, including a four-year-old American girl being a terrorist, profiling works on probabilities. Inserting political correctness here goes against reason in asserting that every person is just as likely to be a terrorist.
    \\

    
      Example \#2:
    \\

    
      The masked individual who committed the crime was about 6’2”, and took down four male security guards by hand.  It is just as likely that the criminal was a woman.
    \\

    
      Explanation: While it is certainly possible that a 6’2” female martial-arts master is the criminal, it is highly unlikely, and it would be a waste of resources to question an even number of men and women based on the desire not to discriminate.
    \\

    
      Example \#3:
    \\

    
      Everyone is entitled to his or her own religious beliefs.  So if dancing in the streets naked is part of their ritual, we must extend them that right.
    \\

    
      Explanation: Are any and all religiously-based behaviors acceptable?  Must we allow all expression of religion?  Where do we draw the line and why?
    \\

    
      Example \#4:
    \\

    
      Sacrificing virgins is part of that tribe's culture and heritage.  Therefore, it is just as acceptable as our culture’s tradition of eating a hot dog at a baseball game.
    \\

    
      Explanation: Here we enter the realm of morality and choose to protect a “cultural belief” over saving the life of a young girl. 
    \\

    
      These examples—and this fallacy—are very controversial.  Like all fallacies, arguments need to be made.  I am making an argument that PC can be a fallacy in many cases.  You might agree, you might disagree.  In either case, be prepared to argue for your position with valid reasons.
    \\

    
      Exception: See above.
    \\

    
      Tip: At its core, being “PC” is the belief that minimizing social and institutional offense is a kind and compassionate thing to do. Don’t confuse this with the fallacy of determining the truth of a claim based on its perceived political correctness. Reality doesn’t care about social and institutional offense.
    \\

  \par \textbf{Tu quoque
    
      (also known as: appeal to hypocrisy, thou too fallacy, you too fallacy, hypocrisy, personal inconsistency)
    \\

  
    Description: Claiming the argument is flawed by pointing out that the one making the argument is not acting consistently with the claims of the argument.

    
      Logical Form:
    \\

    
      Person 1 is claiming that Y is true, but person 1 is acting as if Y is not true.
    \\

    
      Therefore, Y must not be true.
    \\

    
      Example \#1:
    \\

    
      Helga: You should not be eating that... it has been scientifically proven that eating fat burgers are no good for your health.
    \\

    
      Hugh: You eat fat burgers all the time so that can’t be true.
    \\

    
      Explanation: It doesn’t matter (to the truth claim of the argument at least) if Helga follows her own advice or not.  While it might appear that the reason she does not follow her own advice is that she doesn’t believe it’s true, it could also be that those fat burgers are just too damn irresistible.
    \\

    
      Example \#2:
    \\

    
      Jimmy Swaggart argued strongly against sexual immorality, yet while married, he has had several affairs with prostitutes; therefore, sexual immorality is acceptable.
    \\

    
      Explanation: The fact Jimmy Swaggart likes to play a round of bedroom golf with some local entrepreneurial ladies, is not evidence for sexual immorality {\it in general}, only that {\it he is} sexually immoral.
    \\

    
      Exception: If Jimbo insisted that his actions were in line with sexual morality, then it would be a very germane part of the argument.
    \\

    
      Tip: Again, admit when your lack of self-control or willpower has nothing to do with the truth claim of the proposition.  The following is what I remember my dad telling me about smoking (he smoked about four packs a day since he was 14).
    \\

    
      Bo, never be a stupid a--hole like me and start smoking.  It is a disgusting habit that I know will eventually kill me.  If you never start, you will never miss it.
    \\

    
      My dad died at age 69 -- of lung cancer.  I never touched a cigarette in my life and never plan to touch one.
    \\

  }
\par \textbf{I'm not prejudiced, but...}


Friend argument
    
      (also known as: I'm not racist, I have black friends, Some of my best friends are black)
    \\

  

Whataboutism
    
      (also known as: And you are lynching Negroes, The pot calling the kettle black, The Mote and the Beam, clean hands, amoral familism, Whataboutery, Whataboutary)
    \\

  
    
      - **Description:** Whataboutism is a logical fallacy and rhetorical tactic used to deflect criticism by raising a different issue or accusing the critic of similar or worse faults. It aims to divert attention from the original topic by shifting the focus to a perceived hypocrisy or another unrelated issue.
    \\

    
      - **Logical Form:**
    \\

    
        - **P1:** X criticizes Y for a particular fault or action.
    \\

    
        - **P2:** Instead of addressing X's criticism, Y responds with a different fault or action of X, or another unrelated issue.
    \\

    
        - **C:** The focus is shifted away from the original criticism, and the conversation becomes about the new issue or accusation.
    \\

    
      - **Example \#1:** When confronted about a company's poor environmental practices, a representative responds, "Well, other companies also pollute the environment."
    \\

    
      - **Explanation:** The representative is using whataboutism to deflect attention from their company’s practices by pointing out that others have similar issues, rather than addressing the specific criticism at hand.
    \\

    
      - **Example \#2:** During a debate about a politician's corruption, one side responds, "But look at the corruption scandals in the opposing party!"
    \\

    
      - **Explanation:** Here, whataboutism is employed to divert the discussion from the politician's corruption by highlighting alleged corruption in another party, thereby avoiding the scrutiny of the original issue.
    \\

    
      - **Variation:**
    \\

    
        - **Red Herring:** A related but distinct fallacy where the diversion involves introducing an irrelevant topic to mislead or distract from the original issue.
    \\

    
        - **Tu Quoque:** A specific type of whataboutism that focuses on accusing the critic of hypocrisy by pointing out their own similar faults or behavior.
    \\

    
      - **Tip:** To counter whataboutism, calmly steer the conversation back to the original issue or criticism, focusing on addressing the specific points raised without getting sidetracked by unrelated issues.
    \\

    
      - **Exception:** When a comparison is relevant and directly related to the original issue, it may not constitute whataboutism. Ensure the new issue or comparison has a direct and pertinent connection to the discussion.
    \\

    
      - **Fun Fact:** The term "whataboutism" is derived from the Russian phrase "что вы скажете об этом?" (what do you say about this?), reflecting its use in political discourse and propaganda to deflect criticism and avoid accountability.
    \\

  

Appeal to Trust
    
      (also known as: appeal to distrust [opposite], appeal to trustworthiness, Appeal to confidence)
    \\

  
    Description: The belief that if a source is considered trustworthy or untrustworthy, then any information from that source must be true or false, respectively. This is problematic because each argument, claim, or proposition should be evaluated on its own merits.

    
      This doesn't include trusting in someone or something. For example, when we trust our children that they will make the right decisions in a certain situation, we are not using fallacious reasoning. When we trust that our seat belts will work when they need to, we are not using fallacious reasoning. When we trust that our puppy will leave us a gift on the carpet if we don't let him out by 7:00 AM, we are not using fallacious reasoning. When we express trust in someone, we are essentially expressing a degree of confidence, not making an absolute claim of something being true or false.
    \\

    
      Logical Forms:
    \\

    
      Source X is a trusted source of information.
    \\

    
      Claim Y was made by source X.
    \\

    
      Therefore, claim Y must be true.
    \\

    
       
    \\

    
      Source X is a distrusted source of information.
    \\

    
      Claim Y was made by source X.
    \\

    
      Therefore, claim Y must be false.
    \\

    
      Example \#1:
    \\

    
      I read in the Wall Street Journal that pork bellies are a good investment. So could I borrow a million dollars to invest?
    \\

    
      Explanation: The {\it Wall Street Journal} could reasonably be seen as a trusted source (it doesn't matter if you agree or not, as long as the arguer thinks it is). In this case, the information appears to be more of an opinion than a fact—and an investment prediction, which by its nature is risky and therefore its truth value is questionable.
    \\

    
      Example \#2:
    \\

    
      Cindy: I read in the Global Enquirer that Bingo Kelly, the famous movie star, is in rehab.
    \\

    
      Jack: That's poppycock! You can't trust the Global Enquirer any more than you can trust a toddler with a nail gun.
    \\

    
      Explanation: The {\it Global Enquirer} might not be known for its high-quality journalism and truthful reporting. However, even the most untrustworthy sources sometimes share true information. While it might be a good heuristic (rule of thumb) to be highly skeptical of any claims from such an untrustworthy source, such claims cannot be so confidently dismissed without a good reason.
    \\

    
      Exception: As long as one is claiming a degree of confidence instead of assuming true or false, there is no fallacy. Trustworthiness does impact the level of confidence one should have, but not certainty.
    \\

    
      Tip: Try to keep your level of confidence proportionate to your level of trust.
    \\

  

Double Standard
    Description: Judging two situations by different standards when, in fact, you should be using the same standard. This is used in argumentation to unfairly support or reject an argument.

    
      Logical Form:
    \\

    
      Person 1 makes claim X and gives reason Y.
    \\

    
      Person 2 makes claim Z and gives reason Y.
    \\

    
      Person 1 unfairly rejects reason Y, but only for claim Z and not claim X.
    \\

    
      Example \#1:
    \\

    
      Husband: I forbid you to go to that male strip club! That is a completely inappropriate thing for a wife to do!
    \\

    
      Wife: What about when you went to the female strip club last year?
    \\

    
      Husband: That was just for fun, and besides, that's different.
    \\

    
      Explanation: The husband is holding his wife to a different standard without articulating the standard. Most people would also agree that the standard is unfair.
    \\

    
      Example \#2:
    \\

    
      Catholic: I know St. Peter answers prayers because when I pray to him, my prayers are sometimes answered. When they are not, it is because St. Peter knows what is best for me.
    \\

    
      Protestant: Do you realize how foolish that sounds? You can say the same thing about praying to a mailbox.
    \\

    
      Catholic: How do you know God answers prayers?
    \\

    
      Protestant: Well... I... that's different.
    \\

    
      Explanation: It often occurs within religion where the standards applied to one religion or denomination to claim "truth" don't apply to arguments from other religions or denominations. In this example, the Protestant is demanding stronger "evidence" for the Catholic's claim than she would demand for herself explaining how God answers prayers.
    \\

    
      Exception: The fallacy is in the fact that the standards should  be the same, but sometimes there are legitimate different standards. For example, a president's remarks are held to a different standard than a reality television star’s remarks.
    \\

    
      Fun Fact: The default position is equal standards. One should not have to argue for this; the one claiming that standards are not equal has the {\em burden of proof}.
    \\

  

Two Wrongs Make a Right
    
      (also known as: Two wrongs don't make a right)
    \\

  
    Description: When a person attempts to justify an action against another person because the other person did take or would take the same action against him or her.

    
      Logical Forms:
    \\

    
      Person 1 did X to person 2.
    \\

    
      Therefore, person 2 is justified to do X to person 1.
    \\

    
       
    \\

    
      Person 1 believes that person 2 would do X to person 1.
    \\

    
      Therefore, person 1 is justified to do X to person 2.
    \\

    
      Example \#1:
    \\

    
      Jimmy stole Tommy’s lunch in the past.
    \\

    
      Therefore, it is acceptable for Tommy to steal Jimmy’s lunch today.
    \\

    
      Explanation: It was wrong for Jimmy to steal Tommy’s lunch, but it is not good reasoning to claim that Tommy stealing Jimmy’s lunch would make the situation right.  What we are left with, are two kids who steal, with no better understanding of why they shouldn’t steal.
    \\

    
      Example \#2:
    \\

    
      It looks like the waiter forgot to charge us for the expensive bottle of champagne.  Let’s just leave -- after all, if he overcharged us, I doubt he would chase us down to give us our money back that we overpaid.
    \\

    
      Explanation: Here the reasoning is a bit more fallacious because we are making an assumption of what the waiter might do.  Even if that were true, two ripoffs don’t make the situation right.
    \\

    
      Exception: There can be much debate on what exactly is “justified retribution” or “justified preventative measures”.
    \\

    
      Fun Fact: Three lefts make a right.
    \\

  

Bulverism
    Description: This is a combination of {\it circular reasoning}  and the {\it genetic fallacy}. It is the assumption and assertion that an argument is flawed or false because of the arguer's suspected motives, social identity, or other characteristic associated with the arguer's identity.

    
      Logical Form:
    \\

    
      Person 1 makes argument X.
    \\

    
      Person 2 assumes person 1 must be wrong because of their suspected motives, social identity, or other characteristic associated with their identity.
    \\

    
      Therefore, argument X is flawed or not true.
    \\

    
      Example \#1:
    \\

    
      Martin: All white people are not racists.
    \\

    
      Charlie: Yes they are. You just believe that because you are white.
    \\

    
      Explanation: Charlie is making two errors: 1) he is assuming that Martin must be wrong and 2) he is basing that assumption on an accidental feature of Martin—the amount of pigmentation in his skin.
    \\

    
      Example \#2:
    \\

    
      Mom: Remember, dear. Nobody's going to buy the cow if they get the milk for free.
    \\

    
      Daughter: You are only saying that because you are my mother.
    \\

    
      Daughter: Wait... did you just call me a cow?
    \\

    
      Explanation: Mom is doing her best to advise her daughter that she should be a bit more sexually reserved with her male suitors, cautioning her that she is unlikely to get any commitments unless she holds back sex. Although the claim is indeed dubious, the daughter assumes that it is wrong because of the source (her mother) and her mother's suspected motives (to get her married). So the mother must be wrong (assumption) because of her motives, and it is because of her motives that she is wrong ({\it circular reasoning} and the {\it genetic fallacy}).
    \\

    
      Exception: There is no exception; however, in some cases it is fair to cast doubt on the argument based on the identity of the person making the argument. This is a heuristic that may be useful, but problematic in critical argumentation.
    \\

    
      Tip: If you want a glass of milk, just buy the glass of milk, you don't need the whole cow (wait, did I just support prostitution?)
    \\

  

Argumentum ergo decedo
    
      (also known as: therefore leave, then go off, traitorous critic fallacy)
    \\

  
    
      Description: Responding to criticism by attacking a person's perceived favorability to an out-group or dislike to the in-group as the underlying reason for the criticism rather than addressing the criticism itself, and suggesting that they stay away from the issue and/or leave the in-group. This is usually done by saying something such as, "Well, if you don't like it, then get out!"
    \\

    
      Logical Form:
    \\

    
      Person 1 offers criticism against group 1.
    \\

    
      Person 2 responds to the criticism by disingenuously asking them why they don't leave group 1.
    \\

    
      Example \#1:
    \\

    
      Gertrude: I am tired of having to fill out these forms all day. Can't we find a more efficient system?
    \\

    
      Cindy-Lou: If you're not happy with the way we do things, we can find someone who is!
    \\

    
      Explanation: Cindy-Lou did not address the concern, but essentially threatened Gertrude to shut up or lose her job. This example might also be seen as {\it appeal to force} .
    \\

    
      Example \#2:
    \\

    
      Steve: In Sweden, college is free for citizens. How come we can't do that here?
    \\

    
      Ed: If you like Sweden so much, move there. The USA would be glad to be rid of your liberal ass!
    \\

    
      Explanation: Besides {\it begging the question} (Steve did not say he liked Sweden), Ed refused to address the question asked and deflected with a disingenuous question on why Steve does not move to Sweden.
    \\

    
      Exception: Repeated expressions of favoritism for the out-group and dislike of the in-group could justify a why-don't-you-join-the-out-group type of response.
    \\

    
      Tip: Remember the old saying about the grass being greener on the other side.
    \\

  \par \textbf{ad hominem (abusive)
    
      (also known as: argumentum ad personam, personal abuse, personal attacks, abusive fallacy,
    \\

    
      appeal to the person, damning the source, name calling, refutation
    \\

    
      by caricature, against the person, against the man)
    \\

  
    Description: Attacking the person making the argument, rather than the argument itself, when the attack on the person is completely irrelevant to the argument the person is making.

    
      Logical Form:
    \\

    
      Person 1 is claiming Y.
    \\

    
      Person 1 is a moron.
    \\

    
      Therefore, Y is not true.
    \\

    
      Example \#1:
    \\

    
      My opponent suggests that lowering taxes will be a good idea -- this is coming from a woman who eats a pint of Ben and Jerry’s each night!
    \\

    
      Explanation: The fact that the woman loves her ice cream, has nothing to do with the lowering of taxes, and therefore, is irrelevant to the argument.  {\it Ad hominem} attacks are usually made out of desperation when one cannot find a decent counter argument.
    \\

    
      Example \#2:
    \\

    
      Tony wants us to believe that the origin of life was an “accident”.  Tony is a godless SOB who has spent more time in jail than in church, so the only information we should consider from him is the best way to make license plates.
    \\

    
      Explanation: Tony may be a godless SOB.  Perhaps he did spend more time in the joint than in church, but all this is irrelevant to his argument or truth of his claim as to the origin of life.
    \\

    
      Exception: When the attack on the person is relevant to the argument, it is not a fallacy.  In our first example, if the issue being debated was the elimination of taxes only on Ben and Jerry’s ice cream, then pointing out her eating habits would be strong evidence of a conflict of interest.
    \\

    
      Tip: When others verbally attack you, take it as a compliment to the quality of your argument.  It is usually a sign of desperation on their part.
    \\

  }


Name calling

Verbal abuse

Appeal to Stupidity
    Description: Attempting to get the audience to devalue reason and intellectual discourse, or devaluing reason and intellectual discourse based on the rhetoric of an arguer.

    
      Logical Form:
    \\

    
      Person 1 downplays the importance of reason, logic, or science.
    \\

    
      Person 1 makes a claim, argument, or assertion.
    \\

    
      Audience is more likely to accept claim, argument, or assertion.
    \\

    
      Example \#1:
    \\

    
      Anthony: You know what's wrong with us today? We think too much! We need to act more with our heart and gut! Today is the first day of the rest of your life! Sign up for my 30-day program now for just \$999.99!
    \\

    
      Audience: (Cheers uncontrollably).
    \\

    
      Explanation: It is a common persuasion technique to get people in an emotional state and have them make an emotional decision while in that state. This is exactly what Anthony is doing here while undermining the importance of critical thinking.
    \\

    
      Example \#2:
    \\

    
      Politician: The other guy likes to throw statistics and data at us showing how much the economy has improved. But data and statistics don't feed our children. You feel it. The economy has gotten worse! Feelings are more important than facts.
    \\

    
      Explanation: There is a strong emotional appeal here accompanied by the devaluation of statistics and data (i.e., facts) in favor of feelings in order to answer an objective question: has the economy improved?
    \\

    
      Exception: Don’t confuse the appeal to stupidity with an arational argument. Arational arguments are not subject to reason and are properly feeling-based. Thus, asking people to put aside “reason” is not fallacious.
    \\

    
      {\em Mom: Which puppy do you want?} \newline
{\em Kid: They are all so cute and lovable. They all look healthy... I can’t decide!} \newline
{\em Mom: Go with your gut. Which one do you have the strongest feelings for?}
    \\

    
      Tip: If picking from a liter of puppies, don’t choose the craziest one.
    \\

  

Appeal to the Law
    Description: When following the law is assumed to be the morally correct thing to do, without justification, or when breaking the law is assumed to be the morally wrong thing to do, without justification.

    
      Logical Forms:
    \\

    
      X is illegal. Therefore, it is immoral.
    \\

    
      
    \\

    
      Y is legal. Therefore, it is moral.
    \\

    
      Example \#1:
    \\

    
      Tom: I plan on chaining myself to the bulldozer so they can't knock down the senior center.
    \\

    
      Judy: That's just wrong. You'll get arrested. Don't be a bad person!
    \\

    
      Explanation: Civil disobedience is just one example of something that is illegal but does not have to be immoral. Laws are created for many reasons, and only some are created for "moral" reasons (according to someone's moral code).
    \\

    
      Example \#2:
    \\

    
      Lucy: I cheated on my husband the other night.
    \\

    
      Rob: Why did you do that!?
    \\

    
      Lucy: Calm down! It's not like cheating is illegal or anything.
    \\

    
      Explanation: Again, the law and one's moral code are not the same things. While there does tend to be overlap, assuming illegal is immoral or legal is moral without a rational argument connecting the two concepts, is fallacious.
    \\

    
      Exception: Sometimes what is illegal is clearly immoral, and no justification is required. For example, if someone were to say, "How could you kick that old lady while laughing? That is a horrible thing to do, and you can go to jail!"
    \\

    
      Tip: Laws become archaic when they do not keep up with social norms and cultural changes. What was once deemed "immoral" could easily change to "moral" within a few years and vice versa. Never simply assume that laws are good and right; demand justification.
    \\

  

Argument from Age
    
      (also known as: wisdom of the ancients)
    \\

  
    Description: The misconception that previous generations had superior wisdom to modern man, thus conclusions that rely on this wisdom are seen accepted as true or more true than they actually are.

    
      Logical Form:
    \\

    
      Person 1 says that Y is true. 
    \\

    
      Person 1 was an ancient mystic.
    \\

    
      Therefore, Y is true.
    \\

    
      Example \#1:
    \\

    
      Swami Patooty wrote, back in the 6th century, “To know oneself, is to one day self know.”  You don’t find pearls like that today!
    \\

    
      Explanation: There are many sayings today that are just as ambiguous, obscure, and nonsensical as the ones carved in stone 1500 years ago -- the difference is perception.  Especially with “aged wisdom”, we tend to read in meaning to ambiguity where none exists or where the author’s intended meaning is impossible to know.
    \\

    
      Example \#2:
    \\

    
      My Grammy told me that to be healthy I should have bacon and eggs every morning for breakfast.
    \\

    
      Explanation: It seems politically incorrect to suggest that older generations are not “wise,” but the fact is, wisdom is not necessarily a function of age.  We have two generations of scientific knowledge that our grandparents did not have, and the world has changed quite a bit in the last two generations.  While some advice may be timeless, other advice, such as the advice in the example, was based on the beliefs of the day and should be discarded like a container of chunky milk.
    \\

    
      Exception: When the age is directly related to the truth of the claim as in, “Wine tastes better with age”.
    \\

    
      Fun Fact: Even ancient Greeks said stupid things.
    \\

  

Fallacy of Opposition
    Description: Asserting that those who disagree with you must be wrong and not thinking straight, primarily based on the fact that they are the opposition.

    
      Logical Form:
    \\

    
      {\em Person 1 is asserting X. \newline
Person 1 is the opposition. \newline
Therefore, X must be wrong.}
    \\

    
      Example \#1:
    \\

    
      {\em President Trump said that he was proud of the children who participated in this year's Special Olympics. Those kids are a bunch of losers.}
    \\

    
      Explanation: This is an extreme example of a very real example that we have all seen since around early 2016. Those who passionately hate Trump, reflexively disagree with everything he says and does, associating the truth of his statement with the feelings they have for him. This is not reasonable thinking.
    \\

    
      Example \#2:
    \\

    
      {\em The Democrats support more aggressive gun control laws. Can you believe they want to deny repeat offenders and those on the terrorist watch list their rights?}
    \\

    
      Explanation: Very often we see support for reasonable policies rejected based on the party that proposes such policies. We know this because research has been done in this area.
    \\

    
      Exception: There might be a situation where your opposition must say things that are demonstrably wrong, or they wouldn’t be your opposition. For example,
    \\

    
      {\em Only those who disagree with X are my opposition.} \newline
{\em X is demonstrably right.} \newline
{\em Bill is my opposition.} \newline
{\em Therefore, Bill is wrong.}
    \\

    
      It seems strange to suggest that because Bill is my opposition, he is wrong, but this is necessarily true if we hold that “Only those who disagree with X are my opposition” and “X is demonstrably right.” This wouldn’t make logical sense if we didn’t set the conditions so that anyone belonging to the group “opposition” would be wrong.
    \\

    
      Tip: Rejecting information from an opponent known to lie, might be a reasonable heuristic, but it is not a good critical thinking technique.
    \\

  

Identity Fallacy
    
      (also known as: identity politics)
    \\

  
    Description: When one's argument is evaluated based on their physical or social identity, i.e., their social class, generation, ethnic group, gender or sexual orientation, profession, occupation or subgroup when the strength of the argument is independent of identity.

    
      Logical Form:
    \\

    
      Person 1 makes argument X.
    \\

    
      Person 2 dismisses argument X because of the physical or social identity of person 1.
    \\

    
      Example \#1:
    \\

    
      S.J. Sam: Asian people in this country are systematically passed over in the tech field for non-Asians.
    \\

    
      Cindy: Actually, according to most research, employers are biased in favor of Asian technical workers.
    \\

    
      S.J. Sam: Unless you are Asian, keep your mouth shut. You can't possibly know the struggles of the Asian community!
    \\

    
      Explanation: S.J. Sam is making an empirical claim about a hiring preference for non-Asians. Cindy has refuted the claim that is independent of her physical or social identity (i.e., her ability to refute the argument is not dependent upon her being Asian). S.J. Sam rejects her rebuttal because she is not Asian. In addition, he pulls a {\it red herring} by changing the argument to "knowing the struggles" of the Asian community.
    \\

    
      Example \#2:
    \\

    
      The female staff of a large corporation holds a meeting to discuss solutions to reduce discrimination against women at the company. Men are invited but asked just to listen and not contribute to the discussion.
    \\

    
      Explanation: The implication here is that men have nothing to add to the discussion. Ideas to reduce gender discrimination are independent of gender, that is, both men and women can have equally valid arguments.
    \\

    
      Exception: A requirement for this fallacy is "when the strength of the argument is independent of identity." There are arguments that do rely on identity. For example, claims of feeling and perception could be unique to certain groups.
    \\

    
      S.J. Sam: As a gay man, I feel that I am being discriminated against at work.
    \\

    
      Cindy: I don't think people at work discriminate against gays.
    \\

    
      S.J. Sam: You are not gay. I bet your perspective would be different if you were.
    \\

    
      Cindy could ask for evidence of discrimination, which would be reasonable, but she dismisses S.J. Sam's claim when she lacks the insight due to her not being part of the social group (gays).
    \\

    
      Tip: Before you exclude any group from your discussions or ignore their arguments based on their physical or social group, make sure you have a solid reason.
    \\

  

Abusive analogy
    {\bf Description:} is a highly specialized version of the ad hominem argument. Instead of the arguer being insulted directly, an analogy is drawn which is calculated to bring him into scorn or disrepute. The opponent or his behaviour is compared with something which will elicit an unfavourable response toward him from the audience. \newline
 \newline
{\bf logical form} \newline
Person 1 states that X is true. \newline
Person 2 states that Y is true and hasn't property X \newline
Therefore, X must not be true \newline
(but Y really is not too much like X) \newline
 \newline
{\bf Example \#1:} \newline
Smith has proposed we should go on a sailing holiday, though he knows as much about ships as an Armenian bandleader does. \newline
 \newline
(Perhaps you do not need to know all that much for a sailing holiday. Smith can always learn. The point here is that the comparison is deliberately drawn to make him look ridiculous. There may even be several Armenian bandleaders who are highly competent seamen.) \newline
 \newline
{\bf Explanation:}  The analogy may even be a valid one, from the point of view of the comparison being made. This makes it more effective, but no less fallacious, since the purpose is to introduce additional, unargued, material to influence a judgement. \newline
 \newline
{\bf Example \#2:} \newline
If science admits no certainties, then a scientist has no more certain knowledge of the universe than does a Hottentot running through the bush. \newline
 \newline
(This is true, but is intended as abuse so that the hearer will be more sympathetic to the possibility of certain knowledge.) \newline
 \newline
{\bf Explanation:} The fallacy is a subtle one because it relies on the associations which the audience make from the picture presented. Its perpetrator need not say anything which is untrue; he can rely on the associations made by the hearer to fill in the abuse. The abusive analogy is a fallacy because it relies on this extraneous material to influence the argument. \newline
 \newline
{\bf Example \#3:} \newline
In congratulating my colleague on his new job, let me point out tha has no more experience of it than a snivelling boy has on his first da school. \newline
 \newline
(Again, true. But look who’s doing the snivelling.) \newline
 \newline
{\bf Explanation:} While politicians delight in both abuse and analogies, there are surprisingly few good uses of the abusive analogy from that domain. A good one should have an element of truth in its comparison, and invite abuse by its other associations. All other things being equal, it is easier to be offensive by making a comparison which is untrue, than to be clever by using elements of truth. Few have reached the memorable heights of Daniel O’Connell’s description of Sir Robert Peel: \newline
 \newline
{\bf Example \#4:} \newline
…a smile like the silver plate on a coffin. \newline
 \newline
(True, it has a superficial sparkle, but it invites us to think of something rather cold behind it.) \newline
 \newline
{\bf Explanation:}  The venom-loaded pens of literary and dramatic critics are much more promising springs from which abusive analogies can trickle forth {\bf Example \#5:} He moved nervously about the stage, like a virgin awaiting the Sult (And died after the first night.) {\bf Explanation:} Abusive analogies take composition. If you go forth without preparation, you will find yourself drawing from a well-used stock of comparisons which no longer have the freshness to conjure up vivid images. Describing your opponents as being like ‘straightlaced schoolmistresses’ or ‘sleazy strip-club owners’ will not lift you above the common herd. A carefully composed piece of abusive comparison, on the other hand, can pour ridicule on the best-presented case you could find: ‘a speech like a Texas longhorn; a point here, a point there, but a whole lot of bull in between’.
  

argumentum ad hominem (Circumstantial)
    
      (also known as: appeal to bias, appeal to motive, appeal to personal interest, argument from motives, conflict of interest, faulty motives, naïve cynicism, questioning motives, vested interest)
    \\

  
    Description: Suggesting that the person who is making the argument is biased or predisposed to take a particular stance, and therefore, the argument is {\it necessarily} invalid.

    
      Logical Form:
    \\

    
      Person 1 is claiming Y.
    \\

    
      Person 1 has a vested interest in Y being true.
    \\

    
      Therefore, Y is false.
    \\

    
      Example \#1:
    \\

    
      Salesman: This car gets better than average gas mileage and is one of the most reliable cars according to Consumer Reports.
    \\

    
      Will: I doubt it—you obviously just want to sell me that car.
    \\

    
      Explanation: The fact that the salesman has a vested interest in selling Will the car does not mean that he is lying.  He may be, but this is not something you can conclude solely on his interests.  It is reasonable to assume that salespeople sell the products and services they do because they believe in them.
    \\

    
      Example \#2:
    \\

    
      Of course, your minister says he believes in God.  He would be unemployed otherwise.
    \\

    
      Explanation: The fact that atheist ministers are about as in demand as hookers who, “just want to be friends”, does not mean that ministers believe in God just because they need a job.
    \\

    
      Exception: As the bias or conflict of interest becomes more relevant to the argument, usually signified by a lack of other evidence, the argument is seen as less of a fallacy and more as a legitimate motive.  For example, courtesy of Meat Loaf...
    \\

    
      Girl: Will you love me forever?
    \\

    
      Boy: Let me sleep on it!!!
    \\

    
      Girl: Will you love me forever!!!
    \\

    
      Boy: I couldn't take it any longer
    \\

    
      Lord, I was crazed
    \\

    
      And when the feeling came upon me
    \\

    
      Like a tidal wave
    \\

    
      I started swearing to my god and on my mother's grave
    \\

    
      That I would love you to the end of time
    \\

    
      I swore that I would love you to the end of time!
    \\

    
      Tip: When you know you have something to gain from a position you hold (assuming, of course, you are not guilty of this fallacy for holding the position), be upfront about it and bring it up before someone else does.
    \\

    
      Supporting this cause is the right thing to do.  Yes, as the baseball coach, I will benefit from the new field, but my benefit is negligible compared to the benefit the kids of this town will receive.  After all, they are the ones who really matter here.
    \\

    References:

    
      
        
      \\

      
        
          Walton, D. (1998). {\it Ad hominem arguments}. University of Alabama Press.
        
      
    
  \subsection{Appeal to authority
    
      (also known as: argumentum ad verecundiam, argument from authority, ipse dixit, Argumentum ad Reverentiam, Appeal to Reverence)
    \\

  
    Description: Insisting that a claim is true simply because a valid authority or expert on the issue said it was true, without any other supporting evidence offered. Also see the {\it appeal to false authority}  .

    
      Logical Form:
    \\

    
      According to person 1, who is an expert on the issue of Y, Y is true.
    \\

    
      Therefore, Y is true.
    \\

    
      Example \#1:
    \\

    
      Richard Dawkins, an evolutionary biologist and perhaps the foremost expert in the field, says that evolution is true. Therefore, it's true.
    \\

    
      Explanation: Richard Dawkins certainly knows about evolution, and he can confidently tell us that it is true, but that doesn't make it true. What makes it true is the preponderance of evidence for the theory.
    \\

    
      Example \#2:
    \\

    
      How do I know the adult film industry is the third largest industry in the United States? Derek Shlongmiester, the adult film star of over 50 years, said it was. That's how I know.
    \\

    
      Explanation: Shlongmiester may be an industry expert, as well as have a huge talent, but a claim such as the one made would require supporting evidence. For the record, the adult film industry may be large, but on a scale from 0 to 12 inches, it's only about a fraction of an inch.
    \\

    
      Exception: Be very careful not to confuse "deferring to an authority on the issue" with the {\it appeal to authority fallacy}. Remember, a fallacy is an error in reasoning. Dismissing the council of legitimate experts and authorities turns good skepticism into denialism. The {\it appeal to authority} is a fallacy in argumentation, but deferring to an authority is a reliable heuristic that we all use virtually every day on issues of relatively little importance. There is always a chance that any authority can be wrong, that’s why the critical thinker accepts facts {\it provisionally}. It is not at all unreasonable (or an error in reasoning) to accept information as provisionally true by credible authorities. Of course, the reasonableness is moderated by the claim being made (i.e., how extraordinary, how important) and the authority (how credible, how relevant to the claim).
    \\

    
      The {\it appeal to authority} is more about claims that require evidence than about facts. For example, if your tour guide told you that Vatican City was founded February 11, 1929, and you accept that information as true, you are not committing a fallacy (because it is not in the context of argumentation) nor are you being unreasonable.
    \\

    
      Tip: Question authority -- or become the authority that people look to for answers.
    \\

  }


Appeal to accomplishment
    
      (also known as: appeal to success)
    \\

  
    Description: When the argument being made is sheltered from criticism based on the level of accomplishment of the one making the argument.  A form of this fallacy also occurs when arguments are evaluated on the accomplishments, or success, of the person making the argument, rather than on the merits of the argument itself.

    
      Logical Forms:
    \\

    
      Person 1 claims that Y is true.
    \\

    
      Person 1 is very accomplished.
    \\

    
      Therefore, Y is true.
    \\

    
       
    \\

    
      Person 1 presents evidence against claim Y.
    \\

    
      Person 1 is told to shut up until person 1 becomes as accomplished as person 2.
    \\

    
      Example \#1:
    \\

    
      I have been around the block many times, and I have had my share of success.  So believe me when I tell you that there is no better hobby than cat-juggling.
    \\

    
      Explanation: We can all admire accomplishment and success, but this is irrelevant to cat-juggling.  There are many accomplished and successful people who are immoral, mean, insensitive, hateful, liars,  miserable, and just plain wrong about a great many things.
    \\

    
      Example \#2:
    \\

    
      I hold a doctorate in theology, have written 12 books, and personally met the Pope.  Therefore, when I say that Jesus’ favorite snack was raisins dipped in wine, you should believe me.
    \\

    
      Explanation: While the credentials of the one making the statement are certainly impressive, in no way do these credentials lend credibility to the belief that Jesus’ favorite snack was wine-dipped raisins.
    \\

    
      Exception: When one’s accomplishments are directly related to the argument, it is more meaningful.
    \\

    
      I have been around the block many times, and I have had my share of success in real estate.  So believe me when I tell you that, if you know what you are doing, real estate can be a great way to make a great living.
    \\

    
      Tip: Many successful people attempt to use their success as a wildcard to be an authority on everything.  Don’t allow one’s own success to cloud your judgment of the claims they are making.  Evaluate the evidence above all else.
    \\

  

bare assertion fallacy

Courtier's reply\par \textbf{Argument to the Purse
    
      (also known as: appeal to money)
    \\

  
    Description: Concluding that the truth value of the argument is true or false based on the financial status of the author of the argument or the money value associated with the truth. The {\it appeal to poverty} is when the truth is assumed based on a lack of wealth whereas the {\it appeal to wealth} is when the truth is assumed based on an excess of wealth.

    
      Logical Form:
    \\

    
      Person 1 says Y is true.
    \\

    
      Person 1 is very rich.
    \\

    
      Therefore, Y must be true (appeal to wealth) / false (appeal to poverty).
    \\

    
      Example \#1:
    \\

    
      Mike: Did you know that the author of the book, “Logically Fallacious,” made a fortune on the Internet?
    \\

    
      Jon: So?
    \\

    
      Mike: That means that this book must be awesome!
    \\

    
      Explanation: While my financial status might impress the participants at an Amway conference, it has little to do with my knowledge of fallacies.  However, remember the {\it argument from fallacy}; just because the argument is fallacious, does not mean the conclusion is not true, dammit.
    \\

    
      Example \#2:
    \\

    
      Simon is very poor.  Simon says that the secret to life is giving up all your material possessions, and living off the government’s material possessions.  Simon must be very enlightened.
    \\

    
      Explanation: Just like people tend to associate wealth with wisdom, they also associate extreme poverty with wisdom.  Rich people are rich and poor people are poor—which members of those groups have wisdom does not depend on their financial status.
    \\

    
      Exception: If one’s wealth, or lack thereof, is directly related to the truth value of an argument, then it is not a fallacy.
    \\

    
      Mike: Did you know that the author of this book, who does extremely well financially in business, also wrote the book, “Year To Success” that was endorsed by Arnold Schwarzenegger?
    \\

    
      Jon: I did not know that.
    \\

    
      Mike: That means that his book on success is probably worth looking into!
    \\

    
      Jon: I agree, and I am sure Bo will thank you for the cheap plug.
    \\

    
      Tip: There is nothing wrong with a little self-promotion.
    \\

  }


argumentum ad Lazarum
    
      (also known as: Appeal to poverty, Argument from Poverty)
    \\

  
    
      - **Description:** Argumentum ad Lazarum is a logical fallacy where a claim is accepted or rejected based solely on the financial status or socioeconomic condition of the person making the claim. It suggests that a person's argument is valid or invalid depending on their economic situation, rather than the merits of the argument itself.
    \\

    
      - **Logical Form:**
    \\

    
        - **P1:** X is poor or of low socioeconomic status.
    \\

    
        - **P2:** The validity of X's argument is judged based on their poverty or lack of resources.
    \\

    
        - **C:** X’s argument is either accepted or dismissed because of their financial status, rather than its logical or factual merit.
    \\

    
      - **Example \#1:** A homeless person argues for increased social services, and someone dismisses their argument by saying, "What do you expect from someone in your situation?"
    \\

    
      - **Explanation:** This example demonstrates argumentum ad Lazarum by dismissing the homeless person's argument on the basis of their poverty, rather than evaluating the argument’s substance or reasoning.
    \\

    
      - **Example \#2:** A low-income advocate proposes a policy change, and their proposal is rejected with the comment, "Of course you'd support that; you're just looking out for your own interests."
    \\

    
      - **Explanation:** Here, the advocate's financial status is used to undermine the credibility of their argument, instead of considering the policy proposal’s merits.
    \\

    
      - **Variation:**
    \\

    
        - **Argumentum ad Divitias:** The converse fallacy, where the argument's validity is determined based on the speaker’s wealth or high socioeconomic status.
    \\

    
        - **Ad Hominem (Financial Status):** A broader category of ad hominem attacks where a person’s financial status is used to challenge their argument.
    \\

    
      - **Tip:** Evaluate arguments based on their logical structure and evidence rather than the personal or financial status of the individual presenting them. Focus on the content of the argument itself.
    \\

    
      - **Exception:** If the financial status of an individual is directly relevant to the argument (e.g., a person’s experience with poverty is used to inform a policy proposal), discussing their socioeconomic condition may be relevant to understanding their perspective, but it should not solely determine the argument’s validity.
    \\

    
      - **Fun Fact:** The term "Argumentum ad Lazarum" is named after Lazarus, a biblical figure who was poor and lived in a state of destitution, contrasting with the wealthy and ignoring his arguments or perspectives based on his poverty.
    \\

  

Argumentum ad crumenam
    
      (also known as: appeal to wealth, Argument from Wealth)
    \\

  
    
      - **Description:** Argumentum ad Crumenam is a logical fallacy where a claim or argument is accepted or rejected based solely on the wealth or financial status of the person making it. It implies that an argument is more valid or convincing if the person presenting it is wealthy, or less valid if they are not.
    \\

    
      - **Logical Form:**
    \\

    
        - **P1:** X is wealthy or of high socioeconomic status.
    \\

    
        - **P2:** The validity of X's argument is judged based on their wealth or financial status.
    \\

    
        - **C:** X’s argument is either accepted or given more weight because of their financial status, rather than its logical or factual merit.
    \\

    
      - **Example \#1:** A billionaire proposes a new business model and someone agrees with it solely because the billionaire is rich.
    \\

    
      - **Explanation:** In this case, the agreement is based on the billionaire’s wealth rather than the merits of the business model itself, which is a fallacy as the argument’s value should be assessed independently of the speaker’s financial status.
    \\

    
      - **Example \#2:** During a debate about economic policy, a rich individual’s opinion is given undue weight simply because they are wealthy, while similar arguments from less wealthy individuals are dismissed.
    \\

    
      - **Explanation:** This example shows the fallacy by giving undue importance to the rich individual’s opinion based on their financial status rather than the substance of their argument.
    \\

    
      - **Variation:**
    \\

    
        - **Argumentum ad Lazarum:** The converse fallacy, where a person's argument is dismissed or accepted based on their poverty or low socioeconomic status.
    \\

    
        - **Ad Hominem (Wealth):** A broader category of ad hominem attacks that involve evaluating the argument based on the speaker’s financial status.
    \\

    
      - **Tip:** Focus on the logical structure and evidence of arguments rather than the financial status of the individual presenting them. Evaluate the content of the argument independently of the speaker’s wealth.
    \\

    
      - **Exception:** Discussions about financial policies or economic issues where the speaker’s wealth is directly relevant to their perspective may warrant consideration of their financial status, but it should not solely determine the validity of their arguments.
    \\

    
      - **Fun Fact:** The term "Argumentum ad Crumenam" comes from Latin, where "crumenam" refers to a money bag, reflecting the fallacy’s reliance on financial status rather than logical reasoning.
    \\

  

Name-dropping

Dictat (propaganda)

Anonymous authority
    
      (also known as: appeal to anonymous authority)
    \\

  
    Description: When an unspecified source is used as evidence for the claim.  This is commonly indicated by phrases such as “They say that...”, “It has been said...”, “I heard that...”, “Studies show...”, or generalized groups such as, “scientists say...”  When we fail to specify a source of the authority, we can’t verify the source, thus the credibility of the argument.  Appeals to anonymous sources are more often than not, a way to fabricate, exaggerate, or misrepresent facts in order to deceive others into accepting your claim.  At times, this deception is done subconsciously -- it might not always be deliberate.

    
      Logical Form:
    \\

    
      Person 1 once heard that X was true.
    \\

    
      Therefore, X is true.
    \\

    
      Example \#1:
    \\

    
      You know, they say that if you swallow gum it takes seven years to digest.  So whatever you do, don’t swallow the gum!
    \\

    
      Explanation: “They” are wrong as “they” usually are.  Gum passes through the system relatively unchanged but does not hang around for 7 years like a college student terrified to get a job.  “They” is a common form of appeal to {\it anonymous authority}.
    \\

    
      Example \#2:
    \\

    
      The 13.7 billion-year-old universe is a big conspiracy.  I read this article once where these notable scientists found strong evidence that the universe was created 6000 years ago, but because of losing their jobs, they were forced to keep quiet!
    \\

    
      Explanation: Without knowing who these scientists are, or the credibility of the source of the article, we cannot verify the evidence; therefore, we should not accept the evidence.
    \\

    
      Exception: At times, an accepted fact uses the same indicating phrases like the ones used for the fallacy; therefore, if the anonymous authority is actually just a statement of an accepted fact, it should be accepted as evidence.
    \\

    
      Climate change is happening -- and always has been.  Scientists say the earth is certainly in a warming phase, but there is some debate on the exact causes and certainly more debate on what should be done about it politically.
    \\

    
      Tip: Be very wary of “they”.
    \\

  

Appeal to Celebrity
    Description: Accepting a claim of a celebrity based on his or her celebrity status, not on the strength of the argument.

    
      Logical Form:
    \\

    
      Celebrity 1 says to use product Y.
    \\

    
      Therefore, we should use product Y.
    \\

    
      Example \#1:
    \\

    
      Tom Cruise says on TV that Billy Boy Butter is the best tasting butter there is.  Tom Cruise is awesome -- especially in MI4 when he scaled that building with only one suction glove; therefore, Billy Boy Butter is the best tasting butter there is!
    \\

    
      Explanation: Tom Cruise might be awesome, and perhaps he really does think Billy Boy Butter is the best tasting butter there is, but Tom is no more an authority on the taste of butter than anyone else; therefore, to accept the claim without any other evidence or reason is fallacious.
    \\

    
      Example \#2:
    \\

    
      Mike Seaver from that 80’s sitcom, “Growing Pains”, is really cool.  He is now a born-again Christian and apologist for the faith.  Therefore, you should really believe what he has to say!
    \\

    
      Explanation: Mike Seaver is awesome, but Kirk Cameron, the actor that plays that character?  Even if Kirk were super duper (which he might be, I don’t know him), his views on the truth of religion are equally valid as yours, or anyone else's who determines what he or she considers to be the truth through faith.
    \\

    
      Exception: Some celebrity endorsements are authentic, where the celebrities are motivated by the love of the product itself, not the huge check they are getting for pretending to like the product.  When these products are directly related to their celebrity status, then this could be seen as a valid (but not sufficient) reason for wanting the product.
    \\

    
      Honestly, I really can’t think of any examples, but there must be some out there.
    \\

    
      Tip: If you are in business and looking for a celebrity to endorse your product, try not to pick one that is likely to be accused of killing his wife and his wife’s lover, then taking off in a white Bronco.
    \\

  

Appeal to False Authority
    
      (also known as: appeal to doubtful authority, appeal to dubious authority, appeal to improper authority, appeal to inappropriate authority, appeal to irrelevant authority, appeal to misplaced authority, appeal to unqualified authority, argument from false authority)
    \\

  
    \chapter{
      Appeal to False Authority
    }
  
    
    Description: Using an alleged authority as evidence in your argument when the authority is not really an authority on the facts relevant to the argument. As the audience, allowing an irrelevant authority to add credibility to the claim being made. Also see the {\em {\it appeal to authority}}.

    
      Logical Forms:
    \\

    
      According to person 1 (who offers little or no expertise on Y being true), Y is true. \newline
Therefore, Y is true.
    \\

     \newline

    

    
      According to person 1 (who offers little or no expertise on Y being true), Y is true. \newline
Therefore, Y is more likely to be true.
    \\

     \newline

    

    
      Expert A gives her view on issue B. \newline
Expert A’s area of expertise has little or nothing to do with issue B. \newline
Expert A’s opinion influences how people feel about issue B.
    \\

    
      Example \#1:
    \\

    
      My 5th-grade teacher once told me that girls would go crazy for boys if they learn how to dance.  Therefore, if you want to make the ladies go crazy for you, learn to dance.
    \\

    
      Explanation: Even if the 5th-grade teacher were an expert on relationships, her belief about what makes girls “go crazy” for boys is speculative, or perhaps circumstantial, at best. In other words, the teacher's expertise is in dance, not on the psychology of attraction.
    \\

    
      Example \#2:
    \\

    
      The Pope told me that priests could turn bread and wine into Jesus’ body and blood.  The Pope is not a liar. Therefore, priests really can do this.
    \\

    
      Explanation: The Pope may believe what he says, and perhaps the Pope is not a liar, but the Pope is not an authority on the {\it fact}  that the bread and wine are actually transformed into Jesus’ body and blood. After all, how much flesh and blood does this guy Jesus actually have to give? 
    \\

    
      Example \#3:
    \\

    
      Dr. Dean, TV’s hottest new psychologist, says that coffee enemas are the “fountain of youth.” Get me that coffee enema!
    \\

    
      Explanation: Assuming Dr. Dean is actually a licensed psychologist, that does not qualify him to give advice about non-psychology related issues such as coffee enemas. Like the colon, Dr. Dean is most likely full of... obligate anaerobes. Extending his expertise from psychology to issues of the colon is fallacious.
    \\

    
      Example \#4:
    \\

    
      My accountant says that within the next 90 days, the president will be impeached! So we should take this claim seriously!
    \\

    
      Explanation: Unless the accountant has some inside information to the presidency, her expertise in accounting has little to do with the current administration, political, and constitutional law.
    \\

    
      Exception: Don’t pigeonhole people into certain areas of expertise. A medical doctor can also be an expert in sewing. A fly-fisherman can also be an expert in law. And a patent clerk can also be an expert in quantum mechanics.
    \\

    
      Tip: Beware of your {\it confirmation bias}. You may want a person to be an authority on the topic, and this desire will result in your seeking out confirming information and ignoring conflicting information.
    \\

    \chapter{
      Argument from False Authority
    }
  
    

    
      Description: When a person making a claim is presented as an expert who should be trusted when his or her expertise is not in the area being discussed.
    \\

    
      Logical Form:
    \\

    
      Expert A gives her view on issue B.
    \\

    
      Expert A's area of expertise has little or nothing to do with issue B.
    \\

    
      Expert A's opinion influences how people feel about issue B.
    \\

    
      Example \#1:
    \\

    
      Dr. Dean, TV's hottest new psychologist, says that coffee enemas are the "fountain of youth." Get me that coffee enema!
    \\

    
      Explanation: Assuming Dr. Dean is actually a licensed psychologist, that does not qualify him to give advice about non-psychology related issues such as coffee enemas. Extending his expertise from psychology to issues of the colon is fallacious.
    \\

    
      Example \#2:
    \\

    
      My accountant says that within the next 90 days, the president will be impeached! So we should take this claim seriously!
    \\

    
      Explanation: Unless the accountant has some inside information to the presidency, her expertise in accounting has little to do with the current administration, political, and constitutional law.
    \\

    
      Exception: Don't pigeonhole people into certain areas of expertise. A medical doctor can also be an expert in sewing. A fly-fisherman can also be an expert in law. And a patent clerk can also be an expert in quantum mechanics.
    \\

    
      Tip: Become an expert in something.
    \\

  

Blind Authority Fallacy
    
      (also known as: blind obedience, the “team player” appeal, Nuremberg defense, divine authority fallacy [form of], appeal to/argument from blind authority)
    \\

  
    Description: Asserting that a proposition is true solely on the authority making the claim. It is often the case that those who blindly follow an authority ignore any counter-evidence to the authority’s claim, no matter how strong. The authority could be anyone or anything, including parents, a coach, a boss, a military leader, a document, or a god.

    
      Logical Form:
    \\

    
      Person 1 says Y is true.
    \\

    
      Person 1 is seen as the ultimate authority.
    \\

    
      Therefore, Y is true.
    \\

    
      Example \#1:
    \\

    
      During the Nazi war crimes trials at Nuremberg after World War II,  Nazi war criminals were charged with genocide, mass murder, torture, and other atrocities.  Their defense: "I was only following orders".
    \\

    
      Explanation: Most of us begin our lives seeing our parents as the ultimate authority, and we experience their wrath when we question that authority.  Unfortunately, this bad habit is carried over into adulthood where we replace our parents with a coach, a boss, a teacher, a commander, or a god.  Rather than question, we blindly follow.  This fallacy has probably resulted in more deaths, pain, suffering, and misery than all others combined.
    \\

    
      Example \#2:
    \\

    
      Your honor, the Bible clearly says that psychics, wizards, and mediums are to be stoned to death and that it is our responsibility to do so (Leviticus 20:27).  Therefore, I had every right to try to stone Dianne Warwick, and her psychic friends, to death.
    \\

    
      Explanation: Most Americans do see the Bible as the ultimate authority, but that darn, pesky legal system gets in the way.
    \\

    
      Exception: To quote Col. Jessep from {\it A Few Good Men}, “We follow orders, son. We follow orders or people die. It's that simple. Are we clear?”  I have never served in the military, so I cannot say how far I would go when just, “following orders”.  I wouldn’t want anyone to die because I questioned orders, yet I wouldn’t want anyone to die because I followed orders blindly.  I guess this is why I am not in the military.
    \\

    
      Variation: The divine authority fallacy is when the authority referenced is specifically said to be divine.
    \\

    
      Tip: Moral reasoning is difficult, and the consequences of making poor moral decisions can be traumatic, but the more experience you have with moral reasoning, the better you will get at it. Don’t allow an authority to rob you of this growth opportunity.
    \\

  

Appeal to the minority
    
      - **Description:** The appeal to the minority is a logical fallacy in which an argument is considered valid or more credible simply because it is supported by a small or minority group. This fallacy assumes that the less popular or less conventional view must be correct or superior simply because it is held by a minority.
    \\

    
      - **Logical Form:**
    \\

    
        - **P1:** A claim is supported by a minority group or an unconventional opinion.
    \\

    
        - **P2:** Because the claim is supported by a minority, it is presented as being more valid or true.
    \\

    
        - **C:** The claim is considered valid or superior solely based on its minority status, rather than on its logical or factual merits.
    \\

    
      - **Example \#1:** "Only a few scientists support this new theory, so it must be true."
    \\

    
      - **Explanation:** This example illustrates the fallacy by assuming that the theory’s truth is validated because it is supported by a minority of scientists, without evaluating the theory’s evidence or logical basis.
    \\

    
      - **Example \#2:** "Most people disagree with this policy, but that’s exactly why it’s worth supporting."
    \\

    
      - **Explanation:** Here, the policy is being defended not on its merits but because it is opposed by the majority, implying that its minority support is a reason for its validity.
    \\

    
      - **Variation:**
    \\

    
        - **Appeal to Popularity (Majority):** The converse fallacy where a claim is considered valid because it is widely accepted by the majority.
    \\

    
        - **Argumentum ad Novitatem:** A fallacy where something is considered better simply because it is new or novel, which can sometimes overlap with minority views.
    \\

    
      - **Tip:** Assess arguments based on evidence, logic, and reasoning rather than the number of supporters or the popularity of the view. Popularity or rarity alone does not determine the validity of an argument.
    \\

    
      - **Exception:** In some cases, minority views might be valuable if they offer unique insights or challenge prevailing biases, but their value should be judged on their own merits and evidence rather than their minority status.
    \\

    
      - **Fun Fact:** The appeal to the minority fallacy often occurs in debates where unconventional or fringe opinions are promoted as being superior simply due to their rarity, leading to a paradoxical situation where rarity is mistakenly equated with truth.
    \\

  
    
      (Also Known As: Argumentum ad Populum (Minority), Appeal to Unusual Opinion, Minority Opinion Fallacy)
    \\

  \subsection{Appeal to Definition
    
      (also known as: Argumentum ad dictionarium, appeal to the dictionary, victory by definition, Fallacy of definition)
    \\

  
    Description: Using a dictionary’s limited definition of a term as evidence that term cannot have another meaning, expanded meaning, or even conflicting meaning.  This is a fallacy because dictionaries don’t reason; they simply are a reflection of an abbreviated version of the current accepted usage of a term, as determined by argumentation and eventual acceptance.  In short, dictionaries tell you what a word meant, according to the authors, at the time of its writing, not what it meant before that time, after, or what it should mean.

    
      Dictionary meanings are usually concise, and lack the depth found in an encyclopedia; therefore, terms found in dictionaries are often incomplete when it comes to helping people to gain a full understanding of the term.
    \\

    
      Logical Form:
    \\

    
      The dictionary definition of X does not mention Y.
    \\

    
      Therefore, Y must not be part of X.
    \\

    
      Example \#1:
    \\

    
      Ken: Do you think gay marriage should be legalized?
    \\

    
      Paul: Absolutely not!  Marriage is defined as the union between a man and a woman—not between two men or two women!
    \\

    
      Ken: Did you know that in 1828 the dictionary definition of marriage included, “for securing the maintenance and education of children”?  Does that mean that all married couples who can’t or choose not to have children aren’t really married?
    \\

    
      Paul: No, it just means they need to buy updated dictionaries.
    \\

    
      Ken: As do you.  The current Merriam-Webster Dictionary includes as a secondary definition, “the state of being united to a person of the same sex in a relationship like that of a traditional marriage.”
    \\

    
      Explanation: The dictionary does not settle controversial issues such as gay marriage—it simply reports the most current accepted definition of the term itself while usually attempting to remain neutral on such controversial issues.
    \\

    
      Example \#2:
    \\

    
      Armondo: Mrs. Patterson was wrong to knock off 10 points off my oral presentation because I kept using the word, “erection” instead of building.
    \\

    
      Felix: That was hilarious, but did you honestly think you would not get in trouble?
    \\

    
      Armondo:  No, my dictionary says that an erection is a building.
    \\

    
      Explanation: Armondo may be right, but the dictionary is not the final authority on all issues, especially social behavior.  More modern usage, especially in a high school setting, takes precedence in this case.
    \\

    
      Exception: The dictionary works well when the term in question is a result of a misunderstanding or ignorance.  For example:
    \\

    
      Ken: Do you accept evolution?
    \\

    
      Paul: No. Because life cannot come from non-life.
    \\

    
      Ken: Look up “evolution” and you will see that it makes no claims to the origin of life.
    \\

    
      Tip: Don’t be afraid to argue with authority if you believe you are right -- even when that authority is the dictionary.
    \\

  }


Etymological Fallacy
    Description: The assumption that the present-day meaning of a word should be/is similar to the historical meaning.  This fallacy ignores the evolution of language and heart of linguistics.  This fallacy is usually committed when one finds the historical meaning of a word more palatable or conducive to his or her argument. This is a more specific form of the {\it appeal to definition}.

    
      Logical Form:
    \\

    
      X is defined as Y.
    \\

    
      X used to be defined as Z.
    \\

    
      Therefore, X means Z.
    \\

    
      Example \#1:
    \\

    
      Elba: I can’t believe the art critic said my artwork is awful!
    \\

    
      Rowena: He must have meant it in the old sense of the word -- that your artwork inspired awe!
    \\

    
      Elba: Yes!  That makes sense now!
    \\

    
      Explanation: “Awful” did once mean “to inspire awe”, but there are very few, if any, people who continue to use the term in this way.  Just because it makes her feel better, it cannot be assumed.
    \\

    
      Example \#2:
    \\

    
      Steve: I think it is fantastic that you and Sylvia are getting married!
    \\

    
      Chuck: I cannot believe you think my getting married only exists in my imagination!  That is what fantastic means, after all.
    \\

    
      Explanation: Yes, it is true "fantastic" was once most commonly used as existing only in the imagination, but common use of this word has a very different definition.
    \\

    
      Exception: If a bogus, “modern”, definition is made up by a questionable source, that won’t make all other sources “historical”.
    \\

    
      Tip: Don’t call a housewife a “hussy” even though the word “hussy” comes from the word “housewife” and used to refer to the mistress of a household, not the disreputable woman it refers to today. They don’t like that.
    \\

    References:

    
      
        
      \\

      
        
          Wilson, K. G. (1993). {\it The Columbia Guide to Standard American English}. Columbia University Press
        
      
    
  \subsection{Sunk-Cost Fallacy
    
      (also known as: argument from inertia, concorde fallacy, finish the job fallacy)
    \\

  
    
      Description: Reasoning that further investment is warranted on the fact that the resources already invested will be lost otherwise, not taking into consideration the overall losses involved in the further investment.
    \\

    
      Logical Form:
    \\

    
      X has already been invested in project Y.
    \\

    
      Z more investment would be needed to complete project Y, otherwise X will be lost.
    \\

    
      Therefore, Z is justified.
    \\

    
      Example \#1: I have already paid a consultant \$1000 to look into the pros and cons of starting that new business division.  He advised that I shouldn’t move forward with it because it is a declining market.  However, if I don’t move forward, that \$1000 would have been wasted, so I better move forward anyway.
    \\

    
      Explanation: What this person does not realize is that moving forward will most likely result in the loss of much more time and money.  This person is thinking short-term, not long-term, and is simply trying to avoid the loss of the \$1000, which is fallacious thinking.
    \\

    
      Example \#2: There are ministers, priests, pastors, and other clergy all around the world who have invested a significant portion of their lives in theology, who can no longer manage to hold supernatural beliefs -- who have moved beyond faith.  Hundreds of them recognize those sunk costs and are searching for the best way to move on (see http://www.clergyproject.org) whereas many others cannot accept the loss of their religious investment, and continue to practice a profession inconsistent with their beliefs.
    \\

    
      Explanation: Of course, the clergy who have not moved beyond faith and are living consistently with their beliefs have not committed this fallacy.
    \\

    
      Exception: If a careful evaluation of the hypothetical outcomes of continued investment versus accepting current losses and ceasing all further investment have been made, then choosing the former would not be fallacious.
    \\

    
      Tip: Is there any part of your life where you continue to make bad investments because you fear to lose what was already invested? Do something about it.
    \\

    
      
    \\

  }
\par \textbf{Appeal to Tradition
    
      (also known as: argumentum ad antiquitatem, appeal to common practice, appeal to antiquity, appeal to traditional wisdom, proof from tradition, appeal to past practice, traditional wisdom)
    \\

  
    Description: Using historical preferences of the people (tradition), either in general or as specific as the historical preferences of a single individual, as evidence that the historical preference is correct.  Traditions are often passed from generation to generation with no other explanation besides, “this is the way it has always been done”—which is not a reason, {\it it is an absence of a reason}.

    
      Logical Forms:
    \\

    
      We have been doing X for generations.
    \\

    
      Therefore, we should keep doing X.
    \\

    
       
    \\

    
      Our ancestors thought X was right.
    \\

    
      Therefore, X is right.
    \\

    
      Example \#1:
    \\

    
      Dave: For five generations, the men in our family went to Stanford and became doctors, while the women got married and raised children.  Therefore, it is my duty to become a doctor.
    \\

    
      Kaitlin: Do you want to become a doctor?
    \\

    
      Dave: It doesn’t matter -- it is our family tradition.  Who am I to break it?
    \\

    
      Explanation:  Just as it takes people to start traditions, it takes people to end them.  A tradition is not a reason for action -- it is like watching the same movie over and over again but never asking why you should keep watching it.
    \\

    
      Example \#2:
    \\

    
      Marriage has traditionally been between a man and a woman; therefore, gay marriage should not be allowed.
    \\

    
      Explanation:  Very often traditions stem from religious and/or archaic beliefs, and until people question the logic and reasoning behind such traditions, people who are negatively affected by such traditions will continue to suffer.  Just because it was acceptable in past cultures and times, does not mean it is acceptable today.  Think racism, sexism, slavery, and corporal punishment.
    \\

    
      Exception: Victimless traditions that are preserved for the sake of preserving the traditions themselves do not require any other reason.
    \\

    
      Tip: If it weren’t for the creativity of our ancestors, we would have no traditions.  Be creative and start your own traditions that somehow make the world a better place.
    \\

  }


Non-anticipation
    
      - Description: The non-anticipation fallacy assumes that all worthwhile ideas or actions have already been discovered or implemented. It dismisses new ideas on the grounds that if they were valuable, they would have already been part of established wisdom. This fallacy hinders progress by rejecting innovation simply because it hasn't been recognized or adopted in the past.
    \\

    
      
    \\

    
      - Logical Form:
    \\

    
        1. A new idea or proposal (A) is presented.
    \\

    
        2. The idea (A) is dismissed because it has not been previously accepted or recognized.
    \\

    
        3. The rejection is based on the assumption that if (A) were worthwhile, it would already be known and implemented.
    \\

    
      
    \\

    
      - Example \#1:
    \\

    
        - Scenario: "If tobacco really is so harmful, how come people didn’t ban it years ago?"
    \\

    
        - Explanation: This argument assumes that because tobacco wasn't banned in the past, it cannot be harmful. It ignores the fact that past societies may not have had the knowledge or technology to understand tobacco's health effects.
    \\

    
      
    \\

    
      - Example \#2:
    \\

    
        - Scenario: "If breakfast television is all that good, why has it taken so long for it to appear?"
    \\

    
        - Explanation: This statement dismisses the value of breakfast television by suggesting that if it were beneficial, it would have been introduced earlier. It overlooks changes in viewer habits and technological advancements that made breakfast television viable only recently.
    \\

    
      
    \\

    
      - Variation: 
    \\

    
        - Scenario: "People didn’t need these long Christmas holidays years ago, why should they now?"
    \\

    
        - Explanation: This argument assumes that because long holidays weren't common in the past, they are unnecessary now, ignoring changes in work-life balance and societal values over time.
    \\

    
      
    \\

    
      - Tip: When encountering the non-anticipation fallacy, consider whether the rejection of a new idea is based on its own merits or simply because it is novel. Progress often involves adopting new concepts that past generations may not have considered or understood.
    \\

    
      
    \\

    
      - Exception: Some ideas may indeed be rejected repeatedly for valid reasons. It is important to distinguish between innovative ideas and those that have been dismissed after thorough evaluation.
    \\

    
      
    \\

    
      - Fun Fact: The idea of handwashing to prevent infections in hospitals was initially rejected when first proposed by Ignaz Semmelweis in the 19th century. It took many years for the medical community to recognize its importance, highlighting how new, valuable ideas can be overlooked.
    \\

  

Appeal to ancient wisdom
    
      (Also Known As: Appeal to Tradition, Argument from Antiquity, Argumentum ad Antiquitatem,argumentum ad antiquitatis)
    \\

  
    
      - **Description:** The appeal to ancient wisdom is a logical fallacy that asserts a claim or practice is valid or superior because it is old or has been long-standing. This fallacy assumes that because something has been traditionally accepted or practiced for a long time, it must be correct or preferable.
    \\

    
      - **Logical Form:**
    \\

    
        - **P1:** X has been accepted, practiced, or believed for a long time.
    \\

    
        - **P2:** Longevity or tradition is presented as a reason for X’s validity or superiority.
    \\

    
        - **C:** X is considered valid or superior simply because it is ancient or traditional, rather than based on its actual merit or evidence.
    \\

    
      - **Example \#1:** "We should continue using this old medical treatment because it has been used for centuries."
    \\

    
      - **Explanation:** This example assumes that the long history of the treatment automatically makes it effective or valid, without evaluating its current scientific evidence or effectiveness.
    \\

    
      - **Example \#2:** "Our family has always followed this ritual, so it must be the right way to do things."
    \\

    
      - **Explanation:** Here, the argument is based on the idea that the longstanding nature of the ritual makes it correct, rather than assessing its validity or relevance in the present context.
    \\

    
      - **Variation:**
    \\

    
        - **Appeal to Tradition:** Often used interchangeably with appeal to ancient wisdom, but may refer more broadly to any long-standing practice or belief, not just those that are ancient.
    \\

    
        - **Appeal to Novelty:** The opposite fallacy, where something is considered better simply because it is new.
    \\

    
      - **Tip:** Evaluate claims or practices based on their current evidence, logic, and relevance rather than solely on their historical or traditional status. Longevity does not guarantee correctness.
    \\

    
      - **Exception:** Some traditional practices may have valid reasons for their endurance and could be supported by evidence or cultural significance, but this should be assessed on a case-by-case basis rather than assumed based on tradition alone.
    \\

    
      - **Fun Fact:** The term "argumentum ad antiquitatem" translates from Latin as "argument from antiquity," reflecting the fallacy’s reliance on the age or tradition of an idea as its primary justification.
    \\

  

Appeal to nature
    
      (also known as: Argumentum ad Naturam)
    \\

  
    Description: When used as a fallacy, the belief or suggestion that “natural” is better than “unnatural” based on its naturalness. Many people adopt this as a default belief. It is the belief that is what is natural must be good (or any other positive, evaluative judgment) and that which is unnatural must be bad (or any other negative, evaluative judgment).

    
      The {\it appeal to nature fallacy} is often confused with the {\it naturalistic fallacy} and the {\it moralistic fallacy} because they are quite similar. The {\it appeal to nature}, however, specifically references “natural” or “unnatural” and can also make a non-moral judgment such as “beautiful” or “destructive.”
    \\

    
      Logical Forms:
    \\

    
      X is natural. \newline
Y is not natural. \newline
Therefore, X is better than Y. \newline
 \newline

    \\

    
      That which is natural is good, right, beautiful, etc.
    \\

    
      X is natural.
    \\

    
      Therefore, X is good, right, beautiful, etc. \newline
 \newline

    \\

    
      That which is unnatural is bad, wrong, destructive, etc.
    \\

    
      X is unnatural.
    \\

    
      Therefore, X is bad, wrong, destructive, etc.
    \\

    
       \newline

      Example \#1: \newline
 \newline


      
        I shop at Natural Happy Sunshine Store (NHSS), which is much better than your grocery store because at NHSS everything is natural including the 38-year-old store manager’s long gray hair and saggy breasts.
      \\

      
        Explanation: I can appreciate natural food and products as much as the next granola-eating guy, but to make any claim of “betterness”, one needs to establish criteria by which to judge.  Perhaps not paying almost twice as much for the same general foods is “better” for me.  Perhaps I prefer a little insecticide on my apple to insects inside my apple, and maybe I like faux brunettes with perky breasts due to “unnatural” bra support.
      \\

      
        {\it Natural is not always “better”.}
      \\

      
        Example \#2:
      \\

      
        Cocaine is all natural; therefore, it is good for you.
      \\

      
        Explanation: There are very many things in this world that are “all natural” and very bad for you besides cocaine, including, earthquakes, monsoons, and viruses, just to name a few.  Whereas “unnatural” things such as aspirin, pacemakers, and surgery can be very good things.
      \\

      
        Exception: There are many natural things that are better than unnatural, but they must be evaluated based on other criteria besides the “naturalness”.
      \\

      
        Fun Fact: Mother Nature is the kind of mother who wouldn’t hesitate to throw you in a dumpster and leave you there to die.
      \\

      
        Variation: The{\it  naturalistic fallacy} can be seen as a subset of the {\it appeal to nature}, or a more specific version that makes a moralistic value claim rather than the more generic claim of goodness. For example, saying that cocaine is good for you because it is natural is an example of the {\it appeal to nature}. This has nothing to do with morality, but with health. Saying that polyamorous behavior (having multiple sexual partners) is morally good because it seems to be in line with our natural tendencies is an example of the {\it naturalistic fallacy}. If one were to say that the first example was an example of the {\it naturalistic fallacy}, they would be incorrect. If one were to argue that the second was an example of the {\it appeal to nature}, they wouldn't be wrong (if they are equating moral actions with goodness), but they could be more accurate. Just as some dogs are Great Danes but not all dogs are Great Danes, some appeals to nature can be naturalistic fallacies, but not all appeals to nature are naturalistic fallacies.
      \\

    
  

Argument to moderation
    
      (also known as: argumentum ad temperantiam, appeal to moderation, middle ground, false compromise, gray fallacy, golden mean fallacy, fallacy of the mean, splitting the difference, false balance, false equivalency, bothsidesism)
    \\

  
    Description: Asserting that given any two positions, there exists a compromise between them that must be correct.

    
      Logical Form:
    \\

    
      Person 1 says A.
    \\

    
      Person 2 says Z.
    \\

    
      Therefore, somewhere around M must be correct.
    \\

    
      Example \#1:
    \\

    
      So you are saying your car is worth \$20k.  I think it is worth \$1, so let’s just compromise and say it is worth \$10k. (Assuming the car is worth \$20k)
    \\

    
      Explanation: The price of \$20k was a reasonable book value for the car, where the price of \$1 was an unreasonable extreme.  The fact is the car is worth about \$20k -- thinking the car is worth \$1 or \$1,000,000, won’t change that fact[1].
    \\

    
      Example \#2:
    \\

    
      Ok, I am willing to grant that there might not be angels and demons really floating around Heaven or hanging out in Hell, but you must grant that there has to be at least one God.  Is that a fair compromise?
    \\

    
      Explanation: There is no compromise when it comes to truth.  Truth is truth.  If there are angels, demons, and God, there are angels, demons, and God.  If there aren’t, there aren’t.  Compromise and splitting the difference work fine in some cases, but not in determining truth.
    \\

    
      Exception: When the two extremes are equally distanced from the “correct” value -- and there actually {\it is} a correct, or fair, value between the two proposed values.
    \\

    
      So you are saying your car is worth \$40k.  I think it is worth \$1, so let’s just compromise and say it is worth \$20k. (Assuming the car is worth \$20k)
    \\

    
      Tip: If you know you are entering into a negotiation, be prepared to be low-balled, and don’t let those figures change your target figure going into the negotiation. 
    \\

  

Appeal to the stone
    
      (also known as: argumentum ad lapidem)
    \\

  

argument from anecdote

Argument by Pigheadedness
    
      (also known as: argument by stubbornness, invincible ignorance fallacy)
    \\

  
    Description: This is a refusal to accept a well-proven argument for one of many reasons related to stubbornness. It can also be the refusal to argue about a claim that one supports.

    
      Logical Form:
    \\

    
      Argument X is well-argued.
    \\

    
      Person 1 has no objections to the argument, besides just refusing to accept the conclusion.
    \\

    
      Therefore, argument X is not true.
    \\

    
      Example \#1:
    \\

    
      Dad: You are failing math since you moved the Xbox to your room. You have been playing video games for at least 6 hours each day since. Before that, you consistently got A's and B's. Don't you think that the video games are the real problem here?
    \\

    
      Blake: No.
    \\

    
      Explanation: Blake is offering no counter argument or reasoning for rejecting his dad's well-articulated argument. He is simply being stubborn.
    \\

    
      Example \#2:
    \\

    
      Cathy: I hate everything about Michelle Obama!
    \\

    
      Jorge: Do you hate that she launched the national campaign, "Let's Move!," to reduce childhood obesity? \newline
Cathy: Yes. \newline
Jorge: Do you hate that she launched the national veterans' campaign, "Joining Forces," with Dr. Jill Biden? \newline
Cathy: Yes. \newline
Jorge: Do you hate that she traveled to Africa for a week to focus on youth leadership, education, health, and wellness? \newline
Cathy: Yes. \newline
Jorge: Do you hate that she launched the national campaign, "Reach Higher," a higher education initiative? \newline
Cathy: Yes. \newline
Jorge: Do you hate that she launched the national campaign, "Let Girls Learn," a global focus on girls' education? \newline
Cathy: Yes.
    \\

    
      Jorge: Do you hate her well-toned triceps and biceps?
    \\

    
      Cathy: Yes, especially those!
    \\

    
      Explanation: Unreasonable people tend to engage in black-and-white thinking and are committed to an ideological position at any expense—including reason. Cathy is one of those people.
    \\

    
      Exception: Don’t confuse unwillingness to engage with the argument by pigheadedness.
    \\

    
      {\em Street Preacher (to a woman walking by wearing a t-shirt that reads “Thank God I am an atheist!”): You are going to burn in hell!} \newline
{\em Woman: (keeps walking)} \newline
{\em Street Preacher: (frantically quoting Bible verses that support his claim).} \newline
{\em Woman: Yeah, I don’t think so.} \newline
{\em Street Preacher: (Yelling Bible verses louder as the woman gets farther away, while trying to keep up with her).} \newline
{\em Woman: I don’t buy it, sorry! By the way, you just stepped in dog poop.}
    \\

    
      In this example, the woman is not sincerely engaging in the argument. She might have no interest, no time, or simply sees using the Bible to support claims in the Bible as{\em  {\it circular reasoning}}, and not see the street preacher as a worthy interlocutor.
    \\

    
      Tip: As a reminder, avoid absolutes. Instead of saying that you "hate everything" about someone, say something such as, "there's not much I like about..."
    \\

  

Rationalization (psychology)
    
      (also known as: making excuses)
    \\

  
    Description: Offering false or inauthentic excuses for our claim because we know the real reasons are much less persuasive or more embarrassing to share, or harsher than the manufactured ones given.

    
      Logical Form:
    \\

    
      Reason A is given for claim B, although reason A is not the real reason.
    \\

    
      Example \#1:
    \\

    
      I can’t go with you to that opera because I have a deadline at work coming up, plus I need to wash my hair that night.
    \\

    
      Explanation: The real reason is, “I don’t want to go”, but that might hurt some feelings, so manufactured reasons (excuses) are given in place of the authentic and honest reason.
    \\

    
      Example \#2:
    \\

    
      I believe in winged horses because the Koran is historically accurate and would never get such an important fact wrong.
    \\

    
      Explanation: The person actually believes in winged horses out of {\it faith}, but recognizes that is not a persuasive argument -- especially to the non-believer of Islam.  Out of the desire to hold on to his faith, he adopts a common defense (historical accuracy) and gives that as the reason.
    \\

    
      Exception: Is it acceptable to rationalize to protect someone’s feelings?  I will leave that to you to answer, realizing that all situations are unique.
    \\

    
      Tip: Whenever possible, give honest reasons stated in diplomatic ways.
    \\

    References:

    
      
        
      \\

      
        
          Fallacies | Internet Encyclopedia of Philosophy. (n.d.). Retrieved from http://www.iep.utm.edu/fallacy/\#Rationalization
        
      
    
  \section{red herring
    
      (also known as: Ignoratio elenchi, beside the point, misdirection [form of], changing the subject, false emphasis, the Chewbacca defense, irrelevant conclusion, irrelevant thesis, clouding the issue, ignorance of refutation, Irrelevant conclusion, ignoring refutation, Befogging the Issue, Diversion, avoiding the question [form of], missing the point, straying off the subject, digressing, distraction [form of], Avoiding the Issue)
    \\

  
    
      
        Description: When an arguer responds to an argument by not addressing the points of the argument.  Unlike the {\it strawman fallacy}, avoiding the issue does not create an unrelated argument to divert attention, it simply avoids the argument.
      \\

      
        Logical Form:
      \\

      
        Person 1 makes claim X.
      \\

      
        Person 2 makes unrelated statement.
      \\

      
        Audience and/or person 1 forgets about claim X.
      \\

      
        Example \#1:
      \\

      
        Daryl: Answer honestly, do you think if we were born and raised in Iran, by Iranian parents, we would still be Christian, or would we be Muslim?
      \\

      
        Ross: I think those of us raised in a place where Christianity is taught are fortunate.
      \\

      
        Daryl:  I agree, but do you think if we were born and raised in Iran, by Iranian parents, we would still be Christian, or would we be Muslim?
      \\

      
        Ross: Your faith is weak -- you need to pray to God to make it stronger.
      \\

      
        Daryl:  I guess you’re right.  What was I thinking?
      \\

      
        Explanation: Some questions are not easy to answer, and some answers are not easy to accept.  While it may seem, at the time, like avoiding the question is the best action, it is actually an abandonment of reason and honest inquiry; therefore, fallacious.
      \\

      
        Example \#2:
      \\

      
        Molly: It is 3:00 in the morning, you are drunk, covered in lipstick, and your shirt is on backward!  Would you care to explain yourself?
      \\

      
        Rick: I was out with the guys.
      \\

      
        Molly: And the lipstick?
      \\

      
        Rick: You look wonderful tonight, honey!
      \\

      
        Molly: (softening) You think so?  I got my hair cut today!
      \\

      
        Explanation: It is not difficult to digress a line of questioning, so beware of these attempts.
      \\

      
        Exception: At times, a digression is a good way to take the pressure off of a highly emotional argument.  A funny story, a joke,  or anything used as a “break” could be a very good thing at times.  As long as the issue is dealt with again.
      \\

      
        Tip: Don’t avoid questions where you are afraid you won’t like the answers.  Face them head on, and deal with the truth.
      \\

      
        Variation: {\it Distraction} can be a form of {\it avoiding the issue}, but does not have to be just verbal.  For example, being asked a question you can’t answer and pretending your phone rings, saying you need to use the restroom, faking a heart attack, etc.
      \\

    
    \chapter{
      Red Herring
    }
  
    
    

    
      
        Ignoratio elenchi
      \\

      
        (also known as: beside the point, misdirection [form of], changing the subject, false emphasis, the Chewbacca defense, irrelevant conclusion, irrelevant thesis, clouding the issue, ignorance of refutation)
      \\

      
        Description: Attempting to redirect the argument to another issue to which the person doing the redirecting can better respond. While it is similar to the {\it avoiding the issue} fallacy, the {\it red herring }is a deliberate diversion of attention with the intention of trying to abandon the original argument.
      \\

      
        Logical Form:
      \\

      
        Argument A is presented by person 1.
      \\

      
        Person 2 introduces argument B.
      \\

      
        Argument A is abandoned.
      \\

      
        Example \#1:
      \\

      
        Mike: It is morally wrong to cheat on your spouse, why on earth would you have done that?
      \\

      
        Ken: But what is morality exactly?
      \\

      
        Mike: It’s a code of conduct shared by cultures.
      \\

      
        Ken: But who creates this code?...
      \\

      
        Explanation: Ken has successfully derailed this conversation off of his sexual digressions to the deep, existential, discussion on morality.
      \\

      
        Example \#2:
      \\

      
        Billy: How could the universe be 6000 years old when we know the speed of light, the distance of astronomical objects (13+ billion light years away), and the fact that the light has reached us[1]?
      \\

      
        Marty: 6000 years is not a firm number.  The universe can be as old as about 10,000 years.
      \\

      
        Billy: How do you figure that?...
      \\

      
        Explanation: Marty has succeeded in avoiding the devastating question by introducing a new topic for debate... shifting the young-earth creation timeline where it does not necessarily coincide with the Bible.
      \\

      
        Exception: Using a {\it red herring} to divert attention away from your opponent's {\it red herring}, might work, but do two wrongs make a right?
      \\

      
        Variation: {\em Misdirection} is a more generic term for diverting attention away from something. This could be for the purpose of entertainment, avoiding embarrassment, or any other purpose including argumentation.
      \\

      
        Tip: Impress your friends by telling them that there is no such fish species as a "red herring;" rather it refers to a particularly pungent fish—typically a herring but not always—that has been strongly cured in brine and/or heavily smoked.
      \\

    
  }
\subsection{Appeal to Common Belief
    
      (also known as: argumentum ad populum, appeal to accepted belief, appeal to democracy, appeal to widespread belief, appeal to the masses, appeal to belief, appeal to general belief, appeal to the majority, argument by consensus, consensus fallacy, authority of the many, bandwagon fallacy, appeal to the number, argumentum ad numerum, argumentum consensus gentium, appeal to the mob, appeal to the gallery, consensus gentium, mob appeal, social conformance, value of community, vox populi)
    \\

  
    Description: When the claim that most or many people in general or of a particular group accept a belief as true is presented as evidence for the claim. Accepting another person’s belief, or many people’s beliefs, without demanding evidence as to why that person accepts the belief, is lazy thinking and a dangerous way to accept information.

    
      Logical Form:
    \\

    
      {\em A lot of people believe X.}
    \\

    
      {\em Therefore, X must be true.}
    \\

    
      Example \#1:
    \\

    
      Up until the late 16th century, most people believed that the earth was the center of the universe.  This was seen as enough of a reason back then to accept this as true.
    \\

    
      Explanation: The {\it geocentric model} was an observation (limited) and faith-based, but most who accepted the model did so based on the common and accepted belief of the time, not on their own observations, calculations, and/or reasoning.  It was people like Copernicus, Galileo, and Kepler, who refused to {\em appeal to the common belief} and uncovered a truth not obvious to the rest of humanity.
    \\

    
      Example \#2:
    \\

    
      {\em Mark: Do you believe in virgin births?} \newline
{\em Sue: You mean that babies are born virgins?} \newline
{\em Mark: I mean birth without fertilization.} \newline
{\em Sue: No.} \newline
{\em Mark: How could you not believe in virgin births? Roughly two billion people believe in them, don’t you think you should reconsider your position?}
    \\

    
      Explanation: Anyone who believes in virgin births does not have empirical evidence for his or her belief.  This is a claim accepted on faith, which is an individual and subjective form of accepting information, that should not have any effect on your beliefs.  Don’t forget that there was a time that the common beliefs included a flat earth, earth-centered universe, and demon possession as the cause of most illness.
    \\

    
      Exception: Sometimes there are good reasons to think that the common belief is held by people who do have good evidence for believing.  For example, if virtually all of earth scientists accept that the universe is approximately 13.7 billion years old, it is wise to believe them because they will be able to present objective and empirical evidence as to why they believe.
    \\

    
      Tip: History has shown that those who break away from the common beliefs are the ones who change the course of history.  Be a leader, not a follower.
    \\

  }


Argument by Personal Charm
    
      (also known as: Beautiful people, sex appeal, flamboyance, eloquence)
    \\

  
    Description: When an argument is made stronger by the personal characteristics of the person making the argument, often referred to as “charm”.

    
      Logical Form:
    \\

    
      Person 1 says that Y is true.
    \\

    
      Person 1 is very charming.
    \\

    
      Therefore, Y is true.
    \\

    
      Example \#1:
    \\

    
      Hi there, ladies (wink - teeth sparkle). I just want to say that all of you have the right to do what you will with your bodies, including the right to abortion.
    \\

    
      Explanation: The charm of the arguer is irrelevant to the issue of abortion.
    \\

    
      Example \#2:
    \\

    
      Let me start by thanking the wonderful people of this town for hosting this great event.  I would be honored to call you all my friends.  As friends, I want to tell you that streaking should be legalized.
    \\

    
      Explanation: Buttering up the audience is actually a technique that is suggested—because it is effective.  If you know your argument is weak, and compensate by laying on the charm, you are guilty of this fallacious tactic.  If you are letting the charm affect your decision, you are also committing the fallacy.
    \\

    
      Exception: If the argument being made is directly related to the charm of the arguer, as in arguing that he or she would be the better host for a new show where charm does matter, then no fallacy has been committed.
    \\

    
      Tip: If you are a natural charmer don’t be afraid to use it—just not at the expense of valid claims and strong evidence.
    \\

  

Flag-waving

Alleged Certainty
    
      (also known as: Argumentum e consentu gentium, agreement of the people, assuming the conclusion. commoner)
    \\

  
    Description: Asserting a conclusion without evidence or premises, through a statement that makes the conclusion appear certain when, in fact, it is not.

    
      Logical Form:
    \\

    
      Everybody knows that X is true.
    \\

    
      Therefore, X is true.
    \\

    
      Example \#1:
    \\

    
      People everywhere recognize the need to help the starving children of the world.
    \\

    
      Explanation: Actually, people everywhere don’t recognize this.  This may seem like common sense to those who make the claim, and to many who hear the claim, but there are many people on this earth who do not share that view and need to be convinced first.
    \\

    
      Example \#2:
    \\

    
      Everyone knows that, without our culture's religion, we all would be like lost sheep.
    \\

    
      Explanation: Everyone does not know that.  Sometimes, without stepping outside your own social or cultural sphere, it might seem like what you might accept as universal truths are simply truths within your own social or cultural sphere.  Don’t assume universal truths.
    \\

    
      Example \#3:
    \\

    
      There’s no question that our president is a major idiot.
    \\

    
      Explanation: As tempting as a claim such as this one might be for some to accept as true, it must be recognized as the fallacy it is. Simply saying “our president is a major idiot,” then supporting that claim with evidence is very different than replacing the evidence with the statement of certainty (i.e., “there’s no question that...”).
    \\

    
      Exception: Facts that would seem foolish not to assume, can be assumed -- but one should be prepared to support the assumption, no matter how certain one may be.
    \\

    
      We all know that, without water, we cannot survive.
    \\

    
      Tip: Replace the word “certain” in your life with the phrase "very probable" or "very confident."
    \\

  

Appeal to Common Folk
    
      (also known as: appeal to the common man)
    \\

  
    Description: In place of evidence, attempting to establish a connection to the audience based on being a “regular person” just like each of them.  Then suggesting that your proposition is something that all common folk believe or should accept.

    
      Logical Forms:
    \\

    
      X is just common folk wisdom.
    \\

    
      Therefore, you should accept X.
    \\

    
       
    \\

    
      Person 1 is a common man who proposes Y.
    \\

    
      You are also a common man.
    \\

    
      Therefore, you should accept Y.
    \\

    
      Example \#1:
    \\

    
      My fellow Americans, I am just like you.  Sure, I have a few private jets and homes in twelve countries, but I put on my pants one leg at a time, just like you common people.  So believe me when I say, this increase in taxes for the common folk is just what we all need.
    \\

    
      Explanation: There is no valid reason given for the increase in taxes.
    \\

    
      Example \#2:
    \\

    
      You don't want a hot dog and beer?  Eating hot dogs and drinking beer at a baseball game is the American thing to do.
    \\

    
      Explanation: Here the person making the argument is appealing to the {\it tradition} of the common folk.
    \\

    
      Exception: If the “common folk” appeal is made in addition to valid reasons, then it is not a fallacy, although I would argue it is cheap pandering that many people can easily detect.
    \\

    
      Tip: If you are tempted to appeal to some folk, appeal to the folk that made the world a better place. Not only do people love inspirational stories, but these stories are also powerful motivators. Just be sure to use this technique in addition to reason, not in place of it.
    \\

  

Appeal to Popularity
    
      (also known as: argumentum ad numeram, appeal to common belief)
    \\

  
    Description: Using the popularity of a premise or proposition as evidence for its truthfulness.  This is a fallacy which is very difficult to spot because our “common sense” tells us that if something is popular, it must be good/true/valid, but this is not so, especially in a society where clever marketing, social and political weight, and money can buy popularity.

    
      Logical Form:
    \\

    
      Everybody is doing X.
    \\

    
      Therefore, X must be the right thing to do.
    \\

    
      Example \#1: 
    \\

    
      Mormonism is one of the fastest growing sects of Christianity today so that whole story about Joseph Smith getting the golden plates that, unfortunately, disappeared back into heaven, must be true!
    \\

    
      Explanation: Mormonism is indeed rapidly growing, but that fact does not prove the truth claims made by Mormonism in any way.
    \\

    
      Example \#2: 
    \\

    
      A 2005 Gallup Poll found that an estimated 25\% of Americans over the age of 18 believe in astrology—or that the position of the stars and planets can affect people's lives.  That is roughly 75,000,000 people.  Therefore, there must be some truth to astrology!
    \\

    
      Explanation: No, the popularity of the belief in astrology is not related to the truthfulness of astrological claims.  Beliefs are often {\it cultural memes} that get passed on from person to person based on many factors other than truth. 
    \\

    
      Exception: When the claim being made is about the popularity or some related attribute that is a direct result of its popularity.
    \\

    
      People seem to love the movie, {\it The Shawshank Redemption}.  In fact, it is currently ranked \#1 at IMDB.com, based on viewer ratings.
    \\

    
      Tip: Avoid this fallacy like you avoid a kiss from your great aunt with the big cold sore on her lip.
    \\

  

Nutpicking Fallacy
    Description: When someone presents an atypical or weak member of a group as if they are a typical or strong representative.

    
      Logical Form:
    \\

    
      Person X is presented as a typical or strong representative of group Y. \newline
Person X is actually an atypical or weak member of group Y. \newline
Therefore, Person X is seen as a typical or strong representative of group Y.
    \\

    
      Example \#1: Politically ideological individuals on social media consistently post quotes, articles, and stories about the heroes on their side of the political spectrum and villains on the other side in an attempt to influence public perception of their political adversaries. The implied message is, “See, this is what the liberals, are like and this is what the conservatives are like.”
    \\

    
      Example \#2: After the killing of George Floyd, the police officer responsible has been presented by those calling to defund the police as a strong representative of the police in the United States. This representation triggers people’s availability bias resulting in an inaccurate view of police in general. At the same time, FOX News will report on hero cops who save babies, get killed in the line of duty, and replace broken refrigerators for senior citizens. Neither of these portrayals is typical or a strong representation of police in general.
    \\

    
      Exception: As Stephen Woodford points out in his video in the reference, sometimes “nuts” permeate the group to the extent that the “nut” is typical of the group. An example is flat-earthers. I say that unapologetically.
    \\

    
      Variation: A variation of this fallacy is {\it overextended outrage}. Essentially, this is like picking the nuts for the purpose of expressing or inciting outrage toward an entire group.
    \\

    
      Fun Fact: I have heard a couple of people refer to the {\em cherry picking fallacy} as the {\em nutpicking fallacy}. Review the {\em cherry picking fallacy} and you will see that there are notable differences.
    \\

  

Gadarene Swine Fallacy
    
      Description: The assumption that because an individual is not in formation with the group, that the individual must be the one off course. It is possible that the one who appears off course is the only one on the right course.
    \\

    
      
    \\

    
      Logical Form:
    \\

    
      
    \\

    
      Person X stands out from the group.
    \\

    
      
    \\

    
      Therefore, person X is wrong.
    \\

    
      
    \\

    
      Example \#1:
    \\

    
      
    \\

    
      Why can't your daughter fall in line like the other girls?
    \\

    
      
    \\

    
      Explanation: The assumption here is that the "other girls" are doing the right thing. This needs to be established or demonstrated through reason and evidence.
    \\

    
      
    \\

    
      Example \#2:
    \\

    
      
    \\

    
      Many people throughout history started revolutions by taking the morally right action when it was considered morally wrong or even illegal at the time. Consider Rosa Parks.
    \\

    
      
    \\

    
      Exception: It is just as wrong to assume that the one is "out of formation" as it is to assume that all the rest must be "out of formation." While it might be statistically more probable that the one is out of formation, evidence should be sought before making any definitive claim.
    \\

    
      
    \\

    
      Tip: Compare this to the Galileo fallacy . You will see that being the oddball neither makes you right nor wrong.
    \\

  

Imposter Fallacy
    Description: When one suggests or claims, with insufficient evidence, that the group outliers who are viewed as damaging to the group are primarily made up of infiltrators of another group with the purpose of making the infiltrated group look bad.

    
      This is similar to the{\em  {\it nutpicking fallacy} }in that the group outliers and the “nuts” are equally as embarrassing or damaging to the group, but with the {\em imposter fallacy}, the outliers are claimed to be actors or imposters. This is also similar to the {\em no true Scotsman fallacy} in that a foundational claim is that no “true” group member would act in this way, so they must be an imposter.
    \\

    
      Logical Form:
    \\

    
      {\em Group A comprises members X,Y, and Z who act in a way that damages group A’s reputation.} \newline
{\em Person 1 suggests or claims, with insufficient evidence, that members X,Y, and Z are actually part of Group B who are there to make Group A look bad.}
    \\

    
      Example \#1:
    \\

    
      {\em Frieda Freestuff: I can’t believe you support Trump. Didn’t you see the Trump rally with the group of supporters holding up signs for “White Power?” Do you really want to be associated with that message?} \newline
{\em Garry Godznguns: Those aren’t Trump supporters; they are liberals pretending to be Trump supporters just to make Trump supporters look bad.} \newline
{\em Frieda Freestuff: It’s working.}
    \\

    
      Explanation: There is no question that imposters exist. It is not uncommon that political rivals will pretend to be the worst of the other group with the goal of damaging the other group’s reputation. The problem here is that Garry Godznguns has no evidence to back up his claim, and there is plenty of evidence to the contrary that white supremacists and white nationalists overwhelmingly support Trump.
    \\

    
      Example \#2:
    \\

    
      {\em Peter Procops: I can’t believe you support these protests. People are being injured and killed. Property is being destroyed. Stores are being looted. Our city looks like a war zone!} \newline
{\em Patricia Defunddapopo: The protesters are peaceful and lawful. It was just on the news that a guy they arrested for vandalism was a white nationalist who admitted vandalizing to make the protesters look bad.} \newline
{\em Peter Procops: It’s working.}
    \\

    
      Explanation: While it is true that the guy arrested was a white nationalist who admitted vandalizing to make the protesters look bad, this was one case out of thousands. There is video documentation of known activists advocating for looting, vandalism, and even arson. The evidence strongly suggests the majority of destruction is not due to the imposters.
    \\

    
      Exception: This is a fallacy contingent upon evidence. If enough evidence exists that the majority of the group outliers in question are imposters, then there is no fallacy.
    \\

    
      Fun Fact: The imposter fallacy is often committed with the deceptive sharing fallacy when one shares a one-off story about an actual imposter.
    \\

  

Overbelief
    
      - **Description:** Overbelief refers to an excessive or irrational commitment to a belief, often despite conflicting evidence or the lack of reasonable justification. It involves holding a belief with an intensity that surpasses rational assessment or empirical support, frequently leading to an entrenched and unyielding position.
    \\

    
      - **Logical Form:**
    \\

    
        - **P1:** Belief X is held with intense conviction.
    \\

    
        - **P2:** There is substantial evidence contradicting X or lacking support for X.
    \\

    
        - **C:** Despite the evidence, the belief is maintained or intensified.
    \\

    
      - **Example \#1:** "Despite numerous scientific studies showing that homeopathy is ineffective, some individuals continue to assert its benefits with fervent conviction."
    \\

    
      - **Explanation:** This example illustrates how individuals can hold onto a belief in homeopathy despite clear scientific evidence against it, demonstrating overbelief by maintaining the belief strongly even when contradicted by evidence.
    \\

    
      - **Example \#2:** "A person may continue to support a political candidate whose policies are failing to address critical issues effectively, simply because they have an unwavering commitment to the candidate."
    \\

    
      - **Explanation:** This shows overbelief in a political context where the individual persists in their support for a candidate despite evidence of ineffectiveness, driven by their strong personal commitment to the belief.
    \\

    
      - **Variation:**
    \\

    
        - **Confirmation Bias:** This cognitive bias occurs when individuals favor information that confirms their preexisting beliefs, contributing to overbelief.
    \\

    
        - **Cognitive Dissonance:** The psychological discomfort experienced when holding conflicting beliefs may lead individuals to overbelieve in one belief to reduce discomfort.
    \\

    
      - **Tip:** To counteract overbelief, critically evaluate your beliefs by actively seeking out and considering evidence that challenges them. Engage in open-minded discussions and be willing to adjust your views based on new information.
    \\

    
      - **Exception:** Overbelief can be mitigated if an individual is willing to consider and integrate evidence that contradicts their beliefs. Being open to change and questioning one's own convictions helps in avoiding excessive commitment.
    \\

    
      - **Fun Fact:** The term "overbelief" is often used in psychological and philosophical contexts to describe how deeply ingrained beliefs can influence perception and reasoning, sometimes leading to persistent adherence despite clear evidence to the contrary.
    \\

  
    
      (Also Known As: Excessive Belief, Cognitive Overcommitment)
    \\

  

Judgmental language

Logic Chopping
    
      (also known as: quibbling, nit-picking, smokescreen, splitting-hairs, trivial objections)
    \\

  
    Description: Using the technical tools of logic in an unhelpful and pedantic manner by focusing on trivial details instead of directly addressing the main issue in dispute.  Irrelevant over precision.

    
      {\it Pay close attention to this fallacy, because after reading this book, you may find yourself committing this fallacy more than any others, and certainly more often than you did before reading this book.}
    \\

    
      Logical Form:{\it  \newline
}
    \\

    
      {\it A claim is made. \newline
An objection is made regarding a trivial part of the claim, distracting from the main point.}
    \\

    
      Example \#1:
    \\

    
      John: Can you please help me push my car to the side of the road until the tow truck comes?
    \\

    
      Paul: Why push it to the side of the road?  Why not just leave it?
    \\

    
      John: It is slowing down traffic unnecessarily where it is.
    \\

    
      Paul: Many things slow down traffic—do you feel you need to do something about all them?
    \\

    
      John: No, but this was my fault.
    \\

    
      Paul: Was it really? Were you the direct cause of your car breaking down?
    \\

    
      John: Are you going to help me move this damn car or not?!
    \\

    
      Explanation: You can see here that Paul is avoiding the request for assistance by attempting to make a deep philosophical issue out of a simple request.  While Paul may have some good points, not every situation in life calls for deep critical thought.  This situation being one of them.
    \\

    
      Example \#2:
    \\

    
      Service Tech: Your car could use some new tires.
    \\

    
      Bart: You have a financial interest in selling me tires, why should I trust you?
    \\

    
      Service Tech: You brought your car to me to have it checked, sir.
    \\

    
      Bart: I brought my car to the shop where you work.
    \\

    
      Service Tech: So should we forget about the new tires for now?
    \\

    
      Bart: I never suggested that.  Are you trying to use reverse psychology on me, so I will buy the tires?
    \\

    
      Explanation: This kind of fallacy could easily be a result of someone with paranoid behavioral tendencies -- thinking the world is out to get him or her.
    \\

    
      Exception: Of course, there is no clear line between situations that call for critical thought and those that call for reactionary obedience, but if you cross the line, hopefully, you are with people who care about you enough to tell you.
    \\

    
      Tip: People don’t like to be made to feel inferior.  You need to know when tact and restraint are more important than being right.
    \\

    References:

    
      
        
      \\

      
        
          Byerly, H. C. (1973). {\it A primer of logic}. Harper \& Row.
        
      
    
  

Truth by consensus

Irrelevant Humour
    
      - Description: This fallacy occurs when humor is introduced into an argument to divert attention from the main issue. While humor can engage and entertain, using it to sidetrack the discussion detracts from addressing the actual points of the argument. It leverages the audience's amusement to win favor, rather than relying on sound reasoning.
    \\

    
      
    \\

    
      - Logical Form:
    \\

    
        1. Argument A is being discussed.
    \\

    
        2. Humor B is introduced, unrelated to Argument A.
    \\

    
        3. Attention shifts from Argument A to Humor B, distracting the audience from the original issue.
    \\

    
      
    \\

    
      - Example \#1:
    \\

    
        - Scenario: During a parliamentary debate, Thomas Massey-Massey introduces a motion to change the name of Christmas to Christ-tide. An opponent interrupts, asking if Massey-Massey would like to be called 'Thotide Tidey-Tidey'.
    \\

    
        - Explanation: The humorous interruption distracts from the serious debate about the motion, leading to its dismissal amidst the ensuing laughter.
    \\

    
      
    \\

    
      - Example \#2:
    \\

    
        - Scenario: Bishop Wilberforce, debating evolution against Thomas Huxley, asks Huxley if his descent from a monkey was on his grandfather’s or grandmother’s side.
    \\

    
        - Explanation: The joke is used to ridicule Huxley's stance on evolution, diverting attention from the scientific arguments to a humorous but irrelevant question.
    \\

    
      
    \\

    
      - Variation:
    \\

    
        - Scenario: A speaker debating the sale of military planes to authoritarian states is interrupted by a joke suggesting wheelbarrows could also carry nuclear weapons.
    \\

    
        - Explanation: The joke deflects the audience's focus from the serious discussion about arms sales to a humorous, but irrelevant, comparison.
    \\

    
      
    \\

    
      - Tip: Use humor judiciously in debates and arguments. Ensure it enhances the discussion rather than distracting from the main points. If you encounter irrelevant humor, steer the conversation back to the key issues at hand.
    \\

    
      
    \\

    
      - Exception: Humorous anecdotes that are relevant and reinforce the argument can be effective. The fallacy specifically involves humor that is irrelevant and intended to divert attention.
    \\

    
      
    \\

    
      - Fun Fact: In debates, witty retorts can become famous. For instance, when Nancy Astor was asked how many toes a pig has, she replied, "Why don’t you take off your shoes and count them?" This showcased her quick wit and redirected the humor back at the questioner.
    \\

  
    
      (also known as: Humor Red Herring, Jocular Diversion)
    \\

  

fallaca ancidentis
    
      (Also Known As: Fallacy of Accident, Fallacy of Misplaced Concreteness)
    \\

  
    
      - **Description:** Fallacia accidentis occurs when an argument improperly generalizes a specific case or attribute to apply universally. This fallacy involves confusing an accidental property (a feature that is not essential) with a necessary one, leading to incorrect conclusions based on this misunderstanding.
    \\

    
      - **Logical Form:**
    \\

    
        - **P1:** A characteristic or condition X is true in a specific case A.
    \\

    
        - **P2:** X is not essential or necessary for A but is mistaken as such.
    \\

    
        - **C:** X is incorrectly assumed to be true in all cases or for all entities.
    \\

    
      - **Example \#1:** "Since some people are excellent cooks and also introverted, all introverts must be good cooks."
    \\

    
      - **Explanation:** This example demonstrates the fallacy by generalizing a trait (being a good cook) based on an incidental characteristic (introversion) rather than a necessary or inherent trait of all introverts.
    \\

    
      - **Example \#2:** "A successful businessperson often works long hours, so all people who work long hours will be successful in business."
    \\

    
      - **Explanation:** Here, the fallacy arises by assuming that working long hours (an incidental feature) is a necessary condition for business success, ignoring other factors that contribute to success.
    \\

    
      - **Variation:**
    \\

    
        - **Fallacy of Hasty Generalization:** This occurs when a broad conclusion is drawn from a small or unrepresentative sample, similar to generalizing an incidental feature to all cases.
    \\

    
        - **Post Hoc Reasoning:** Attributing a result directly to an incidental factor without proper evidence, often confusing correlation with causation.
    \\

    
      - **Tip:** To avoid fallacia accidentis, ensure that any generalizations or conclusions are based on essential and necessary attributes rather than incidental or accidental features. Carefully analyze whether the characteristic in question is fundamental to the context.
    \\

    
      - **Exception:** The fallacy may not apply if the characteristic in question is genuinely universal and necessary within the specific context being discussed, and not merely an incidental feature.
    \\

    
      - **Fun Fact:** The term "fallacia accidentis" is rooted in classical logic and philosophy, illustrating how ancient thinkers sought to identify and correct errors in reasoning that persist in modern discussions and debates.
    \\

  

Straw man Fallacy
    
      (Strawman Fallacy)
    \\

  
    Description: Substituting a person’s actual position or argument with a distorted, exaggerated, or misrepresented version of the position of the argument.

    
      Logical Form:
    \\

    
      {\em Person 1 makes claim Y.}
    \\

    
      {\em Person 2 restates person 1’s claim (in a distorted way).}
    \\

    
      {\em Person 2 attacks the distorted version of the claim.}
    \\

    
      {\em Therefore, claim Y is false.}
    \\

    
      Example \#1:
    \\

    
      {\em Ted: Biological evolution is both a theory and a fact.}
    \\

    
      {\em Edwin: That is ridiculous!  How can you possibly be absolutely certain that we evolved from pond scum!}
    \\

    
      {\em Ted: Actually, that is a gross misrepresentation of my assertion.  I never claimed we evolved from pond scum.  Unlike math and logic, science is based on empirical evidence and, therefore, a scientific fact is something that is confirmed to such a degree that it would be perverse to withhold provisional consent.  The empirical evidence for the fact that biological evolution does occur falls into this category.}
    \\

    
      Explanation: Edwin has ignorantly mischaracterized the argument by a) assuming we evolved from pond scum (whatever that is exactly), and b) assuming “fact” means “certainty”.
    \\

    
      Example \#2:
    \\

    
      {\em Zebedee: What is your view on the Christian God?}
    \\

    
      {\em Mike: I don’t believe in any gods, including the Christian one.}
    \\

    
      {\em Zebedee: So you think that we are here by accident, and all this design in nature is pure chance, and the universe just created itself?}
    \\

    
      {\em Mike: You got all that from me stating that I just don’t believe in any gods?}
    \\

    
      Explanation: Mike made one claim: that he does not believe in any gods.  From that, we can deduce a few things, like he is not a theist, he is not a practicing Christian, Catholic, Jew, or a member of any other religion that requires the belief in a god, but we cannot deduce that he believes we are all here by accident, nature is chance, and the universe created itself.  Mike might have no beliefs about these things whatsoever.  Perhaps he distinguishes between “accident” and natural selection, perhaps he thinks the concept of design is something we model after the universe, perhaps he has some detailed explanation based on known physics as to how the universe might have first appeared, or perhaps he believes in some other supernatural explanation.  Regardless, this was a gross mischaracterization of Mike’s argument.
    \\

    
      Exception: At times, an opponent might not want to expand on the implications of his or her position, so making assumptions might be the only way to get the opponent to point out that your interpretation is not accurate, then they will be forced to clarify.
    \\

    
      Tip: Get in the habit of {\em steelmanning} the argument. The opposite of the {\em strawman} is referred to as the {\em steelman}, which is a productive technique in argumentation where the one evaluating the argument makes the strongest case for the argument, assuming the best intentions of the interlocutor. This technique prevents pointless, time-wasting bickering and demonstrates respect for both the interlocutor and the process of critical argumentation. Consider the following dialog:
    \\

    
      {\em Johan: The progressive left is making it more difficult for me to vote blue this coming election.} \newline
{\em Sebastian: Why is that?} \newline
{\em Johan: It primarily has to do with the endorsement of the riots.} \newline
{\em Sebastian: So if I understand you correctly, you are saying that the democratic establishment appears to be supporting riots rather than condemning them?} \newline
{\em Johan: Yes, that is exactly it.}
    \\

    
      At this point, Sebastian can present a case for why Johan’s argument fails—his actual argument. Sebastian will also likely be taken more seriously by Johan since he demonstrated goodwill in his attempt to accurately portray Johan’s argument.
    \\

    References:

    
      
        
      \\

      
        Hurley, P. J. (2011). A Concise Introduction to Logic. Cengage Learning.
      \\

    
  \section{special pleading
    Description: Applying standards, principles, and/or rules to other people or circumstances, while making oneself or certain circumstances exempt from the same critical criteria, without providing adequate justification.  Special pleading is often a result of strong emotional beliefs that interfere with reason.

    
      Logical Form:
    \\

    
      {\em If X then Y, but not when it hurts my position.}
    \\

    
      Example \#1:
    \\

    
      {\em Yes, I do think that all drunk drivers should go to prison, but your honor, he is my son!  He is a good boy who just made a mistake!}
    \\

    
      Explanation: The mother in this example has applied the rule that all drunk drivers should go to prison.  However, due to her emotional attachment to her son, she is fallaciously reasoning that he should be exempt from this rule, because, “he is a good boy who just made a mistake”, which would hardly be considered adequate justification for exclusion from the rule.
    \\

    
      Example \#2:
    \\

    
      {\em Superstition is a belief or practice resulting from ignorance, fear of the unknown, trust in magic or chance, or a false conception of causation -- unless it is astrology.}
    \\

    
      Explanation: It has been said that one’s superstition is another’s faith.  The standard of superstition has been defined by the person and violated by astrology.  However, while the person in the example rejects all other sources of superstition using certain criteria, the superstitious belief of their preference is exempt from these criteria.
    \\

    
      Exception: “Adequate justification” is subjective, and can be argued.
    \\

    
      Tip: If you are accused of special pleading, take the time to consider honestly if the accusation is warranted.  This is a fallacy that is easy to spot when others make it yet difficult to spot when we make it.
    \\

    References:

    
      
        
      \\

      
        
          Walton, D. (1999). {\it One-Sided Arguments: A Dialectical Analysis of Bias}. SUNY Press.
        
      
    
  }


Subjectivist Fallacy
    
      (also known as: relativist fallacy)
    \\

  
    Description: Claiming something is true for one person, but not for someone else when, in fact, it is true for everyone (objective) as demonstrated by empirical evidence.

    
      Logical Form:
    \\

    
      Person 1 claims that Y is true.
    \\

    
      Person 2 claims that Y is true for some people, but not for everyone (even though empirical evidence demonstrates otherwise).
    \\

    
      Example \#1:
    \\

    
      Jane: You know, smoking might not be the most healthy habit to start.
    \\

    
      Terry: Smoking is unhealthy for most people, but not for me.
    \\

    
      Explanation: Sorry Terry, smoking is unhealthy for everyone -- you are no different.
    \\

    
      Example \#2:
    \\

    
      Jack: Sorry, your argument is full of contradictions.
    \\

    
      Ted: Contradictions only apply to the carnal mind, not the spiritual one.
    \\

    
      Explanation: Besides being a case of the {\it subjectivist fallacy}, Ted is also moving outside the realm of reason and logic.
    \\

    
      Exception: Many things are actually true or false, depending on the person to which the rule may or may not apply.
    \\

    
      While Twinkies may be horrible to you, I find them delicious—baked, spongy sunshine with a white, creamy, cloud-like center, with the power to make any problem go away—even if just for a brief, magical moment.
    \\

    
      Tip: Stay away from Twinkies.
    \\

    References:

    
      
        
      \\

      
        
          Peacocke, C. (2005). {\it The Realm of Reason}. Clarendon Press.
        
      
    
  

I'm entitled to my opinion\section{Argument from Ignorance
    
      (also known as: argumentum ad ignorantiam, appeal to ignorance, appeal to mystery [form of], black swan fallacy [form of], toupee fallacy [form of], untestability, Unfalsifiability)
    \\

  
    Description: The assumption of a conclusion or fact based primarily on lack of evidence to the contrary.  Usually best described by, “absence of evidence is not evidence of absence.”

    
      Logical Forms:
    \\

    
      X is true because you cannot prove that X is false.
    \\

    
      X is false because you cannot prove that X is true.
    \\

    
      Example \#1: 
    \\

    
      Although we have proven that the moon is not made of spare ribs, we have not proven that its core cannot be filled with them; therefore, the moon’s core is filled with spare ribs.
    \\

    
      Explanation: There is an infinity of things we cannot prove -- the moon being filled with spare ribs is one of them.  Now you might expect that any “reasonable” person would know that the moon can’t be filled with spare ribs, but you would be expecting too much.  People make wild claims, and get away with them, simply on the fact that the converse cannot otherwise be proven.
    \\

    
      Example \#2: 
    \\

    
      To this very day (at the time of this writing), science has been unable to create life from non-life; therefore, life must be a result of divine intervention.
    \\

    
      Explanation: Ignoring the {\it false dilemma}, the fact that we have not found a way to create life from non-life is not evidence that there is no way to create life from non-life, nor is it evidence that we will some day be able to; it is just evidence that we do not know how to do it.  Confusing ignorance with impossibility (or possibility) is fallacious.
    \\

    
      Exception: The assumption of a conclusion or fact deduced from evidence of absence, is not considered a fallacy, but valid reasoning. 
    \\

    
      Jimbo: Dude, did you spit your gum out in my drink?
    \\

    
      Dick: No comment.
    \\

    
      Jimbo: (after carefully pouring his drink down the sink looking for gum but finding none...)  Jackass!
    \\

    
      Tip: Look at all your existing major beliefs and see if they are based more on the lack of evidence than evidence.  You might be surprised as to how many actually are.
    \\

    
      Variations: The {\em Black Swan Fallacy} is committed when one claims, based on past experience, contradictory evidence or claims must be rejected. It is treating the heuristic of induction like an algorithm. The name comes from the claim that “all swans are white” because nobody has ever seen a black swan before... until they did. The reasonable position to hold, assuming you existed in a pre-black-swan world, would be that “all swans that we currently know of are white.” Leave room for discovery unless it has been demonstrated that the contradictory evidence, or claims cannot possibly exist or such claims would be impossible. For example, claiming “all triangles have three sides” is both accurate and reasonable.
    \\

    
      The {\em Toupee Fallacy} is a cleverly-named variation of the appeal to ignorance where the absence of evidence is the result of the claim made being false. Consider the argument, “all toupées look fake; I've never seen one that I couldn't tell was fake.” The reason the person has never seen one they couldn’t tell was fake is because when they did see one they couldn’t tell was fake, they couldn’t tell was fake. The same goes for penile enlargements and boob jobs.
    \\

    
      The {\em Appeal to Mystery} is a specific claim stating that the reason we cannot prove something is because “it is a mystery.” Rather than question if the claim is true, we accept that it is true and forego any more investigation by writing it off as a mystery. Why is it that I smell just fine after working out, but everyone else thinks I stink? It’s a mystery!
    \\

  }


Evidence of absence
    
      (Also Known As: Absence of Evidence Fallacy, Lack of Evidence Fallacy)
    \\

  
    
      - **Description:** The Evidence of Absence fallacy occurs when one concludes that something does not exist or is not true solely because there is a lack of evidence to prove its existence or truth. It is the mistaken belief that if evidence has not yet been found, it does not exist or cannot exist.
    \\

    
      - **Logical Form:**
    \\

    
        - **P1:** No evidence has been found to support the existence of X.
    \\

    
        - **P2:** If X existed, there would be evidence for it.
    \\

    
        - **C:** Therefore, X does not exist.
    \\

    
      - **Example \#1:** "There is no evidence of extraterrestrial life on Earth, so aliens must not exist."
    \\

    
      - **Explanation:** This example commits the fallacy by assuming that the absence of evidence on Earth conclusively proves that extraterrestrial life does not exist anywhere in the universe.
    \\

    
      - **Example \#2:** "We have not found evidence of a cure for a specific disease, so it must be impossible to find one."
    \\

    
      - **Explanation:** This example assumes that the lack of current evidence for a cure means that discovering one is impossible, ignoring the potential for future discoveries and research.
    \\

    
      - **Variation:**
    \\

    
        - **Absence of Evidence as Evidence of Absence:** Using the lack of evidence as a direct proof of non-existence or falsehood.
    \\

    
        - **Appeal to Ignorance:** Similar to the fallacy, where the lack of evidence is used to argue for or against a claim.
    \\

    
      - **Tip:** Evaluate the possibility of future evidence and consider that current evidence might be incomplete or yet to be discovered. Avoid concluding absolute non-existence based on the current lack of evidence.
    \\

    
      - **Exception:** In some cases, a complete lack of evidence may be reasonable grounds for skepticism, especially if exhaustive searches and studies have been conducted. However, it’s crucial to differentiate between reasonable skepticism and outright denial of possibility.
    \\

    
      - **Fun Fact:** The term "Evidence of Absence" is often contrasted with "Absence of Evidence," which refers to the idea that the absence of proof does not prove non-existence, but rather indicates a lack of proof.
    \\

  

Appeal to censorship
    
      (Also Known As: Argument from Silence, Censorship Fallacy)
    \\

  
    
      - **Description:** The Appeal to Censorship occurs when an argument assumes that because a topic is censored, suppressed, or restricted, it must be invalid, false, or unworthy of consideration. This fallacy involves relying on the act of censorship as a means to discredit or avoid discussing a particular idea or argument.
    \\

    
      - **Logical Form:**
    \\

    
        - **P1:** A topic, idea, or argument is censored or suppressed.
    \\

    
        - **P2:** The act of censorship implies that the topic, idea, or argument is false or invalid.
    \\

    
        - **C:** Therefore, the topic, idea, or argument should not be considered or is automatically discredited.
    \\

    
      - **Example \#1:** "The government has banned the book, so it must be full of lies and misinformation."
    \\

    
      - **Explanation:** This example illustrates the fallacy by assuming that censorship (the book being banned) is a valid indicator of the book’s content being false or unreliable, without examining the actual content of the book.
    \\

    
      - **Example \#2:** "You shouldn’t listen to that podcast because it was removed from the streaming platform; it must be promoting harmful or false ideas."
    \\

    
      - **Explanation:** Here, the argument relies on the removal of the podcast from a platform as evidence of its negative value or falsehood, rather than critically evaluating the ideas presented in the podcast.
    \\

    
      - **Variation:**
    \\

    
        - **Ad Hominem Fallacy:** Attacking the credibility of a source or proponent of an idea instead of addressing the idea itself, often involving claims of censorship or suppression.
    \\

    
        - **Red Herring:** Introducing irrelevant information (such as censorship) to divert attention from the actual argument or issue at hand.
    \\

    
      - **Tip:** When evaluating arguments, focus on the content and evidence presented rather than relying on external factors like censorship. Censorship does not inherently determine the validity of an argument or idea.
    \\

    
      - **Exception:** The fallacy may not apply if the censorship is directly relevant to the validity of the argument or idea (e.g., if it involves legally prohibited content), but the focus should still be on the argument's merits rather than the censorship itself.
    \\

    
      - **Fun Fact:** The term "Appeal to Censorship" is often used in discussions about freedom of speech and the role of media in society, highlighting the complex relationship between censorship and the credibility of information.
    \\

  \subsection{Argument from incredulity
    
      (also known as: Argumentum ad iudicium, argument from personal astonishment, argument from personal incredulity, personal incredulity, appeal to common sense, divine fallacy)
    \\

  
    \chapter{
      Appeal to Common Sense
    }
  
    Description: Asserting that your conclusion or facts are just “common sense” when, in fact, they are not. We must argue as to {\it why} we believe something is common sense if there is any doubt that the belief is not common, rather than just asserting that it is. This is a more specific version of {\it alleged certainty}.

    
      Logical Form:
    \\

    
      {\em It's common sense that X is true. \newline
Therefore, X is true.}
    \\

    
      Example \#1:
    \\

    
      {\em It's common sense that if you smack your children, they will stop the bad behavior. So don't tell me not to hit my kids.}
    \\

    
      Explanation: What is often accepted as "common sense" is factually incorrect or otherwise problematic. While hitting your kids may stop their current bad behavior, the long-term psychological and behavioral negative effects can far outweigh the temporary benefits. Logically speaking, the example simply appeals to "common sense" rather than makes an attempt at a strong argument.
    \\

    
      Example \#2:
    \\

    
      {\em Don: If you drink alcohol, it will kill any virus you might have.} \newline
{\em Tony: What evidence do you have to support that? } \newline
{\em Don: I don’t need evidence. It is common sense.}
    \\

    
      Explanation: Since we use alcohol to clean viruses from our skin, it might be “common sense” that it would clean our insides too. But this is not the case. First, the alcohol needs to be high concentration, like 60\% or more to be effective. Second, it does nothing to kill a virus in your body. Third, Don does need evidence if he is trying to convince Tony he is right, and he should demand evidence for himself before engaging in a risky behavior because of what he believes is “common sense.”
    \\

    
      Example \#3:
    \\

    
      {\em FlatSam: Common sense tells us that if the earth were a sphere, people on the bottom would fall off.} \newline
{\em ReasonEric: I don’t think you get to reference something you clearly don’t have.}
    \\

    
      Explanation: FlatSam is attempting to justify his implied argument (the earth is flat) by appealing to “common sense.” He appears to be offering a reason (i.e., people on the bottom would fall off), but that reason is itself a claim “supported” by the appeal to common sense.
    \\

    
      Exception: What is "common sense" to one might not be to another. It is possible one might not accept something that is "common sense," so it could be argued that the error in reasoning falls on the person rejecting the assertion of common sense.
    \\

    
      Tip: It's all about good communication. Keep your assumptions to a minimum when attempting to make a persuasive argument.
    \\

    
    

    \chapter{
      Argument from Incredulity
    }
  
    

    
      (also known as:  argument from personal astonishment, argument from personal incredulity, personal incredulity)
    \\

    
      Description: Concluding that because you can't or refuse to believe something, it must not be true, improbable, or the argument must be flawed. This is a specific form of the {\it argument from ignorance}.
    \\

    
      Logical Form:
    \\

    
      Person 1 makes a claim.
    \\

    
      Person 2 cannot believe the claim.
    \\

    
      Person 2 concludes, without any reason besides he or she cannot believe or refuses to believe it, that the claim is false or improbable.
    \\

    
      Example \#1:
    \\

    
      Marty: Doc, I'm from the future. I came here in a time machine that you invented. Now, I need your help to get back to the year 1985.
    \\

    
      Doc: I've had enough practical jokes for one evening. Good night, future boy!
    \\

    
      Explanation: Clearly Marty is making an extraordinary claim, but the doc's dismissal of Marty's claim is based on pure incredulity. It isn't until Marty provides the Doc with extraordinary evidence (how he came up with the Flux Capacitor) that the Doc accepts Marty's claim. Given the nature of Marty's claim, it could be argued that Doc's dismissal of Marty's claim (although technically fallacious) was the more reasonable thing to do than entertain its possibility with good questions.
    \\

    
      Example \#2:
    \\

    
      NASA: Yes, we really did successfully land men on the moon.
    \\

    
      TinFoilHatGuy1969: Yeah, right. And Elvis is really dead.
    \\

    
      Explanation: The unwillingness to entertain ideas that one finds unbelievable is fallacious, especially when the ideas are mainstream ideas made by a reputable source, such as a NASA and the truthfulness of the moon landings.
    \\

    
      Exception: We can't possibly entertain every crackpot with crackpot ideas. People with little credibility or those pushing fringe ideas need to provide more compelling evidence to get the attention of others.
    \\

    
      Fun Fact: YouTube is not a reliable source. But this doesn’t mean that very reputable sources don’t use YouTube for content distribution. The problem is, so does TinFoilHatGuy1969.
    \\

    
      
    \\

    
      References:
    \\

    
      
        
      \\

      
        
          Bebbington, D. (2011). Argument from personal incredulity. {\it Think}, {\it 10}(28), 27–28. https://doi.org/10.1017/S1477175611000030
        
      
    
  }


Amazing Familiarity
    
      (also known as: argument from omniscience, “how the hell can you possibly know that?”)
    \\

  
    Description: The argument contains information that seems impossible to have obtained—like it came from an omniscient author. This kind of writing/storytelling is characteristic of fiction, so when it is used in an argument it should cast doubt.

    
      Logical Form:
    \\

    
      {\em Claim X is made that nobody could possibly know.}
    \\

    
      Example \#1:
    \\

    
      {\em The president is a good man and would have never cheated on his wife, and has never cheated in anything in the past.}
    \\

    
      Explanation: Clearly the arguer could not know this unless the arguer was with the president all the time. We might assume that the arguer has some special knowledge and find this argument credible when we should only accept it as an opinion from someone who can't possibly know what he or she claims to know.
    \\

    
      Example \#2:
    \\

    
      {\em God wants us to love each other, but he is okay with us killing each other if we are defending our land—we will still go to Heaven.}
    \\

    
      Explanation: Claims of knowing the mind of God are highly dubious. While we cannot rule out "divine revelation," we would need to weigh that possibility against the likelihood of a false belief.
    \\

    
      Example \#3: 
    \\

    
      {\em Larry is pure evil.}
    \\

    
      Explanation: Larry may be an ass. He might have done many bad things. Heck, he might have even have ripped the “do not remove under penalty of law” tag off his mattress. However, to make claims of “pure” evil, one would have to be Larry. While making absolute claims about a person such as being “pure evil” might just be hyperbole, if we take the claim at face value, knowing the claim to be true would require the kind of omniscience reflected in this fallacy.
    \\

    
      Exception: "Seems impossible" is not "impossible." It might be possible that someone actually has the detailed knowledge they claim. We need to keep that option open when thinking probabilistically.
    \\

    
      Tip: By simply adding an “I believe” to non-factual arguments and claims, you can avoid many fallacies and be more honest while exercising more humility.
    \\

  

Argument from omniscience
    
      (also known as: allness, absolute thinking, Omniscience Fallacy)
    \\

  
    
      - **Description:** The Argument from Omniscience fallacy occurs when someone assumes that because a person or entity is omniscient (all-knowing), they are necessarily correct in all their claims or judgments. This fallacy overlooks the possibility that even an omniscient being might still face challenges or that their assertions might be misinterpreted by others.
    \\

    
      - **Logical Form:**
    \\

    
        - **P1:** A person or entity is omniscient.
    \\

    
        - **P2:** Omniscient beings are always correct.
    \\

    
        - **C:** Therefore, any claim or assertion made by this omniscient being is correct.
    \\

    
      - **Example \#1:** "Since God is omniscient, the interpretation of the scriptures given by religious leaders must be correct."
    \\

    
      - **Explanation:** This example assumes that because God is believed to be all-knowing, the interpretations made by those claiming to understand God’s will are necessarily accurate, which may not account for human error or misinterpretation.
    \\

    
      - **Example \#2:** "If a person claims to have divine knowledge and is assumed to be omniscient, then their statement about the future must be true."
    \\

    
      - **Explanation:** This assumes that divine knowledge guarantees accuracy in all predictions or statements about the future, disregarding potential fallibility or miscommunication.
    \\

    
      - **Variation:**
    \\

    
        - **Divine Omniscience Fallacy:** Specifically involves religious or divine claims.
    \\

    
        - **Intellectual Authority Fallacy:** Broader form where expertise or assumed superior knowledge is taken as infallible.
    \\

    
      - **Tip:** Question the basis for the claim of omniscience and consider that even if someone or something is assumed to be all-knowing, human interpretation and understanding can still introduce errors or biases.
    \\

    
      - **Exception:** The fallacy does not apply if the omniscience is not assumed to imply infallibility or if the context involves a demonstrated track record of accuracy in the claims being made.
    \\

    
      - **Fun Fact:** The concept of omniscience is often discussed in philosophy of religion and theology, particularly in debates about the nature of divine attributes and their implications for human understanding.
    \\

  

Argument from silence
    
      (also known as: argumentum ex silentio)
    \\

  
    
      Description: Drawing a conclusion based on the silence of the opponent, when the opponent is refusing to give evidence for any reason.
    \\

    
      Logical Form:
    \\

    
      Person 1 claims X is true, then remains silent.
    \\

    
      Person 2 then concludes that X must be true.
    \\

    
      Example \#1:
    \\

    
      Jay: Dude, where are my car keys?
    \\

    
      Silent Bob: (says nothing)
    \\

    
      Jay: I KNEW you took them!
    \\

    
      Explanation: Refusal to share evidence is not necessarily evidence for or against the argument. Silent Bob’s silence does not mean he took the keys.  Perhaps he did, or perhaps he knows who did, or perhaps he saw a Tyrannosaurus eat them and was threatened by the king of the pixies not to say anything, or perhaps he just felt like not answering. 
    \\

    
      Example \#2:
    \\

    
      Morris: Oh youthful spirit, you have so much to learn.  I know for a fact that there are multiple dimensions that beings occupy.
    \\

    
      Clifton: How can you possibly *know* that for a fact?
    \\

    
      Morris: (raises one eyebrow, stares deeply into the eyes of Clifton and says nothing)
    \\

    
      Clifton: Wow. You convinced me!
    \\

    
      Explanation: The reason this technique works so well, is because {\it imagined reasons are often more persuasive than real reasons}.  If someone wants to be convinced, this technique works like a charm.  However, to the critical thinker, this will not fly.  Silence is not a valid substitute for reason or evidence.
    \\

    
      Exception: Generally speaking, absence of evidence is not evidence; however, there are many cases where the {\it reason} evidence is being held back can be seen as evidence.  In the first example, prompting Silent Bob to share a reason for his silence could result in a statement from Silent Bob that can be used as evidence.
    \\

    
      Tip: Silence can be very powerful. In public speaking, knowing when to pause and let the audience digest what you said helps them comprehend your message. In argumentation, a pause after making a solid point can increase the odds your interlocutor(s) will accept the point.
    \\

  

Holmesian fallacy
    (also known as: Sherlock Holmes fallacy, process of elimination fallacy,far-fetched hypothesis, arcane explanation)
  
    Description: Offering a bizarre (far-fetched) hypothesis as the correct explanation without first ruling out more mundane explanations.

    
      Logical Form:
    \\

    
      {\em Far-fetched hypothesis is proposed.} \newline
{\em Mundane, probable hypotheses are ignored.}
    \\

    
      Example \#1:
    \\

    
      {\em Seth: How did my keys get in your coat pocket?} \newline
{\em Terrence: Honestly, I don’t know,  but I have a theory.  Last night, a unicorn was walking through the neighborhood.  The local leprechauns did not like this intrusion, so they dispatched the fairies to make the unicorn go away.  The fairies took your keys and dropped them on the unicorn, scaring the unicorn back from where he came.  The fairies then returned your keys but accidentally put them in my coat pocket.}
    \\

    
      Explanation: When creating a hypothesis, there are infinite possibilities, but far fewer probabilities.  Skipping over the probabilities is fallacious reasoning.  We should start with the fact that Terrence is lying, then go from there.  There are many theories between lying and the mythical creature caper theory.
    \\

    
      Example \#2:
    \\

    
      {\em The rainbow represents a special covenant or promise of protection from another worldwide flood. The rainbow's appearance to Noah may have been its first occurrence in the sky (Gen. 9:8-17). Typical raindrops of sufficient size to cause a rainbow require atmosphere instability. Prior to the Flood, weather conditions were probably very stable. (Donald B. DeYoung, Weather and The Bible, Grand Rapids, Eerdmans, 1992, pp. 112,113).}
    \\

    
      Explanation: This is part of an attempt by a young-earth creationist to make Genesis a literal, historical fact.  Of course, the mundane explanation is that Genesis is not meant to be taken as a literal, historical fact -- it is not meant to be read as a science book.
    \\

    
      Exception: If mundane explanations can be ruled out first, usually through falsification, then we can move on to more bizarre hypotheses.
    \\

    
      Fun Fact: The {\em Principle of Parsimony} states that the most acceptable explanation of an occurrence, phenomenon, or event is the simplest, involving the fewest entities, assumptions, or changes.
    \\

    References:

    
      
        
      \\

      
        
          Dennett, D. C. (2006). {\it Breaking the Spell: Religion as a Natural Phenomenon}. Penguin.
        
      
    
  

Science doesn't know everything

Science was wrong before

Toupée fallacy
    
      (Also Known As: Hairpiece Fallacy)
    \\

  
    
      - **Name:** Toupée Fallacy
    \\

    
       Survivorship Bias
    \\

    
      - **Description:** The Toupée Fallacy is an informal logical fallacy that arises from selection bias and the problem of induction. It asserts that because some toupées (hairpieces) are obviously fake, all toupées must look fake. This fallacy occurs when someone generalizes about a group based on a limited and non-representative sample, ignoring the possibility of exceptions.
    \\

    
      - **Logical Form:**
    \\

    
        - **P1:** All toupées that have been observed are fake.
    \\

    
        - **P2:** No toupée that looks convincing has been observed.
    \\

    
        - **C:** Therefore, all toupées look fake.
    \\

    
      - **Example \#1:** "Every toupée I've ever seen looks obviously fake, so all toupées must be fake."
    \\

    
      - **Explanation:** This example illustrates the fallacy by generalizing about all toupées based solely on the subset of toupées that are obviously fake, ignoring the existence of well-made toupées that could look realistic.
    \\

    
      - **Example \#2:** "Since I've only seen bad toupées, it must be true that no toupée looks real."
    \\

    
      - **Explanation:** This statement generalizes from personal experience with only poor-quality toupées to a conclusion about all toupées, without considering the possibility of high-quality toupées that might not be noticeable as fake.
    \\

    
      - **Variation:**
    \\

    
        - **Inverted Toupée Fallacy:** "Despite advances in forensic science, undetected murders still occur," generalizing about all forensic advancements based on cases of undetected crimes.
    \\

    
      - **Tip:** To avoid this fallacy, ensure your generalizations are based on a comprehensive and representative sample, and consider the possibility of exceptions that might not be immediately visible.
    \\

    
      - **Exception:** The fallacy does not apply if you have examined a sufficiently large and representative sample that includes high-quality examples, or if there's concrete evidence to support the generalization.
    \\

    
      - **Fun Fact:** The term "Toupée Fallacy" was popularized around 2006, and it humorously highlights how selection bias and confirmation bias can lead to erroneous conclusions in everyday observations.
    \\

  \subsection{Appeal to eye
    
      (also known as: Argumentum ad oculos, argument to eye, Appeal to Appearances, Appeal to the Visible)
    \\

  
    
      - **Description:** Argumentum ad Oculos is a rhetorical fallacy where an argument is made based on the visual appearance or perception of something rather than its actual merit or evidence. It assumes that if something looks a certain way, it must be that way, without further scrutiny.
    \\

    
      - **Logical Form:**
    \\

    
        - **P1:** Person X appears a certain way or makes a claim based on appearance.
    \\

    
        - **P2:** Because of this appearance, the claim or nature of Person X is accepted as true or false.
    \\

    
        - **C:** Therefore, the claim or nature of Person X is true or false based solely on the appearance.
    \\

    
      - **Example \#1:** "You don’t know how to make mistakes in the exam like your neighbor. All has been officially written off."
    \\

    
      - **Explanation:** This example assumes that because a neighbor's mistakes are visible or obvious, the official status or correctness of the neighbor’s situation is clear, without considering other factors.
    \\

    
      - **Example \#2:** "They have sworn an oath, that they are in the middle of the day. However, in two additional instances, they seemed to be absent from the location during the stated time."
    \\

    
      - **Explanation:** This example illustrates the fallacy by focusing on appearances or perceived evidence (absence during specific times) to judge the truthfulness of an oath, without considering other possible explanations or evidence.
    \\

    
      - **Variation:** Argumentum ad Iudicium (Appeal to Judgment) – A related fallacy where judgments are based on superficial impressions rather than a thorough examination.
    \\

    
      - **Tip:** To avoid falling into this fallacy, it’s important to base conclusions on comprehensive evidence and logical reasoning, rather than relying solely on appearances or perceptions.
    \\

    
      - **Exception:** This fallacy is less relevant if appearances consistently match underlying facts or if thorough evidence supports the observed appearance.
    \\

    
      - **Fun Fact:** The term “Argumentum ad Oculos” reflects a focus on visible or apparent qualities, highlighting how visual impressions can sometimes mislead reasoning if not properly scrutinized.
    \\

  }


El Greco fallacy
    
      (Also Known As: The Astigmatism Fallacy)
    \\

  
    
      - **Description:** The El Greco Fallacy is a perceptual fallacy where it is mistakenly assumed that perceptual abnormalities experienced by an individual will affect their representation of the world in a similar manner. It is named after the artist El Greco, whose vertically distorted painting style was erroneously attributed to a supposed astigmatism that distorted his view of the world.
    \\

    
      - **Logical Form:**
    \\

    
        - **P1:** If an individual has a perceptual abnormality, they will represent their surroundings according to that abnormality.
    \\

    
        - **P2:** El Greco's paintings are vertically distorted.
    \\

    
        - **C:** Therefore, El Greco must have had a perceptual abnormality (like astigmatism) that caused him to perceive and paint the world in a distorted manner.
    \\

    
      - **Example \#1:** "El Greco's paintings are elongated vertically because he must have had astigmatism that made him see the world in a distorted way."
    \\

    
      - **Explanation:** This example incorrectly attributes the visual distortion in El Greco's art to a physical perceptual defect, ignoring that such a defect would also affect his view of his canvases, thereby nullifying the supposed distortion.
    \\

    
      - **Example \#2:** "Research suggests that holding rods horizontally makes doorways look narrower because people adapt their perception to match their physical actions."
    \\

    
      - **Explanation:** This example assumes that perceptual distortions from physical actions (like holding rods) directly alter visual perception, without considering alternative explanations or the potential for experimental bias.
    \\

    
      - **Variation:** Misinterpretation of perceptual adaptation effects – assuming that all perceptual changes are directly caused by physical abnormalities or situational factors without considering other influences.
    \\

    
      - **Tip:** Be cautious of attributing perceptual phenomena solely to physical or perceptual abnormalities without considering alternative explanations or the experimental setup.
    \\

    
      - **Exception:** The fallacy does not apply if empirical evidence consistently supports a direct relationship between perceptual abnormalities and specific perceptual distortions.
    \\

    
      - **Fun Fact:** The term "El Greco Fallacy" highlights a common misconception in perception research, where assumptions about perceptual experiences can lead to incorrect conclusions about the nature of perception and its effects.
    \\

  

appeal to calm down
    
      (also known as: argumentum a tuto, appeasement argument, Calm Down Fallacy)
    \\

  
    
      - **Description:** The Appeal to Calm Down is a logical fallacy where someone dismisses or undermines an argument or concern by telling the other person to "calm down" or "take it easy." This tactic is used to invalidate the emotional response rather than addressing the actual issue or argument being presented.
    \\

    
      - **Logical Form:**
    \\

    
        - **P1:** Person A expresses concern or argument about an issue.
    \\

    
        - **P2:** Person B responds by telling Person A to calm down or relax.
    \\

    
        - **C:** Therefore, Person A's concern or argument is invalid or not worth addressing.
    \\

    
      - **Example \#1:** "I think we should discuss the budget cuts; they might affect our team's resources significantly." — "Just calm down, it's not the end of the world."
    \\

    
      - **Explanation:** Instead of addressing the implications of the budget cuts, the response minimizes the concern by telling the speaker to calm down, thereby avoiding a substantive discussion.
    \\

    
      - **Example \#2:** "I'm really upset about the new policy changes; they feel unfair." — "You need to calm down; it's not a big deal."
    \\

    
      - **Explanation:** The response dismisses the speaker's feelings by suggesting that their emotional reaction is exaggerated, rather than addressing the fairness of the policy changes.
    \\

    
      - **Variation:** Appeal to Emotion (Calm Down) – Dismissing an argument by focusing on the emotional state of the person making the argument rather than engaging with the argument itself.
    \\

    
      - **Tip:** When faced with the Appeal to Calm Down, try to refocus the conversation on the substance of the argument rather than getting sidetracked by emotional responses.
    \\

    
      - **Exception:** The fallacy may not apply if the emotional reaction is genuinely disproportionate and interferes with a constructive discussion. However, it's important to address the underlying concerns rather than dismissing them outright.
    \\

    
      - **Fun Fact:** The Appeal to Calm Down is often used in heated debates or discussions as a way to deflect from the issues at hand and can sometimes escalate tensions rather than diffuse them.
    \\

  

Appeal to Self-evident Truth
    Description: Making the claim that something is "self-evident" when it is not self-evident in place of arguing a claim with reason. In everyday terms, something is "self-evident" when understanding what it means immediately results in knowing that it is true, such as 2+2=4. The concept of self-evidence is contentions and argued among philosophers based on their ideas of epistemology. This means that what is "self-evident" to one person is not necessarily self-evident to another. However, some ideas are clearly self-evident and some are not.

    
      Logical Form:
    \\

    
      {\em Person 1 claims Y without evidence.} \newline
{\em Person 2 asks for evidence.} \newline
{\em Person 1 claims that Y is self-evident.}
    \\

    
      Example \#1:
    \\

    
      {\em Richie: Lord Xylon is the one true ruler of the universe.} \newline
{\em Toby: Why do you think that?} \newline
{\em Richie: It is self-evident.}
    \\

    
      Explanation: People often confuse their own subjective feelings and interpretations with self-evidence. Richie may believe that Lord Xylon is the one true ruler of the universe, but his belief cannot be used in place of evidence.
    \\

    
      Example \#2:
    \\

    
      {\em Sara: No human should ever kill another human being.} \newline
{\em Dottie: Why not?} \newline
{\em Sara: It's self-evident.}
    \\

    
      Explanation: The fallacy is in the implied claim that the argument needs no evidence or explanation because it is "self-evident."
    \\

    
      Exception: This fallacy depends on the claim of “self-evidence” not being self-evident. A claim that it reasonably self-evident would not be fallacious:
    \\

    
      {\em Richie: I exist.} \newline
{\em Toby: Why do you think that?} \newline
{\em Richie: It is self-evident.}
    \\

    
      Tip: If you can't explain something, that doesn't mean you are dealing with something that is self-evident; it could just be your failure to explain something.
    \\

  

Just Because Fallacy
    
      (also known as: trust me, mother knows best fallacy, because I said so, you’ll see)
    \\

  
    Description: Refusing to respond to give reasons or evidence for a claim by stating yourself as the ultimate authority on the matter.  This is usually indicated by the phrases, “just trust me”, “because I said so”, “you’ll see”, or “just because”.  The {\it just because fallacy}  is not conducive to the goal of argumentation -- that is coming to a mutually agreeable solution.  Nor is it helpful in helping the other person understand why you are firm on your position. “Just because” is not a reason that speaks to the question itself; it is simply a deflection to authority (legitimate or not).

    
      Logical Form:
    \\

    
      X is true because I said so.
    \\

    
      Example \#1:
    \\

    
      Trebor: Mom, can David sleepover tonight?
    \\

    
      Mom: No.
    \\

    
      Trebor:  Why not?
    \\

    
      Mom: Because.
    \\

    
      Trebor: Because why?
    \\

    
      Mom: Because I said so!  End of discussion!
    \\

    
      Explanation: “Because I said so” is not a valid reason for why a friend can’t sleep over.  Maybe the real reason is that sleepovers give mom a headache.  Maybe mom wants Trebor to go to bed early because he is cranky the next day if he doesn’t, or perhaps David is just a little brat that drives mom crazy.
    \\

    
      Example \#2:
    \\

    
      Slick Rick: The best I can do for ya is \$25,000.
    \\

    
      Prospect: Why can’t you do any better?
    \\

    
      Slick Rick: Just because that is the lowest I can go.
    \\

    
      Prospect: But why.
    \\

    
      Slick Rick: Because.
    \\

    
      Explanation: In this case, it is clear that there is some underlying reason about which Slick Rick does not want the prospect to know.  This reason, almost certainly, has something to do with the fact that Slick Rick can go lower if needed.
    \\

    
      Exceptions: There is really no exception to this rule in argumentation or serious discussion.  Perhaps this is acceptable in situations where you have the authority to choose not to make an argument out of a command, like in a parent-child relationship—or perhaps your significant other has planned a surprise for you, and the “you’ll see” is meant to deflect your inquiry for your own benefit.
    \\

    
      Tip: Don’t let yourself off the hook with “just because” excuses.  Keep asking yourself, “what is the real reason?”  The answer could uncover an issue that needs your attention.
    \\

  \subsection{Appeal to Faith
    Description: This is an abandonment of reason in an argument and a call to faith, usually when reason clearly leads to disproving the conclusion of an argument.  It is the assertion that one must have (the right kind of) faith in order to understand the argument. 

    
      Even arguments that heavily rely on reason that ultimately require faith, abandon reason.
    \\

    
      Logical Form:
    \\

    
      X is true.
    \\

    
      If you have faith, you will see that.
    \\

    
      Example \#1:
    \\

    
      Jimmie: Joseph Smith, the all American prophet, was the blond-haired, blue-eyed voice of God.
    \\

    
      Hollie: What is your evidence for that?
    \\

    
      Jimmie: I don't need evidence—I only need {\it faith}.
    \\

    
      Explanation: There are some things, some believe, that are beyond reason and logic.  Fair enough, but the moment we accept this, absent of any objective method of telling what is beyond reason and why {\it anything goes,} anything can be explained away without having to explain anything.
    \\

    
      Example \#2:
    \\

    
      Tom: Did you know that souls ("Thetans") reincarnate and have lived on other planets before living on Earth, and Xenu was the tyrant ruler of the Galactic Confederacy?
    \\

    
      Mike: No, I did not know that.  How do you know that?
    \\

    
      Tom: I know this through my {\it faith}.  Do you think everything can be known by science alone?  Your faith is weak, my friend.
    \\

    
      Explanation: It should be obvious that reason and logic are not being used, but rather “faith”.  While Tom might be right, there is still no valid reason offered.  The problem also arises in the vagueness of the {\it appeal to faith}.  Tom’s answer can be used to answer virtually any question imaginable, yet the answer is really a deflection.
    \\

    
      St. Bingo: You need to massage my feet.
    \\

    
      Tina: Why?
    \\

    
      St. Bingo: My child, you will only see that answer clearly through the eyes of faith.
    \\

    
      Exception: No exceptions -- the {\it appeal to faith} is always a fallacy when used to justify a conclusion in the absence of reason.
    \\

    
      Tip: Atheist and theist debate often reduces to the concept of faith, sometimes after many hours. It is often a good idea to start with the question, “Is faith a reliable method to know things?” 
    \\

  }


Appeal to Heaven
    
      (also known as: deus vult, gott mit uns, manifest destiny, special covenant)
    \\

  
    Description: Asserting the conclusion must be accepted because it is the “will of God” or “the will of the gods”.  In the mind of those committing the fallacy, and those allowing it to pass as a valid reason, the will of God is not only knowable, but the person making the argument knows it, and no other reason is necessary.

    
      Logical Form:
    \\

    
      God wants us to X.
    \\

    
      Therefore, we should X.
    \\

    
      Example \#1:
    \\

    
      Judge: So why did you chop those people into little pieces and put the pieces in a blender?
    \\

    
      Crazy Larry: Because God told me to do it.
    \\

    
      Judge: Good enough for me.  Next case!
    \\

    
      Explanation: We should all be thankful that our legal system does not work this way, but human thinking does.  Every day, people do things or don’t do things according to what they believe is the will of their god.  Fortunately, most of the time, this does not include a blender.
    \\

    
      Example \#2:
    \\

    
      Ian: Why is the story of Abraham and Isaac regarded as such a “beautiful” Christian story?  The guy was about to burn his son alive as a human sacrifice!
    \\

    
      Wallace: Because it was the will of God that Abraham was following, no matter how difficult it was for him.  Isn’t that beautiful?
    \\

    
      Ian: I guess as long as it was the will of God, being asked to burn children alive is a beautiful thing.
    \\

    
      Explanation: One needs to ask, how do you know it is the will of God?  Satan is said to be the great deceiver -- he would only be great if those being deceived couldn’t tell the difference between God and Satan.  In reality, appealing to Heaven, or God, is an abandonment of logic and reason, and as we have seen, potentially extremely dangerous.
    \\

    
      Exception: When the supposed, "will of God", is in line with what someone would already do or believe based on reason, no fallacy is committed.
    \\

    
      I choose not to kill other people because I would not want them choosing to kill me, plus, I believe that God wouldn’t like it if I did.
    \\

    
      Fun Fact: Sometimes the only difference between faithfulness and insanity is adherence to the law.
    \\

  \subsection{Appeal to Closure
    
      (also known as: appeal to justice [form of])
    \\

  
    Description: Accepting evidence on the basis of wanting closure—or to be done with the issue. While the desire for closure is a real psychological phenomenon that does have an effect on the well-being of individuals, using "closure" as a reason for accepting evidence that would otherwise not be accepted, is fallacious. This is similar to the {\it argument from ignorance} where one makes a claim based on the lack of information because not knowing is too psychologically uncomfortable. However, the {\it appeal to closure} focuses on accepting evidence and for the reason of closure.

    
      Logical Form:
    \\

    
      Evidence X is presented, and found to be insufficient (or evaluated with a heavy bias due to the desire for closure).
    \\

    
      Closure is desired.
    \\

    
      Therefore, evidence X is accepted.
    \\

    
      Example \#1:
    \\

    
      After the terrorist attack on the city, the citizens were outraged and wanted justice. So they arrested a Muslim man with no alibi who looked suspicious then charged him with the crime.
    \\

    
      Example \#2:
    \\

    
      Art: Why didn’t it work out between you and Marci? \newline
Steve: It turns out that she was a lesbian. \newline
Art: Did you know she and Jack just got married? \newline
Steve: [experiences a hard hit to his ego]
    \\

    
      Explanation: I heard from Suzi who told Tanya who told Jennifer that Marci simply thought Steve was a jerk but fed him the classic break up line, “it’s not you, it’s me.” Steve, being the alpha male he is, couldn’t deal with this psychological trauma of not having a firm answer to why Marci didn’t want to be with him, so based on the fact that Marci once commented that a supermodel had “pretty eyes,” Steve put the issue to rest by writing Marci off as a lesbian. Steve discovered that calling Marci a lesbian was actually just him projecting his own homosexual desires on Marci. Today, Steve is living happily with his partner, Raúl, and their white Persian cat, Mr. Muffins.
    \\

    
      Explanation: Unfortunately, unsolved crimes are bad politically for those in charge and based on the number and percentage of false arrests, it is clear that appealing to closure has some serious consequences for many innocent people.
    \\

    
      Exception: It has been stated elsewhere that "agree to disagree" falls under the {\it appeal to closure}. This is not the case because agreeing to disagree does not mean that either party is accepting the evidence of the other, in fact, it's the opposite. People can agree to "move on" or "table the issue," for many logical reasons. This is similar to negotiation and compromise. When people compromise, they usually do not agree to accept evidence they wouldn't otherwise accept. For example, if an atheist and theist are debating the existence of the Biblical God, they wouldn't say, "Okay, I'll agree that some kind of creator god exists if you agree that this god does not currently interfere in the universe."
    \\

    
      Variation: The {\em appeal to justice} is also about closure, but “just” closure. The concept of justice is the focus where the facts and reasoning are secondary. “Do you want to live in a world where people can rob banks and get away with it? Then when it comes time for your verdict, vote Bad Boy Billy guilty!”
    \\

    
      Tip: Remember that justice is a subjective term. When people cry “justice,” they often have created a narrative with very clear heroes and villains, when the truth is far more often unclear.
    \\

  }


Appeal to Equality
    
      (also known as: appeal to egalitarianism, appeal to equity)
    \\

  
    Description: An assertion is deemed true or false based on an assumed pretense of equality, where what exactly is "equal" is not made clear, and not supported by the argument.

    
      Logical Form:
    \\

    
      {\em A equals B (when A does not equal B)} \newline
{\em Y applies to A or B.} \newline
{\em Therefore, Y applies to both A and B.}
    \\

    
      Example \#1:
    \\

    
      {\em If women get paid maternity leave, so should men.}
    \\

    
      Explanation: There are some good reasons that men should get some form of paid paternity leave, but they are not present in this argument. There is an unstated assumption that what benefits women get, for whatever reason, men should get the same. This is {\it begging the question}.
    \\

    
      Example \#2:
    \\

    
      {\em Gay marriage should be the law of the land because gays should have the same rights as heterosexuals.}
    \\

    
      Explanation: What do the "same rights" mean? Before gay marriage, men had the right to marry women, and women had the right to marry men. A man marrying a man is a different form of the "right of marriage" and needs to be argued as such. Again, there are many excellent arguments for gay marriage, but this isn't one of them.
    \\

    
      Example \#3:
    \\

    
      {\em Why should fetuses not have human rights, yet when they exit the womb as babies, they do have human rights? Clearly, fetuses deserve to have the same human rights as the rest of us.}
    \\

    
      Explanation: This argument implies that fetuses and babies outside the womb are equal, and ignores the fact that one is dependent on the resources of the mother's body to survive. Like with gay marriage, there are many excellent arguments for extending human rights to fetuses, but due to the {\it appeal to equality}, this isn't one of them.
    \\

    
      Exception: There is quite a bit of subjectivity in the analysis of this fallacy. Is what is made equal, “clear?” Is this equality supported by the argument? Consider the following:
    \\

    
      {\em People of all races are equal. Human rights apply to everyone, so members of the HootchyCootchie Tribe on the Island of PiddlyWiddly have the right to marriage and family.}
    \\

    
      Unless we are talking about two or more of the same thing (i.e., A=A), then we really mean “equal in some way.” Although not explicitly stated, “equality” is referring to having human rights, which seems pretty clear. The right to marriage and family is a universal human right, so this is supported by the argument.
    \\

    
      Tip: Don't get outraged when one questions what you believe should be "equal." Very often, either you really mean "similar" or you and your opponent have different concepts of what exactly should be equal.
    \\

  \subsection{Pragmatic Fallacy
    
      Description: Claiming that something is true because the person making the claim has experienced, or is referring to someone who has experienced, some practical benefit from believing the thing to be true. The practical benefit is often summarized as “it works.” The person is confusing the truth-value of the claim with the results from believing the claim to be true.
    \\

    
      Logical Form:
    \\

    
      I believe X is true. \newline
Believing in X results in practical benefit Y. \newline
Therefore, X is true.
    \\

    
      Example \#1:
    \\

    
      Starbeam: Of course, astrology is true! \newline
Nate: How do you know this? \newline
Starbeam: Because on the days I forget to consult my horoscope, things always go wrong.
    \\

    
      Explanation: Astrology is a pseudoscience that claims divine information about human affairs and terrestrial events by studying the movements and relative positions of celestial objects. Starbeam is almost certainly experiencing a host of cognitive biases, including the {\em confirmation bias}, where she notices the things that go wrong and ignores all that goes right on the days she forgets to read her horoscope. There is also likely some s{\em elf-fulfilling prophecy} going on where she interprets things that happen to her in a negative light, allowing her to maintain the belief in the power of horoscopes. Starbeam is committing the pragmatic fallacy because she is claiming astrology is true due to her “evidence” that horoscopes work for her.
    \\

    
      Example \#2: People all over the world under different religions that believe in all kinds of gods claim that their particular religion is true because of how believing in their religion makes them feel, the practical benefits they get from being a member of that religion (e.g., community support, social programs, etc.), their belief that they are loved, and more.
    \\

    
      Explanation: All of these reasons as to why religion “works” might be sufficient for why believing in a religion can be good  but these reasons do not address the truth-claims the religion makes about the existence of gods, angels, an afterlife, a soul, and similar claims of existence.
    \\

    
      Exception: This is not a fallacy when what is claimed to be true is the fact that something “works,” and works is defined subjectively (i.e., works for the person and not for everyone). For example, it is fair to say that prayer “works” when “works” is defined as giving (some) people a sense of peace and comfort.
    \\

    
      Seth: Prayer works! \newline
Tina: What do you mean by “works” and how do you know this? \newline
Seth: I mean that it gives me a sense of peace and comfort. \newline
Tina: Will it work for me? \newline
Seth: I don’t know.
    \\

    References:

    
      http://skepdic.com/pragmatic.html
    
  }


Apex fallacy
    
      (Also Known As: Semantic Apex Fallacy)
    \\

  
    
      - **Name:** Apex Fallacy
    \\

    
      
    \\

    
      - **Description:** The Apex Fallacy occurs when someone evaluates a group based on the performance of its top members, rather than a representative sample of all members. This fallacy can lead to misleading conclusions about the overall characteristics or abilities of the entire group.
    \\

    
      - **Logical Form:**
    \\

    
      P1: Of entities in set X, all entities in subset Y are Z.
    \\

    
      P2: (unstated) Y is representative of X.
    \\

    
      C: All entities in set X are Z.
    \\

    
      - **Example \#1:** "The world's best long-distance runners come from Africa. Therefore, all Africans must be excellent long-distance runners."
    \\

    
      - **Explanation:** This is an Apex Fallacy because the conclusion assumes that the exceptional performance of a few top athletes (subset Y) is representative of all Africans (set X), ignoring that the majority of Africans are not top long-distance runners.
    \\

    
      - **Example \#2:** "All the most powerful leaders are men. Therefore, all men must be powerful."
    \\

    
      - **Explanation:** This is an Apex Fallacy because it assumes that the power held by a few prominent men is representative of all men, ignoring that most men are not in positions of power.
    \\

    
      - **Variation:** Nadir Fallacy – This occurs when someone evaluates a group based on the worst-performing members rather than a representative sample.
    \\

    
      - **Tip:** When evaluating a group or making generalizations, ensure that the sample is representative of the entire group rather than focusing only on extreme or exceptional cases.
    \\

    
      - **Exception:** The fallacy may not apply if the exceptional cases are statistically significant and representative, or if the analysis explicitly controls for anomalies in the sample.
    \\

    
      - **Fun Fact:** The term "Apex Fallacy" is frequently discussed in online forums related to gender issues, particularly in critiques of feminism or in discussions about gender disparities in power and success.
    \\

  \par \textbf{Lack of proportion}


Disregarding known science

Exaggeration

Argument by Selective Reading
    Description: When a series of arguments or claims is made and the opponent acts as if the weakest argument was the best one made. This is a form of{\it  cherry picking }and very similar to the {\it selective attention }fallacy.

    
      Logical Form:
    \\

    
      Person 1 makes arguments X, Y, and Z.
    \\

    
      Argument Z is the weakest.
    \\

    
      Person 2 responds as if argument Z was the best person 1 has made.
    \\

    
      Example \#1:
    \\

    
      Kevin: I think there is good evidence that God exists because of the fine-tuning argument, the teleological argument, and perhaps because over 2 billion believe it as well.
    \\

    
      Sydney: It is ridiculous to believe in God just because a lot of other people do too!
    \\

    
      Explanation: Kevin gave three reasons for his belief in God, two are worthy of debate, and one is not. Sydney focused on the one that is not and responded as if that were the only one he made.
    \\

    
      Example \#2:
    \\

    
      Jona: Yes, man did walk on the moon. There is overwhelming evidence that does not come from either NASA or the United States government. Besides, I personally know one of the astronauts involved in one of the Apollo missions, and he confirms that they really did send men to the moon.
    \\

    
      Biff: Your friend is just being paid to perpetuate the lie.
    \\

    
      Explanation: Biff focused on the weaker of the two arguments and ignored the other.
    \\

    
      Exception: One can start by dismissing the weakest arguments first, as long as they get to the strongest one.
    \\

    
      Tip: If your interlocutor begins rattling off several arguments,  politely interrupt them and request that they begin with their strongest argument, and allow you to address that one before proceeding to the next argument.
    \\

  

Willed Ignorance
    
      (also known as: Willful ignorance)
    \\

  
    Description: Refusing to change one’s mind or consider conflicting information based on a desire to maintain one's existing beliefs.

    
      Logical Form:
    \\

    
      I believe X.
    \\

    
      You have evidence for Y.
    \\

    
      I don’t want to see it because I don't want to stop believing in X, so X is still true.
    \\

    
      Example \#1:
    \\

    
      I don’t want anything coming in the way of me and my beliefs; therefore, I will only socialize with people who share my beliefs.
    \\

    
      Explanation: This is a common form of the fallacy -- excluding oneself from society as a whole to smaller subgroups where the same general opinions are shared.
    \\

    Example \#2:

    
      
    
    
      {\em Carl: Exercise causes cancer.} \newline
{\em Janet: That is not true. I have mountains of evidence I can show you that demonstrates the opposite.} \newline
{\em Carl: You keep your exercise propaganda to yourself. I know what I know. Now if you will excuse me, I have to binge watch Baywatch.}
    
    
      
    
    
      Explanation: Carl is blissfully ignorant in his belief that allows him to avoid the wonderful pain of exercise. Perhaps Carl does suspect that he is wrong, but feels he does not have to change his belief until he is proven wrong. Thus, he will not allow Janet the opportunity to prove him wrong.
    
    
      
    
    
      Exception: There may be circumstances where ignorance is truly bliss, and it is better to maintain a positive illusion than to be exposed to a hard truth that one is not psychologically prepared to accept.
    
    
      
    
    
      Fun Fact: This fallacy is similar to the {\em confirmation bias}, but as a fallacy, it is used in argumentation.
    
  

Missing Data Fallacy
    
      (also known as: missing information fallacy)
    \\

  
    Description: Refusing to admit ignorance to the hypothesis and/or the conclusion, but insisting that your ignorance has to do with missing data that validate both the hypothesis and conclusion.

    
      Logical Form:
    \\

    
      {\em Hypothesis H is put forward.}
    \\

    
      {\em Fatal Flaw F is pointed out.}
    \\

    
      {\em Rather than change the hypothesis to match the data, it is simply assumed that there must be data missing that will eliminate flaw F.}
    \\

    
      Example \#1: 
    \\

    
      {\em Jeremy: Drinking Diet Cosie Cola will result in the reversal of male-pattern baldness.}
    \\

    
      {\em Rick: This has never been established scientifically.}
    \\

    
      {\em Jeremy: That is because it must be mixed with another ingredient.}
    \\

    
      {\em Rick: Which is...?}
    \\

    
      {\em Jeremy: They haven’t found it yet.}
    \\

    
      Explanation: Jeremy is assuming the theory is correct based on some unknown missing data (the secret ingredient), rather than admitting that the whole theory is invalid.
    \\

    
      Example \#2: 
    \\

    
      {\em Gil: Scientists have no idea what the appendix is for because they refuse to accept that its function is the source of psychic powers in humans that we have forgotten how to use.}
    \\

    
      {\em John: Scientists actually now know that the appendix serves an important role in the fetus and in young adults.  This is well documented and empirically tested.}
    \\

    
      {\em Gil: This does not mean that it still is not the source of psychic powers—this just has not been tested yet.}
    \\

    
      Explanation: In order to protect the hypothesis from error, it is assumed, without evidence that the answer does exist, but is beyond current scientific understanding. 
    \\

    
      Exception: When the data does exist, especially when it is empirically verified, but you just don't know what it is, it is acceptable to stick with your hypothesis and admit you don’t know the missing data off hand, but you can get it.  For example:
    \\

    
      {\em John: The Shroud of Turin was found many years back.  This is physical proof that Jesus existed.}
    \\

    
      {\em Gil: You know, John, there is plenty of evidence against the authenticity of this.}
    \\

    
      {\em John: Yeah? What specifically?}
    \\

    
      {\em Gil: I honestly don’t know the details off the top of my head, but I can e-mail you when I get home.}
    \\

    
      Tip: Think of the {\em missing data fallacy} like having four cards of a royal flush and a seven of another suit. You don’t have a royal flush; in fact, you have an essentially worthless hand. The difference is, with a full deck of cards, we know that the card to complete the royal flush exists.
    \\

  

Notable Effort
    
      (also known as: “E” is for effort)
    \\

  
    Description: Accepting good effort as a valid reason to accept the truth of the conclusion, even though the effort is unrelated to the truth.

    
      Logical Form:
    \\

    
      Person 1 made a notable effort to prove Y.
    \\

    
      Therefore, Y is true.
    \\

    
      Example \#1:
    \\

    
      Judge: In all my years as a federal judge I have never seen a defendant make such a good effort to prove his innocence.  As a result, I rule for the defendant.
    \\

    
      Explanation: The fact that the defendant made a good effort to prove his innocence means nothing to the fact that he is actually innocent or not—unless he {\it succeeded }in his efforts.  The judge's ruling would be based on emotion and not reason.
    \\

    
      Example \#2:
    \\

    
      How can you possibly deny his claim?  William wrote an entire book trying to explain why he thinks his claim is true.  Therefore, it must be true.
    \\

    
      Explanation: The fact that William made a {\it notable effort}  to prove his claim does not mean that he did.
    \\

    
      Exception: As long as the effort is unrelated to the truth of the claim, there are no exceptions.
    \\

    
      Tip: Enough with the “everyone’s a winner” mentality.  As long as we keep rewarding {\it all} effort, we devalue the effort that leads to {\it successful results}.  The world needs losers as well -- just don’t be one of them.
    \\

  

Just world fallacy
    
      (also known as: Just world hypothesis, Belief in a Just World)
    \\

  
    
      - **Description:** The Just World Fallacy is the cognitive bias that assumes that the world is fundamentally fair, and that people get what they deserve. This fallacy leads to the belief that bad things happen to people because they somehow deserve it, while good things happen to those who are deserving or virtuous.
    \\

    
      - **Logical Form:**
    \\

    
        - **P1:** The world is fair and just.
    \\

    
        - **P2:** Individuals receive outcomes that reflect their moral character or actions.
    \\

    
        - **C:** Therefore, if something bad happens to someone, it must be because they deserved it, and if something good happens, it is because they were deserving.
    \\

    
      - **Example \#1:** "She must have done something to deserve her illness; otherwise, why would such a thing happen to her?"
    \\

    
      - **Explanation:** This example illustrates the fallacy by attributing an undeserved illness to the person’s supposed moral failings, rather than considering other factors or random chance.
    \\

    
      - **Example \#2:** "He got promoted because he worked hard and is a good person. People who are unsuccessful must not be trying hard enough or are not as capable."
    \\

    
      - **Explanation:** This example reflects the fallacy by assuming that success is solely due to personal merit, ignoring other factors such as luck, privilege, or systemic issues.
    \\

    
      - **Variation:**
    \\

    
        - **Blaming the Victim:** Holding victims responsible for their suffering or misfortune based on the belief that they must have done something to deserve it.
    \\

    
        - **Moral Luck:** The idea that individuals are morally responsible for outcomes that are beyond their control, due to a perceived fairness in the world.
    \\

    
      - **Tip:** Recognize that misfortune or success can result from a variety of factors, including luck and systemic issues, rather than solely personal merit or fault.
    \\

    
      - **Exception:** The fallacy may not apply in cases where there is clear, direct evidence linking actions to outcomes (e.g., criminal behavior leading to legal consequences), but it is still important to consider the broader context and multiple contributing factors.
    \\

    
      - **Fun Fact:** The term "Just World Fallacy" was popularized by social psychologist Melvin Lerner in the 1960s, who conducted experiments demonstrating how people are inclined to believe in a just world to maintain their sense of security and fairness.
    \\

  

Appeal to Complexity
    Description: Concluding that because you don't understand something, it must not be true, it's improbable, or the argument must be flawed. This is a specific form of the {\it argument from ignorance}.

    
      Logical Form:
    \\

    
      I don't understand argument X.
    \\

    
      Therefore, argument X cannot be true / is flawed / improbable.
    \\

    
      Example \#1:
    \\

    
      Bill the Eye Guy: The development of the eye is monophyletic, meaning they have their origins in a proto-eye that evolved around 540 million years ago. Multiple eye types and subtypes developed in parallel. We know this partly because eyes in various animals show adaption to their requirements.
    \\

    
      Toby: Uh, that sounds made up. I don't think the eye could have evolved.
    \\

    
      Explanation: Yes, the evolution of the eye is confusing to non-biologists and those who are not familiar with evolutionary theory and natural selection. But the complexity of this argument is not a reason to reject it or find it less credible than a simpler claim (e.g. Zeus created eyes from clay).
    \\

    
      Example \#2: If a layperson criticizes a complex policy about which they know nothing or very little, they are probably {\it appealing to complexity}.
    \\

    
      Explanation: Black and white thinking is found where one has a lack of knowledge about a topic. Policies such as national healthcare are incredibly complex where each change has benefits and drawbacks. Non-experts often will dismiss a 1000-page document and say something such as, "Look, it's simple. Do this, this, and this. Problem solved." No, it's not simple, and that wouldn't solve the problem.
    \\

    
      Exception: When the one making the argument doesn't understand what they are actually saying, they are committing the {\it argument by gibberish}.
    \\

    
      Tip: It is the job of the arguer to make the argument as clear as possible, and use language and terms that the audience can understand. This is a major problem in science communication. The {\it curse of knowledge} often leads to those trying to explain a complex topic in such a way where they assume the audience has as much knowledge in the fields as they do. If your audience fails to understand your argument, don't blame the audience;  explain your argument differently.
    \\

  

Unfalsifiability
    
      (also known as: untestability)
    \\

  
    
      
        Description: Confidently asserting that a theory or hypothesis is true or false even though the theory or hypothesis cannot possibly be contradicted by an observation or the outcome of any physical experiment, usually without strong evidence or good reasons.
      \\

      
        Making unfalsifiable claims is a way to leave the realm of rational discourse, since unfalsifiable claims are often faith-based, and not founded on evidence and reason.
      \\

      
        Logical Form:
      \\

      
        {\em X is true (when X is cannot possibly be demonstrated to be false)}
      \\

      
        Example \#1:
      \\

      
        {\em I have tiny, invisible unicorns living in my anus.  Unfortunately, these cannot be detected by any kind of scientific equipment.}
      \\

      
        Explanation: While it may actually be a fact that tiny, invisible, mythological creatures are occupying this person’s opening at the lower end of the alimentary canal, it is a theory that is constructed so it cannot be falsified in any way; therefore, should not be seriously considered without significant evidence.
      \\

      
        Example \#2:
      \\

      
        {\em Priests can literally turn wine into the blood of Jesus.}
      \\

      
        Explanation: Surely, we can examine the liquid and see if it at least changes chemically, can we not?  No.  Because transubstantiation is not about a physical or chemical change, but a change in “substance” -- which, of course, is not a material change and, therefore, impossible to falsify.  Furthermore, the claim is not that it “might be” happening, but it certainly is happening, adding to the fallaciousness of the claim.  The only evidence for this is some ambiguous verses in the Bible -- so ambiguous that over a billion Christians don’t subscribe to the belief that transubstantiation occurs.  So we have {\it unfalsifiability}, belief of certainty, and very weak evidence.
      \\

      
        Exception: All unfalsifiable claims are not fallacious; they are just unfalsifiable.  As long as proper skepticism is retained and proper evidence is given, it could be a legitimate form of reasoning.
      \\

      
        Tip: Never assume you must be right simply because you can’t be proven wrong.
      \\

      References:

      
        Flanagan, O. J. (1991). {\it The Science of the Mind}. MIT Press.
      
    
  

Argument from fallacy
    (also known as: argumentum ad logicam, disproof by fallacy, argument to logic, fallacy fallacy, fallacist's fallacy, bad reasons fallacy [form of])
  
    Description: Concluding that the truth value of an argument is false based on the fact that the argument contains a fallacy.

    
      Logical Form:
    \\

    
      Argument X is fallacious.
    \\

    
      Therefore, the conclusion or truth claim of argument X is false.
    \\

    
      Example \#1:
    \\

    
      Ivan: You cannot borrow my car because it turns back into a pumpkin at midnight.
    \\

    
      Sidney: If you really think that, you’re an idiot.
    \\

    
      Ivan: That is an ad hominem; therefore, I can’t be an idiot.
    \\

    
      Sidney: I beg to differ.
    \\

    
      Explanation: While it is true that Sidney has committed the {\it ad hominem (abusive)} by calling Ivan an idiot rather than providing reasons why Ivan’s car won’t turn into a pumpkin at midnight, that fallacy is not evidence against the claim.
    \\

    
      Example \#2:
    \\

    
      Karen: I am sorry, but if you think man used to ride dinosaurs, then you are obviously not very well educated.
    \\

    
      Kent:  First of all, I hold a PhD in creation science, so I am well-educated.  Second of all, your ad hominem attack shows that you are wrong, and man did use to ride dinosaurs.
    \\

    
      Karen:  Getting your PhD in a couple of months, from a “college” in a trailer park, is not being well-educated.  My fallacy in no way is evidence for man riding on dinosaurs, and despite what you may think, the Flintstone’s was not a documentary!
    \\

    
      Explanation: Karen’s {\it ad hominem (abusive)} in her initial statement has nothing to do with the truth value of the argument that man used to ride dinosaurs.
    \\

    
      Exception: At times, fallacies are used by those who can’t find a better way to support the truth claims of their argument -- it could be a sign of desperation.  This can be evidence for {\it them not being able to defend their claim}, but not against the claim itself.
    \\

    
      Variation: The {\it bad reasons fallacy} is similar, but the argument does not have to contain a fallacy -- it could just be a bad argument with bad evidence or reasons.  Bad arguments do not automatically mean that the conclusion is false; there can be much better arguments and reasons that support the truth of the conclusion.
    \\

    
      I have never seen God; therefore, he does not exist.
    \\

    
      This is a terrible reason to support a very strong conclusion, but this doesn’t mean that God does exist; it simply means the argument is weak.
    \\

    
      Tip: It may be futile dealing with people who consistently present fallacious arguments. If you find they are simply not very good at reasoning, you can help them learn. However, if they are using fallacious arguments as a form of deception, this is a strong indicator of acting in bad faith.
    \\

  

Double-barreled question
    
      (Also Known as: Double-Direct Question)
    \\

  
    
      - **Description:** A Double-Barreled Question is an informal fallacy that occurs when a question combines two or more distinct issues, but only allows for one answer. This can lead to unclear or inaccurate responses, as the respondent cannot address each issue separately.
    \\

    
      - **Logical Form:**
    \\

    
        - **P1:** Question asks about multiple issues or topics.
    \\

    
        - **P2:** Respondent can only provide one answer.
    \\

    
        - **C:** The answer may not accurately reflect the respondent's views on each individual issue.
    \\

    
      - **Example \#1:** "Do you think that students should have more classes about history and culture?"
    \\

    
      - **Explanation:** This question combines two distinct issues (history and culture) into one query. Respondents may agree with one part (e.g., more history classes) and disagree with the other (e.g., more culture classes), but they can only provide a single response, which may not accurately reflect their opinions on each issue.
    \\

    
      - **Example \#2:** "How satisfied are you with your pay and job conditions?"
    \\

    
      - **Explanation:** This question asks about satisfaction with two separate aspects of employment (pay and job conditions) in one question. A respondent may be satisfied with one aspect but not the other, making it difficult to gauge satisfaction with each individually.
    \\

    
      - **Variation:** Trible (Triple, Treble)-Barreled Question – Similar to a double-barreled question, but involves three or more issues.
    \\

    
      - **Tip:** To avoid double-barreled questions, break down complex queries into separate, simpler questions that address each issue individually.
    \\

    
      - **Exception:** If a question is designed to elicit a single unified response about interconnected issues, it may not be fallacious. However, care should be taken to ensure that the question accurately captures the respondent's views on each component.
    \\

    
      - **Fun Fact:** The term “double-barreled question” is also used in legal contexts, where such questions are referred to as “compound questions” and can lead to objections in court proceedings.
    \\

  

Shaggy defense
    
      (Also Known As: It Wasn't Me Defense)
    \\

  
    
      - **Description:** The Shaggy Defense is a legal strategy where a person denies an accusation with a simple claim of "it wasn't me," despite substantial evidence against them. The term is named after reggae musician Shaggy's 2000 song "It Wasn't Me," which features a narrative of denying wrongdoing despite clear evidence.
    \\

    
      - **Logical Form:**
    \\

    
        - **P1:** There is substantial evidence that the accused is involved in wrongdoing.
    \\

    
        - **P2:** The accused denies involvement with a simple statement, "It wasn’t me."
    \\

    
        - **C:** The defense relies on the denial rather than addressing the evidence.
    \\

    
      - **Example \#1:** During the R. Kelly trial, the musician was accused of being the person in a video depicting illegal activities. His defense was based on the claim that it was not him in the video.
    \\

    
      - **Explanation:** Despite the clear identification of R. Kelly in the video, he maintained that he was not the person depicted, using a denial as his primary defense strategy.
    \\

    
      - **Example \#2:** In the 2010 Virginia court case Preston v. Morton, the defendant denied being the driver of a truck involved in an accident, even though evidence suggested otherwise.
    \\

    
      - **Explanation:** The defendant’s primary defense was the assertion that he was not the driver, regardless of the evidence pointing to his involvement.
    \\

    
      - **Variation:** Trivial Denial - When a person denies involvement in wrongdoing with a simple assertion, even when the evidence is weak or non-existent.
    \\

    
      - **Tip:** When using the Shaggy Defense, it is crucial to recognize that denying the accusation without addressing the evidence often undermines the credibility of the defense.
    \\

    
      - **Exception:** If evidence truly does not support the accusation and the denial is based on factual inaccuracies in the evidence, the defense may be justified.
    \\

    
      - **Fun Fact:** The term "Shaggy Defense" became widely recognized after Slate writer Josh Levin coined it in 2008, inspired by Shaggy's hit song and its portrayal of denial despite clear evidence.
    \\

  

The squeaky wheel gets the grease
    
      - **Description:** This aphorism suggests that issues or concerns that are vocalized or brought to attention are more likely to be addressed compared to those that remain unnoticed or unvoiced. It implies that making noise or drawing attention to a problem increases the likelihood of it being resolved.
    \\

    
      - **Logical Form:**
    \\

    
        - **P1:** An issue or concern is brought to attention through vocalization or complaints.
    \\

    
        - **P2:** Issues that are vocalized receive attention and action.
    \\

    
        - **C:** Issues that are not vocalized are less likely to receive attention.
    \\

    
      - **Example \#1:** A customer repeatedly complains about a defect in a product to the company’s support team, resulting in a quick resolution or refund.
    \\

    
      - **Explanation:** By making their issue known through persistent complaints, the customer draws attention to their problem, which prompts the company to address it promptly.
    \\

    
      - **Example \#2:** A student who frequently raises their concerns about a lack of resources in class may receive additional support or materials, while quieter students might not receive the same level of attention.
    \\

    
      - **Explanation:** The student's vocalization of the problem leads to a direct response from the educational institution, showing that attention-seeking behavior can lead to solutions.
    \\

    
      - **Variation:** "The squeaky wheel gets the grease" can be contrasted with proverbs from other cultures that emphasize the consequences of standing out or making oneself too noticeable.
    \\

    
      - **Tip:** If you want an issue to be addressed, make sure to raise it clearly and persistently. However, balance is key, as excessive complaining can sometimes be counterproductive.
    \\

    
      - **Exception:** The principle does not always hold if the complaint is perceived as unreasonable or if the system for addressing complaints is ineffective.
    \\

    
      - **Fun Fact:** The proverb's origins are attributed to American humorist Josh Billings, who popularized it in the 1870s, though its exact origins remain uncertain. Similar sentiments are found in various cultures, reflecting a common understanding of the importance of advocacy and vocalization in problem-solving.
    \\

  
    
      (Also Known As: The Loudest Wheel Gets the Grease)
    \\

  \section{blaming}


Victim blaming
    
      - **Description:** Blaming the victim is a defense mechanism where the responsibility for a crime or abuse is shifted onto the victim. This often involves suggesting that the victim’s actions, choices, or characteristics are to blame for the abuse they suffered, rather than the perpetrator’s actions.
    \\

    
      - **Logical Form:**
    \\

    
        - **P1:** A crime or abuse has occurred.
    \\

    
        - **P2:** The victim's actions or characteristics are highlighted as contributing factors.
    \\

    
        - **C:** The victim is held responsible for the crime or abuse rather than the perpetrator.
    \\

    
      - **Example \#1:** A rape victim is criticized for wearing provocative clothing, implying that their attire was an invitation for the assault.
    \\

    
      - **Explanation:** This shifts focus from the perpetrator's wrongdoing to the victim's appearance, wrongfully suggesting that their clothing justified the assault.
    \\

    
      - **Example \#2:** After a natural disaster, people blame the victims for not being prepared enough or for living in a risky area, rather than addressing systemic issues or failures in disaster preparedness.
    \\

    
      - **Explanation:** This diverts attention from the lack of adequate infrastructure or support systems to the supposed shortcomings of the victims, minimizing the responsibility of authorities or systems.
    \\

    
      - **Variation:** Denying the victim, which involves asserting that the supposed victims are actually the real aggressors or that their suffering is exaggerated or not valid.
    \\

    
      - **Tip:** It is important to focus on the perpetrator’s actions and address systemic issues rather than blaming victims, which can exacerbate their trauma and discourage them from seeking justice.
    \\

    
      - **Exception:** In some legal contexts, assessing the victim's actions might be relevant to understanding the full context of the crime. However, this should not shift responsibility away from the perpetrator.
    \\

    
      - **Fun Fact:** The concept of blaming the victim has been explored in various fields including psychology and sociology, highlighting its impact on how society perceives and handles crimes and injustices.
    \\

  
    
      (Also known as: Blaming the Victim)
    \\

  

Scapegoating
    Description: Unfairly blaming an unpopular person or group of people for a problem or a person or group that is an easy target for such blame.

    
      Logical Form:
    \\

    
      Nobody likes or cares about X.
    \\

    
      Therefore, X is to blame for Y.
    \\

    
      Example \#1:
    \\

    
      I know I got drunk, slapped the waitress on the behind, then urinated in the parking lot... from inside the restaurant, but that was Satan who had a hold of me.
    \\

    
      Explanation: The person is avoiding personal responsibility and blaming “Satan” for his actions.  Satan is an easy target -- he does not show up to defend himself, and a surprising number of people believe he exists and actually does cause immoral behavior.
    \\

    
      Example \#2:
    \\

    
      The reason New Orleans was hit so hard with the hurricane was because of all the immoral people who live there.
    \\

    
      Explanation: This was an actual argument seen in the months that followed hurricane Katrina.  Ignoring the validity of the claims being made, the arguer is blaming a natural disaster on a group of people.
    \\

    
      Exception: There is no exception when people are being unfairly blamed.
    \\

    
      Fun Fact: {\em Scapegoating} meets a deep psychological need for justice, or more accurately, the {\em belief} that justice has been served.
    \\

    References:

    
      
        
      \\

      
        
          Douglas, T. (2002). {\it Scapegoats: Transferring Blame}. Routledge.
        
      
    
  \section{reverse appeal to emotion
    
      (also known as: pathos gambit)
    \\

  }


Appeal to anger
    
      (also known as: argumentum ad iram, appeal to hatred, loathing, appeal to outrage)
    \\

  
    Description: When the emotions of anger, hatred, or rage are substituted for evidence in an argument.

    
      Logical Forms:
    \\

    
      Person 1 claims that X is true.
    \\

    
      Person 1 is outraged.
    \\

    
      Therefore, X is true.
    \\

    
       
    \\

    
      Claim A is made.
    \\

    
      You are outraged by claim A.
    \\

    
      Therefore, claim A is true/false.
    \\

    
      Example \#1:
    \\

    
      Are you tired of being ignored by your government?  Is it right that the top 1\% have so much when the rest of us have so little?  I urge you to vote for me today!
    \\

    
      Explanation: This is a common tactic to play on the emotions of others to get them to do what you want them to do.  The fact is, no evidence was given or claim was made linking your vote with the problems going away.  The politician will hope you will make the connection while she can claim innocence down the road when the people attempt to hold the politician to a promise she really never made.
    \\

    
      Example \#2:
    \\

    
      How can you possibly think that humans evolved from monkeys!  Does my nanna look like a flippin' monkey to you?
    \\

    
      Explanation: Ignoring the fact that we didn’t evolve from monkeys (we share a common ancestor with modern African apes), the fact that the arguer is obviously offended is irrelevant to the facts.
    \\

    
      Exception: Like all appeals to emotion, they work very well when used, in addition to a supported conclusion, not in place of one.
    \\

    
      Are you tired of being ignored by your government?  Is it right that the top 1\% have so much when the rest of us have so little?  I urge you to vote for me today, and I will spend my career making America a place where the wealth is more evenly distributed!
    \\

    
      Fun Fact: The great Yoda once said, “Fear leads to anger, anger leads to hate, hate leads to suffering.” With all due respect to the cute, little, green guy, anger can be very powerful and effective, as well as lead to great things.  Think of Martin Luther King, Jr.
    \\

    
      By the way, Yoda’s statement actually commits the{\it  slippery slope  fallacy}.
    \\

  

Appeal to Normality
    Description: Using social norms to determine what is good or bad.  It is the idea that normality is the standard of goodness.  This is fallacious because social norms are not the same as norms found in nature or norms that are synonymous with the ideal function of a created system.  The conclusion, "therefore, it is good" is often unspoken, but clearly implied.

    
      Logical Forms:
    \\

    
      {\em X is considered normal behavior. \newline
Therefore, X is good behavior.}
    \\

    
      {\em X is not considered normal behavior. \newline
Therefore, X is bad behavior.}
    \\

    
      {\em X is considered normal behavior. \newline
Therefore, we should strive for X (normality).}
    \\

    
      Example \#1:
    \\

    
      {\em I am only slightly obese.  That is perfectly normal here in America.}
    \\

    
      Explanation:  The person is correct in that being slightly obese is considered normal in America.  In no way is this a good thing by virtually any measure of goodness.  Athletes and those who make their health and fitness a priority are far from normal, but viewing that level of health and fitness as bad is clearly fallacious.
    \\

    
      Example \#2:
    \\

    
      {\em Why doesn't Tim get a real job like normal people instead of trying to launch that Internet business from home?}
    \\

    
      Explanation:  Tim is not like normal people when it comes to work—he is part of the minority who dream big and follow their dreams.  Tim might make it big, or he might not.  Without the Tim's of the world, the normal people would have no place to get a "real job."
    \\

    
      Exception: There are circumstances where eccentric or unusual behavior is clearly problematic. These are situations where even slight deviations from the norm have been demonstrated to have negative results, and the implied “badness” of the behavior needs no justification.
    \\

    
      {\em My dad got arrested again sunbathing naked in a public park while tripping on acid... during a snowstorm. That’s not normal.}
    \\

    
      Another exception is when "normal" is used in such a way to balance negative social behavior.  For example, a mother may yell at her misbehaving child to "act normal" at a school open house.  There is no implication here that being "normal" is the ideal behavior, just an immediate and realistic improvement from the current behavior.
    \\

    
      Tip: For the most part, being "normal" or "average" is nothing to be proud of.  Be better than average.
    \\

  \section{Argument by assertion
    
      Here's the information about "Proof by Assertion" formatted as requested:
    \\

    
      
    \\

    
      - **Name:** Proof by Assertion
    \\

    
      
    \\

    
      - **Description:** Proof by assertion is a logical fallacy in which a proposition is repeatedly stated as if it were self-evident or proven, without providing substantive evidence or reasoning. This method relies on the repetition of the claim rather than engaging in proper argumentation or evidence-based support.
    \\

    
      - **Logical Form:**
    \\

    
        - **P1:** A claim is made with an assertion.
    \\

    
        - **P2:** The claim is repeated several times.
    \\

    
        - **C:** The repeated assertion is considered to be proven or accepted as true without further evidence.
    \\

    
      - **Example \#1:** "It is clear that climate change is not a serious issue. It’s just not a big deal. We should focus on more important problems instead."
    \\

    
      - **Explanation:** The speaker asserts that climate change is not serious and repeats this claim without providing evidence or engaging with counterarguments. The repetition of the assertion is used in place of proper proof.
    \\

    
      - **Example \#2:** "Everyone knows that our product is the best on the market. It’s obvious because we say it is. We’ve always been the best."
    \\

    
      - **Explanation:** The speaker continuously asserts that their product is the best without presenting specific comparisons, data, or evidence. The repeated assertion is meant to convince the audience without any substantive support.
    \\

    
      - **Variation:** Proof by assertion can be seen in various contexts, including advertising, debates, and political discourse. It may also involve making repetitive claims that are not supported by evidence.
    \\

    
      - **Tip:** To avoid falling for proof by assertion, look for concrete evidence and logical reasoning supporting the claims made. Repeated statements should be scrutinized for their validity and substantiation.
    \\

    
      - **Exception:** In some cases, repeated assertions might be used strategically to reinforce a point or emphasize a widely accepted truth, though this should still be supported by evidence.
    \\

    
      - **Fun Fact:** Proof by assertion is often criticized for its lack of intellectual rigor. It contrasts with more valid forms of argumentation, which rely on evidence, logical reasoning, and critical analysis to support claims.
    \\

  
    
      (Also Known As: Assertion Fallacy, Argument by Repetition, Proof by assertion)
    \\

  }
\subsection{Argument from repetition
    
      (also known as: argumentum ad nauseam, argument from nagging, proof by assertion)
    \\

  
    Description: Repeating an argument or a premise over and over again in place of better supporting evidence.

    
      Logical Form:
    \\

    
      X is true. X is true. X is true. X is true. X is true. X is true... etc.
    \\

    
      Example \#1:
    \\

    
      That movie, “Kill, Blood, Gore” deserves the Oscar for best picture.  There are other good movies, but not like that one.  Others may deserve an honorable mention, but not the Oscar, because “Kill, Blood, Gore” deserves the Oscar.
    \\

    
      Explanation: There are no reasons given for why, {\it Kill, Blood, Gore} deserves the Oscar, not even any opinion shared.  All we have is a repeated claim stated slightly differently each time.
    \\

    
      Example \#2:
    \\

    
      Saul: At one time, all humans spoke the same language.  Then because of the Tower of Babel, God got angry and created all the different languages we have today -- or at least some form of them.
    \\

    
      Kevin: I studied linguistics in college, and I can pretty much guarantee you that’s not what happened.  Besides the short story in the Bible, what other evidence do you have to support this theory?
    \\

    
      Saul: We know, because of the Word of God, that God got angry and created all the different languages we have today -- or at least some form of them.
    \\

    
      Kevin: You said that already.  What other evidence do you have to support this theory?
    \\

    
      Saul: In the Bible, it says that all humans once spoke the same language.  Then because of the Tower of Babel, God got angry and created all the different languages we have today -- or at least some form of them.
    \\

    
      Kevin: (nauseated from the repetition, hurls all over Saul’s slacks)
    \\

    
      Explanation: Restating the same claims, even rearranging the words or substituting words, is not the same as making new claims, and certainly does not make the claims any more true.
    \\

    
      Exception: When an opponent is attempting to misdirect the argument, repeating the argument to get back on track is a wise play.
    \\

    
      Tip: Repetition can be a good strategy when your interlocutor does not seem to be acknowledging your point. Rather than repeat yourself, rephrase and repeat. Make your same point but in a different way.
    \\

  }


Sealioning
    
      (Also Known As: Persistent Questioning, Troll Interrogation)
    \\

  
    
      - **Description:** Sealioning is a tactic in which an individual engages in an unreasonable, persistent, and insincere questioning of someone, often with the intent to harass, derail, or exhaust the other party rather than seek genuine answers. It involves a repetitive pattern of asking for evidence or clarification in a manner that is often disingenuous.
    \\

    
      - **Logical Form:**
    \\

    
        - **P1:** Person A makes a claim or expresses an opinion.
    \\

    
        - **P2:** Person B repeatedly asks for evidence or clarification in an unrelenting manner.
    \\

    
        - **P3:** Person B's questioning is not aimed at understanding but at causing frustration or discrediting Person A.
    \\

    
        - **C:** The conversation is derailed, and Person A is pressured into an endless cycle of responses without constructive dialogue.
    \\

    
      - **Example \#1:** "Can you provide sources for your claim that climate change is accelerating? I need to see the data. Where are your sources? How do you know they’re accurate? Can you explain this again? What about this detail? Can you provide a peer-reviewed study to support this?"
    \\

    
      - **Explanation:** The individual asking questions is not genuinely interested in the answers but is instead using this tactic to repeatedly challenge the other person in a manner that is meant to be overwhelming and disruptive.
    \\

    
      - **Example \#2:** "You say this policy will benefit everyone. Can you provide a detailed breakdown of how it will help each demographic? How will it address specific concerns? Can you show evidence from multiple studies? What about the long-term effects?"
    \\

    
      - **Explanation:** The questioning is excessive and focused on undermining the claim rather than engaging in a meaningful discussion. The aim is to keep the discussion going in circles without reaching a productive outcome.
    \\

    
      - **Variation:** Sealioning can vary in its level of aggression and persistence. It may involve multiple tactics, such as shifting goalposts, making unreasonable demands for proof, or ignoring responses to continue questioning.
    \\

    
      - **Tip:** To handle sealioning, set clear boundaries for the discussion and avoid engaging in an endless cycle of responses. If necessary, disengage from the conversation and address the behavior directly if it becomes disruptive.
    \\

    
      - **Exception:** Not all persistent questioning is sealioning. Genuine inquiries and constructive debate involve asking questions with the intent to understand and engage in meaningful dialogue.
    \\

    
      - **Fun Fact:** The term "sealioning" originated from a comic strip by David Malki!, in which a sealion repeatedly asks for evidence and clarification in a manner that is both unreasonable and disruptive.
    \\

  

Proof by assertion\section{ is–ought problem
    
      (also known as: is-ought fallacy, arguing from is to ought, is-should fallacy, Hume's law)
    \\

  
    Description: When the conclusion expresses what ought to be, based only on what is, or what ought not to be, based on what is not. This is very common, and most people never see the problem with these kinds of assertions due to accepted social and moral norms. This bypasses reason and we fail to ask why something that is,{\it  ought} to be that way.

    
      This is the opposite of the {\it moralistic fallacy}.
    \\

    
      A more traditional use of the {\it naturalistic fallacy} is committed when one attempts to define “good” as anything other than itself. The philosopher G. E. Moore (1873-1958) argued that it is a mistake to try to define the concept “good” in terms of some natural property (thus, the name “naturalistic”). Defining the concept “good,” Moore argued, is impossible since it is a simple concept; a concept that cannot be defined in terms of any other concept. Not all philosophers agree that this is a fallacy. Some have argued that ethical terms, such as “good” can be defined in nonethical natural terms. They believe that ethical judgments directly follow from facts, i.e., it is possible to argue from a fact to a value.
    \\

    
      Logical Forms:
    \\

    
      X is. \newline
Therefore, X ought to be. \newline
 \newline

    \\

    
      X is not.
    \\

    
      Therefore, X ought not to be.
    \\

    
       
    
    
      Example \#1:
    
    
       
    
    
      Homosexuality is / ought to be morally wrong (moral property) because it is not normal (natural property). \newline
or \newline
Homosexuality is not normal (natural property); therefore, it is / ought to be morally wrong (moral property).
    \\

    
       
    \\

    
      
        Explanation: If we break this down, we can say the claim is that homosexuality (X) is not normal (X is not). We are arguing that homosexuality is morally wrong (X ought not to be) because it is not normal (X is not). The claim that homosexuality is not normal is based on defining normality as “commonly occurring.” We can see the flaw in this argumentation through a simple analogy: lying, cheating, and stealing are normal (in that most people do it at some time in their lives), but this doesn’t make those actions morally good.
      \\

      
        Example \#2:
      \\

      
        Nature gives people diseases and sickness; therefore, it is morally wrong to interfere with nature and treat sick people with medicine.
      \\

      
        Explanation: If we break this down, we can say that the claim that nature gives people diseases and sickness is a declaration of what is (i.e., a natural property of the world). From this, we are deriving an ought (i.e., we ought not interfere...). The wording and order of these arguments can be confusing, but remember that the underlying fallacy here is the deduction of an ought from an is.
      \\

      
        We go against nature (or what is) all the time. We cannot sometimes use nature as a moral baseline and at other times condemn her for her careless attitude and indifference toward the human race.
      \\

      
        Exception: At times, our morality will be in line with what is—but if we are justifying a moral action, we need to use something besides simply what is.
      \\

      
        Fun Fact: The {\it naturalistic} and the {\it moralistic}  fallacies are often confused with the appeal to nature fallacy. One reason, perhaps, is because what is “natural” is another way of saying what is, is. But with the {\it naturalistic} and the {\it moralistic}  fallacies, the conclusion does not have to be based on what is “natural;” it just has to be based on what is. For example,
      \\

      
        Since wars have taken place since the beginning of recorded history, then they can’t be morally wrong.
      \\

      
        This is another example of the {\it naturalistic fallacy} but not an {\it appeal to nature}.
      \\

      
        Social Media Share Image: Drag and drop this to your desktop to use freely in social media or just share this page.
      \\

    
    References:

    
      
        
      \\

      
        
          Pinker, S. (2003). {\it The Blank Slate: The Modern Denial of Human Nature}. Penguin.
        
        
          Tanner, J. (2006). The naturalistic fallacy. The Richmond Journal of Philosophy, 13, 1–6.
        
      
    
  }
\subsection{Moralistic fallacy
    
      (also known as: moral fallacy)
    \\

  
    Description: When the conclusion expresses what is, based only on what one believes ought to be, or what isn’t is based on what one believes ought not to be.

    
      This is the opposite of the {\it naturalistic fallacy}.
    \\

    
      In his 1957 paper, Edward C. Moore defined the {\it moralistic fallacy}  as the assertion that moral judgments are of a different order from factual judgements. Over the years, this concept has been simplified to deriving an “is” from an “ought.”
    \\

    
      Logical Forms:
    \\

    
      X ought to be. \newline
Therefore, X is. \newline
 \newline

    \\

    
      X ought not to be.
    \\

    
      Therefore, X is not.
    \\

    
       \newline

      Example \#1: \newline
 \newline


      
        Adultery, as well as philandering, is wrong.
      \\

      
        Therefore, we have no biological tendency for multiple sex partners.
      \\

      
        Explanation: While, morally speaking, adultery and philandering may be wrong, this has no bearing on the biological aspect of the desire or need. In other words, what we shouldn’t do (according to moral norms), is not necessarily the same as what we are biologically influenced to do. Also note that moral judgments are more commonly stated as facts (e.g., “philandering is wrong”) than expressed as “oughts” (e.g., “philandering ought to be wrong”). This causes people to confuse the naturalistic and moralistic fallacies.
      \\

      
        Example \#2:
      \\

      
        Being mean to others is wrong.
      \\

      
        Therefore, it cannot possibly be part of our nature.
      \\

      
        Explanation: While, morally speaking, being mean to others may be wrong, this has no bearing on the biological aspect of the desire or need. Again, what we shouldn’t do (according to moral norms), is not necessarily the same as what we are biologically influenced to do.
      \\

      
        Exception: An argument can certainly be made that an ought is the same as an is, but it just cannot be assumed.
      \\

      
        Fun Fact: The {\it naturalistic} and the {\it moralistic}  fallacies are often confused with the {\it appeal to nature}  fallacy. One reason, perhaps, is because what is “natural” is another way of saying what is, is. But with the {\it naturalistic} and the {\it moralistic}  fallacies, the conclusion does not have to be based on what is “natural;” it just has to be based on what is. For example,
      \\

      
        Men and women ought to be equal. Therefore, women are just as strong as men and men are just as empathetic as women.
      \\

      
        This is another example of the {\it moralistic fallacy} but not an {\it appeal to nature}.
      \\

    
    References:

    
      
        
      \\

      
        
          Moore, E. C. (1957). The Moralistic Fallacy. {\it The Journal of Philosophy}, 54(2), 29–42. https://doi.org/10.2307/2022356
        
        
          Pinker, S. (2003). {\it The Blank Slate: The Modern Denial of Human Nature}. Penguin.
        
      
    
  }


Demoralization

Virtue signalling
    
      - **Description:** Virtue signalling refers to the act of expressing opinions or sentiments intended to demonstrate one's good character or moral correctness. It is often criticized when these expressions are perceived as insincere or superficial, and are done primarily to enhance the speaker's social status or self-image, rather than to contribute to meaningful action or change.
    \\

    
      - **Logical Form:**
    \\

    
        - **P1:** An individual or group makes a public statement or action expressing a moral or ethical stance.
    \\

    
        - **P2:** The primary intention behind the statement or action appears to be the enhancement of their own social image or status.
    \\

    
        - **C:** Therefore, the statement or action may be considered virtue signalling rather than a genuine effort to address the issue.
    \\

    
      - **Example \#1:** A celebrity posts a photo on social media with a hashtag supporting a social cause, but their overall actions do not reflect any substantial contribution to that cause.
    \\

    
      - **Explanation:** The celebrity's social media post may be seen as virtue signalling if it is perceived as an attempt to gain social approval or enhance their public image, without corresponding substantive actions or commitments to the cause they are endorsing.
    \\

    
      - **Example \#2:** A politician publicly condemns a controversial issue to gain favor with voters, but their legislative record shows little effort to address the issue in practice.
    \\

    
      - **Explanation:** The politician’s public condemnation may be viewed as virtue signalling if it is seen as a strategic move to appeal to constituents, rather than a reflection of genuine concern or a commitment to effecting real change.
    \\

    
      - **Variation:** Virtue signalling can also occur in various forms, including public statements, symbolic gestures, or performative acts that are more about showcasing one’s moral stance rather than engaging in meaningful action.
    \\

    
      - **Tip:** To avoid being perceived as virtue signalling, ensure that public expressions of moral or ethical stances are supported by tangible actions and efforts that contribute to addressing the issues discussed.
    \\

    
      - **Exception:** Virtue signalling can sometimes be a starting point for genuine engagement if it leads to concrete actions and sustained efforts to address the underlying issues.
    \\

    
      - **Fun Fact:** The term "virtue signalling" gained popularity in the early 2010s and is often used in discussions about social media and public relations, highlighting the complexities of online identity and activism.
    \\

  \subsection{Economic fallacies}


Hyperbolic discounting
    
      - **Description:** Hyperbolic discounting is a cognitive bias that describes the tendency for people to prefer smaller, more immediate rewards over larger, delayed rewards. This bias results in people disproportionately valuing immediate gratification, often at the expense of long-term benefits. It contrasts with exponential discounting, which assumes a constant rate of time preference.
    \\

    
      - **Logical Form:**
    \\

    
        - **P1:** People often face choices between smaller rewards available sooner and larger rewards available later.
    \\

    
        - **P2:** People tend to prefer the smaller, sooner rewards, even if the delayed reward is significantly larger.
    \\

    
        - **C:** Therefore, people demonstrate hyperbolic discounting by placing higher value on immediate gratification relative to future benefits.
    \\

    
      - **Example \#1:** Choosing to spend \$50 now on a new gadget rather than saving that money to invest in a larger purchase or investment opportunity in the future.
    \\

    
      - **Explanation:** The immediate pleasure of acquiring the gadget outweighs the greater long-term value of saving or investing the money, illustrating the preference for immediate rewards over future gains.
    \\

    
      - **Example \#2:** Opting to watch a TV show now rather than spending that time working on a project that could yield significant future benefits, like career advancement.
    \\

    
      - **Explanation:** The immediate enjoyment of watching TV is prioritized over the potential long-term benefits of completing the project, reflecting the tendency to undervalue future rewards.
    \\

    
      - **Variation:** Hyperbolic discounting can vary in strength depending on the context and the individual’s self-control and long-term planning ability. It is often more pronounced in situations where immediate rewards are more tangible or emotionally satisfying.
    \\

    
      - **Tip:** To mitigate the effects of hyperbolic discounting, create structured plans with specific goals and rewards for achieving long-term objectives. Setting up commitment devices can also help in sticking to long-term plans.
    \\

    
      - **Exception:** In some cases, individuals may exhibit less hyperbolic discounting if they have strong future-oriented goals or if they experience a high level of immediate urgency that overrides their usual preference for short-term rewards.
    \\

    
      - **Fun Fact:** Hyperbolic discounting is often discussed in behavioral economics and psychology, and it helps explain various phenomena like procrastination and impulsive buying. It contrasts with the more traditional economic assumption of exponential discounting, which suggests a constant rate of time preference.
    \\

  

Sunk-Cost Fallacy
    
      (also known as: argument from inertia, concorde fallacy, finish the job fallacy)
    \\

  
    Description: Reasoning that further investment is warranted on the fact that the resources already invested will be lost otherwise, not taking into consideration the overall losses involved in the further investment.

    
      Logical Form:
    \\

    
      X has already been invested in project Y.
    \\

    
      Z more investment would be needed to complete project Y, otherwise X will be lost.
    \\

    
      Therefore, Z is justified.
    \\

    
      Example \#1:
    \\

    
      I have already paid a consultant \$1000 to look into the pros and cons of starting that new business division.  He advised that I shouldn’t move forward with it because it is a declining market.  However, if I don’t move forward, that \$1000 would have been wasted, so I better move forward anyway.
    \\

    
      Explanation: What this person does not realize is that moving forward will most likely result in the loss of much more time and money.  This person is thinking short-term, not long-term, and is simply trying to avoid the loss of the \$1000, which is fallacious thinking.
    \\

    
      Example \#2: There are ministers, priests, pastors, and other clergy all around the world who have invested a significant portion of their lives in theology, who can no longer manage to hold supernatural beliefs -- who have moved beyond faith.  Hundreds of them recognize those sunk costs and are searching for the best way to move on (see http://www.clergyproject.org) whereas many others cannot accept the loss of their religious investment, and continue to practice a profession inconsistent with their beliefs.
    \\

    
      Explanation: Of course, the clergy who have not moved beyond faith and are living consistently with their beliefs have not committed this fallacy.
    \\

    
      Exception: If a careful evaluation of the hypothetical outcomes of continued investment versus accepting current losses and ceasing all further investment have been made, then choosing the former would not be fallacious.
    \\

    
      Tip: Is there any part of your life where you continue to make bad investments because you fear to lose what was already invested? Do something about it.
    \\

    References:

    
      
        
      \\

      
        
          Besanko, D., \& Braeutigam, R. (2010). {\it Microeconomics}. John Wiley \& Sons.
        
      
    
  

Broken window fallacy
    
      (also known as: glazier's fallacy)
    \\

  
    Description: The illusion that destruction and money spent in recovery from destruction, is a net-benefit to society.  A broader application of this fallacy is the general tendency to overlook opportunity costs or that which is unseen, either in a financial sense or other.

    
      This fallacy goes far beyond just looking for the silver lining, thinking positive, or making the best of a bad situation.  It is the incorrect assumption that the net benefit is positive.
    \\

    
      Logical Form:
    \\

    
      Disaster X occurred, but this is a good thing because Y will come, as a result.
    \\

    
      Example \#1:
    \\

    
      Dad, I actually did America a favor by crashing your car.  Now, some auto shop will have more work, their employees will make more money, those employees will spend their money, and who knows, they might just come to your store and buy some of your products!
    \\

    
      Explanation: I actually tried a variation of this argument when I was a kid -- it didn’t work, but not only did it not work, it is fallacious reasoning, and here is why: by crashing the car, a produced good is destroyed, and resources have to go into replacing that good as opposed to creating new goods.
    \\

    
      Example \#2:
    \\

    
      The Holocaust was a good thing overall.  It educated future generations about the evils of genocide.
    \\

    
      Explanation: This is a real argument, I kid you not.  People tend to overvalue their own gain (the education) and devalue the losses that are unseen (the unimaginable suffering of the victims and their families). 
    \\

    
      Exception: It might be the case when some kind of destruction actually can benefit society -- like in lightning striking the local crack house, and a soup kitchen being reconstructed in its place.
    \\

    
      Tip: Be sensitive when looking for the best of a bad situation, keeping in mind all those who may have suffered. In your statement of optimism or hope, be sure to show compassion as well.
    \\

    
      {\em The Holocaust was a horrible event in human history, and the damage that resulted will never be forgotten. As with most tragedies, they can be used to educate us, helping us to prevent similar future events.}
    \\

  \chapter{uncategorized
  }


Cartesian circle
    
      (Also known as: Arnauld's Circle)
    \\

  
    
      - Description: The Cartesian Circle is a term used to describe the fallacy of circular reasoning attributed to René Descartes. It highlights a logical flaw in Descartes' argument where he uses the existence of a benevolent God to validate the reliability of clear and distinct perceptions, while simultaneously using these clear and distinct perceptions to prove God's existence.
    \\

    
      
    \\

    
      - Logical Form:
    \\

    
        1. Descartes claims that clear and distinct perceptions are reliable because God, a non-deceiver, guarantees them.
    \\

    
        2. Descartes uses these reliable clear and distinct perceptions to argue for the existence of God.
    \\

    
      
    \\

    
      - Example \#1:
    \\

    
        - Scenario: Descartes argues that whatever is perceived clearly and distinctly is true because a benevolent God ensures this.
    \\

    
        - Explanation: This forms a circular argument because Descartes' proof of God's existence relies on the reliability of clear and distinct perceptions, which in turn is supposed to be guaranteed by God.
    \\

    
      
    \\

    
      - Example \#2:
    \\

    
        - Scenario: Descartes claims that his knowledge of being a thinking thing depends on the clear knowledge of an existing God.
    \\

    
        - Explanation: This argument is circular as it assumes God's existence to validate the clear and distinct perception of oneself as a thinking entity, while simultaneously using this perception to prove God's existence.
    \\

    
      
    \\

    
      - Variation: Circular reasoning can occur in various contexts where the conclusion is assumed in one of the premises. For example, using the Bible to prove God's existence while asserting that God's existence makes the Bible true.
    \\

    
      
    \\

    
      - Tip: To avoid circular reasoning, ensure that your premises are independently verifiable and do not rely on the conclusion to support them.
    \\

    
      
    \\

    
      - Exception: Some philosophers argue that foundational beliefs can be self-evident and do not require external validation, which can make circular reasoning seem less problematic in those contexts.
    \\

    
      
    \\

    
      - Fun Fact: The term "Cartesian Circle" was first popularized by Descartes' contemporaries, such as Marin Mersenne and Antoine Arnauld, who critiqued his work for this logical flaw.
    \\

    
      
    \\

    
      Additional Information
    \\

    
      - Modern Commentators: Some modern philosophers, like Bernard Williams and Harry Frankfurt, have tried to defend Descartes by suggesting different interpretations of his arguments, such as the idea that his goal was to show the reliability of reason rather than to prove conclusions through syllogistic logic.
    \\

    
      
    \\

  

argument from the contrary
    
      (also known as: Argumentum e contrario, Argumentum ex contrario, Appeal from the Contrary)
    \\

  
    
      - Description: In logic and law, an argumentum e contrario is used to argue that a proposition is correct because it is not disproven by a certain case. It contrasts with analogy, where similarity between cases is used to argue for a certain conclusion. In legal contexts, this argument helps address issues not explicitly covered by existing laws by inferring that if the law does not mention something, it is not intended to be included.
    \\

    
      
    \\

    
      - Logical Form:
    \\

    
        1. Law A states that X must do Y.
    \\

    
        2. It does not state that Z must do Y.
    \\

    
        3. Therefore, Z does not need to do Y.
    \\

    
      
    \\

    
      - Example \#1:
    \\

    
        - Scenario: § 123 of the X-Law says that green cars need to have blue tires.
    \\

    
        - Argument: Therefore, red cars don't have to have blue tires.
    \\

    
        - Explanation: The argument is based on the fact that the law specifies green cars but does not mention red cars. Thus, it is inferred that the law does not apply to red cars.
    \\

    
      
    \\

    
      - Example \#2:
    \\

    
        - Scenario: § 456 of the Y-Law says that it's irrelevant whether a message is sent by letter or by telegraph.
    \\

    
        - Argument: Therefore, messages cannot be sent by fax machines.
    \\

    
        - Explanation: This argument incorrectly infers that because the law does not mention faxes, they must be excluded. It misinterprets the law's intent, as fax machines likely did not exist when the law was written.
    \\

    
      
    \\

    
      - Variation: Arguments e contrario can vary in their application, especially in legal contexts where the interpretation of the law's intent and the specifics of unmentioned cases can lead to different conclusions. It often contrasts with arguments from analogy.
    \\

    
      
    \\

    
      - Tip: When using an argumentum e contrario, consider the law's intent and whether the unmentioned case might reasonably be included if the law were updated. Be cautious of overextending the argument to situations the law did not explicitly intend to exclude.
    \\

    
      
    \\

    
      - Exception: Argumentum e contrario is not always a fallacy. In some cases, it is a legitimate interpretative tool in law where it is clear that the law deliberately excludes certain cases. However, if the exclusion is not deliberate or clear, the argument can become fallacious.
    \\

    
      
    \\

    
      - Fun Fact: The Latin maxim "ubicumque lex voluit dixit, ubi tacuit noluit" translates to "if the legislator wished to say something, he would do that expressly." This highlights the principle behind argumentum e contrario in legal interpretation.
    \\

    
      
    \\

  

Attack ad
    
      (Also known as: Negative Campaigning, Smear Campaign)
    \\

  
    
      - Description: An attack ad is a type of advertisement used in political campaigns to launch personal attacks against an opposing candidate or party to gain support for the attacking candidate. These ads often form part of negative campaigning and are disseminated via mass media, criticizing an opponent's political platform, character, or policy ideas, often using innuendo and opposition research.
    \\

    
      
    \\

    
      - Logical Form:
    \\

    
        1. Candidate A has a specific flaw or policy issue.
    \\

    
        2. This flaw or issue is highlighted in an attack ad.
    \\

    
        3. Therefore, voters should not support Candidate A and should support Candidate B instead.
    \\

    
      
    \\

    
      - Example \#1:
    \\

    
        - Scenario: The "Daisy" advertisement from the 1964 U.S. presidential election.
    \\

    
        - Explanation: The ad featured a young girl picking daisy petals, followed by a countdown and a nuclear explosion, implying that Barry Goldwater's aggressive Cold War policies could lead to nuclear war. This ad effectively played on voters' fears and painted Goldwater as a dangerous choice.
    \\

    
      
    \\

    
      - Example \#2:
    \\

    
        - Scenario: The "Willie Horton" ad from the 1988 U.S. presidential election.
    \\

    
        - Explanation: This ad highlighted a furlough program supported by Michael Dukakis that allegedly allowed a criminal to commit further crimes. The ad suggested that Dukakis was weak on crime, swaying voters by presenting a stark and emotional narrative.
    \\

    
      
    \\

    
      - Variation: Attack ads can target a candidate's character, policies, or associates, and can range from subtle innuendo to overt accusations. They can appear on television, online platforms, or other media.
    \\

    
      
    \\

    
      - Tip: While crafting or analyzing an attack ad, consider the potential for backlash if the ad is perceived as overly harsh or unfair. Balance criticism with evidence to maintain credibility.
    \\

    
      
    \\

    
      - Exception: If an attack ad is too personal or perceived as unfair, it can backfire, damaging the reputation of the candidate who sponsored it rather than the intended target.
    \\

    
      
    \\

    
      - Fun Fact: One of the earliest and most famous attack ads, "Daisy," aired only once but had a profound impact on the 1964 U.S. presidential election, showcasing the power of televised political advertisements.
    \\

  

Greedy reductionism
    
      - Description: Greedy reductionism, identified by Daniel Dennett in his 1995 book *Darwin's Dangerous Idea*, is an erroneous form of reductionism. It occurs when scientists and philosophers, in their eagerness to explain phenomena, underestimate complexities and attempt to skip essential layers or levels of theory. This contrasts with "good" reductionism, which methodically explains phenomena in terms of their parts and interactions.
    \\

    
      
    \\

    
      - Logical Form:
    \\

    
        1. Identify a complex phenomenon.
    \\

    
        2. Attempt to explain it by reducing it to simpler components.
    \\

    
        3. Ignore or underestimate intermediate layers of complexity.
    \\

    
        4. Draw conclusions that may be overly simplistic or incorrect.
    \\

    
      
    \\

    
      - Example \#1:
    \\

    
        - Scenario: Behaviorism by B.F. Skinner
    \\

    
        - Explanation: Skinner's radical behaviorism attempted to explain all mental processes through operant conditioning alone, ignoring the potential contributions of neurological states. He claimed that one fundamental process could account for all mental activity, which oversimplified the complexities of human cognition.
    \\

    
      
    \\

    
      - Example \#2:
    \\

    
        - Scenario: Consciousness Explained by Dennett
    \\

    
        - Explanation: In *Consciousness Explained*, Dennett argued that human consciousness arises from the coordinated activity of many unconscious brain components. Critics accused him of "explaining away" consciousness by not accounting for the full complexity of conscious experience, which led Dennett to later distinguish between good and greedy reductionism.
    \\

    
      
    \\

    
      - Variation: Nonreductive physicalism opposes greedy reductionism by arguing that certain phenomena, such as consciousness, cannot be fully explained by reductionist analysis. Nonreductive physicalists claim that some characteristics of conscious systems are emergent properties that require more than just a reductionist approach.
    \\

    
      
    \\

    
      - Tip: When employing reductionism, ensure to consider all levels of complexity and avoid skipping essential layers of theory. Recognize that some phenomena may require a multi-layered explanatory approach.
    \\

    
      
    \\

    
      - Exception: In some cases, a comprehensive reductionist approach may be infeasible due to the intrinsic complexity of the phenomena. Acknowledging the limits of reductionism can help mitigate the risk of greedy reductionism.
    \\

    
      
    \\

    
      - Fun Fact: The term "nothing-buttery" emerged in the 1950s to criticize overly simplistic explanations that reduce complex phenomena to just one aspect. The phrase "nothing-but-ism" was used even earlier, in the 1930s, to describe similar reductionist thinking.
    \\

  
    
      (Also known as: Nothing-buttery, Nothing-but-ism)
    \\

  

Junkyard tornado
    
      (Also known as: Hoyle's Fallacy)
    \\

  
    
      - Description: The junkyard tornado is an argument against abiogenesis, comparing the probability of life arising by chance to the likelihood of a tornado assembling a Boeing 747 from junkyard debris. Originally posited by English astronomer Fred Hoyle, it argues that the formation of complex life is so improbable that it couldn't happen by random chance alone. This analogy is often used by those rejecting evolutionary theory, despite its flawed assumptions.
    \\

    
      
    \\

    
      - Logical Form:
    \\

    
        1. Define the complexity of life (e.g., the formation of enzymes or proteins).
    \\

    
        2. Calculate the probability of this complexity arising by random chance.
    \\

    
        3. Conclude that the probability is exceedingly low, suggesting the need for an alternative explanation.
    \\

    
      
    \\

    
      - Example \#1:
    \\

    
        - Scenario: Formation of Enzymes
    \\

    
        - Explanation: Hoyle calculated the probability of obtaining all of life's approximate 2000 enzymes in a random trial to be one in 10\^40,000, arguing that such a low probability implies that life couldn't have arisen by chance.
    \\

    
      
    \\

    
      - Example \#2:
    \\

    
        - Scenario: Cellular Biochemistry
    \\

    
        - Explanation: The argument suggests that the probability of a protein molecule achieving a functional sequence of amino acids by chance is too low to be realistic, likening it to the improbable scenario of blind men solving Rubik's Cubes simultaneously.
    \\

    
      
    \\

    
      - Variation: The argument is sometimes extended to cellular biochemistry, claiming that the formation of functional proteins by chance alone is akin to the junkyard tornado analogy.
    \\

    
      
    \\

    
      - Tip: When assessing arguments against evolution, consider the role of natural selection and gradual processes over time. Evolutionary theory explains complexity through numerous small, cumulative changes rather than one improbable event.
    \\

    
      
    \\

    
      - Exception: The junkyard tornado argument ignores natural selection, which significantly increases the probability of complex structures evolving through intermediate stages.
    \\

    
      
    \\

    
      - Fun Fact: Despite Hoyle himself being an atheist, his argument has been adopted by religious groups to support creationism and intelligent design.
    \\

  \section{Mind projection fallacy
    
      - Description: The mind projection fallacy is an informal fallacy first described by physicist and Bayesian philosopher E. T. Jaynes. It occurs when someone projects their subjective perceptions or ignorance onto the external world, mistakenly believing these mental states to be intrinsic properties of reality. This fallacy has two forms: the positive form, where one's subjective judgments are assumed to be inherent properties of objects, and the negative form, where one's lack of knowledge about a phenomenon is taken to mean the phenomenon is inherently unknowable.
    \\

    
      
    \\

    
      - Logical Form:
    \\

    
        1. Observe or imagine a characteristic or property.
    \\

    
        2. Assume this characteristic or property is a real, inherent aspect of the external world.
    \\

    
        3. Conclude that others should share the same perception, or assume they are irrational or misinformed if they do not.
    \\

    
      
    \\

    
      - Example \#1:
    \\

    
        - Scenario: Imagined Objects
    \\

    
        - Explanation: Someone sees the world through their subjective lens and assumes their view reflects reality. For example, a person who believes a particular color is ugly might assume it is inherently ugly, rather than understanding that this is a personal preference.
    \\

    
      
    \\

    
      - Example \#2:
    \\

    
        - Scenario: Knowledge Assumptions
    \\

    
        - Explanation: A person might assume that because they do not understand quantum mechanics, the phenomenon itself is inherently indeterminate or unknowable, rather than acknowledging their personal lack of knowledge.
    \\

    
      
    \\

    
      - Variation: This fallacy can be extended to any situation where subjective experiences or personal ignorance are wrongly attributed as inherent qualities of the external world.
    \\

    
      
    \\

    
      - Tip: Always differentiate between subjective perceptions and objective reality. Recognize that others may have different perspectives and that a lack of personal understanding does not reflect the nature of reality itself.
    \\

    
      
    \\

    
      - Exception: The mind projection fallacy is not applicable when subjective experiences are clearly recognized as such and not confused with objective properties.
    \\

    
      
    \\

    
      - Fun Fact: E. T. Jaynes used the mind projection fallacy to critique the Copenhagen interpretation of quantum mechanics, arguing that statistical properties often described as inherent to nature are actually reflections of our own ignorance or imagination.
    \\

  }


Psychologist's fallacy
    
      - Description: The psychologist's fallacy is an informal fallacy that occurs when an observer assumes that their own subjective experience reflects the true nature of an event or mental state. Named by William James, the fallacy involves confusing one's own perspective or reaction with the inherent characteristics of the phenomenon being studied. This error can lead to misinterpretation of the mental states or behaviors of others, based on one's own experience.
    \\

    
      
    \\

    
      - Logical Form:
    \\

    
        1. Observer experiences a mental state or reaction.
    \\

    
        2. Observer assumes this experience reflects the true nature of the event or mental state being studied.
    \\

    
        3. Observer incorrectly applies their personal perspective to others.
    \\

    
      
    \\

    
      - Example \#1:
    \\

    
        - Scenario: Personal Reaction to Stress
    \\

    
        - Explanation: A psychologist feels overwhelmed by stress and assumes that all individuals will react to similar stressful situations in the same way. This fallacy leads to the incorrect assumption that their personal experience of stress is universal.
    \\

    
      
    \\

    
      - Example \#2:
    \\

    
        - Scenario: Emotional Responses
    \\

    
        - Explanation: An experimenter who is highly emotional might assume that the participants in their study will have similar emotional responses to stimuli. This may lead to biased conclusions about how different individuals react emotionally.
    \\

    
      
    \\

    
      - Variation: The fallacy can also manifest as assuming that others will respond to stimuli or situations in the same way as the observer would, based on personal biases or stereotypes.
    \\

    
      
    \\

    
      - Tip: Be aware of the difference between personal experience and universal truths. When studying mental states or behaviors, ensure that findings are based on objective data and not solely on one's subjective experiences.
    \\

    
      
    \\

    
      - Exception: The fallacy is less likely to occur when the observer acknowledges and controls for their own biases, using rigorous methods to account for subjective differences.
    \\

    
      
    \\

    
      - Fun Fact: William James, who first identified this fallacy, was a pioneering psychologist and philosopher. His work laid foundational concepts in psychology, including the exploration of subjective experience and its impact on understanding mental phenomena.
    \\

  \subsection{Historian's fallacy
    
      (also known as: Presentism (literary and historical analysis))
    \\

  
    Description: Judging a person's decision in the light of new information not available at the time.

    
      Logical Form:
    \\

    
      Claim X was made in the past.
    \\

    
      Those who made the claim did not take into consideration Y, which was not available to them at the time.
    \\

    
      Therefore, this was a foolish claim.
    \\

    
      Example \#1:
    \\

    
      You should have never taken the back roads to the concert.  If you had taken the main roads, you would not have been stuck in all that traffic due to the accident.
    \\

    
      Explanation: “Thanks for that!” is the usual sarcastic response to this fallacy.  Of course, had we known about the accident, the main road would have been the better choice—but nobody could have reasonably predicted that accident.  It is fallacious, and somewhat pointless, to suggest that we “should have” taken the other way.
    \\

    
      Example \#2:
    \\

    
      Judas was an idiot to turn Jesus over to the authorities.  After all, he ended up committing suicide out of guilt.
    \\

    
      Explanation: It is easy for us to blame Judas as people who know the whole story and how it played out.  We have information Judas did not have at the time.  Besides, if Judas never turned in Jesus, and Jesus was never killed, but died while walking on water as an old man after tripping over a wave, would Christianity exist?
    \\

    
      Exception: Sometimes, it’s funny to commit this fallacy on purpose at the expense of your friends’ dignity.
    \\

    
      Hey, nice going on that decision to buy stock in the company that was shut down a week later by the FBI for the prostitution ring.  Do you have any stock tips for me?
    \\

    
      Tip: Practice forgiveness. We all make mistakes, and most of us learn from our mistakes and become better people. Don’t be so quick to crucify someone for something they did in the past, especially if you are doing so to virtue signal.
    \\

    References:

    
      
        
      \\

      
        
          Arp, R. (2013). {\it 1001 Ideas That Changed the Way We Think}. Simon and Schuster.
        
      
    
  }


Baconian fallacy
    
      - Description: The Baconian fallacy is the erroneous belief that historians can derive the "whole truth" about historical events through induction from individual pieces of evidence. This fallacy assumes that by accumulating and analyzing sufficient individual facts or evidence, one can achieve a complete and comprehensive understanding of history. In reality, historians can only aim to gain a partial understanding of historical events due to the limitations in evidence and interpretation.
    \\

    
      
    \\

    
      - Logical Form:
    \\

    
        1. Historians collect individual pieces of evidence about historical events.
    \\

    
        2. It is assumed that analyzing this evidence will lead to a complete and accurate understanding of the entire historical context.
    \\

    
        3. The fallacy lies in the belief that this process will reveal "the whole truth" about history, ignoring the inherent limitations and complexities of historical analysis.
    \\

    
      
    \\

    
      - Example \#1:
    \\

    
        - Scenario: Analyzing Ancient Documents
    \\

    
        - Explanation: A historian studies a collection of ancient documents and concludes that they have uncovered a complete and accurate picture of a particular historical period. This assumption overlooks gaps in the historical record and the possibility of differing interpretations.
    \\

    
      
    \\

    
      - Example \#2:
    \\

    
        - Scenario: Compiling Historical Data
    \\

    
        - Explanation: A historian gathers extensive data on economic conditions from various time periods and concludes that they fully understand the economic history of a region. This overlooks the fact that data may be incomplete and that historical understanding is often limited by factors such as bias, missing evidence, and the complexity of historical phenomena.
    \\

    
      
    \\

    
      - Variation: The fallacy may also appear in the form of believing that comprehensive knowledge can be achieved by merely aggregating all available evidence without considering the limitations and potential biases in the evidence.
    \\

    
      
    \\

    
      - Tip: Acknowledge the limitations of historical evidence and the interpretative nature of historical analysis. Recognize that while evidence can provide valuable insights, it cannot always deliver a complete or definitive account of historical events.
    \\

    
      
    \\

    
      - Exception: The fallacy is less relevant in cases where historians are explicitly aware of and account for the limitations of their evidence and methods, focusing on well-supported interpretations rather than claiming to uncover the absolute truth.
    \\

    
      
    \\

    
      - Fun Fact: The term "Baconian fallacy" is named after Sir Francis Bacon, who advocated for the use of induction in scientific inquiry. However, Bacon's approach is sometimes misapplied in historical analysis, leading to the fallacy of assuming that induction can provide a complete historical truth.
    \\

  

Dixiecrat fallacy
    
      - Description: The Dixiecrat fallacy is an informal fallacy used to frame political arguments in a way that casts modern Republicans in a positive light while portraying Democrats negatively. This fallacy exploits the historical shifts in party positions on civil rights issues to create a misleading narrative that suggests a fundamental and enduring difference in party ideologies.
    \\

    
      
    \\

    
      - Logical Form:
    \\

    
        1. Present a historical civil rights issue or topic, such as the Fourteenth Amendment or the actions of Martin Luther King, Jr.
    \\

    
        2. Highlight how the topic was historically supported by Republicans and opposed by Democrats.
    \\

    
        3. Imply or suggest that this historical alignment reflects the current ideological positions of the parties.
    \\

    
        4. Draw a conclusion that supports the idea that the Republican Party is the party of civil rights and the Democratic Party is not, ignoring the historical shifts in party ideologies and policies.
    \\

    
      
    \\

    
      - Example \#1:
    \\

    
        - Scenario: Discussion of the Civil Rights Act of 1964
    \\

    
        - Explanation: A modern Republican might argue that the Republican Party has always been the party of civil rights because Republicans were instrumental in passing the Civil Rights Act of 1964. This overlooks the fact that the political alignments and party platforms have shifted over time, and the Democratic Party, once the party of segregation, has since become a strong advocate for civil rights.
    \\

    
      
    \\

    
      - Example \#2:
    \\

    
        - Scenario: Mention of the Dixiecrats
    \\

    
        - Explanation: The fallacy may involve referencing the Dixiecrats, a splinter group of Southern Democrats who opposed civil rights legislation in the 1940s and 1950s, and suggesting that this opposition reflects the views of the Democratic Party as a whole. This ignores the fact that many of these Dixiecrats eventually became Republicans and that party ideologies have evolved significantly since that time.
    \\

    
      
    \\

    
      - Variation: The fallacy can also manifest in discussions about historical figures or events that are used selectively to support current partisan arguments, without acknowledging the changes in party platforms or ideologies.
    \\

    
      
    \\

    
      - Tip: When evaluating historical claims about political parties, consider the historical context and acknowledge that party platforms and ideologies can change over time. Assess claims critically and avoid drawing overly simplistic or misleading conclusions based on selective historical evidence.
    \\

    
      
    \\

    
      - Exception: The fallacy may be less relevant in discussions where both historical and current party positions are accurately represented and analyzed within the context of evolving political landscapes.
    \\

    
      
    \\

    
      - Fun Fact: The Dixiecrat fallacy is named after the Dixiecrats, a faction of the Democratic Party that opposed civil rights legislation in the mid-20th century. This term highlights the misleading use of historical facts to serve contemporary political arguments.
    \\

  

Historical fallacy
    
      - Description: The historical fallacy is a logical error described by philosopher John Dewey. It occurs when someone assumes that the process used to achieve a result was necessary for that result, leading to the belief that the outcome would not have occurred without the specific process. This fallacy involves misattributing the outcome to the process that brought it about, rather than recognizing that the outcome could have happened through other means or by chance.
    \\

    
      
    \\

    
      - Logical Form:
    \\

    
        1. An outcome is achieved through a specific process.
    \\

    
        2. The outcome is then assumed to be dependent on the exact process used.
    \\

    
        3. The process is incorrectly considered essential to the result, ignoring the possibility of other methods achieving the same outcome.
    \\

    
      
    \\

    
      - Example \#1:
    \\

    
        - Scenario: Finding a Lost Wallet
    \\

    
        - Explanation: A man loses his wallet and searches for it in a location he suspects it might be. He finds the wallet there and concludes that his initial suspicion about the location was correct and essential for finding it. He fails to consider that the wallet might have been found through other means or by chance.
    \\

    
      
    \\

    
      - Example \#2:
    \\

    
        - Scenario: A Successful Experiment
    \\

    
        - Explanation: A scientist conducts an experiment using a particular method and achieves a successful result. The scientist then assumes that the success was solely due to the chosen method, disregarding the possibility that other methods might have also led to the same result.
    \\

    
      
    \\

    
      - Variation: The fallacy can also manifest as the assumption that the historical context or specific conditions leading to an outcome were crucial, rather than recognizing the possibility of alternative scenarios or methods.
    \\

    
      
    \\

    
      - Tip: When evaluating outcomes, consider multiple potential causes or methods, and avoid attributing success solely to the specific process used. Recognize that outcomes may be achieved through various means.
    \\

    
      
    \\

    
      - Exception: The historical fallacy may not apply if the process used is unique and cannot be replicated or if it is demonstrated that no other process would have led to the same result.
    \\

    
      
    \\

    
      - Fun Fact: John Dewey, who described this fallacy, was a prominent American philosopher and educator known for his work on pragmatism and progressive education. His insights into logical errors like the historical fallacy continue to influence critical thinking and analysis.
    \\

  

Fantasy Projection
    Description: Confusing subjective experiences, usually very emotionally charged, with objective reality, then suggesting or demanding that others accept the subjective experience as objective reality.

    
      New Terminology: In this context, {\em subjective experience}  is the way one interprets some external stimuli. {\em Objective reality}  is independent of our interpretations; it is a collection of facts about the world we all share.
    \\

    
      Logical Form:
    \\

    
      {\em Person 1 has subjective experience X.} \newline
{\em Person 1 incorrectly believes that experience X represents objective reality.} \newline
{\em Therefore, person 1 insists that others accept that X represents objective reality.}
    \\

    
      Example \#1:
    \\

    
      {\em Freddie: People are mean to me wherever I go. It is clear that we live in a cruel world with people who are mostly nasty. If you don’t see that, something is wrong with you!}
    \\

    
      Explanation: Perhaps people are mean to Freddie because Freddie is mean to others, and it’s Freddie’s behavior that is resulting in the “mean” behaviors of others (this is known as a {\em self-fulfilling prophecy}). Freddie is projecting his experience, which is unique to him, onto the world at large. He is insisting that other people see humanity the way he does. We don’t deny that Freddie is experiencing the world in the way he is; we just don’t accept that Freddie’s experience represents objective reality. To accurately determine if the world is, indeed, made up of “people who are mostly nasty,” we would need to conduct research using the scientific method.
    \\

    
      Example \#2:
    \\

    
      {\em I feel that we are all surrounded by Narggles. These are spiritual beings who help us through life. We know they exist because they are the ones that give us the confidence to move forward in a decision.}
    \\

    
      Explanation: Ignoring the circular reasoning (how we "know" Narggles exist), one person's fantasy might be their own reality, but not everyone else's.
    \\

    
      Exception: One can argue that one's subjective experience is part of objective reality as long as they don’t insist that you interpret the stimuli the same way they did.
    \\

    
      {\em Freddie: It is a fact that this world consists of people who feel like most people they interact with are mean to them.}
    \\

    
      Fun Fact: Everyone knows Kerplunkers, not Narggles, are spiritual beings who help us through life.
    \\

  

Misology

Missing dollar riddle
    
      (Also known as: Missing Dollar Paradox)
    \\

  
    
      - Description: The missing dollar riddle is a famous puzzle that involves a logical fallacy by presenting a scenario where a dollar seems to be missing after a series of transactions. It plays on the confusion between different categories of money and how they are accounted for.
    \\

    
      
    \\

    
      - Logical Form:
    \\

    
        1. Three guests check into a hotel and pay a total of \$30.
    \\

    
        2. The manager realizes the bill should have been \$25 and gives \$5 to the bellhop to return to the guests.
    \\

    
        3. The bellhop decides to keep \$2 as a tip and gives \$1 back to each guest.
    \\

    
        4. The guests end up paying \$27 (\$9 each), and the bellhop keeps \$2, which totals \$29.
    \\

    
        5. The riddle asks where the missing \$1 is, suggesting a discrepancy in the total amount.
    \\

    
      
    \\

    
      - Example \#1:
    \\

    
        - Scenario: Three guests initially pay \$30, but the corrected bill is \$25. The bellhop returns \$3 and keeps \$2 as a tip.
    \\

    
        - Explanation: The confusion arises from adding \$27 (what the guests effectively paid) to \$2 (kept by the bellhop), which wrongly implies it should total \$30. In reality, the \$27 includes the \$2 tip; the correct breakdown is \$25 (hotel) + \$3 (returned to guests) + \$2 (bellhop tip) = \$30.
    \\

    
      
    \\

    
      - Example \#2:
    \\

    
        - Scenario: Consider a modified version where the bill is \$10. Guests pay \$30, the manager gives \$20 to the bellhop to return. The bellhop keeps \$2 and gives \$6 back to each guest.
    \\

    
        - Explanation: Here, the guests effectively paid \$12 (\$4 each), and the bellhop kept \$2. Adding \$12 (paid) and \$2 (kept by bellhop) incorrectly leads to \$14, whereas the correct sum should be \$10 (hotel) + \$6 (returned to guests) + \$2 (bellhop tip) = \$18. This demonstrates that the added amounts are not directly comparable.
    \\

    
      
    \\

    
      - Variation: Variants of the riddle may involve different amounts or scenarios, such as using shillings or different numbers of people, but the underlying fallacy remains the same: incorrectly summing amounts that are not directly related.
    \\

    
      
    \\

    
      - Tip: To avoid confusion, always account for all categories of money separately. Ensure that any sums of money include all assets and liabilities in the correct context.
    \\

    
      
    \\

    
      - Exception: The riddle does not apply if all transactions and amounts are correctly accounted for and reconciled. It specifically targets misunderstandings in the way the total amounts are presented.
    \\

    
      
    \\

    
      - Fun Fact: The riddle has appeared in various forms in literature and media, including a 1933 mathematical fallacy by Cecil B. Read and adaptations in pop culture, such as in the BBC comedy series "Help" and in Abbott and Costello routines.
    \\

  

Sunday school answer
    
      (Also known as: Trite Answer, Simplistic Answer)
    \\

  
    
      - Description: The term "Sunday school answer" is a pejorative expression used primarily in Evangelical Christianity to describe a simplistic or overly obvious response to a complex question. It refers to answers that are commonly expected in a Sunday school setting, regardless of the specific question being asked. Such answers often include concepts like Jesus, sin, and the cross.
    \\

    
      
    \\

    
      - Logical Form:
    \\

    
        1. A complex or specific question is posed.
    \\

    
        2. An overly simplistic or generic answer is given (often a religious reference).
    \\

    
        3. The answer fails to adequately address the question, leading to criticism.
    \\

    
      
    \\

    
      - Example \#1:
    \\

    
        - Scenario: A Sunday school teacher asks, "What is brown and furry and collects nuts for the winter?"
    \\

    
        - Response: A student answers, "It sounds like a squirrel, but is it Jesus?"
    \\

    
        - Explanation: The answer is inappropriate because it does not address the question and relies on a common religious reference rather than a thoughtful response.
    \\

    
      
    \\

    
      - Example \#2:
    \\

    
        - Scenario: During a discussion about personal struggles, someone asks, "What should we do when we face difficult times?"
    \\

    
        - Response: Someone replies, "We should just pray and trust in God."
    \\

    
        - Explanation: This response is seen as a "Sunday school answer" because it is a simplistic solution to a complex issue, lacking depth and consideration for the nuances of the situation.
    \\

    
      
    \\

    
      - Variation:
    \\

    
        - In Mormonism: The term is similarly used within the culture of the Church of Jesus Christ of Latter-day Saints (LDS Church) to refer to trite answers that are commonly given in Sunday School classes. Examples include "reading the scriptures," "praying daily," and "serving others." Such responses can be seen as inadequate if they do not genuinely engage with the challenges being discussed.
    \\

    
      
    \\

    
      - Tip: To avoid providing a "Sunday school answer," consider the context of the question and provide a more nuanced, thoughtful response that addresses the complexities involved rather than resorting to cliché or overly simplistic answers.
    \\

    
      
    \\

    
      - Exception: Not all answers that might be labeled as "Sunday school answers" are invalid. In some cases, simple truths can be effective and appropriate, especially when they resonate with the audience or offer genuine encouragement.
    \\

    
      
    \\

    
      - Fun Fact: Despite its pejorative connotation, some argue that answers dismissed as "Sunday school answers" are often the most fundamental and truthful solutions to life’s challenges, highlighting the tension between simplicity and depth in addressing complex issues.
    \\

  

False-Uniqueness Effect
    
      (Also known as: Uniqueness bias, Illusion of Uniqueness)
    \\

  
    
      - Description: The false-uniqueness effect is a cognitive bias where individuals perceive their qualities, traits, and behaviors as more unique and exceptional compared to others, even though these attributes are actually quite common. This bias often arises in the context of positive traits or behaviors.
    \\

    
      
    \\

    
      - Logical Form:
    \\

    
        1. Individuals rate their own traits or behaviors as unique or better than average.
    \\

    
        2. They assume that fewer people share these traits or behaviors than is actually the case.
    \\

    
        3. This overestimation of uniqueness helps maintain or enhance self-esteem.
    \\

    
      
    \\

    
      - Example \#1:
    \\

    
        - Scenario: A person believes they are an exceptionally safe driver compared to others.
    \\

    
        - Explanation: Despite statistics showing that most drivers consider themselves safe, this individual thinks their driving habits are more cautious and unique. This belief may enhance their self-esteem, making them feel superior.
    \\

    
      
    \\

    
      - Example \#2:
    \\

    
        - Scenario: A student thinks they are more diligent in group projects than their peers.
    \\

    
        - Explanation: The student believes their work ethic is exceptional compared to the average student, even though many others might exhibit similar levels of diligence. This perception helps the student feel better about their contributions.
    \\

    
      
    \\

    
      - Variation: The false-uniqueness effect can be contrasted with the false-consensus effect, where individuals overestimate the extent to which their attitudes and behaviors are typical and shared by others. The false-consensus effect typically applies to negative traits or behaviors, while the false-uniqueness effect applies to positive traits.
    \\

    
      
    \\

    
      - Tip: Be aware of this bias when assessing your own attributes and behaviors. Try to consider a wider perspective and use objective benchmarks to evaluate how common your traits truly are.
    \\

    
      
    \\

    
      - Exception: The false-uniqueness effect may be less pronounced when individuals are aware of broad statistics and actual prevalence rates of traits or behaviors. Accurate self-assessment often requires considering empirical data rather than personal perception alone.
    \\

    
      
    \\

    
      - Fun Fact: The term “false-uniqueness effect” was formally introduced by psychologists Suls and Wan in 1987, building on earlier concepts like the “illusion of uniqueness” described by Snyder and Shneckel in 1975.
    \\

  

Virtuality fallacy
    
      (Also known as: Digital Denial Fallacy)
    \\

  
    
      - Description: The virtuality fallacy is an informal fallacy that occurs when it is asserted that things existing in a virtual context, such as cyberspace, are not real or do not have real effects. This fallacy involves dismissing the impact or reality of virtual phenomena based on their digital nature.
    \\

    
      
    \\

    
      - Logical Form:
    \\

    
        1. Premise 1: X exists in cyberspace.
    \\

    
        2. Premise 2: Cyberspace is virtual.
    \\

    
        3. Conclusion: X (or the effect of X) is not real.
    \\

    
      
    \\

    
      - Example \#1:
    \\

    
        - Scenario: An individual posts harmful or offensive comments online and dismisses concerns by claiming, "It’s just the internet; it doesn’t really matter."
    \\

    
        - Explanation: This example illustrates the virtuality fallacy by assuming that because the comments are made in a virtual space, their impact or harm is not real.
    \\

    
      
    \\

    
      - Example \#2:
    \\

    
        - Scenario: A person cheats in a video game and justifies their actions by saying, "It’s just a game; it doesn’t have real consequences."
    \\

    
        - Explanation: This example reflects the fallacy by downplaying the effects of cheating, such as affecting other players' experiences or the game's integrity, by arguing that the context is virtual and therefore inconsequential.
    \\

    
      
    \\

    
      - Variation: The virtuality fallacy is related to the broader category of fallacies where digital or virtual actions are perceived as having no real-world implications, similar to the "it’s just a joke" fallacy where the impact of an action is minimized based on its context.
    \\

    
      
    \\

    
      - Tip: Recognize that virtual actions and contexts can have tangible effects on individuals and systems. Just because something occurs in a digital space does not mean it lacks real-world significance.
    \\

    
      
    \\

    
      - Exception: The fallacy does not apply in cases where virtual contexts are explicitly understood to have no real-world impact, such as certain fictional scenarios or harmless digital simulations with no real consequences.
    \\

    
      
    \\

    
      - Fun Fact: The concept of the virtuality fallacy is increasingly relevant in discussions about the impact of online behavior, particularly as virtual environments and digital interactions become more integral to daily life and societal norms.
    \\

  \section{Demonizing the enemy
    
      (Also known as: Demonization of the Enemy, Dehumanization of the Enemy)
    \\

  
    
      \#\#\# Demonizing the Enemy
    \\

    
      
    \\

    
      - Name: Demonizing the Enemy
    \\

    
      
    \\

    
      - Also known as: Demonization of the Enemy, Dehumanization of the Enemy
    \\

    
      
    \\

    
      - Description: Demonizing the enemy is a propaganda technique used to portray an opponent as a malevolent and threatening force with only destructive intentions. This tactic aims to generate hatred and fear towards the enemy, making them seem more dangerous and justifying aggressive actions against them, while also consolidating support among allies and demoralizing the enemy.
    \\

    
      
    \\

    
      - Logical Form:
    \\

    
        1. Premise 1: The enemy is depicted as purely evil and destructive.
    \\

    
        2. Premise 2: The portrayal is supported by media and state narratives.
    \\

    
        3. Conclusion: The enemy's actions and character are perceived as entirely negative, justifying hostility and aggression.
    \\

    
      
    \\

    
      - Example \#1:
    \\

    
        - Scenario: During World War I, Kaiser Wilhelm II was portrayed in Russian media as a villainous figure responsible for the war's devastation.
    \\

    
        - Explanation: By personalizing the enemy as an evil leader, this portrayal simplified complex geopolitical conflicts into a binary of good versus evil, mobilizing public sentiment against the enemy and justifying continued conflict.
    \\

    
      
    \\

    
      - Example \#2:
    \\

    
        - Scenario: Propaganda during World War II depicted the Axis powers as barbaric and inhumane, often showing enemy leaders as ruthless and villainous in state-approved media.
    \\

    
        - Explanation: This characterization aimed to reinforce the perception of the Axis powers as an existential threat, thus encouraging national unity and resolve against a perceived evil.
    \\

    
      
    \\

    
      - Variation: Demonization can vary from portraying a single leader as evil (e.g., Kaiser Wilhelm II) to dehumanizing entire populations or political groups (e.g., portraying enemy nations as barbaric). It can also manifest in more subtle forms, such as media portrayals that emphasize the enemy's most negative traits while ignoring complexities.
    \\

    
      
    \\

    
      - Tip: Be critical of sources that consistently frame an opponent or group in exclusively negative terms. Consider the motivations behind such portrayals and seek out balanced perspectives to avoid falling into the trap of demonization.
    \\

    
      
    \\

    
      - Exception: While demonization is a common propaganda technique, it is not always effective or applicable in every conflict. In some cases, efforts to humanize or understand the enemy can lead to more constructive dialogue and resolution.
    \\

    
      
    \\

    
      - Fun Fact: The concept of demonizing an enemy is not limited to modern media or politics. Historical records show that even ancient civilizations used similar techniques to mobilize support and justify conflicts, as noted by Thucydides in his accounts of Ancient Greece.
    \\

  }


False accusions
    
      (Also known as: Groundless Accusation, Unfounded Accusation, False Allegation, False Claim)
    \\

  
    
      - Description: A false accusation is a claim or allegation of wrongdoing that is not true or is unsupported by facts. Such accusations can occur in various contexts, including everyday life, quasi-judicial settings, and formal judicial proceedings. They can result from intentional deceit, misunderstandings, or errors in communication or investigation.
    \\

    
      
    \\

    
      - Logical Form:
    \\

    
        1. Premise 1: X is accused of wrongdoing.
    \\

    
        2. Premise 2: There is no factual evidence supporting the wrongdoing or the accusation.
    \\

    
        3. Conclusion: The accusation against X is false or unfounded.
    \\

    
      
    \\

    
      - Example \#1:
    \\

    
        - Scenario: A person falsely accuses their colleague of theft to avoid reprimand for their own mistakes.
    \\

    
        - Explanation: The accuser's false claim is designed to deflect blame and avoid personal consequences. The absence of evidence and the fact that the alleged theft did not occur reveal the accusation as false.
    \\

    
      
    \\

    
      - Example \#2:
    \\

    
        - Scenario: During a workplace dispute, an employee falsely accuses another of harassment to gain an advantage in a promotion contest.
    \\

    
        - Explanation: The accusation is baseless and intended to damage the accused's reputation. It is a strategic move to undermine the accused's standing, rather than a genuine claim of misconduct.
    \\

    
      
    \\

    
      - Variation: False accusations can vary in their nature, including completely false allegations where the event never occurred, false allegations where the event did occur but not involving the accused, and mixed allegations that combine true and false elements. They can also arise from deliberate deceit or unintentional errors.
    \\

    
      
    \\

    
      - Tip: When faced with accusations, verify the evidence and consider the context. Look for corroborating details and be cautious of accusations that seem to serve the accuser's interests or coincide with their personal gain.
    \\

    
      
    \\

    
      - Exception: In cases where false accusations are the result of mental illness or suggestive questioning, the accuser may not be intentionally lying but may be influenced by psychological factors or improper investigative techniques.
    \\

    
      
    \\

    
      - Fun Fact: The term "false accusation" has been the subject of numerous high-profile legal cases and media stories. For instance, the wrongful conviction of Juan Catalan in the mid-2000s, later exonerated by evidence from a television show, highlights how false accusations can dramatically impact lives and lead to significant legal and social consequences.
    \\

  

Alphabet Soup
    Description: The deliberate and excessive use of acronyms and abbreviations to appear more knowledgeable in the subject or confuse others.

    
      Logical Form:
    \\

    
      Person 1 uses acronyms and abbreviations.
    \\

    
      Therefore, person 1 knows what he or she is talking about.
    \\

    
      Example \#1:
    \\

    
      In programming CGI, a WYSIWYG interface doesn't handle PHP or CSS very well. If you sign up for my personal consulting, I will show you how to program effectively.
    \\

    
      Explanation: Simply overusing acronyms is not the problem here; it's the deliberate overuse for the purpose of making people think the speaker is very knowledgeable in this area, or perhaps to use terms the audience is unaware of, making the audience think they need the consulting service more than they thought they did.
    \\

    
      Example \#2:
    \\

    
      Am I good at public speaking? Let's see. I have a CC, AC-B, AC-S, AC-G, CL, AL-B, AL-S, and a DTM. What do you think?
    \\

    
      Explanation: These are all designations from Toastmasters International. In fact, many of them have more to do with leadership than speaking, but the average audience member would never know that. Besides, getting all these awards just means the person did the work needed, not that they are necessarily good at public speaking—kind of like certificates given out in fifth-grade gym class to all the kids who do more than six sit-ups.
    \\

    
      Exception: "Excessive" is subjective. Acronyms and abbreviations are perfectly acceptable in many situations.
    \\

    
      Tip: Don't be so quick to assume nefarious intentions. Sometimes people simply are unaware that they are overusing this type of language.
    \\

  

Alternative Advance
    
      (also known as: lose-lose situation)
    \\

  
    Description: When one is presented with just two choices, both of which are essentially the same, just worded differently.  This technique is often used in sales.  Fallacious reasoning would be committed by the person accepting the options as the only options, which would most likely be on a subconscious level since virtually anyone—if they thought about it—would recognize other options exist.

    
      Example \#1:
    \\

    
      Max: If you’re not a witch, you have nothing to fear.  If you’re not a witch, you are not made of wood; therefore, you will sink and drown after we tie you up and throw you in the well.  If you do float, then you are made of wood, you are a witch, and we will hang you.
    \\

    
      Glinda: Wait, how is it I have nothing to worry about if I am not a witch?
    \\

    
      Explanation: The argument is created so that any woman accused of being a witch will die, which is certainly a lose-lose situation.
    \\

    
      Example \#2:
    \\

    
      Guy working a booth in the mall: Excuse me, but you look like you can use a vacation!  Do you have a few minutes to chat about vacation destinations, or would you prefer I just send you some information by e-mail?
    \\

    
      Explanation: Of course, other options include just ignoring the guy and keep walking; telling the guy, “no thank you,” and keep walking; or respond, “I have some time to chat. My rate is \$10 per minute. Do you prefer to pay me by cash or check?”
    \\

    
      Exception: If you engage your critical thinking and realize other options exist and still choose one of the given options, you would not be guilty of fallacious reasoning.
    \\

    
      Tip: Whenever you are presented with options, carefully consider the possibility of other options not mentioned, and propose them.
    \\

  

Commutation of Conditionals
    
      (also known as: the fallacy of the consequent, converting a conditional)
    \\

  
    Description: Switching the antecedent and the consequent in a logical argument.

    
      Logical Form:
    \\

    
      If P then Q.
    \\

    
      Therefore, if Q then P.
    \\

    
      Example \#1:
    \\

    
      If I have a PhD, then I am smart.
    \\

    
      Therefore, if I am smart, then I have a PhD.
    \\

    
      Explanation: There are many who could, rightly so, disagree with the first premise, but assuming that premise is true, does not guarantee that the conclusion is true.  There are many smart people without PhDs.
    \\

    
      Example \#2:
    \\

    
      If I have herpes, then I have a strange rash.
    \\

    
      Therefore, if I have a strange rash, then I have herpes.
    \\

    
      Explanation: I am glad this is not true.  One can have non-herpes rashes.
    \\

    
      Exception: If p=q, then it is necessarily true that q=p.
    \\

    
      Tip: If you think might herpes, see your doctor.
    \\

    References:

    
      
        
      \\

      
        
          Pickard, W. A., \& Aristotle. (2006). {\it On Sophistical Refutations}. ReadHowYouWant.com, Limited.
        
      
    
  

Confusing an Explanation with an Excuse
    
      (also known as: confusing and explanation with justification, confusing elucidation with justification)
    \\

  
    Logical Form:

    
      {\em Person 1 wants claim X be justified. \newline
Person 2 explains claim X in detail. \newline
Therefore, claim X is justified / true.}
    \\

    
      Example \#1:
    \\

    
      {\em Barto: If masks don’t work, how do you explain the almost perfect correlation between mask-wearing communities and lower transmission rates? \newline
Tikki: All this means is that in communities where more people where masks, the virus is less-likely to spread. It is not proof that masks are the reason.}
    \\

    
      Explanation: Not only did Tikki not answer the question asked, she created an answer based on elucidation of what Barto had said. Tikki explained what a correlation is (i.e., not “proof”) but came no closer to explaining the reason for the correlation.
    
    
      
    
    
      Example \#2: \newline
 \newline

    
    
      {\em Virgil: How do you justify the claim that Bigfoot is the missing link between the great apes and humans? \newline
Marshall: Well, a "missing link" is the intermediary species between the two in the evolutionary process. \newline
 \newline
}
    
    
      Explanation: Marshall simply explained what a missing link is; he did not give a valid reason for why he believes that Bigfoot is the missing link.
    \\

    
      Exception: If it is clear to both parties that no justification attempt is being made, but rather just stating a fact, then this fallacy is not being committed.
    \\

    
      Tip: If you are unsure if someone is trying to make an excuse or simply stating a fact, ask them.  Don’t assume.
    \\

  

Confusing Currently Unexplained with Unexplainable
    Description: Making the assumption that what cannot currently be explained is, therefore, unexplainable (impossible to explain). This is a problem because we cannot know the future and what conditions might arise that offer an explanation. It is also important to note that we cannot assume the currently unexplained is explainable.

    
      Logical Form:
    \\

    
      Claim X is currently unexplained.
    \\

    
      Therefore, claim X is unexplainable.
    \\

    
      Example \#1:
    \\

    
      Teri: I don't know why that stuffed animal flew off my dresser this morning. I guess some things in life will forever remain a mystery!
    \\

    
      Explanation: The fact that Teri could not explain something, does not make it unexplainable. Later, it might be revealed that a family member was playing a trick and tied a string to the stuffed animal. Maybe Teri will read about a slight earthquake that happened at the same time. Maybe Teri will later discover that there is a rat in her room that made a home in her stuffed animals. The point is, what is currently unexplained is not necessarily unexplainable.
    \\

    
      Example \#2:
    \\

    
      While we may be able to explain how humans got here, we will never be able to explain why.
    \\

    
      Explanation: Besides {\it begging the question} (this assumes there is a "why"), we don't know if we will never be able to explain why. Perhaps aliens created us, and they will tell us one day that created us for one big social experiment. Perhaps they wanted to see, after seeding Earth with life about 3 1/2 billion years ago, if one of the members of the human species, Alfredo, scratched his left ear at precisely 3:46 PM on December 12, 2023. After which time they will hit the reset button and try a different experiment.
    \\

    
      Exception: Like other claims, expressing probability is a way around this fallacy.
    \\

    
      While we may be able to explain how humans got here, we may  never be able to explain why (assuming there is a why).
    \\

    
      Tip: Be careful using terms such as "impossible" and "possible." The casual use of both terms is often incorrect within argumentation.
    \\

    References:

    
      This a logical fallacy frequently used on the Internet. No academic sources could be found.
    
  

Conspiracy Theory
    
      (also known as: canceling hypothesis, canceling hypotheses, coverups)
    \\

  
    Description: Explaining that your claim cannot be proven or verified because the truth is being hidden and/or evidence destroyed by a group of two or more people.  When that reason is challenged as not being true or accurate, the challenge is often presented as just another attempt to cover up the truth and presented as further evidence that the original claim is true.

    
      Logical Form:
    \\

    
      A is true.
    \\

    
      B is why the truth cannot be proven.
    \\

    
      Therefore, A is true.
    \\

    
      Example \#1:
    \\

    
      Noah’s ark has been found by the Russian government a long time ago, but because of their hate for religion, they have been covering it up ever since.
    \\

    
      Example \#2:
    \\

    
      Geologists and scientists all over the world are discovering strong evidence for a 6000-year-old earth, yet because of the threat of ruining their reputation, they are suppressing the evidence and keeping quiet.
    \\

    
      Explanation: The psychology behind conspiracy theories is quite complex and involves many different cognitive biases and fallacies discussed in this book.  In general, people tend to overlook the incredible improbabilities involved in a large-scale conspiracy, as well as the potential risks for all involved in the alleged cover-up.  In the above examples, those who stick with a literal interpretation of the Bible often experience {\it cognitive dissonance}, or the mental struggle involved when one’s beliefs contradict factual claims.  This cognitive dissonance causes people to create conspiracy theories, like the ones above, to change facts to match their beliefs, rather than changing their beliefs to match facts.
    \\

    
      Exception: Sometimes, there really are conspiracies and cover-ups.  The more evidence one can present for a cover-up, the better, but we must remember that possibility does not equal probability.
    \\

    
      Tip: Take time to question any conspiracy theories in which you believe are true.  Do the research with an open mind.
    \\

    References:

    
      
        
      \\

      
        
          Barkun, M. (2006). {\it A Culture of Conspiracy: Apocalyptic Visions in Contemporary America}. University of California Press.
        
      
    
  

Distinction Without a Difference
    
      Description: The assertion that a position is different from another position based on the language when, in fact, both positions are the same -- at least in practice or practical terms.
    \\

    
      
    \\

    
      Logical Form:
    \\

    
      
    \\

    
      Claim X is made where the truth of the claim requires a distinct difference between A and B.
    \\

    
      There is no distinct difference between A and B.
    \\

    
      Therefore, claim X is true.
    \\

    
      
    \\

    
      Example \#1:
    \\

    
      
    \\

    
      Sergio: There is no way I would ever even consider taking dancing lessons.
    \\

    
      Kitty: How about I ask my friend from work to teach you?
    \\

    
      Sergio: If you know someone who is willing to teach me how to dance, then I am willing to learn, sure.
    \\

    
      
    \\

    
      Explanation: Perhaps it is the stigma of “dancing lessons” that is causing Sergio to hold this view, but the fact is, someone teaching him how to dance is the same thing.  Sergio has been duped by language.
    \\

    
      
    \\

    
      Example \#2:
    \\

    
      
    \\

    
      We must judge this issue by what the Bible says, not by what we think it says or by what some scholar or theologian thinks it says.
    \\

    
      
    \\

    
      Explanation: Before you say, “Amen!”, realize that this is a clear case of distinction without a difference.  There is absolutely no difference here because the only possible way to read the Bible is through interpretation, in other words, what we think it says.  What is being implied here is that one's own interpretation (what he or she thinks the Bible says) is what it really says, and everyone else who has a different interpretation is not really reading the Bible for what it says.
    \\

    
      
    \\

    
      Exception: It is possible that some difference can be very minute, exist in principle only, or made for emphasis, in which case the fallacy could be debatable.
    \\

    
      
    \\

    
      Coach:  I don’t want you to try to get the ball; I want you to GET the ball!
    \\

    
      
    \\

    
      In practical usage, this means the same thing, but the effect could be motivating, especially in a non-argumentative context.
    \\

    
      
    \\

    
      Tip: Replace the phrase, “I’ll try” in your vocabulary with, “I’ll do my best”.  While the same idea in practice, perceptually it means so much more.
    \\

    
      
    \\

  

Fact-to-Fiction Fallacy
    Description: Attempting to support a narrative or argument with facts that don't support the narrative or argument. The distinguishing characteristic of the fact-to-fiction fallacy is the accusation that those who reject your conclusion are rejecting the facts.

    
      Logical Form:
    \\

    
      Facts are stated and made clear they are facts.
    \\

    
      Therefore, some conclusion is true that is not supported by the facts.
    \\

    
      If you reject the conclusion, you are rejecting the facts.
    \\

    
      Example \#1:
    \\

    
      FACT: On average, 150,000 people die every day around the world. It is crazy to panic about a virus that, at its peak, was killing 8,000 people per day. But what else can be expected from a moron like you who rejects facts?
    \\

    
      Explanation: It is a fact that, on average, 150,000 people die every day around the world. It is also a fact that at its peak in the spring of 2020, the Coronavirus was killing about 8,000 people per day. What hasn't been established is what justifies "panic." "Panic," or perhaps more accurately "serious concern," can be justified in many other ways besides the total number of historical deaths. Given this, it doesn't follow that it is "crazy to panic." Further, saying or writing "FACT," followed by an accusation of rejecting the fact, makes this more fallacious than a standard non-sequitur.
    \\

    
      Example \#2: In May of 2020, it was common to hear accusations of "anti-science" hurled at people who wanted the country to reopen after being closed due to the Covid-19 pandemic.
    \\

    
      Explanation: "Anti-science" is similar to "anti-fact." There are many facts that point to ways the virus could spread, and reopening the country would unquestionably increase the odds that the virus would spread. Science doesn't make value judgments, however. For example, would an extra X number of deaths per week justify people getting back to work? Science {\em informs} political decisions such as these; it doesn't answer them. It can be perfectly consistent to agree with the science (facts) and still reject the argument that we are reopening the country "too soon."
    \\

    
      Exception: If there is no accusation of rejecting facts, we might have a non-sequitur, but not a fact-to-fiction fallacy. It is not unreasonable to state a fact such as the 150,000 deaths per day stat, then ask why we are taking the actions we are for just 8000 deaths? These kinds of questions are likely to lead to answers that expose the nuance of the arguments.
    \\

    
      Tip: Be careful not to read too much into facts. Our minds tend to “connect the dots” in order to support the narratives we already accept.
    \\

  

False Effect
    Description: Claiming that the cause is true or false based on what we know about the effect in a claim of causality that has not been properly established. The cause is often an implied claim, and it is this claim that is being deemed true or false, right or wrong.

    
      Logical Forms:
    \\

    
      X apparently causes Y.
    \\

    
      Y is wrong.
    \\

    
      Therefore, X is wrong.
    \\

    
       
    \\

    
      X apparently causes Y.
    \\

    
      Y is right.
    \\

    
      Therefore, X is right.
    \\

    
      Example \#1:
    \\

    
      Mom: Watching TV that close will make you go blind, so move back! \newline
Jonny: That is B.S., Mom. Sorry, I am not moving.
    \\

    
      Explanation: The {\em false effect} of watching TV too closely is going blind. For the most part, the threat that you will “ruin” your eyesight is an old wives’ tale. Almost certainly, nobody is going blind from sitting too close unless they ram their eyes into the protruding knobs. Regardless of the {\em false effect} (i.e., blindness), watching TV too closely (the cause) has been shown to have some adverse effects on vision, so concluding that one shouldn’t “move back” from sitting too close is fallacious.
    \\

    
      Example \#2:
    \\

    
      The Church used to claim that giving 10\% of your income to the Church will free a child’s soul from Limbo into Heaven, so clearly giving money to the Church is a scam!
    \\

    
      Explanation: Centuries ago, the Church did accept “contributions” to get loved ones out of “Limbo,” and it wasn’t until 2007 when the Church made it clear that Limbo was a theory and not an official doctrine of the Church, separating the Church from that belief. As for the argument, the false effect of “freeing a child’s soul from Limbo” does not warrant the conclusion that giving your money to the Church is a scam.
    \\

    
      Example \#3: 
    \\

    
      {\em The presence of police at protests cause an escalation of violence. It was the case that at the protest last night attended by uniformed police, there was an escalation of violence. Therefore, police should not be at protests.}
    \\

    
      Explanation: We begin with a causal claim: The presence of police at protests cause an escalation of violence (X apparently causes Y). This is just a claim at this point and has not been established, but because there was an escalation of violence (Y is right), it is concluded that the implied claim (police should not be at protests) is true (therefore, X is right). In reality, we don’t know what caused the escalation of violence, so we cannot conclude anything about the claim. In addition, we are assuming we have agreed to the condition:
    \\

    
      {\em If the presence of police at protests cause an escalation of violence, then there should be no presence of police at protests.}
    \\

    
      To which we may not want to agree if we reason that the presence of police at protests results in a greater good.
    \\

    
      Exception: It can be difficult to parse the claim that is being said to be wrong or right, in which case the causal link might be what is really being rejected. In our first example, it is fallacious to conclude that there is no reason we shouldn’t watch TV close up. We parsed this from “Watching TV that close will make you go blind.” If one is claiming that the causal claim is what is false after showing the effect to be false, then there is no fallacy.
    \\

    
      {\em Mom: Watching TV that close will make you go blind, so move back!} \newline
{\em Jonny: That is B.S., Mom. Watching TV too close does NOT make someone go blind.}
    \\

    
      Fun Fact: This fallacy is different from the {\em false cause}  fallacy (listed under {\em questionable cause}).
    \\

  

Gadarene Swine Fallacy
    Description: The assumption that because an individual is not in formation with the group, that the individual must be the one off course. It is possible that the one who appears off course is the only one on the right course.

    
      Logical Form:
    \\

    
      Person X stands out from the group.
    \\

    
      Therefore, person X is wrong.
    \\

    
      Example \#1:
    \\

    
      Why can't your daughter fall in line like the other girls?
    \\

    
      Explanation: The assumption here is that the "other girls" are doing the right thing. This needs to be established or demonstrated through reason and evidence.
    \\

    
      Example \#2:
    \\

    
      Many people throughout history started revolutions by taking the morally right action when it was considered morally wrong or even illegal at the time. Consider Rosa Parks.
    \\

    
      Exception: It is just as wrong to assume that the one is "out of formation" as it is to assume that all the rest must be "out of formation." While it might be statistically more probable that the one is out of formation, evidence should be sought before making any definitive claim.
    \\

    
      Tip: Compare this to the {\it Galileo fallacy} . You will see that being the oddball neither makes you right nor wrong.
    \\

    References:

    
      
        
      \\

      
        
          Laing, R. D. (1990). {\it The Politics of Experience and The Bird of Paradise}. Penguin Books Limited.
        
      
    
  

Having Your Cake
    
      (also known as: failure to assert, diminished claim, failure to choose sides)
    \\

  
    Description: Making an argument, or responding to one, in such a way that it does not make it at all clear what your position is.  This puts you in a position to back out of your claim at any time and go in a new direction without penalty, claiming that you were “right” all along. 

    
      Logical Form:
    \\

    
      I believe X is a strong argument.
    \\

    
      Y is also a very strong argument.
    \\

    
      Example \#1:
    \\

    
      Reporter: Mr. Congressman, where do you stand on the clean water vs. new factory issue?
    \\

    
      Congressman: Of course, I want our state to have the cleanest water possible.  I can appreciate the petition against the new factory as I can also appreciate the new jobs introduced in our community as a result of the new factory.
    \\

    
      Explanation: This type of “non-decision” or refusal to choose a side often eludes those looking for an answer but getting more of a non-answer in return.  In our example, the congressman can later choose a side based on the outcome, looking like the guy who knew the right answer all along.
    \\

    
      Example \#2:
    \\

    
      Scott: So do you think the earth has only been here for 6-10 thousand years?
    \\

    
      Sam: The evidence for an old earth is very strong, but we cannot discount some of the claims made by the creationists.
    \\

    
      Scott: So what are you saying?
    \\

    
      Sam: I am saying that a 4.7 billion-year-old earth makes a lot of sense, but the 6000-year-old theory does, as well.
    \\

    
      Explanation: We all know and want to shoot people like Sam.  Sam is failing to assert his position.  If Sam’s opinion is respected in this area, no doubt people on both sides will use his statement to their advantage.  This ambiguity is not helpful and is misleading.
    \\

    
      Exception: Wishy-washy statements are sometimes acceptable to demonstrate your uncertainty on a given issue, and if these kinds of statements are followed with admissions of uncertainty or ignorance,  then they are not fallacious; they are honest.
    \\

    
      Tip: If you don’t have an opinion, say that you don’t have an opinion.  If you don’t know, say that you don’t know.  It’s that simple.
    \\

  

Hedging
    Description: Refining your claim simply to avoid counter evidence and then acting as if your revised claim is the same as the original.

    
      Logical Form:
    \\

    
      Claim X is made.
    \\

    
      Claim X is refuted.
    \\

    
      Claim Y is then made and is made to be the same as claim X when it is not.
    \\

    
      Example \#1:
    \\

    
      Freddie: All women are evil, manipulative, man-haters.
    \\

    
      Wade: Including your mother and best friend?
    \\

    
      Freddie: Not them, but all the others.
    \\

    
      Wade: How can you say that, when you only know maybe a hundred or so women?
    \\

    
      Freddie: Obviously, I am talking about the ones I know.
    \\

    
      Explanation: The claim changed quite drastically from about 3.5 billion women to about 100, yet there was no admission by Freddie of this drastic change in his argument.  Freddie is guilty of committing this fallacy, and those who see Freddie’s initial argument as still valid, are guilty, as well.
    \\

    
      Example \#2:
    \\

    
      Adam: The story of Noah’s ark is very probable, and almost certainly a historical and scientific fact.
    \\

    
      Greg: So you think it is very probable that two of each animal came from around the globe, including the animals that cannot survive for very long outside their natural environments?
    \\

    
      Adam: Well, that part did require God’s help.
    \\

    
      Greg: You think it is very probable even though virtually every geologist and natural scientist today reject the idea of a global flood?
    \\

    
      Adam: Probability exists on many levels.
    \\

    
      Greg: Do you really still think this story is, “very probable”?
    \\

    
      Adam: Yes.
    \\

    
      Explanation: Besides the multiple {\it ad hoc}  explanations used by Adam to answer the counterclaims, each counterclaim was evidence against the initial claim, specifically the “very probable” nature of the story.  Rather than concede the argument or revise the claim, Adam let his insistence to be right come before logical thought and refused to change his original claim.
    \\

    
      Exception: If the point of argumentation is really to arrive closer to the truth, then there is no shame in revising claims.  If this is done, there is no fallacy.
    \\

    
      Fun Fact: Every time you acknowledge that you are wrong, you are one step closer to actually being right.
    \\

    References: Dowden, B. (n.d.). Fallacies | Internet Encyclopedia of Philosophy. Retrieved from http://www.iep.utm.edu/fallacy/
  

Inconsistency
    
      (also known as: internal contradiction, logical inconsistency)
    \\

  
    Description: In terms of a fallacious argument, two or more propositions are asserted that cannot both possibly be true.  In a more general sense, holding two or more views/beliefs that cannot all be true together.  Quotes from Yogi Berra (even if apocryphal) are great examples of fallacies, especially inconsistencies.

    
      Logical Form:
    \\

    
      {\em Proposition 1 is logically inconsistent with proposition 2.} \newline
{\em Proposition 1 and proposition 2 are both asserted or implied to be true.}
    \\

    
      Example \#1: 
    \\

    
      "I never said most of the things I said." - {\it Yogi Berra}
    \\

    
      Explanation: I know this requires no explanation, and I don't mean to insult your intelligence, but for consistency's sake, I will explain.  If he had said those things, then he said them, which is a contradiction to his claim that he never said them. This is both an {\it internal inconsistency} and a {\it logical inconsistency}. It is internal because the inconsistency is contained within the statement itself; it doesn't require any other premises or arguments.
    \\

    
      Example \#2: 
    \\

    
      "Nobody goes there anymore.  It's too crowded." - {\it Yogi Berra}
    \\

    
      Explanation: Again, I apologize, but here it goes... If "nobody" went there, then it could not possibly be crowded, since "crowded" implies too many people are there. This is both an {\it internal inconsistency} and a {\it logical inconsistency}.
    \\

    
      Exception: One needs to be able to explain how the beliefs are not inconsistent.
    \\

    
      Tip: Think about your beliefs.  Are there any inconsistent with each other?  Any inconsistent with how you act and what you do?
    \\

  

Inflation of Conflict
    Description: Reasoning that because authorities cannot agree precisely on an issue, no conclusions can be reached at all, and minimizing the credibility of the authorities, as a result.  This is a form of black and white thinking -- either we know the exact truth, or we know nothing at all. 

    
      Logical Form:
    \\

    
      Authority A disagrees with authority B on issue X.
    \\

    
      Therefore, we can say nothing meaningful about issue X.
    \\

    
      Example \#1:
    \\

    
      My mom says that I should study for at least 2 hours each night, and my dad says just a half hour should be fine.  Neither one of them knows what they are talking about, so I should just skip studying altogether.
    \\

    
      Explanation: A disagreement among experts does not mean that both are wrong, the answer is a compromise, or that there is no answer to be known; it simply means that there is disagreement -- that is all we can infer.
    \\

    
      Example \#2:
    \\

    
      Scientists cannot agree on the age of the universe.  Some say it is 13.7 billion years old, some say it is only about 13 billion years old.  That’s a difference of almost a billion years!  It should be clear that because there is so much disagreement, then the 6000-year-old universe should be carefully considered, as well.
    \\

    
      Explanation: Scientists who “disagree” with the estimated age of the universe do so primarily on slightly different interpretations of the same objectively valid dating methods.  The difference is fairly minute in terms of percentage.  Suggesting 6000 years is valid is one thing, but doing so based on the difference in interpretation from mainstream science is completely fallacious.  The differences have no bearing on the truth claim of the argument (the actual age).
    \\

    
      Exception: When the difference in professional disagreement is critical, it should be carefully examined. For example, if two doctors were debating on what medicine to give a patient, and both were claiming that the other medicine would kill the patient.
    \\

    
      Fun Fact: When experts have very different views on scientific issues, it is very often the case that one or more of them are sharing views outside their realm of professional expertise.
    \\

  

Least Plausible Hypothesis
    Description: Choosing more unreasonable explanations for phenomena over more defensible ones.  In judging the validity of hypotheses or conclusions from observation, the scientific method relies upon the {\it Principle of Parsimony}, also known as {\it Occam’s Razor}, which states, {\it all things being equal, the simplest explanation of a phenomenon that requires the fewest assumptions is the preferred explanation until it can be disproved.  }

    
      This is very similar to the {\it far-fetched hypothesis}, but the hypotheses are generally more within reason (i.e., no leprechauns involved).
    \\

    
      Logical Form:
    \\

    
      Hypothesis X is used to explain Y, but hypothesis X is the least plausible.
    \\

    
      Example \#1:
    \\

    
      Here is why I think my date never showed up: her father had a heart-attack, and she had to rush him to the hospital.  In her state of panic, she forgot her cell phone and while at the hospital she was too concerned about her dad to worry about standing me up.
    \\

    
      Explanation: While possible, it is not {\it probable}.  It is much more probable that his date just forgot or has purposely stood him up.  People tend to believe in the least probable hypotheses out of desire, emotion, or faith -- not out of reason.
    \\

    
      Example \#2: Creationists have written volumes of books explaining how, given some divine intervention, a few broken natural laws, and accepting the {\it inconsistency} of nature, it could be possible that the universe is only 6000 years old.  Accepting these theories would require the abandonment or radical reformation of virtually every science we have, as well us a new understanding of the term, “fact”.  So either all of that is true, or, the Biblical creation story, like hundreds of others in cultures all around the world, are simply mythology.
    \\

    
      Explanation: Given the incomprehensible number and severity of the assumptions that would need to be made for creationism to be true, the explanation that the creation story is mythology, by far, is the most economical explanation.
    \\

    
      Exception: This isn't about what might be possible; {\it Occam’s Razor} is all about probabilities.
    \\

    
      Fun Fact: “Plausibility” technically means “believability,” which is highly subjective. A Flat-earther would find the flat-earth theory more plausible than the spherical earth. Clearly, who is making the claim of plausibility, matters. 
    \\

    References:

    
      
        
      \\

      
        
          Wormeli, R. (2001). {\it Meet Me in the Middle: Becoming an Accomplished Middle-level Teacher}. Stenhouse Publishers.
        
      
      
        
      \\

    
    
      
    \\

  

Limited Depth
    Description: Failing to appeal to an underlying cause, and instead simply appealing to membership in a category.  In other words, simply asserting what you are trying to explain without actually explaining anything.

    
      Logical Form:
    \\

    
      {\em Claim X is made about Y.} \newline
{\em Claim X is true because Y is a member of category Z.}
    \\

    
      Example \#1:
    \\

    
      {\em My dog goes through our garbage because he is a dog.}
    \\

    
      Explanation: We know your dog is a dog, but what about him being a dog makes him go through the garbage?  By referring to your dog as a member of the category “dog”, this fails to explain anything.
    \\

    
      Example \#2:
    \\

    
      {\em Mormons are really, really nice because they go to Mormon church.}
    \\

    
      Explanation: {\it Question begging} aside, simply stating that Mormons are a member of the group, “Mormon churchgoers” does not explain why they are nice.  A reasonable explanation would need to include a valid causal relationship between niceness and Mormon-church-going.
    \\

    
      Exception: At times, limited depth can be used as a shorthand when assumptions are made that no deeper explanation is needed. 
    \\

    
      {\em I need oxygen because I am human!}
    \\

    
      Fun Fact: Mormons getting their own planet in the Mormon afterlife is actually a misconception, not official Mormon doctrine.
    \\

    References:

    
      
        
      \\

      
        
          Farha, B. (2013). {\it Pseudoscience and Deception: The Smoke and Mirrors of Paranormal Claims}. University Press of America.
        
      
    
  

Limited Scope
    Description: The theory doesn't explain anything other than the phenomenon it explains (that one thing), and at best, is likely to be incomplete.  This is often done by just redefining a term or phrase rather than explaining it.

    
      Logical Form:
    \\

    
      {\em Theory X is proposed to explain Y.} \newline
{\em Theory X explains nothing else but Y.}
    \\

    
      Example \#1:
    \\

    
      {\em My car broke down because it is no longer working.}
    \\

    
      Explanation: “It isn’t working” is just another way of saying “broke down”, and fails to explain {\it why} it broke down.
    \\

    
      Example \#2:
    \\

    
      {\em People often make hasty decisions because they don’t take enough time to consider their choices.}
    \\

    
      Explanation: Not taking enough time to consider choices is precisely what a hasty decision is.  Again, no explanation is offered, just a definition in place of an explanation.
    \\

    
      Exception: If “because” is replaced with a phrase like, “in other words”, then it is a deliberate clarification and not a fallacy.
    \\

    
      Fun Fact: {\em {\it Limited depth}} and {\em limited scope}  fallacies are sometimes known as {\em fallacies of explanation}.
    \\

    References:

    
      
        
      \\

      
        
          Farha, B. (2013). {\it Pseudoscience and Deception: The Smoke and Mirrors of Paranormal Claims}. University Press of America.
        
      
    
  

Multiple Comparisons Fallacy
    
      (also known as: multiple comparisons, multiplicity, multiple testing problem, the look-elsewhere effect)
    \\

  
    Description: Claiming that unexpected trends that occur through random chance alone in a data set with a large number of variables are meaningful.

    
      In inductive arguments, there is always a chance that the conclusion might be false, despite the truth of the premises. This is often referred to as “confidence level.” In any given study or poll, there is a confidence level of less than 100\%. If a confidence level is 95\%, then one out of 20 similar studies will have a false conclusion. If you make multiple comparisons (either in the same study or compare multiple studies), say 20 or more where there is a 95\% confidence level, you are likely to get a false conclusion. This becomes a fallacy when that false conclusion is seen as significant rather than a statistical probability.
    \\

    
      This fallacy can be overcome by proper testing techniques and procedures that are outside the scope of this book.
    \\

    
      Logical Forms:
    \\

    
      Out of N studies, A produced result X and B produced result Y.
    \\

    
      Tomorrow’s headlines read, “Studies show Y”.
    \\

    
      
    \\

    
      The study’s significance level was X.
    \\

    
      The study compared multiple variables until some significant result was found.
    \\

    
      Example \#1:
    \\

    
      100 independent studies were conducted comparing brain tumor rates of those who use cell phones to those who don’t.
    \\

    
      90 of the tests showed no significant difference in the rates.
    \\

    
      5 of the tests showed that cell phone users were more than twice as likely to develop tumors than those who don’t use cell phones.
    \\

    
      5 of the tests showed that cell phone users were half as likely to develop tumors than those who don’t use cell phones.
    \\

    
      FunTel Mobile’s new ad, “Studies show: Cell phone users are half as likely to develop brain tumors!”
    \\

    
      Explanation: Because we did multiple tests, i.e., compared multiple groups, statistically we are likely to get results that fall within the acceptable margin of error.  These must be disregarded as anomalies or tested further, but not taken to be meaningful while ignoring the other results.
    \\

    
      Example \#2:
    \\

    
      In our study, we looked at 100 individuals who sang right before going to bed, and 100 individuals who did not sing.  Here is what we found: Over 90\% of the individuals who sang slept on their backs, and just 10\% slept on their stomachs or sides.  This is compared to 50\% of those who did not sing, sleeping on their backs and 50\% sleeping on their stomachs or sides.  Therefore, singing has something to do with sleeping position.
    \\

    
      Explanation: What this study did not report, is that over 500 comparisons were done between the two groups, on everything from quality of sleep to what they ate for breakfast the next day.  Out of all the comparisons, most were meaningless, thus were discarded—but as expected via the law statistics and probability, there were some anomalies, the sleeping position being the most dramatic. 
    \\

    
      Exception: Only proper testing and accurate representation of the results would lead to non-fallacious conclusions.
    \\

    
      {\em Fun Fact:} In a group of 23 random people, it is more likely than not that at least two of the people in the group have the same birthday. This is referred to a the {\em birthday paradox} and it is a classic example of the {\em multiple comparisons fallacy}.
    \\

    
      References:
    \\

    
      
        
      \\

      
        
          Walsh, J. (1996). {\it True Odds: How Risk Affects Your Everyday Life}. Silver Lake Publishing.
        
      
    
  

Overextended Outrage
    
      (also known as: overextended moral outrage, overextended, political outrage)
    \\

  
    Description: This is a form of poor statistical thinking where one or more statistically rare cases are implied to be the norm or the trend (without evidence) for the purpose of expressing or inciting outrage toward an entire group. It is a form of extreme {\it stereotyping (the fallacy)}, based on the cognitive bias known as the {\it group attribution error}.

    
      Logical Form:
    \\

    
      Person 1 does something bad.
    \\

    
      Person 1 is a member of group X.
    \\

    
      Outrage is expressed towards group X.
    \\

    
      Example \#1:
    \\

    
      FAUX News runs a story about an illegal immigrant who committed a horrible crime. The commentators talk about this case for weeks, expressing outrage about the serious danger illegal immigrants pose to the good people of the United States.
    \\

    
      Explanation: Violent crime by illegal immigrants is rarer than violent crime committed by U.S. citizens [1]. However, if the narrative a media outlet is trying to sell is that illegal immigrants are dangerous, then they can influence public opinion by inferring that one example of such violence is characteristic of the group. Expressing outrage is a way to make the influence even more effective.
    \\

    
      Example \#2:
    \\

    
      The Huffaluf Post runs a story about a Republican who assaulted a Muslim woman and told her to "go back where she came from." The story is shared millions of times and picked up by other liberal media outlets. Liberals are discussing this story on social media saying how outraged they are at Republicans for their hatred of Muslims.
    \\

    
      Explanation: People and the media (biased media) tend to associate a physical or social identity to the perpetrator of a crime for the purpose of damaging the group's public perception. Why a "Republican" man? How many Republicans are assaulting Muslim women? How many Democrats are? The data are ignored for the benefit of the narrative being sold.
    \\

    
      Exception: There is no exception. If it is "overextended," then the problem is being exaggerated, and a group of people is unfairly demonized.
    \\

    
      Tip: Next time you read about a story that makes you feel outraged, direct your outrage to the individuals directly involved in the story. Don't demonize an entire physical or social identity.
    \\

    References:

    
      This is an original logical fallacy named by the author.

      
        
      \\

      
        1 Adelman, R., Reid, L. W., Markle, G., Weiss, S., \& Jaret, C. (2017). Urban crime rates and the changing face of immigration: Evidence across four decades. {\it Journal of Ethnicity in Criminal Justice}, {\it 15}(1), 52–77. https://doi.org/10.1080/15377938.2016.1261057
      \\

    
  

Post-Designation
    
      (also known as: fishing for data)
    \\

  
    Description: Drawing a conclusion from correlations observed in a given sample, but only after the sample has already been drawn, and without declaring in advance what correlations the experimenter was expecting to find.

    
      Logical Form:
    \\

    
      {\em A sample from a data set is drawn.} \newline
{\em A correlation is found that was not looked at nor is it statistically surprising.} \newline
{\em The correlation is seen as being meaningful.}
    \\

    
      Example \#1:
    \\

    
      {\em In looking at the records of my students, I have found that 9 out of 10 are an only child.  Therefore, society is moving towards one-child families.}
    \\

    
      Explanation: When you start looking at data with no expectations, anything goes, and any data due to random, statistical anomalies will stand out as “odd”.  In this case, the fact that 9 out of 10 kids don’t have siblings is outside of the norm, but that is the nature of probability.  If you were hypothesizing that most kids don’t have siblings, and you found this data, then it would provide more of a reason to do further research in making a more justified conclusion. 
    \\

    
      Example \#2:
    \\

    
      {\em In looking at the difference between 100 Christians and 100 atheists, we found that Christians were significantly more likely to eat tuna fish.}
    \\

    
      Explanation: When you fish for data, you are bound to catch something -- in this case, tuna.  Notice that because we were looking for anything, we are bound to find it.
    \\

    
      Exception: At times, truth is revealed in data whether we look for it or not, but we need to realize that meaningless statistical anomalies are to be expected when looking at data.
    \\

    
      Fun Fact: {\em Post-designation} appears to be just a different name for the {\it multiple comparisons fallacy}. I have kept both entries because I have expanded the {\it multiple comparisons fallacy}  to include multiple comparisons between studies, which doesn’t seem to fit the {\em post-designation} fallacy’s definition.
    \\

  

Proof by Intimidation
    
      (also known as: argumentum verbosium, proof by verbosity, fallacy of intimidation)
    \\

  
    Description: Making an argument purposely difficult to understand in an attempt to intimidate your audience into accepting it, or accepting an argument without evidence or being intimidated to question the authority or {\it a priori} assumptions of the one making the argument.

    
      Logical Form:
    \\

    
      Claim A is made by person 1.
    \\

    
      Person 1 is very intimidating.
    \\

    
      Therefore, claim A is true.
    \\

    
      Example \#1:
    \\

    
      Professor Xavier says that the egg certainly came before the chicken.  He won the Nobel prize last year for his work in astronomy, and the MMA world championship -- so I don’t dare question his claim. 
    \\

    
      Explanation: Professor X sure sounds like a brilliant and tough guy, but that is not evidence for his claim.
    \\

    
      Example \#2:
    \\

    
      Dr. Professor Pete said, with the utmost eloquence, masterful stage presence, and unshakable confidence, that 1+1=3.  Therefore, 1+1=3.
    \\

    
      Explanation: Despite the intellectually intimidating presence of Dr. Professor Pete, 1+1 still equals 2.
    \\

    
      Exception: If you live in a state where you can be killed for asking questions, then this is not a fallacy, but a survival technique.
    \\

    
      Tip: If you live in a state where you can be killed for asking questions, move.
    \\

    References:

    
      
        
      \\

      
        
          Terrell, D. B. (1967). {\it Logic: A Modern Introduction to Deductive Reasoning}. Holt, Rinehart and Winston.
        
      
    
  

Proof Surrogate
    Description: A claim masquerading as proof or evidence, when no such proof or evidence is actually being offered.

    
      Logical Form:
    \\

    
      Claim X is made.
    \\

    
      Claim X is expressed in such a way where no evidence is forthcoming, or no requests for evidence are welcome.
    \\

    
      Therefore, X is true.
    \\

    
      Example \#1:
    \\

    
      Jose writes that "people are mostly good at heart." The author is simply wrong.
    \\

    
      Explanation: The arguer states that the author is "simply wrong" yet offers no reasons. Words and phrases such as "simply," "obviously," "without question," etc., are indicators that no such evidence will be presented.
    \\

    
      Example \#2:
    \\

    
      Politician X is crooked—this is an indisputable fact known by everyone except politician X's supporters.
    \\

    
      Explanation: The language "this in an indisputable fact" is a surrogate for the evidence showing that politician X is crooked.
    \\

    
      Exception: Claims that are universally accepted as self-evident truths don't apply.
    \\

    
      If you put your penis in a wood chipper, it's going to hurt.
    \\

    
      Tip: If you have a penis, don't put it in a wood chipper.
    \\

    References:

    
      
        
      \\

      
        
          Dowden. (1993). {\it Logical Reasoning Im}. Thomson Learning EMEA, Limited.
        
      
    
  

Proving Non-Existence
    Description: Demanding that one proves the non-existence of something in place of providing adequate evidence for the existence of that something.  Although it may be possible to prove non-existence in special situations, such as showing that a container does not contain certain items, one cannot prove universal or absolute non-existence.

    
      Logical Form:
    \\

    
      I cannot prove that X exists, so you prove that it doesn’t.
    \\

    
      If you can’t, X exists.
    \\

    
      Example \#1:
    \\

    
      God exists.  Until you can prove otherwise, I will continue to believe that he does.
    \\

    
      Explanation: There are decent reasons to believe in the existence of God, but, “because the existence of God cannot be disproven”, is not one of them.
    \\

    
      Example \#2:
    \\

    
      Sheila: I know Elvis’ ghost is visiting me in my dreams.
    \\

    
      Ron: Yeah, I don’t think that really is his ghost.
    \\

    
      Sheila: Prove that it’s not!
    \\

    
      Explanation: Once again we are dealing with confusion of probability and possibility.  The inability to, “prove”, in any sense of the word, that the ghost of Elvis is not visiting Sheila in her dreams is an impossible request because there is no test that proves the existence and presence of a ghost, so no way to prove the negative or the non-existence.  It is up to Sheila to provide proof of this claim, or at least acknowledge that actually being visited by Elvis’ ghost is just a {\it possibility}, no matter how slim that possibility is.
    \\

    
      Exception: If Ron were to say, “That is impossible”, “there is no way you are being visited”, or make some other claim that rules out any possibility no matter how remote (or crazy), then Sheila would be in the right to ask him for proof -- as long as she is making a point that he cannot know that for certain, and not actually expecting him to produce proof.
    \\

    
      Tip: If you think you are being visited by aliens, gods, spirits, ghosts, or any other magical beings, just ask them for information that you can verify, specifically with a neutral third-party that would prove their existence.  This would be simple for any advanced alien race, any god or heavenly being.  Some ideas of things to ask for:
    \\

    
      future lottery numbers (of course you will give all your winnings to charity)
    \\

    
      answers to scientific problems that do have scientific answers, but aren’t yet known
    \\

    
      exact details of major future events
    \\

    
      But if these beings just tell you things such as:
    \\

    
      passages / ideas from the Bible
    \\

    
      whether you should take that new job or not
    \\

    
      where you left your car keys
    \\

    
      that they really exist, and others will continue to doubt you
    \\

    
      that you should never question their existence
    \\

    
      ...or anything else which is just as likely to come from your imagination that is untestable and {\it unfalsifiable} , then you might want to reconsider the fact that your being of choice is really paying you visits.
    \\

    References:

    
      
        
      \\

      
        
          You Can Prove a Negative. (n.d.). Retrieved from http://www.psychologytoday.com/blog/believing-bull/201109/you-can-prove-negative
        
      
    
  

Psychogenetic Fallacy
    
      Description: Inferring some psychological reason why an argument is made then assuming it is invalid, as a result.
    \\

    
      Logical Form:
    \\

    
      Person 1 makes argument X.
    \\

    
      Person 1 made argument X because of the psychological reason Y.
    \\

    
      Therefore, X is not true.
    \\

    
      Example \#1:
    \\

    
      George: Man, those girls are smokin' hot!
    \\

    
      Derek: No, they're not. You're a victim of the cheerleader effect. When girls are together in a group, each girl looks a lot better than if you were to see her without the other girls.
    \\

    
      Explanation: Besides the fact that "smokin' hot" is a subjective evaluation, meaning that the girls could be hot to George but not Derek, Derek is assuming George's evaluation is wrong because of the psychological effect known as the {\it cheerleader effect}.
    \\

    
      Example \#2:
    \\

    
      Lucas: I remember when I was about three years old my mother saved me from almost being eaten by a shark.
    \\

    
      Katie: I doubt that. What you are experiencing is what cognitive psychologists refer to as a "false memory."
    \\

    
      Explanation: There are two problems here that make Katie guilty of fallacious reasoning. First, she is assuming Lucas' story is not true because of a false memory. Second, she is phrasing this objection with unwarranted confidence.
    \\

    
      Exception: The more extraordinary the argument or claim, the more reasonable it is to assume a psychological effect is involved. However, even in the most extraordinary of claims, the effect should only be proposed, not assumed.
    \\

    
      Marge: Last night in bed, aliens visited me. They paralyzed me for about 30 seconds while they stood over me, then they disappeared right in front of my eyes while I regained the ability to move and speak.
    \\

    
      Kristine: That sounds like a classic case of what is called a hypnogogic hallucination with sleep paralysis. It is a dream-like state where you are not quite sleeping or awake. Your brain shuts down your ability to move and talk while you are unconscious during a normal sleep cycle, but in the case of sleep paralysis, you retain consciousness—but only for a brief time. So the odds are, you weren't really visited by aliens.
    \\

    
      Fun Fact: As video recorders on cell phones became more ubiquitous, reports of alien encounters have dropped considerably.
    \\

    References:

    
      
        
      \\

      
        
          Segal, R. A. (1980). The social sciences and the truth of religious belief. {\it Journal of the American Academy of Religion}, 403–413.
        
      
    
  

Quantum Physics Fallacy
    
      (also known as: appeal to quantum physics)
    \\

  
    
      
        Description: Using quantum physics in an attempt to support your claim, when in no way is your claim related to quantum physics.  One can also use the weirdness of the principles of quantum physics to cast doubt on the well-established laws of the macro world.
      \\

      
        Perhaps the greatest mind in quantum physics, Richard Feynman, once said, “I think I can safely say that nobody understands quantum mechanics,” and he is probably right.  People recognize that this is perhaps the most bizarre, paradoxical, and incomprehensible area of study, that is also a respectable science.  So, if you can manage to connect the truth of your argument to quantum physics, it would be unlikely that there would be many people who know enough about quantum physics to assert that your connection is invalid. Thus your argument gains credibility out of ignorance.
      \\

      
        The mysterious nature of quantum physics is a breeding ground for superstition, religious claims, “proof” of God, universal consciousness, and many other {\it unfalsifiable} claims. 
      \\

      
        Logical Form:
      \\

      
        Quantum physics supports the idea that X is Y.
      \\

      
        Therefore, X is Y.
      \\

      
        (although quantum physics supports no such thing)
      \\

      
        Example \#1:
      \\

      
        Depook: Quantum physics provides evidence that a cosmic consciousness exists.
      \\

      
        Sam: ???
      \\

      
        Explanation: Sam knows nothing about quantum physics, so really cannot respond, yet Depook did not establish an argument as to how it provides evidence, he just made the assertion. 
      \\

      
        Example \#2:
      \\

      
        Depook: Quantum physics is the language of God.  It has been shown that quantum particles contain information that can instantly communicate information over any distance, anywhere in or outside the universe.
      \\

      
        Sam: ???
      \\

      
        Explanation: Sam knows nothing about quantum physics, so really cannot respond.  Depook did expand on his assertion here, relied on the {\it argument by gibberish} in order to make what sounded like scientific claims which, in fact, were not.  According to everything we know about quantum physics, information cannot travel faster than light -- otherwise, it could create a {\it time travel paradox}.
      \\

      
        Exception: Making a scientific claim about quantum physics, using the scientific method, is not fallacious.
      \\

      
        Tip: Pick up an introductory book on quantum physics, it is not only a fascinating subject, but you will be well prepared to ask the right questions and expose this fallacy when used.
      \\

    
  

Retrogressive Causation
    Description: Invoking the cause to eliminate the effect, or calling on the source to relieve the effect of the source.

    
      Logical Form:
    \\

    
      {\em X causes/is the source of Y.}
    \\

    
      {\em In order to eliminate or relieve Y, do more of X.}
    \\

    
      Example \#1:
    \\

    
      {\em Jen: Don’t you realize that all this drinking you are doing is making your family miserable?}
    \\

    
      {\em Bridget: Yes, I do.}
    \\

    
      {\em Jen: Then what are you doing about it?}
    \\

    
      {\em Bridget: Drinking to forget.}
    \\

    
      Explanation: Bridget has a drinking problem that she is dealing with by drinking some more -- because the effects of drinking make her (temporarily) forget/not worry about the greater scale effects of her drinking.  Her reasoning that this is a good idea is fallacious.
    \\

    
      Example \#2:
    \\

    
      David: We have way too much police presence in this city.
    \\

    
      Pete: What are you going to do about it?
    \\

    
      David: Vandalize, loot, and perhaps a little arson.
    \\

    
      Explanation: The primary role of the police is to enforce laws. David is suggesting that breaking the laws will facilitate his goal of having the police force reduced.
    
    
       \newline

      

      
        Exception: In some cases, one may not be trying to eliminate the effect, but rather continue the cycle for some higher purpose. For example, if one learns to feel constant guilt by going to church and is relieved of that guilt by going to confession, they might find meaning in the constant “spiritual cleansing” ritual. While this might seem like irrational thinking to some, it would not fit under this fallacy.
      \\

      
        Fun Fact: People are easily persuaded to act against their self-interests. If acting against their self-interest is for some higher purpose, the behavior would not be considered irrational.
      \\

    
  \section{Righteousness Fallacy
    Description: Assuming that just because a person's intentions are good, they have the truth or facts on their side. Also see {\it self-righteousness fallacy}.

    
      Logical Form:
    \\

    
      Person 1 made claim X.
    \\

    
      Person 1 has good intentions.
    \\

    
      Therefore, X is true.
    \\

    
      Example \#1:
    \\

    
      Ricki: Do you think aborted fetuses have feelings?
    \\

    
      Jenni: I follow the lead of my grandmother who is the most honorable and kind person I know. She says they do have feelings.
    \\

    
      Explanation: Jenni's grandmother might be the queen of honor with kindness oozing from her orthopedic shoes, but these qualities are independent of one's ability to know facts or come to an accurate conclusion based on available data.
    \\

    
      Example \#2:
    \\

    
      The president wants to bomb that country because he thinks they are preparing to launch a nuclear attack against us. I know the president wants to do the right thing for the good of the American people, so if he says there have nukes, they have nukes!
    \\

    
      Explanation: The good intentions of the president are separate from the president's ability to get solid intelligence on foreign affairs. If we are convinced of the president's good intentions, the best we can do is claim that we believe that the president believes he is doing the right thing.
    \\

    
      Exception: This relates to facts, not subjective truth. We can use the idea of righteousness to conclude how we feel about a person.
    \\

    
      Fun Fact: This is related to the cognitive bias, the {\em halo effect}.
    \\

  }


Self-righteousness Fallacy
    Description: Assuming that just because your intentions are good, you have the truth or facts on your side. Also see {\it righteousness fallacy}.

    
      Logical Form:
    \\

    
      {\em You make claim X.}
    \\

    
      {\em You have good intentions.}
    \\

    
      {\em Therefore, X is true.}
    \\

    
      Example \#1:
    \\

    
      {\em Ricki: Do you think aborted fetuses have feelings?} \newline
{\em Jenni: Any honorable and kind person would have to say they do have feelings. So yes.}
    \\

    
      Explanation: Jenni might be the queen of honor with kindness oozing from her puppy-dog eyes, but these qualities are independent of one’s ability to know facts or come to an accurate conclusion based on available data.
    \\

    
      Example \#2:
    \\

    
      {\em Jenni: Is a fetus a human being?} \newline
{\em Ricki: No, because I am not a monster and would never suggest killing an unwanted human being is okay.}
    \\

    
      Explanation: Ricki is making a claim about a fetus and using the fact that she is “not a monster” to support the claim, which is independent of her intentions.
    \\

    
      Exception: This relates to facts, not subjective truth. We can use the idea of righteousness to conclude how we feel about something or someone. For example,
    \\

    
      {\em Jenni: Do you consider a fetus to be as valuable as a human being?} \newline
{\em Ricki: No, because I am not a monster and would never suggest killing an unwanted human being is okay.}
    \\

    
      Fun Fact: {\em Self-righteous} is defined as having or characterized by a certainty, especially an unfounded one, that one is totally correct or morally superior. The {\em self-righteousness fallacy} follows a more generic definition of being correct because of “good intentions.”
    \\

  

Rights To Ought Fallacy
    
      (also known as: constitutional rights fallacy)
    \\

  
    
      Description: When one conflates a reason for one's rights (constitutional or other) with what one {\it should} do.  This is common among staunch defenders of "rights" who fail to see that rights are not the same as optimal courses of action.  It can be a way of attempting to hide the fact that the "should" is based on one's subjective moral values (or at least values that are not shared by the opponent) rather than a more objective law to which virtually everyone acknowledges.
    \\

    
      Logical Form:
    \\

    
      Person A should not have done X.
    \\

    
      Person A had every right to do X; therefore, person A should have done X.
    \\

    
      Example \#1:
    \\

    
      Carl: Hi Billy, it is great to meet you! I think you will be happy here at Friendly Manufacturing, Inc.
    \\

    
      Billy: Hey, you're Irish!  Irish people make great factory workers—that is where they are happiest.  I am surprised to see you in management. 
    \\

    
      Carl: Excuse me??
    \\

    
      Billy: Don't mind me.  I am just expressing my constitutional right to freedom of speech.  Do you have a problem with our Constitution?  Do you hate America?
    \\

    
      Explanation: Billy is clearly ignorant when it comes to the realities of cultural differences, and he seriously lacks social skills.  He is correct that he has every right to express his opinions, but he does not seem to mind offending and hurting others by making his opinions known.  Constitutional rights do not exist in a vacuum—they are part of the larger system that includes social conventions such as tact, appropriate behavior, and kindness.
    \\

    
      Example \#2:
    \\

    
      A top reality TV superstar from the hit show "Goose Galaxy" recently did an interview with GM magazine (Geese Monthly).  In this interview, he told the interviewer that, according to his beliefs, the Galactic Emperor has decreed that all MAC users are "sinful" and MAC use leads to having sex with computers.  When many MAC users and non-MAC users alike expressed their outrage at what they felt was an offensive and demonstrably false proposition, defenders of "Goose Galaxy" screamed that the TV superstar had every right to say those things as his speech is protected by the First Amendment.
    \\

    
      Explanation: The claim made was that the comments were offensive and demonstrably false (as no research has been able to demonstrate that MAC use leads to having sex with computers), and the reality TV star {\it should not} have said those things, yet the "Goose Galaxy" supporters countered with the fact that the reality TV star had the right to say those things; therefore, he {\it should have}.  Notice that no claim was made about rights—this is a {\it strawman}.  The fallacy extends beyond the strawman because the defenders of "Goose Galaxy" are conflating the reality TV star's constitutional rights with the claim that he {\it should}  have said those things.
    \\

    
      Exception: When one's values are in line with the rights, then claiming one "should" exercise his or her rights is not fallacious—as long as the reason given does not have to do with rights: \newline

    \\

    
      I feel that it is morally wrong to use a MAC; therefore, I {\it should}  speak out against MAC users; and yes, it is my constitutional right to do so.
    \\

    
      Tip: Just because you have the right to be a galactic jackass, doesn’t mean you should.
    \\

  

Selective Attention
    Description: Focusing your attention on certain aspects of the argument while completely ignoring or missing other parts.  This usually results in irrelevant rebuttals, {\it strawman fallacy}, and unnecessarily drawn-out arguments.

    
      Logical Form:
    \\

    
      {\em Information is presented.} \newline
{\em Response addresses only some of the information, completely ignoring the rest.}
    \\

    
      Example \#1:
    \\

    
      {\em News Anchor on TV: The Dow Jones was up 200 points today, NASDAQ closed up 120 points, unemployment is and has been declining steadily, but foreclosures have not budged. \newline
Jimbo: Did you hear that?  Our economy is in the crapper!}
    \\

    
      Explanation: While there are many problems with the reasoning of Jimbo, due to his {\it selective attention}, and possible pessimism when it comes to the economy, he did not let the good news register and/or did not take that information into consideration before concluding that our economy is still in the “crapper”, based on that one piece of news on foreclosures.
    \\

    
      Example \#2: Most of us are guilty of {\it selective attention}  when the information is about us.  We tend to embrace the information that makes us feel good and ignore the information that does not.
    \\

    
      Exception: Ignoring irrelevant information is a good thing when evaluating arguments.  The key is to know what is irrelevant.
    \\

    
      Fun Fact: {\em Selective attention} is mostly a cognitive bias in that it happens subconsciously. It becomes a fallacy when it shows up in argumentation.
    \\

  

Self-Sealing Argument
    
      (also known as: vacuous argument)
    \\

  
    Description: An argument or position is self-sealing if and only if no evidence can be brought against it no matter what.

    
      Logical Form:
    \\

    
      {\em Claim X is made.} \newline
{\em Reason Y is given for claim X.} \newline
{\em Reason Y can not possibly be refuted.}
    \\

    
      Example \#1:
    \\

    
      {\em Wherever you go, there you are.}
    \\

    
      Explanation: You can’t argue against that position, and as a result, it is {\it vacuous}, or meaningless. 
    \\

    
      Example \#2:
    \\

    
      {\em Tina: My life is guided by destiny.}
    \\

    
      {\em Mary: How do you know that?}
    \\

    
      {\em Tina: Whatever comes my way is what was meant to be.}
    \\

    
      Explanation: We have the same vacuity problem here, except this one is less obvious and protected by a philosophical belief system.  There is no possible way we can know what "destiny may have in store for us," thus no way to argue against it.  As a result, it is meaningless -- it is the equivalent of saying everything happens because it happens.
    \\

    
      Exception: Holding beliefs that are unfalsifiable is not fallacious, especially when stated as beliefs or opinions. This becomes fallacious when an unfalsifiable claim is presented as evidence in argumentation.
    \\

    
      Tip: Realize that most superstitious beliefs are centered around self-sealing or vacuous arguments, that is why so many people refuse to let go of superstitious beliefs -- because they cannot be proven false.
    \\

    References:

    
      
        
      \\

      
        
          Blair, J. A. (2011). {\it Groundwork in the Theory of Argumentation: Selected Papers of J. Anthony Blair}. Springer Science \& Business Media.
        
      
    
  

Shifting of the Burden of Proof
    
      (also known as: onus probandi, burden of proof [general concept], burden of proof fallacy, misplaced burden of proof, shifting the burden of proof)
    \\

  
    Description: Making a claim that needs justification, then demanding that the opponent justifies the opposite of the claim. The burden of proof is a legal and philosophical concept with differences in each domain. In everyday debate, the burden of proof typically lies with the person making the claim, but it can also lie with the person denying a well-established fact or theory. Like other non-black and white issues, there are instances where this is clearly fallacious, and those which are not as clear.

    
      Logical Form:
    \\

    
      Person 1 is claiming Y, which requires justification.
    \\

    
      Person 1 demands that person 2 justify the opposite of Y.
    \\

    
      Person 2 refuses or is unable to comply.
    \\

    
      Therefore, Y is true.
    \\

    
      Example \#1:
    \\

    
      Jack: I have tiny, invisible unicorns living in my anus.
    \\

    
      Nick: How do you figure?
    \\

    
      Jack: Can you prove that I don't?
    \\

    
      Nick: No.
    \\

    
      Jack: Then I do.
    \\

    
      Explanation: Jack made a claim that requires justification. Nick asked for the evidence, but Jack shifted the burden of proof to Nick. When Nick was unable to refute Jack's ({\it unfalsifiable}) claim, Jack claimed victory.
    \\

    
      Example \#2:
    \\

    
      Audrey: I am a human being. I am not a cyborg from the future here to destroy humanity.
    \\

    
      Fred: Prove that you are human! Cyborgs don't pass out when they lose a lot of blood. Here's a knife.
    \\

    
      Audrey: Get to bed, Freddie. And no more SYFY channel before bed!
    \\

    
      Explanation: Audrey is making a claim of common knowledge, perhaps sparked by Fred's suspicions. Fred is asking Audrey to prove the claim when he is the one that should be justifying his objection to the claim of common knowledge.
    \\

    
      Exception: Again, the question of who has the burden of proof is not always as simple as demonstrated in these examples. Often, this is an argument itself.
    \\

    
      Tip: If possible, justify your argument with evidence even if you might not have the burden of proof. The only time you might not want to do this is when it gives credibility to an outrageous accusation or claim.
    \\

    References:

    
      
        
      \\

      
        
          Bunnin, N., \& Yu, J. (2008). {\it The Blackwell Dictionary of Western Philosophy}. John Wiley \& Sons.
        
      
    
  

Shoehorning
    Description: The process of force-fitting some current affair into one's personal, political, or religious agenda.  Many people aren't aware of how easy it is to make something look like confirmation of a claim after the fact, especially if the source of the confirmation is something in which they already believe, like Biblical prophecies, psychic predictions, astrological horoscopes, fortune cookies, and more.

    
      Logical Form:
    \\

    
      {\em Current event X is said to relate to agenda Y.} \newline
{\em Agenda Y has no rational connection to current event X.}
    \\

    
      Example \#1: This example is taken from the {\it Skeptic’s Dictionary} (http://www.skepdic.com/shoehorning.html). 
    \\

    
      {\em After the terrorist attacks on the World Trade Center and the Pentagon on September 11, 2001, fundamentalist Christian evangelists Jerry Falwell and Pat Robertson shoehorned the events to their agenda. They claimed "liberal civil liberties groups, feminists, homosexuals and abortion rights supporters bear partial responsibility...because their actions have turned God's anger against America." According to Falwell, God allowed "the enemies of America...to give us probably what we deserve." Robertson agreed. The American Civil Liberties Union has "got to take a lot of blame for this," said Falwell and Robertson agreed. Federal courts bear part of the blame, too, said Falwell, because they've been "throwing God out of the public square." Also, "abortionists have got to bear some burden for this because God will not be mocked," said Falwell and Robertson agreed.}
    \\

    
      Explanation: It should be very clear how these religious leaders attempted to profit from the September 11 attacks by {\it shoehorning}.
    \\

    
      Example \#2: For thousands of years people have been rushing to scripture to try to make sense out of a current situation.  Without a doubt, the same verses have been used over and over again for centuries as a prophecy of a current event.  This is {\it shoehorning}.  A great example of this is the BP oil spill in April of 2010.  It has been suggested that the verses from Revelation 8:8–11 predicted this environmental disaster:
    \\

    
      {\em “The second angel blew his trumpet, and something like a great mountain, burning with fire, was thrown into the sea. A third of the sea became blood, a third of the living creatures in the sea died, and a third of the ships were destroyed … A third of the waters became wormwood, and many died from the water, because it was made bitter.” }
    \\

    
      With over 31,000 verses, the probability of NOT finding a verse in the Bible that can be made to fit virtually any modern-day situation is next to zero, but what if you had 2,000 years of history to play with?  It’s not difficult to see how quickly these “fulfilled prophecies” can add up.
    \\

    
      Exception: Explaining events is legitimate when reason is being used -- and sometimes it may actually fit into your ideological agenda.
    \\

    
      Fun Fact: Did you every notice website with crazy conspiricy theories tend to be horribly designed? {\em Haig’s Law} states: “The awfulness of a website's design is directly proportional to the insanity of its contents and creator.”
    \\

    
      
    \\

    References:

    
      
        
      \\

      
        
          shoehorning - The Skeptic’s Dictionary - Skepdic.com. (n.d.). Retrieved from http://www.skepdic.com/shoehorning.html
        
      
    
  

Spiritual Fallacy
    
      (also known as: spiritual excuse)
    \\

  
    Description: Insisting that something meant to be literal is actually “spiritual” as an explanation or justification for something that otherwise would not fit in an explanation.

    
      Logical Form:
    \\

    
      X makes no sense; therefore, X was meant in a “spiritual” sense.
    \\

    
      Example \#1:
    \\

    
      Of course, the Koran is not a history or science book, but each and every story in it does contain a spiritual truth.
    \\

    
      Explanation: Because we cannot define or prove a “spiritual truth”, anything can be a spiritual truth.
    \\

    
      Example \#2:
    \\

    
      Harold Camping, the preacher who predicted the rapture in 2011, said that the rapture actually did come, but it was a "spiritual" rapture.  Of course, there is no way to demonstrate this.
    \\

    
      Explanation: We can’t use “spiritual” as a get-out-of-jail-free card to cover up an apparent contradiction.
    \\

    
      Exception: It is not a fallacy when it is specifically referred to as “spiritual”.
    \\

    
      “and drank the same spiritual drink; for they drank from the spiritual rock that accompanied them, and that rock was Christ.” (1 Cor 10:4)
    \\

    
      Tip: Next time you get pulled over for speeding, tell the cop you were only “spiritually” speeding. See if that works.
    \\

  

Spin Doctoring
    
      (also known as: spinning)
    \\

  
    Description: Presenting information in a deceptive way that results in others interpreting the information in such a way that does not reflect reality but is how you want the information to be interpreted.

    
      Logical Form:
    \\

    
      X represents reality.
    \\

    
      Information is presented in such a way that Y appears to represent reality.
    \\

    
      Example \#1:
    \\

    
      Senator Elizabeth Warren was recently under attack because it was discovered that the men on her staff were paid, on average, considerably more than the women on her staff—an issue that Warren has campaigned on many times. While the facts are true (men are paid more on her staff), in many analyses, relevant data was excluded such as the criteria necessary to prove the claim that women on Senator Warren's staff were paid less than their male counterparts for equivalent work [1]. In fairness to some conservative outlets that reported this story, some used this as an example to show that there is more to the story than just raw numbers.
    \\

    
      Example \#2:
    \\

    
      Simon: It is pretty darn clear that God is against homosexuality. According to Leviticus 20:13, {\it “If a man lies with a male as with a woman, both of them shall be put to death for their abominable deed; they have forfeited their lives.”}
    \\

    
      {\it Bret: You don't understand. This was before Jesus. After Jesus, God was okay with it. Besides, these were specific instructions to a specific people during a specific historical period.}
    \\

    
      Explanation: Very often, people's ideas of God are a result of their values, not the other way around. This is made clear by the cultural shifts on moral issues that correlate with people's interpretation of the Bible (one example being Christian's views on homosexuality). Bret may genuinely believe his narrative, but it was most likely a result of the spin doctoring of another person or organization.
    \\

    
      Exception: They are situations where there is no objective truth to a view and data can be looked at in multiple ways, such as the classic "is the glass half-full or half-empty" question.
    \\

    
      Tip: Consider the source's biases. This will help with detecting spin doctoring.
    \\

    References:

    
      
        
      \\

      
        This a logical fallacy frequently used on the Internet. No academic sources could be found.
      \\

      
        
          1 Elizabeth Warren Pays Female Staffers Less Than Their Male Counterparts? (2017, April 6). Retrieved April 13, 2017, from http://www.snopes.com/elizabeth-warren-staff-pay/
        
      
    
  

Statement of Conversion
    Description:  Accepting the truth of a claim based on a conversion story without considering any evidence for the truth of the claim.

    
      Logical Form:
    \\

    
      I used to believe in X.
    \\

    
      Therefore, X is wrong.
    \\

    
      Example \#1:
    \\

    
      I used to be a Christian, now I know better.
    \\

    
      Explanation: All this tells us is that the arguer changed his mind.  We don’t know why. Accepting this as evidence against Christianity would be fallacious reasoning.
    \\

    
      Example \#2:
    \\

    
      There used to be a time when I didn’t believe, now I see the light and have accepted Jesus as my savior!
    \\

    
      Explanation: All this tells us is that the arguer changed his mind.  We don’t know why.  Accepting this as evidence for Christianity would be fallacious reasoning.
    \\

    
      Exception: It might be the case where the person with the conversion story has some expertise or direct experience related to the claim, so their conversion story is reasonable evidence for the truth of the claim. \newline

    \\

    
      {\em I used to think that the earth was flat, then Elon Musk took me to the space station in one of his ships. I am no longer a flat-earther, and I can’t believe what an idiot I was listening to YouTube videos of 40-year-old guys living in their parent’s basement over NASA.} \newline

    \\

    
      Tip: Remember not to confuse people’s interpretations of their experiences as actual experiences. People who were unbelievers in aliens visiting earth and changed their mind based on “aliens visiting them in their dreams” are almost certainly misinterpreting their experiences.
    \\

    References:

    
      
        
      \\

      
        
          MAC, M. J. T., PhD, CSAC. (2006). {\it Critical Thinking for Addiction Professionals}. Springer Publishing Company.
        
      
    
  

Stereotyping (the fallacy)
    
      
        Description: The general beliefs that we use to categorize people, objects, and events while assuming those beliefs are accurate generalizations of the whole group.
      \\

      
        Logical Form:
      \\

      
        All X’s have the property Y (this being a characterization, not a fact).
      \\

      
        Z  is an X.
      \\

      
        Therefore, Z has the property Y.
      \\

      
        Example \#1:
      \\

      
        French people are great at kissing.  Julie is French.  Get me a date!
      \\

      
        Explanation: “French people are great at kissing” is a stereotype, and believing this to be so is a fallacy.  While it may be the case that {\it some} or even {\it most} are great at kissing, we cannot assume this without valid reasons.
      \\

      
        Example \#2:
      \\

      
        Atheists are morally bankrupt.
      \\

      
        Explanation: This isn’t an argument, but just an assertion, one not even based on any kind of facts.  Stereotypes such as these usually arise from prejudice, ignorance, jealousy, or even hatred.
      \\

      
        Exception: Statistical data can reveal properties of a group that are more common than in other groups, which can affect the probability of any individual member of the group having that property, but we can never assume that all members of the group have that property.
      \\

      
        Tip: Remember that people are individuals above being members of groups or categories.
      \\

    
  

Subverted Support
    Description: The attempt to explain some phenomenon that does not actually occur or there is no evidence that it does.  It is a form of {\it begging the question}.

    
      Logical Form:
    \\

    
      X happens because of Y (when X doesn’t really even happen)
    \\

    
      Example \#1:
    \\

    
      The reason billions of children starve to death each year is because we live in a world that does not care.
    \\

    
      Explanation: Billions of children don’t starve to death each year -- not even close.  If it were close, it might be better categorized as an exaggeration, but this would be more of an attempt to get the audience to accept the assertion as a fact while focusing more on the reason rather than the assertion itself.
    \\

    
      Example \#2:
    \\

    
      The reason the firmament, a tent-like structure that kept the “waters above” from flooding the earth as described in the Bible, is no longer there today, is because it was destroyed during Noah’s flood.
    \\

    
      Explanation: The reason the firmament isn’t there today is because it never existed.  To attempt to explain it is to get the audience to assume it existed.
    \\

    
      Exception: If the argument is preceded with a declaration that the phenomenon does occur, then what would be the {\it subverted support }is simply a reason given.
    \\

    
      The firmament, a tent-like structure that kept the “waters above” from flooding the earth as described in the Bible, once covered the earth.  It is no longer there today because it was destroyed during Noah’s flood.
    \\

    
      Tip: Exaggeration is a risky technique you want to avoid. On the one hand, it can make your argument more compelling (technically, by misrepresenting the truth). On the other hand, if you are called out for your exaggeration, it will damage the credibility of your argument as well as your own credibility.
    \\

  

Blinding with science
    
      - Description: The fallacy of blinding with science occurs when technical jargon and scientific terminology are used to obscure the true meaning of a statement, giving it an unearned sense of authority and validity. This tactic aims to deceive the audience into believing that the speaker’s assertions are supported by rigorous scientific evidence, even when they are not.
    \\

    
      
    \\

    
      - Logical Form:
    \\

    
        1. Speaker uses complex scientific language to present a point.
    \\

    
        2. Audience is unable to understand the jargon.
    \\

    
        3. Audience assumes the point is valid due to its scientific appearance.
    \\

    
      
    \\

    
      - Example \#1:
    \\

    
        - Scenario: "The amotivational syndrome is sustained by peer group pressure except where achievement orientation forms a dominant aspect of the educational and social milieu."
    \\

    
        - Explanation: This statement uses complex terminology to assert that people are influenced by their friends unless they have personal goals. The jargon intimidates the audience into accepting the statement without questioning its validity.
    \\

    
      
    \\

    
      - Example \#2:
    \\

    
        - Scenario: "The transportational flow charts for the period following the postmeridian peak reveal a pattern of decantation of concentrated passenger units in cluster formations surrounding the central area."
    \\

    
        - Explanation: This statement uses complicated language to describe people gathering in the city center after work. The technical jargon obscures the simple reality, making it seem more profound and scientifically grounded than it is.
    \\

    
      
    \\

    
      - Variation:
    \\

    
        - Scenario: "The small, domesticated carnivorous quadruped positioned itself in sedentary mode in superior relationship to the coarse-textured rushwoven horizontal surface fabric."
    \\

    
        - Explanation: This overly complex description of a cat sitting on a mat serves no purpose other than to confuse the reader and obscure the simplicity of the observation.
    \\

    
      
    \\

    
      - Tip: To avoid being blinded by science, focus on understanding the core message. Ask for clarification in simpler terms if the language seems needlessly complex. Evaluate the argument based on evidence rather than the sophistication of the terminology used.
    \\

    
      
    \\

    
      - Exception: This fallacy is not committed if the use of technical language is necessary due to the complexity of the subject and if the audience is expected to understand the terminology.
    \\

    
      
    \\

    
      - Fun Fact: The term "blinding with science" captures the dazzling effect of overly complex language, much like how bright lights can obscure vision. This tactic is often employed in pseudo-sciences and disciplines trying to gain unwarranted credibility by mimicking the language of genuine scientific inquiry.
    \\

  
    
      (Also known as: Obscurantism, Technobabble)
    \\

  

Thatcher’s blame
    
      - Description: The fallacy of Thatcher’s Blame occurs when blame is assigned regardless of the outcome, illustrating a form of criticism where the target is faulted no matter what happens. It is named after the tendency to blame Margaret Thatcher for any and all issues during her tenure as Prime Minister, regardless of the specific circumstances.
    \\

    
      
    \\

    
      - Logical Form:
    \\

    
        1. A person or policy is criticized or blamed.
    \\

    
        2. The blame persists regardless of the outcome or evidence.
    \\

    
        3. The criticism is applied universally to all possible outcomes.
    \\

    
      
    \\

    
      - Example \#1:
    \\

    
        - Scenario: "If a new policy is introduced in Scotland before England, the Scots are being used as guinea pigs. If it’s introduced in England first, the Scots are being left out. If it’s introduced simultaneously, it shows a lack of understanding of regional differences."
    \\

    
        - Explanation: No matter how the policy is introduced, the blame shifts but remains constant, illustrating that the criticism is not about the policy itself but about the inevitable fault-finding.
    \\

    
      
    \\

    
      - Example \#2:
    \\

    
        - Scenario: "If a celebrity takes a public stand, they are criticized for seeking attention. If they remain silent, they are accused of being irrelevant or aloof."
    \\

    
        - Explanation: The celebrity is blamed regardless of their actions, showing that the criticism is predetermined and not based on the actual merits or shortcomings of their behavior.
    \\

    
      
    \\

    
      - Variation:
    \\

    
        - Scenario: "In parliament, if the government acts quickly, they are accused of rushing through decisions recklessly. If they delay, they are criticized for intolerable procrastination."
    \\

    
        - Explanation: The government faces criticism no matter how they handle the situation, demonstrating that the fault-finding is inherent rather than based on the specific actions or results.
    \\

    
      
    \\

    
      - Tip: To use Thatcher’s Blame effectively, predict multiple negative outcomes for any proposed action or decision. This allows you to cover every possible scenario with a predetermined negative judgment, making it appear that any outcome is flawed.
    \\

    
      
    \\

    
      - Exception: This fallacy does not apply if the criticism is based on substantive evidence related to the specific outcome or context of the action, rather than an overarching predetermined judgment.
    \\

    
      
    \\

    
      - Fun Fact: The term “Thatcher’s Blame” reflects a specific historical context where political opponents frequently criticized Margaret Thatcher’s policies regardless of their outcomes, illustrating how political figures can become symbols for generalized criticism.
    \\

  
    
      (Also known as: Omnipresent Blame)
    \\

  

Unaccepted enthyrnemes
    
      (Also known as: Missing Premise Fallacy)
    \\

  
    
      - Description: An enthymeme is an argument where one premise is implied rather than stated explicitly. The fallacy of unaccepted enthymemes occurs when the unstated premise is not accepted by the listener, leading to an argument that lacks proper support. 
    \\

    
      
    \\

    
      - Logical Form:
    \\

    
        1. Argument is presented with an implied premise. 
    \\

    
        2. The implicit premise is not accepted or is unknown to the audience. 
    \\

    
        3. The conclusion drawn is flawed due to the missing or unaccepted premise. 
    \\

    
      
    \\

    
      - Example \#1:
    \\

    
        - Scenario: "Bill must be stupid. You have to be stupid to fail a driving test." 
    \\

    
        - Explanation: The argument assumes that Bill failed his driving test, which is the unstated premise. If Bill did not fail the test, the argument falls apart because the assumption is not accepted by the listener. 
    \\

    
      
    \\

    
      - Example \#2:
    \\

    
        - Scenario: "I hope to repay the bank soon, Mr. Smith. My late aunt said she would leave a reward to everyone who had looked after her." 
    \\

    
        - Explanation: The argument implies that the speaker looked after their aunt, which is the missing premise. If this premise is not accepted or is incorrect, the argument lacks support. 
    \\

    
      
    \\

    
      - Variation:
    \\

    
        - Scenario: "Darling, I'm sorry. Busy people tend to forget such things as anniversaries." 
    \\

    
        - Explanation: The unstated assumption is that the speaker has been busy and therefore forgot the anniversary. If the listener finds out the speaker has been free, the argument fails due to the unaccepted enthymeme. 
    \\

    
      
    \\

    
      - Tip: When presenting arguments or excuses, ensure that all relevant premises are either stated or known to your audience. Verify that your listener accepts the implicit assumptions to avoid fallacious reasoning. 
    \\

    
      
    \\

    
      - Exception: Unaccepted enthymemes do not apply if both parties explicitly agree on the unstated premises. The fallacy arises only when the missing premise is not acknowledged or accepted. 
    \\

    
      
    \\

    
      - Fun Fact: The term "enthymeme" comes from the Greek "enthymema," which means "in the mind" or "thought"—a fitting description for arguments that rely on mental gaps rather than stated logic. 
    \\

  

Feedback fallacy

Lump of labour fallacy
    
      (Also known as: Lump of Jobs Fallacy, Fallacy of Labour Scarcity, Fixed Pie Fallacy, Zero-Sum Fallacy)
    \\

  
    
      \#\#\# Lump of Labour Fallacy
    \\

    
      
    \\

    
      - Name: Lump of Labour Fallacy
    \\

    
      
    \\

    
      - Also known as: Lump of Jobs Fallacy, Fallacy of Labour Scarcity, Fixed Pie Fallacy, Zero-Sum Fallacy
    \\

    
      
    \\

    
      - Description: The lump of labour fallacy is the incorrect belief that there is a fixed amount of work available in an economy. According to this fallacy, if some people work fewer hours or if productivity increases, it will lead to a reduction in the number of jobs or increase in unemployment. This misconception suggests that employment is a zero-sum game, where the gain of one worker is the loss of another.
    \\

    
      
    \\

    
      - Logical Form:
    \\

    
        1. Premise 1: There is a fixed amount of work to be done in the economy.
    \\

    
        2. Premise 2: Reducing work hours or increasing productivity leads to fewer jobs.
    \\

    
        3. Conclusion: Measures such as reducing working hours or increasing immigration will decrease employment or increase unemployment.
    \\

    
      
    \\

    
      - Example \#1:
    \\

    
        - Scenario: Advocates argue that increasing the minimum wage will lead to higher unemployment because employers will reduce their workforce to offset the increased labor costs.
    \\

    
        - Explanation: This argument assumes that the total amount of work is fixed, and that higher wages will lead to fewer jobs. In reality, higher wages can increase worker productivity and demand for goods, potentially creating more jobs.
    \\

    
      
    \\

    
      - Example \#2:
    \\

    
        - Scenario: Some people claim that allowing more immigrants into the workforce will take jobs away from native-born workers.
    \\

    
        - Explanation: This view is based on the belief that the number of available jobs is fixed. However, immigrants can increase demand for goods and services, which can lead to the creation of new jobs and economic growth.
    \\

    
      
    \\

    
      - Variation: The fallacy can be applied to various contexts, such as debates over working hours, immigration, and technological advancements. It is also known as the "zero-sum fallacy" because it incorrectly assumes that gains in employment for some result in losses for others.
    \\

    
      
    \\

    
      - Tip: To counter the lump of labour fallacy, consider that economic growth and job creation are not zero-sum. Expanding productivity, technological advancement, and increasing the labor pool can lead to a more dynamic job market with new opportunities.
    \\

    
      
    \\

    
      - Exception: Critics argue that the fallacy does not account for all nuances, such as the real administrative and operational costs associated with changing labor policies or the transitional impacts of technological changes. These factors can sometimes lead to temporary disruptions or adjustments in the job market.
    \\

    
      
    \\

    
      - Fun Fact: The term "lump of labour fallacy" was popularized by economist David Frederick Schloss in 1891. His work aimed to debunk the misconception that the amount of work is fixed, highlighting the dynamic nature of labor markets and economic growth.
    \\

  

Referential fallacy
    
      \#\#\# Referential Theory of Meaning
    \\

    
      
    \\

    
      - **Name**: Referential Theory of Meaning
    \\

    
      
    \\

    
      - **Also known as**: Direct Reference Theory, Referential Realism
    \\

    
      
    \\

    
      - **Description**: A referential theory of meaning posits that the meaning of a word or expression lies in the real-world object it refers to. For instance, the word "tree" points to an actual tree in the world, making the tree its referent. Critics argue that this view is limited as it assumes all words must refer to real objects and overlooks the nuances of language use and context.
    \\

    
      
    \\

    
      - **Logical Form**:
    \\

    
        1. Premise 1: The meaning of a word is determined by the object it refers to.
    \\

    
        2. Premise 2: Words have meaning only if they point to real-world objects.
    \\

    
        3. Conclusion: The meaning of words resides within the objects they denote.
    \\

    
      
    \\

    
      - **Example \#1**:
    \\

    
        - **Scenario**: The word "Pegasus" refers to a mythical flying horse.
    \\

    
        - **Explanation**: According to the referential theory, "Pegasus" lacks meaning because it does not refer to a real object. However, in practice, "Pegasus" is understood as a concept from mythology, indicating that meaning can exist without a real referent.
    \\

    
      
    \\

    
      - **Example \#2**:
    \\

    
        - **Scenario**: The phrase "nobody was in the room."
    \\

    
        - **Explanation**: The referential theory would struggle with this phrase, as "nobody" does not refer to a specific object. Nevertheless, the phrase is meaningful in indicating an empty room, showing that meaning can arise from usage and context.
    \\

    
      
    \\

    
      - **Variation**: The theory can be contrasted with mediated reference theory, which suggests that meaning is derived through an intermediary sense or description rather than direct reference.
    \\

    
      
    \\

    
      - **Tip**: Understand that while referential theory emphasizes the link between words and their real-world counterparts, it is essential to consider the context and use of language, as words can have meaning even without direct referents.
    \\

    
      
    \\

    
      - **Exception**: The theory falls short in explaining abstract, fictional, or non-existent entities, such as "unicorn" or "justice," which still hold meaning despite lacking a concrete referent.
    \\

    
      
    \\

    
      - **Fun Fact**: Philosopher John Stuart Mill, an early advocate of this theory, distinguished between connotation (attributes implied by a term) and denotation (the actual object referred to), contributing significantly to the development of referential theory.
    \\

    
      
    \\

    
      \#\#\# Notable Philosophers
    \\

    
      
    \\

    
      - **John Stuart Mill**: Introduced the distinction between connotation and denotation, arguing that the meaning of terms involves the attributes they imply and the objects they denote.
    \\

    
      
    \\

    
      - **Bertrand Russell**: Although often associated with mediated reference theory, Russell's views also included elements of direct reference theory, emphasizing the relationship between language and reality.
    \\

    
      
    \\

    
      - **Saul Kripke**: Defended direct reference theory, particularly for proper names, arguing that names are "rigid designators" that refer to the same object in all possible worlds, regardless of the object's properties or context.
    \\

  \section{Appeal to Consequences
    
      (also known as: argumentum ad consequentiam, appeal to consequences of a belief, argument to the consequences, argument from [the] consequences, appeal to convenience [form of], appeal to utility)
    \\

  
    Description: Concluding that an idea or proposition is true or false because the consequences of it being true or false are desirable or undesirable.  The fallacy lies in the fact that the desirability is not related to the truth value of the idea or proposition.  This comes in two forms: the positive and negative. 

    
      Logical Forms: 
    \\

    
      {\em X is true because if people did not accept X as being true then there would be negative consequences.}
    \\

    
      {\em X is false because if people did not accept X as being false, then there would be negative consequences.}
    \\

    
      {\em X is true because accepting that X is true has positive consequences.}
    \\

    
      {\em X is false because accepting that X is false has positive consequences.}
    \\

    
      Example \#1 (positive):
    \\

    
      {\em If there is objective morality, then good moral behavior will be rewarded after death.  I want to be rewarded; therefore, morality must be objective.}
    \\

    
      Example \#2 (negative):
    \\

    
      {\em If there is no objective morality, then all the bad people will not be punished for their bad behavior after death.  I don’t like that; therefore, morality must be objective.}
    \\

    
      Explanation: The fact that one wants to be rewarded, or wants other people to suffer, says nothing to the truth claim of objective morality.  These examples are also {\it begging the question}  that there is life after death.
    \\

    
      Example \#3:
    \\

    
      {\em If there is no freewill, then we are not ultimately in control of our actions. If this is true, our entire system of justice would be seriously flawed. This would be very bad; therefore, freewill must exist.}
    \\

    
      Explanation: The “freewill” argument has been around for thousands of years, and we may never know how free we really are to make decisions. Many philosophers recognize this problem as well as the consequences of not believing in freewill. For this reason, some suggest that we should just accept freewill as being true. While acting as if we have freewill might lead to a better outcome, actually believing freewill to be true because of the consequences is a compromise of one’s rational integrity. \newline

    \\

    
      Exception: If the consequences refer to actions taken or not taken, it would be more of a warning than an argument, thus not fallacious.
    \\

    
      {\em If you continue to reflect the sun in my eyes using your watch, I will take your watch and shove it in a place the sun don’t shine. Therefore, you should really stop doing that.}
    \\

    
      Variation: The {\em appeal to convenience} is accepting an argument because its conclusion is convenient, not necessarily true. This is very similar to the {\em appeal to consequences} except that consequences are only positive and reasonably referred to as “convenient.”
    \\

    
      Tip: Realize that you can deal with reality, no matter what that reality turns out to be.  You don’t need to hide from it—face it and embrace it.
    \\

  }


Appeal to force
    (also known as: argumentum ad baculum, argument to the cudgel, appeal to the stick, Threat of force)
  
    Description:  When force, coercion, or even a threat of force is used in place of a reason in an attempt to justify a conclusion.

    
      Logical Form:
    \\

    
      If you don’t accept X as true, I will hurt you.
    \\

    
      Example \#1:
    \\

    
      Melvin: Boss, why do I have to work weekends when nobody else in the company does?
    \\

    
      Boss: Am I sensing insubordination?  I can find another employee very quickly, thanks to Craigslist, you know.
    \\

    
      Explanation: Melvin has asked a legitimate question to which he did not get a legitimate answer, rather his question was deflected by a threat of force (as being forced out of his job).
    \\

    
      Example \#2:
    \\

    
      Jordan: Dad, why do I have to spend my summer at Jesus camp?
    \\

    
      Dad: Because if you don’t, you will spend your entire summer in your room with nothing but your Bible!
    \\

    
      Explanation: Instead of a reason, dad gave Jordan a description of a punishment that would happen.
    \\

    
      Exception:  If the force, coercion, or threat of force is not being used as a reason but as a fact or consequence, then it would not be fallacious, especially when a legitimate reason is given with the “threat”, direct or implied. 
    \\

    
      Melvin: Boss, why do I have to wear this goofy-looking hardhat?
    \\

    
      Boss: It is state law; therefore, company policy.  No hat, no job.
    \\

    
      Tip: Unless you are an indentured servant (slave) or still living with your parents (slave), do not allow others to force you into accepting something as true.
    \\

  \subsection{Wishful thinking
    Description: When the desire for something to be true is used in place of/or as evidence for the truthfulness of the claim.  Wishful thinking, more as a cognitive bias than a logical fallacy, can also cause one to evaluate evidence very differently based on the desired outcome.

    
      Logical Form:
    \\

    
      I wish X were true.
    \\

    
      Therefore, X is true.
    \\

    
      Example \#1:
    \\

    
      I know in my heart of hearts that our home team will win the World Series.
    \\

    
      Explanation: No, you don’t know that, and what the heck is your “heart of hearts” anyway?  This is classic {\it wishful thinking}  -- wanting the home team to win so pretending that it is/has to be true.
    \\

    
      Example \#2:
    \\

    
      I believe that when we die, we are all given new, young, perfect bodies, and we spend eternity with those whom we love.  I can’t imagine the point of life if it all just ends when we die!
    \\

    
      Explanation: The fact that one doesn’t like the idea of simply not existing is not evidence for the belief.  Besides, nobody seemed to mind the eternity they didn’t exist before they were born.
    \\

    
      Exception: When {\it wishful thinking} is expressed as a hope, wish, or prayer and no belief is formed as a result, then it is not a fallacy because no direct or indirect argument is being made.
    \\

    
      I really hope that I don’t have to spend my eternity with my Aunt Edna, who really loved me, but she drove me nuts with her constant jabbering.
    \\

    
      Tip: Wishing for something to be true is a powerful technique when and only when, a) you have influence on what it is you want to be true and b) you take action to make it come true -- not just wish for it to be true.
    \\

    References:

    
      
        
      \\

      
        
          Andolina, M. (2002). {\it Practical Guide to Critical Thinking}. Cengage Learning.
        
      
    
  }


Magical thinking
    
      (also known as: superstitious thinking)
    \\

  
    
      Description: Making causal connections or correlations between two events not based on logic or evidence, but primarily based on superstition.  Magical thinking often causes one to experience irrational fear of performing certain acts or having certain thoughts because they assume a correlation with their acts and threatening calamities.
    \\

    
      
    \\

    
      Logical Form:
    \\

    
      
    \\

    
      Event A occurs.
    \\

    
      Event B occurs.
    \\

    
      Because of superstition or magic, event A is causally connected to or correlated with event B.
    \\

    
      
    \\

    
      Example \#1:
    \\

    
      
    \\

    
      Mr. Governor issues a proclamation for the people of his state to pray for rain.  Several months later, it rains.  Praise the gods!
    \\

    
      
    \\

    
      Explanation: Suggesting that appealing to the gods for rain via prayer or dance is just the kind of thing crazy enough to get you elected president of the United States, but there is absolutely no logical reason or evidence to support the claim that appealing to the gods will make it rain.
    \\

    
      
    \\

    
      Example \#2:
    \\

    
      
    \\

    
      I refuse to stay on the 13th floor of any hotel because it is bad luck.  However, I don’t mind staying on the same floor as long as we call it the 14th floor.
    \\

    
      
    \\

    
      Explanation: This demonstrates the kind of magical thinking that so many people engage in, that, according to Dilip Rangnekar of Otis Elevators, an estimated 85\% of buildings with elevators did not have a floor numbered “13”.  There is zero evidence that the number 13 has any property that causes bad luck -- of course, it is the superstitious mind that connects that number with bad luck.
    \\

    
      
    \\

    
      Example \#3:
    \\

    
      
    \\

    
      I knew I should have helped that old lady across the road.  Because I didn’t, I have been having bad Karma all day.
    \\

    
      
    \\

    
      Explanation: This describes how one who believes that they deserve bad fortune, will most likely experience it due to the confirmation bias and other self-fulfilling prophecy-like behavior.  Yet there is no logical or rational basis behind the concept of Karma.
    \\

    
      
    \\

    
      Exception: If you can empirically prove your magic, then you can use your magic to reason.
    \\

    
      
    \\

    
      Tip: Magical thinking may be comforting at times, but reality is always what’s true.
    \\

    
      
    \\

  

Argumentum ab utili
    
      \#\#\# Will is More Effective Than Insight
    \\

    
      
    \\

    
      - **Name**: Will is More Effective Than Insight
    \\

    
      
    \\

    
      - **Description**: This technique involves influencing your opponent's will by appealing to their interests and motives, rather than engaging their intellect with logical arguments. It is based on the idea that people are more likely to abandon a viewpoint if they see it as harmful to their interests.
    \\

    
      
    \\

    
      - **Logical Form**:
    \\

    
        1. Premise 1: If a belief is contrary to a person's interests, they are likely to abandon it.
    \\

    
        2. Premise 2: Demonstrate that the opponent's belief is harmful to their interests.
    \\

    
        3. Conclusion: The opponent will abandon their belief.
    \\

    
      
    \\

    
      - **Example \#1**:
    \\

    
        - **Scenario**: A clergyman defends a philosophical dogma.
    \\

    
        - **Explanation**: You point out that the dogma contradicts a fundamental doctrine of his church, making him likely to abandon his defense to avoid conflict with his religious beliefs.
    \\

    
      
    \\

    
      - **Example \#2**:
    \\

    
        - **Scenario**: A landowner supports the use of machinery in agriculture because it increases efficiency.
    \\

    
        - **Explanation**: You suggest that the advancement of steam-powered carriages will depreciate the value of his horses, leading him to reconsider his support for agricultural machinery to protect his financial interests.
    \\

    
      
    \\

    
      - **Variation**: This technique can also be applied to sway bystanders who share common interests with you, even if the primary opponent remains unconvinced.
    \\

    
      
    \\

    
      - **Tip**: Highlight how the opponent’s belief or stance negatively impacts their personal or shared interests to effectively persuade them to change their viewpoint.
    \\

    
      
    \\

    
      - **Exception**: This method is only effective under specific circumstances where the opponent's interests are clearly and directly impacted by their belief.
    \\

    
      
    \\

    
      - **Fun Fact**: The Latin phrase "quam temere in nosmet legem sancimus iniquam" translates to "how thoughtlessly we sanction a law unjust to ourselves," illustrating the self-defeating nature of supporting beliefs contrary to one’s own interests.
    \\

  
    
      (Also known as: Argumentum ab utili, Argument from Utility, Will is More Effective Than Insight)
    \\

  

Unreasonable Inclusion Fallacy
    Description: Attempting to broaden the criteria for inclusion in an ill-famed group or associated with a negative label to the point where the term's definition is changed substantially to condemn or criminalize a far less malicious or deleterious behavior.

    
      Logical Form:
    \\

    
      Person A is accused of bad behavior X.
    \\

    
      Group Y traditionally does not include individuals with bad behavior X
    \\

    
      Person A is said to be a part of group Y for bad behavior X.
    \\

    
      
    \\

    
      Person A is accused of bad behavior X.
    \\

    
      Label Y traditionally does not include bad behavior X
    \\

    
      Person A is given label Y for bad behavior X.
    \\

    
      Example \#1: Tony does not agree that every black person in America should be compensated financially for the history of slavery. Therefore, Tony is a racist.
    \\

    
      Explanation: The term "racist" is an extremely pejorative term that has traditionally been associated with those holding beliefs of racial supremacy—the belief that one's race is superior to other races. By expanding the definition to include "those who don't agree with policies that benefit another race," we fallaciously equate the behavior with the label's or the group's far more malicious roots. Not only does this unreasonably characterize the person who is against the policy, but it also waters down the meaning of "racist" to the point where traditional racists are unreasonably seen more favorably because the term "racism" has been expanded to include a spectrum of far less malicious behaviors.
    \\

    
      Example \#2: Suzie and Patty went on a date. Patty told herself and her friends that she was not going to have sex on the first date, but she did anyway because she was caught up in the moment. The next day, Patty said that Suzie was very seductive, and Patty couldn't resist her advances. Patty's friends convince her that she was raped.
    \\

    
      Explanation: Patty's friends' goal was to show sympathy to Patty. They did this by giving her victim status and criminalizing Suzie's behavior by unreasonably viewing Suzie as a "rapist." Universally, if Suzie's behavior is accepted as "rape," the definition has radically changed, which not only demeans victims of "traditional" rape by associating them with those who were seduced by their date but conflates actual rapists with those who are just seductive.
    \\

    
      Example \#3: Ricardo had a hamburger for lunch. Antonio, Ricardo's vegan friend, argues that he is a murderer.
    \\

    
      Explanation: "Murder" is traditionally defined as the unlawful killing of another human being. A cow isn't a human being, killing a cow isn't unlawful, and Ricardo didn't kill the cow he ate. It is unreasonable to put Ricardo, a guy who at a hamburger, in the same category as Jeffrey Dahmer, a guy who ate his human victims he killed. Antonio is using the term "murderer" in a fallacious attempt to associate a similar level of malice with Ricardo's behavior.
    \\

    
      Exception: See the {\it appeal to definition}. Terms change and evolve as do criteria. What makes this a fallacy is a) the definition is changed substantially and b) for the purpose of condemning or criminalizing a far less malicious or deleterious behavior. 
    \\

    
      Fun Fact: This is not unlike the "everyone gets a trophy" phenomenon where the (implied) definition of "winner" is changed substantially to include all those who lost as well ("you're all winners," says the coach). This may be good for lifting spirits, but it is antithetical to reason.
    \\

  

 Exception That Proves The Rule
    
      - Description: This fallacy occurs when an exception to a rule is dismissed as somehow proving the validity of that rule. Instead of acknowledging that the exception disproves the rule, it is claimed that the exception actually supports or reinforces the rule.
    \\

    
      
    \\

    
      - Logical Form:
    \\

    
        1. Rule R is stated.
    \\

    
        2. An exception E to Rule R is presented.
    \\

    
        3. E is dismissed as the exception that proves R, thereby claiming the rule is still valid.
    \\

    
      
    \\

    
      - Example \#1:
    \\

    
        - Scenario: "You never find songs written about any towns in Britain apart from London."
    \\

    
        - Objection: "What about 'Scarborough Fair'?"
    \\

    
        - Response: "That’s the exception that proves the rule."
    \\

    
        - Explanation: The fallacy lies in dismissing "Scarborough Fair" as an exception that somehow validates the rule that no other British towns are mentioned in songs, rather than acknowledging it as a counter-example that challenges the rule.
    \\

    
      
    \\

    
      - Example \#2:
    \\

    
        - Scenario: "No fictional character ever attracted fan clubs in distant countries like pop stars do."
    \\

    
        - Objection: "Sherlock Holmes has fan clubs worldwide."
    \\

    
        - Response: "He’s simply the exception that proves the rule."
    \\

    
        - Explanation: The fallacy is in claiming that Sherlock Holmes’s global fan clubs support the rule that fictional characters don’t have international fan clubs, ignoring that his existence is a clear counter-example.
    \\

    
      
    \\

    
      - Variation:
    \\

    
        - Scenario: "Medical advances are made through meticulous research, not by chance. Penicillin was a fluke, a rare exception."
    \\

    
        - Explanation: Here, the exception (penicillin) is used to argue that research, rather than chance, is the rule for medical breakthroughs, even though the exception itself might highlight the validity of chance in some cases.
    \\

    
      
    \\

    
      - Tip: When evaluating rules and exceptions, check if the exception genuinely challenges the rule. If it does, the rule might need re-evaluation rather than the exception being dismissed as supportive.
    \\

    
      
    \\

    
      - Exception: The fallacy does not apply if the rule is stated with the understanding that exceptions might occur. In cases where a rule is not universally applied but generally true, exceptions might still be valid and informative.
    \\

    
      
    \\

    
      - Fun Fact: The term "exception that proves the rule" comes from an older use of the word "prove," which meant to "test" or "try out." It does not imply that exceptions confirm the rule but rather that they challenge the rule to prove its robustness.
    \\

  

Refuting the example

argumentum ad Temperantlam
    
      - Description: The argumentum ad temperantiam is the fallacy that asserts the moderate or middle position in a debate is always the correct one. It presumes that taking a moderate stance inherently makes a position more valid or reasonable, regardless of the actual merits of the arguments.
    \\

    
      
    \\

    
      - Logical Form: 
    \\

    
        1. Position A is extreme.
    \\

    
        2. Position B is also extreme.
    \\

    
        3. Therefore, a compromise or middle position (C) is correct.
    \\

    
      
    \\

    
      - Example \#1:
    \\

    
        - Scenario: "The unions have asked for a 6 percent raise, and the management has offered 2 percent. Couldn’t we agree on 4 percent to avoid a strike?"
    \\

    
        - Explanation: The fallacy lies in assuming that the middle point (4 percent) is the correct solution simply because it is a compromise between two extreme positions (6 percent and 2 percent), without considering whether 4 percent is justified or beneficial.
    \\

    
      
    \\

    
      - Example \#2:
    \\

    
        - Scenario: "All bureaucrats are meddlesome, and all petty tyrants are meddlesome. Therefore, all bureaucrats are petty tyrants."
    \\

    
        - Explanation: Here, the fallacy suggests that because both bureaucrats and petty tyrants share a common trait (meddlesome), they must be the same thing. This ignores the fact that the shared trait does not prove an equivalence or justify a middle ground.
    \\

    
      
    \\

    
      - Variation:
    \\

    
        - Scenario: "One side supports full privatization, and the other side wants complete government control. We should implement a balanced approach with some privatization and some government control."
    \\

    
        - Explanation: This argument assumes that a middle-ground approach is inherently correct without evaluating whether the specific balance is effective or appropriate.
    \\

    
      
    \\

    
      - Tip: Be cautious of arguments that assume moderation automatically means correctness. Evaluate the merits of each position on its own rather than assuming that a compromise is the best solution.
    \\

    
      
    \\

    
      - Exception: The fallacy does not apply if the moderate position is derived from valid reasoning and evidence rather than simply being a middle ground. Moderation can be reasonable if it is supported by arguments and facts.
    \\

    
      
    \\

    
      - Fun Fact: The argumentum ad temperantiam is humorously associated with England due to its historical tendency to value moderation and compromise. This is reflected in the British political landscape, where parties often position themselves in the middle to appeal to the largest audience, such as the "Third Way" adopted by New Labour.
    \\

  
    
      (Also known as: The Middle Way Fallacy, The Moderation Fallacy)
    \\

  

The Undistributed middle
    
      (Also known as: Undistributed Term Fallacy, Fallacy of the Undistributed Middle Term)
    \\

  
    
      - Description: The undistributed middle fallacy occurs in a syllogistic argument where the middle term is not distributed in either of the premises, leading to an invalid conclusion. The middle term, which is supposed to link the two premises, fails to cover its entire class, allowing for logical gaps in the argument.
    \\

    
      
    \\

    
      - Logical Form:
    \\

    
        1. All A are B.
    \\

    
        2. Some B are C.
    \\

    
        3. Therefore, some A are C.
    \\

    
      
    \\

    
      - Example \#1:
    \\

    
        - Scenario: "All men are mammals. Some mammals are rabbits, therefore some men are rabbits."
    \\

    
        - Explanation: The middle term "mammals" is not distributed in either premise to cover all mammals. Therefore, the argument incorrectly concludes that some men are rabbits without a proper logical basis.
    \\

    
      
    \\

    
      - Example \#2:
    \\

    
        - Scenario: "All bureaucrats are meddlesome. All petty tyrants are meddlesome. Therefore, all bureaucrats are petty tyrants."
    \\

    
        - Explanation: The middle term "meddlesome" is not distributed to cover all petty tyrants or all bureaucrats. The argument fails because it assumes that the shared characteristic "meddlesome" means that all bureaucrats must be petty tyrants.
    \\

    
      
    \\

    
      - Variation:
    \\

    
        - Scenario: "All nurses are really great people, but some really great people are not properly rewarded. Therefore, some nurses are not properly rewarded."
    \\

    
        - Explanation: Here, "really great people" is not distributed as a universal or negative term, leading to a fallacious conclusion about the reward status of nurses.
    \\

    
      
    \\

    
      - Tip: Ensure that the middle term in a syllogism is properly distributed at least once across the premises. For a valid conclusion, the middle term should cover the entire class it represents.
    \\

    
      
    \\

    
      - Exception: The fallacy of the undistributed middle does not apply if the argument is constructed correctly with the middle term properly distributed and covering the full scope of its class.
    \\

    
      
    \\

    
      - Fun Fact: The undistributed middle fallacy is a classic example of how logical structures can be manipulated to create seemingly convincing but ultimately flawed arguments. The term "undistributed" refers to the failure of the middle term to adequately cover all its possible instances, leading to the fallacy.
    \\

  

fallacia non causae ut causae
    
      (Also known as: Claim Victory Despite Defeat, Fallacy of non causae ut causae, Impudent Triumph)
    \\

  
    
      - **Description**: This technique involves asserting that you have won an argument or proved a point, even when the evidence or answers provided do not support your conclusion. It relies on the bold proclamation of victory to convince others, particularly if the opponent is not confident or astute enough to challenge the claim.
    \\

    
      
    \\

    
      - **Logical Form**:
    \\

    
        1. Premise 1: Your opponent answers several of your questions, none of which support your desired conclusion.
    \\

    
        2. Premise 2: You declare the desired conclusion as if it logically follows from the discussion.
    \\

    
        3. Conclusion: You assert that you have proved your point.
    \\

    
      
    \\

    
      - **Example \#1**:
    \\

    
        - **Scenario**: In a debate, you ask your opponent multiple questions about economic policies.
    \\

    
        - **Explanation**: Although the answers do not favor your stance, you boldly claim that the responses clearly demonstrate your position's superiority and announce your conclusion as if it were a logical outcome of the discussion.
    \\

    
      
    \\

    
      - **Example \#2**:
    \\

    
        - **Scenario**: During a public discussion, you question your opponent about various social issues.
    \\

    
        - **Explanation**: Despite receiving neutral or unrelated answers, you loudly declare that these answers prove your argument, leveraging your assertiveness to make it seem convincing to the audience.
    \\

    
      
    \\

    
      - **Variation**: This technique can also involve misrepresenting or exaggerating the significance of minor points in your favor, presenting them as decisive victories.
    \\

    
      
    \\

    
      - **Tip**: Confidence and assertiveness are key. Speak loudly and with conviction to overshadow any logical inconsistencies and to make your proclamation of victory more believable.
    \\

    
      
    \\

    
      - **Exception**: This tactic is less effective against well-informed, confident opponents who can easily expose the fallacy. It can also backfire if the audience is discerning and aware of the logical disconnect.
    \\

    
      
    \\

    
      - **Fun Fact**: This trick is similar to the fallacy of non causae ut causae, which means "non-cause as cause," where something that is not the cause is incorrectly presented as the cause.
    \\

  

Argumentum ad amicitiam
    
      (Also known as:  Appeal to Friendship)
    \\

  
    
      - **Description**: This argument leverages the relationship or friendship between individuals to influence the acceptance of a claim or proposition. Rather than relying on logical evidence or reasoning, it appeals to the emotional bond and loyalty inherent in a friendship.
    \\

    
      
    \\

    
      - **Logical Form**:
    \\

    
        1. Premise 1: Person A and Person B are friends.
    \\

    
        2. Premise 2: Person A makes a claim.
    \\

    
        3. Conclusion: Person B should accept Person A's claim because they are friends.
    \\

    
      
    \\

    
      - **Example \#1**:
    \\

    
        - **Scenario**: Alice asks Bob to agree with her political opinion because they have been best friends for years.
    \\

    
        - **Explanation**: Alice is not providing any factual evidence to support her political stance but is instead relying on their friendship to convince Bob to agree with her.
    \\

    
      
    \\

    
      - **Example \#2**:
    \\

    
        - **Scenario**: John tells his friend Sarah that she should trust his investment advice because they have known each other since childhood.
    \\

    
        - **Explanation**: John is appealing to their long-standing friendship rather than providing solid financial evidence or reasoning to back up his investment advice.
    \\

    
      
    \\

    
      - **Variation**: This can also appear as asking someone to do a favor or make a decision based on friendship rather than merit or logic. For example, “You should hire me for the job because we’ve been friends since college.”
    \\

    
      
    \\

    
      - **Tip**: To counteract this fallacy, focus on the merits of the argument itself rather than the personal relationship. Ask for evidence and logical reasoning that supports the claim independent of the friendship.
    \\

    
      
    \\

    
      - **Exception**: In some situations, personal relationships and trust can play a legitimate role, such as when seeking personal advice or support where subjective judgment is acceptable.
    \\

    
      
    \\

    
      - **Fun Fact**: The term "Argumentum ad amicitiam" comes from Latin, where "amicitiam" means "friendship." This fallacy is often effective in personal contexts but can be seen as manipulative in formal debates or logical discussions.
    \\

  

Argumentum a simili
    
      (Also known as: Argument by Similarity)
    \\

  
    
      - **Description**: This argument draws a conclusion based on the similarity between two or more situations. It posits that because one situation is like another in some respects, the same conclusions can be applied to both.
    \\

    
      
    \\

    
      - **Logical Form**:
    \\

    
        1. Premise 1: Situation A has characteristics X, Y, and Z.
    \\

    
        2. Premise 2: Situation B has characteristics X, Y, and Z.
    \\

    
        3. Conclusion: What is true for Situation A is also true for Situation B.
    \\

    
      
    \\

    
      - **Example \#1**:
    \\

    
        - **Scenario**: A teacher argues that because a previous class benefited from a new teaching method, the current class will benefit from it as well.
    \\

    
        - **Explanation**: The teacher is assuming that the current class is similar enough to the previous one in relevant aspects (such as learning styles or engagement levels) for the same method to be effective.
    \\

    
      
    \\

    
      - **Example \#2**:
    \\

    
        - **Scenario**: A lawyer argues that because a court ruled in favor of a similar case, the court should rule the same way in the present case.
    \\

    
        - **Explanation**: The lawyer is drawing a parallel between the two cases, suggesting that the legal principles applied previously should apply again due to the similarities.
    \\

    
      
    \\

    
      - **Variation**: Sometimes known as reasoning by analogy, this variation involves drawing parallels between more abstract concepts rather than direct situations. For example, "Just as a gardener must prune a plant for it to grow stronger, a person must face challenges to become resilient."
    \\

    
      
    \\

    
      - **Tip**: To strengthen an argument by similarity, ensure that the situations being compared are similar in all relevant and significant aspects. Be prepared to address potential dissimilarities that could weaken the argument.
    \\

    
      
    \\

    
      - **Exception**: This type of argument can fail if the similarities are superficial or if there are critical differences between the situations that affect the conclusion. Always scrutinize the depth and relevance of the similarities.
    \\

    
      
    \\

    
      - **Fun Fact**: Analogical reasoning, which underlies argumentum a simili, is a fundamental aspect of human cognition and is often used in everyday decision-making as well as scientific reasoning.
    \\

  \section{argumentum a fortiori
    
      (Also known as: Argument from the stronger reason)
    \\

  
    
      \#\#\# Argumentum a fortiori
    \\

    
      
    \\

    
      - **Name**: Argumentum a fortiori
    \\

    
      
    \\

    
      - **Also known as**: Argument from the stronger reason
    \\

    
      
    \\

    
      - **Description**: Argumentum a fortiori is a form of reasoning that asserts if something is true in a stronger or more significant case, it must also be true in a weaker or less significant case. It draws upon existing confidence in a stronger proposition to argue in favor of a second proposition that is even more certain.
    \\

    
      
    \\

    
      - **Logical Form**:
    \\

    
        1. Premise 1: Proposition A is true and is a stronger case.
    \\

    
        2. Premise 2: Proposition B is a weaker case than Proposition A.
    \\

    
        3. Conclusion: Therefore, Proposition B is also true.
    \\

    
      
    \\

    
      - **Example \#1**:
    \\

    
        - **Scenario**: If driving 10 mph over the speed limit is punishable by a fine of \$50, it can be inferred a fortiori that driving 20 mph over the speed limit is also punishable by a fine of at least \$50.
    \\

    
        - **Explanation**: Since the punishment applies to a less severe case (10 mph over), it must logically apply to the more severe case (20 mph over).
    \\

    
      
    \\

    
      - **Example \#2**:
    \\

    
        - **Scenario**: If a person is dead, then one can, with equal or greater certainty, argue a fortiori that the person is not breathing.
    \\

    
        - **Explanation**: Being dead is a stronger condition that implies not breathing, thus confirming the weaker condition (not breathing) through the stronger condition (being dead).
    \\

    
      
    \\

    
      - **Tip**: Ensure that the stronger case truly encompasses the weaker case, making the inference logical and undeniable.
    \\

    
      
    \\

    
      - **Exception**: The argument fails if the stronger and weaker cases do not share a relevant relationship that makes the inference valid. For example, a characteristic true of a larger quantity does not always apply to a smaller quantity if the relationship is not proportional.
    \\

    
      
    \\

    
      - **Fun Fact**: In Jewish law, a fortiori arguments are regularly used under the name "kal va-chomer" (light and heavy), which infers from a lighter case to a heavier case, highlighting the deep historical roots and cross-cultural relevance of this form of reasoning.
    \\

  }


a maiori ad minus
    
      - **Variation**:
    \\

    
        - **A maiore ad minus**: This describes an inference from a greater to a lesser condition. For example, "If a door is big enough for a person two meters high, then a shorter person may also come through."
    \\

  

a minori ad maius
    
      - **Variation**:
    \\

    
        - **A minore ad maius**: This less common variation infers from a lesser to a greater condition. For example, "If a rope can tow a car, it can also tow a truck."
    \\

  

Argumentum ad exemplum
    
      (Also known as: Argument from example)
    \\

  
    
      - **Description**: Argumentum ad exemplum is a rhetorical strategy where an example is used to illustrate and support a point. This form of argumentation uses specific instances to demonstrate the validity of a general claim or principle.
    \\

    
      
    \\

    
      - **Logical Form**:
    \\

    
        1. General Claim: Proposition A is true.
    \\

    
        2. Example: Specific instance B demonstrates Proposition A.
    \\

    
        3. Conclusion: Therefore, Proposition A is supported by the example B.
    \\

    
      
    \\

    
      - **Example \#1**:
    \\

    
        - **Scenario**: To show that exercise improves mental health, one might cite an example of a study where participants who engaged in regular physical activity reported lower levels of stress and anxiety.
    \\

    
        - **Explanation**: The specific instance of the study provides concrete evidence to support the general claim that exercise benefits mental health.
    \\

    
      
    \\

    
      - **Example \#2**:
    \\

    
        - **Scenario**: To argue that renewable energy sources can effectively power a city, one might reference the case of a city that successfully implemented wind and solar power to meet its energy needs.
    \\

    
        - **Explanation**: The example of the city serves as a real-world illustration that strengthens the argument for the viability of renewable energy.
    \\

    
      
    \\

    
      - **Variation**:
    \\

    
        - **Analogical Argument**: Uses examples that draw parallels between similar cases to argue that what is true for one case is likely true for another.
    \\

    
        - **Illustrative Example**: Uses an example purely to clarify a point rather than to prove it.
    \\

    
      
    \\

    
      - **Tip**: Choose examples that are highly relevant and representative of the general claim. Ensure that the examples are accurate and verifiable to maintain the argument’s credibility.
    \\

    
      
    \\

    
      - **Exception**: This argument can be weak if the example used is an outlier or not representative of the general claim. Avoid using exceptional cases that might not generalize well.
    \\

    
      
    \\

    
      - **Fun Fact**: The use of examples is a common teaching technique in various disciplines, as it helps in concretizing abstract concepts and principles, making them easier to understand and remember.
    \\

  \section{Argumentum ad vertiginem
    
      (Also known as: Argument from dizziness, Appeal to dizziness)
    \\

  
    
      - **Description**: Argumentum ad vertiginem is a fallacious rhetorical strategy that seeks to overwhelm or confuse an opponent with complex, intricate, or convoluted arguments, thereby inducing a sense of dizziness or disorientation in order to win the debate. The idea is to make the argument so difficult to follow that the opponent cannot respond effectively.
    \\

    
      
    \\

    
      - **Logical Form**:
    \\

    
        1. Present a complex and convoluted argument (Proposition A).
    \\

    
        2. Opponent becomes confused and disoriented by the complexity of Proposition A.
    \\

    
        3. Claim victory based on the opponent's inability to effectively counter Proposition A.
    \\

    
      
    \\

    
      - **Example \#1**:
    \\

    
        - **Scenario**: During a debate on economic policy, one debater presents a long-winded, highly technical argument filled with jargon and intricate statistical data.
    \\

    
        - **Explanation**: The opponent, unable to follow the convoluted argument and lacking the expertise to dissect it, concedes the point or appears ineffective, allowing the original speaker to claim victory.
    \\

    
      
    \\

    
      - **Example \#2**:
    \\

    
        - **Scenario**: In a discussion about a legal case, a lawyer presents an elaborate series of legal precedents and complex hypothetical scenarios.
    \\

    
        - **Explanation**: The opposing lawyer or judge, overwhelmed by the sheer volume and complexity of the information, may struggle to find a coherent counter-argument, thereby making the original argument seem stronger.
    \\

    
      
    \\

    
      - **Variation**:
    \\

    
        - **Bafflement with Expertise**: Overwhelming the opponent with specialized knowledge or technical details that are beyond their understanding.
    \\

    
        - **Information Overload**: Presenting so much information at once that the opponent cannot process it all, leading to confusion and inability to respond.
    \\

    
      
    \\

    
      - **Tip**: Be aware of the complexity of your own arguments and aim for clarity. If you find yourself becoming overwhelmed by an argument, request simplification or clarification to avoid being manipulated by this fallacy.
    \\

    
      
    \\

    
      - **Exception**: In fields where detailed and complex arguments are necessary and understood by the audience, such as academic or technical discussions among experts, this approach may not be fallacious but rather expected and appropriate.
    \\

    
      
    \\

    
      - **Fun Fact**: The term "vertiginem" comes from the Latin word for dizziness, highlighting the disorienting effect this type of argument aims to produce in its audience or opponent.
    \\

  }


argumentum ad vanitatem
    
      (Also known as: Appeal to vanity)
    \\

  
    
      - **Description**: Argumentum ad vanitatem is a fallacy that seeks to win an argument by appealing to the vanity, pride, or self-esteem of the audience or opponent. This strategy involves flattery or praise to make the audience more receptive to the argument being presented.
    \\

    
      
    \\

    
      - **Logical Form**:
    \\

    
        1. Compliment the audience or opponent (Proposition A).
    \\

    
        2. Present an argument (Proposition B).
    \\

    
        3. Audience or opponent, feeling flattered, is more likely to accept Proposition B.
    \\

    
      
    \\

    
      - **Example \#1**:
    \\

    
        - **Scenario**: A salesperson says, "You clearly have great taste and an eye for quality. That's why I know you'll appreciate this premium product."
    \\

    
        - **Explanation**: The salesperson flatters the customer's sense of taste and quality, making them more inclined to buy the product.
    \\

    
      
    \\

    
      - **Example \#2**:
    \\

    
        - **Scenario**: During a team meeting, a manager says, "You're one of the most dedicated and hardworking members of our team, so I know you'll agree with my proposal to work extra hours this week."
    \\

    
        - **Explanation**: The manager uses flattery to make the employee more likely to agree to work additional hours, appealing to their sense of dedication and hard work.
    \\

    
      
    \\

    
      - **Variation**:
    \\

    
        - **Appeal to Flattery**: Directly complimenting someone's intelligence, skills, or attributes to win them over.
    \\

    
        - **Appeal to Ego**: Highlighting someone's status or achievements to gain their support.
    \\

    
      
    \\

    
      - **Tip**: Recognize when flattery is being used to manipulate your decision-making process. Focus on the merits of the argument rather than how it makes you feel about yourself.
    \\

    
      
    \\

    
      - **Exception**: Compliments or praise that are genuine and relevant to the discussion can enhance rapport and cooperation without necessarily being manipulative.
    \\

    
      
    \\

    
      - **Fun Fact**: The term "vanitatem" is derived from the Latin word for vanity or emptiness, emphasizing the superficial nature of this type of appeal.
    \\

  

argumentum ad superstitionem
    
      \#\#\# Argumentum ad Superstitionem
    \\

    
      
    \\

    
      - **Name**: Argumentum ad Superstitionem
    \\

    
      
    \\

    
      - **Also known as**: Appeal to superstition
    \\

    
      
    \\

    
      - **Description**: Argumentum ad superstitionem is a fallacy where one uses superstitious beliefs or irrational fears to persuade others to accept a conclusion. This fallacy exploits people's tendencies to believe in supernatural causation or omens to validate an argument.
    \\

    
      
    \\

    
      - **Logical Form**:
    \\

    
        1. Propose a superstitious belief (Proposition A).
    \\

    
        2. Connect the belief to a desired conclusion (Proposition B).
    \\

    
        3. Use the fear or respect for the superstition to convince others of Proposition B.
    \\

    
      
    \\

    
      - **Example \#1**:
    \\

    
        - **Scenario**: "You shouldn't walk under that ladder. It's bad luck, and you might fail your exams if you do."
    \\

    
        - **Explanation**: The argument connects the act of walking under a ladder with the superstition of bad luck, leveraging fear of failure to dissuade the behavior.
    \\

    
      
    \\

    
      - **Example \#2**:
    \\

    
        - **Scenario**: "If we don't follow the tradition of throwing salt over our shoulders, we'll bring misfortune to our business."
    \\

    
        - **Explanation**: This argument uses the superstition about salt and bad luck to influence business practices, implying that failure to comply will result in misfortune.
    \\

    
      
    \\

    
      - **Variation**:
    \\

    
        - **Appeal to Omens**: Using signs or omens to justify decisions (e.g., seeing a black cat as a sign to avoid a certain action).
    \\

  

argumentum ad socordiam
    
      (Also known as: Appeal to Laziness)
    \\

  
    
      - **Description**: Argumentum ad socordiam is a fallacy in which one argues for a conclusion based on the avoidance of effort, work, or complexity. This fallacy exploits the tendency to prefer easier, lazier options, even if they are less valid or effective.
    \\

    
      
    \\

    
      - **Logical Form**:
    \\

    
        1. Present a situation requiring effort (Proposition A).
    \\

    
        2. Suggest an easier alternative (Proposition B).
    \\

    
        3. Argue that Proposition B is preferable because it requires less effort.
    \\

    
      
    \\

    
      - **Example \#1**:
    \\

    
        - **Scenario**: "Why bother reading the whole book when you can just read the summary online?"
    \\

    
        - **Explanation**: This argument suggests that reading a summary is better simply because it requires less effort, ignoring the depth and understanding gained from reading the full book.
    \\

    
      
    \\

    
      - **Example \#2**:
    \\

    
        - **Scenario**: "We don't need to implement a new software system; the old one works well enough and changing it would be too much trouble."
    \\

    
        - **Explanation**: Here, the argument favors maintaining the status quo due to the effort involved in change, regardless of the potential benefits of the new system.
    \\

    
      
    \\

    
      - **Variation**:
    \\

    
        - **Appeal to Convenience**: Arguing that a decision is better because it is more convenient, even if it is not the best option.
    \\

    
        - **Appeal to Simplicity**: Preferring simpler solutions purely because they are easier to understand or implement, not because they are more effective.
    \\

    
      
    \\

    
      - **Tip**: Recognize when laziness or convenience is being used to justify decisions and consider whether the easier option truly serves the best interest or just avoids effort.
    \\

    
      
    \\

    
      - **Exception**: There are instances where the simpler or easier option might genuinely be the best choice due to other valid factors such as cost, time, or resource constraints.
    \\

    
      
    \\

    
      - **Fun Fact**: The phrase "lazy reasoning" can be used to describe arguments that rely on the avoidance of effort or complexity, similar to the concept of "lazy thinking," where minimal mental effort is applied.
    \\

  

Argumentum ad quietem
    
      - **Description**: Argumentum ad quietem is a fallacy where an argument is accepted or promoted not based on its merits but because it leads to peace, quiet, or an end to conflict. This fallacy exploits the desire for harmony and avoidance of confrontation, suggesting that agreeing to the argument will result in a more peaceful or agreeable outcome, regardless of its truthfulness or logical soundness.
    \\

    
      
    \\

    
      - **Logical Form**:
    \\

    
        1. Proposition A is controversial or causing conflict.
    \\

    
        2. Accepting Proposition A will lead to peace or an end to conflict.
    \\

    
        3. Therefore, Proposition A should be accepted.
    \\

    
      
    \\

    
      - **Example \#1**:
    \\

    
        - **Scenario**: "Let's just agree that the Earth is flat so we can stop arguing and get along."
    \\

    
        - **Explanation**: This argument suggests accepting the false claim that the Earth is flat solely to end the argument and achieve peace, without regard for the overwhelming evidence that the Earth is round.
    \\

    
      
    \\

    
      - **Example \#2**:
    \\

    
        - **Scenario**: "We should agree with the new policy change even if it's flawed because it will stop the ongoing disputes among the team."
    \\

    
        - **Explanation**: Here, the argument advocates for accepting a potentially flawed policy to end team disputes, rather than addressing the policy's merits or flaws directly.
    \\

    
      
    \\

    
      - **Variation**:
    \\

    
        - **Appeal to Avoidance**: Suggesting that accepting an argument is better because it avoids confrontation or uncomfortable discussions.
    \\

    
        - **Appeal to Conformity**: Promoting an argument because conforming to it will result in social harmony.
    \\

    
      
    \\

    
      - **Tip**: Evaluate arguments based on their evidence and logical reasoning rather than the desire to avoid conflict or maintain peace. Striving for truth and correctness is essential even if it temporarily leads to disagreement.
    \\

    
      
    \\

    
      - **Exception**: In certain contexts, such as mediation or diplomacy, finding common ground to achieve peace might take precedence over debating the absolute truth. However, this should be recognized as a pragmatic choice rather than a logical conclusion about the argument's validity.
    \\

    
      
    \\

    
      - **Fun Fact**: The term "Argumentum ad Quietem" is less commonly discussed in formal logic compared to other fallacies, but it is often encountered in everyday life when people prioritize harmony over rigorous debate or truth-seeking.
    \\

  
    
      (Also known as: Appeal to Peace, Appeal to Tranquility)
    \\

  \chapter{Faulty generalization/Inductive fallacies}


Argument from Hearsay
    
      (also known as: the telephone game, Chinese whispers, anecdotal evidence, anecdotal fallacy/Volvo fallacy [form of])
    \\

  
    Description: Presenting the testimony of a source that is not an eyewitness to the event in question.  It has been conclusively demonstrated that with each passing of information, via analog transmission, the message content is likely to change.  Each small change can and often does lead to many more significant changes, as in the {\it butterfly effect} in{\it  chaos theory}.

    
      Hearsay is generally considered very weak evidence if it is considered evidence at all.  Especially when such evidence is {\it unfalsifiable}  (not able to be proven false).
    \\

    
      Logical Form:
    \\

    
      Person 1 told me that he saw Y.
    \\

    
      Therefore, I must accept that Y is true.
    \\

    
      Example \#1:
    \\

    
      Lolita: Bill stole the money from the company petty cash fund.
    \\

    
      Byron: How do you know?
    \\

    
      Lolita: Because Diane told me.
    \\

    
      Byron: How does she know?
    \\

    
      Lolita: Julian told her.
    \\

    
      Byron: Did anyone actually see Bill steal the money?
    \\

    
      Lolita:  I don’t know, we could ask Morris.
    \\

    
      Byron: Who’s he?
    \\

    
      Lolita: The guy who told Julian.
    \\

    
      Explanation: Lolita is making a bold claim about Bill, based on hearsay.  Not only did Lolita not see Bill steal the money, but neither did Diane, Julian, and who knows about Morris.
    \\

    
      Example \#2:
    \\

    
      There is life after death!  I once heard this story from my friend’s sister, that her maid of honor’s niece knew this guy who had a friend who heard from his camp counselor a story where some guy was in a coma and saw his grandparents in a tunnel of light, and they told him the winning lottery numbers!  I swear to God it’s true!
    \\

    
      Explanation: The validity of the testimony of a coma patient aside, in all likelihood, stories like these are either pure fabrications or exaggerations of some much less interesting story.  Due to something called the {\it confirmation bias} and the {\it wishful thinking}  fallacy, those who already believe in such phenomena are likely to accept such stories as evidence for their truthfulness when, in fact, such stories are not evidence.
    \\

    
      Exception: When you trust the source, and trust that the source is accurately representing the facts, you can at least partially accept the claim, depending on the consequences of accepting or rejecting the claim.  For example, if your best friend told you that her best friend told her about an amazing one day sale at the mall, risking a 10-minute drive to the mall might be justified based on the sources.
    \\

    
      Fun Fact: People are often egregiously wrong in their interpretation of events.  As time passes, imagination is confused with actual events.  You might be able to trust that your best friend is telling you the truth, but only the truth so far as she recalls from her initial interpretation.
    \\

    
      Variation: The {\it anecdotal fallacy}, or V{\it olvo fallacy}, is allowing a specific instance of anecdotal evidence to lend much more weight to an argument than it should.
    \\

  \section{Sampling bias}
\subsection{Cherry picking
    
      (also known as: ignoring inconvenient data, suppressed evidence, fallacy of incomplete evidence, argument by selective observation, argument by half-truth, card stacking, fallacy of exclusion, ignoring the counter evidence, one-sided assessment, slanting, one-sidedness)
    \\

  
    Description: When only select evidence is presented in order to persuade the audience to accept a position, and evidence that would go against the position is withheld.  The stronger the withheld evidence, the more fallacious the argument.

    
      Logical Form:
    \\

    
      Evidence A and evidence B is available.
    \\

    
      Evidence A supports the claim of person 1.
    \\

    
      Evidence B supports the counterclaim of person 2.
    \\

    
      Therefore, person 1 presents only evidence A.
    \\

    
      Example \#1:
    \\

    
      Employer: It says here on your resume that you are a hard worker, you pay attention to detail, and you don’t mind working long hours.
    \\

    
      Andy: Yes sir.
    \\

    
      Employer: I spoke to your previous employer.  He says that you constantly change things that should not be changed, you could care less about other people’s privacy, and you had the lowest score in customer relations.
    \\

    
      Andy: Yes, that is all true, as well.
    \\

    
      Employer: Great then.  Welcome to our social media team!
    \\

    
      Explanation: Resumes are a classic example of {\it cherry picking}  information.  A resume can be seen as an argument as to why you are qualified for the job.  Most employers are wise enough to know that resumes are one-sided and look for more evidence in the form of interviews and recommendations to make a decision.
    \\

    
      Example \#2:
    \\

    
      My political candidate gives 10\% of his income to the needy, goes to church every Sunday, and volunteers one day a week at a homeless shelter.  Therefore, he is honest and morally straight.
    \\

    
      Explanation: What information was left out of the example is that this same candidate gives 10\% of his income to needy prostitutes in exchange for services, goes to the bar every Sunday after church (and sometimes before), and only works at the homeless shelter to get clients for his drug dealing business.
    \\

    
      Exception: If the parts of the truth being suppressed do not affect the truth of the conclusion, or can reasonably be assumed, they could be left out of the argument.  For example, political candidates are not committing this fallacy when they leave out the fact that they will need about 8 hours of sleep each night.
    \\

    
      Tip: If you suspect people are only telling you a half-truth, don’t be afraid to ask, “Is there anything you are not telling me?”
    \\

  }


Survivorship Fallacy
    
      (also known as: survivorship bias, Success Bias, Failure Bias)
    \\

  
    Description: This is best summed up as "dead men don't tell tales." In its general form, the survivorship fallacy is basing a conclusion on a limited number of "winner" testimonies due to the fact we cannot or do not hear the testimonies of the losers. This is based on the cognitive bias called the {\it survivorship bias}.

    
      Logical Form:
    \\

    
      There are X winners and Y losers.
    \\

    
      We only hear the testimonies of the winners.
    \\

    
      Therefore, our conclusion is based on X winners.
    \\

    
      Example \#1:
    \\

    
      Let's use the example of automobile accidents since we have relatively good data on these. According to the National Highway Traffic Safety Administration, roughly 32,000 people die each year on the roads in the United States. The number of people involved in fatal accidents is roughly double that amount, meaning that about half of the people survive, and half die. Let's say that 80\% of people involved in a fatal accident reach out for supernatural help. Given an even distribution among survivors and non survivors this would mean that 25,600 people who reach out for supernatural help die and 25,600 who reach out for supernatural help live to tell about it. Now we need to add in the 20\% of survivors who did not reach out for supernatural help that lived: 6,400. So what we end up with is a group of 32,000 survivors, 80\% of whom appear to have been saved by supernatural intervention. Of course, dead men don't tell tales, so we forget about the 80\% of those who died and reached out for supernatural help and didn't get it. Because of the {\it survivorship bias}, we have a radically biased sample that leads to a fallacious conclusion.
    \\

    
      Example \#2:
    \\

    
      The survivorship bias is used by scammers and con artists who take advantage of the "statistically ignorant" public. One common scam is something I call the "prophetic investor." The scammer will send an e-mail to a very large group of people (say 10 million) with a claim that they have a perfect track record for picking winning investments. But they tell you not to take their word for it, let them prove it you by picking a stock a day for 7 days in a row that increases in value. Then, they say when you are convinced, call them and invest with them. Here's how the scam works:
    \\

    
      Day 1: Five different stocks are chosen, and each stock is sent to 2 million people as the winning pick. Let's say three of those stocks make money, and two don't. The 4 million people who received the stock pick that lost money are removed from the list (kind of like dying). \newline
 \newline
Day 2: Another 5 stocks are chosen. This time, each stock is only sent to the "survivors," about 1.2 million people each get a new pick. Out of that group, perhaps just two stocks are winners. That means that 2.4 million people got the winning stock: two days in a row! You can see where this is headed... \newline
 \newline
... \newline
 \newline
Day 7: After the final day of sending, about 100,000 "survivors" remain on the list. These are people who have been sent the winning picks 7 days in a row and are convinced that the "investor" must be legitimate. After all, what are the chances that anyone would pick 7 winning stocks in a row with that much confidence?
    \\

    
      Tip: Whether someone is selling you investment services or a religion, think about the survivorship bias and how you might be jumping to an inaccurate conclusion.
    \\

    References: Swedroe, L. E. (2005). The Only Guide to a Winning Investment Strategy You’ll Ever Need: The Way Smart Money Preserves Wealth Today. St. Martin’s Press.
  

Casualty fallacy
    
      (also known as: Reverse survivorship)
    \\

  \par \textbf{McNamara Fallacy
    
      (also known as: quantitative fallacy, Skittles fallacy)
    \\

  
    Description: When a decision is based solely on quantitative observations (i.e., metrics, hard data, statistics) and all qualitative factors are ignored.

    
      Logical Form:
    \\

    
      Measure whatever can be easily measured. \newline
Disregard that which cannot be measured easily. \newline
Presume that which cannot be measured easily is not important. \newline
Presume that which cannot be measured easily does not exist.
    \\

    
      Example \#1:
    \\

    
      Donald Trump Jr. Tweeted:
    \\

    
      If I had a bowl of skittles and I told you just three would kill you. Would you take a handful? That's our Syrian refugee problem.
    \\

    
      Explanation: Let's ignore the gross statistical inaccuracy of this quote for a moment (i.e., 1 out of every 100 or so Syrian refugees is not going to kill you). The actual quantitative data about how many Syrian refugees are likely to be terrorists is some number greater than zero. The downside of letting Syrian refugees in the U.S. can be measured quantitatively, perhaps your risk of getting killed by a terrorist will increase from 3.46 billion to one to 3.4 billion to one. The upside, for the most part, is qualitative, that is, cannot be measured easily. What is a human life worth? How do we measure the suffering of others? Since these cannot easily be measured, we ignore them and conclude that taking in Syrian refugees is a bad decision.
    \\

    
      Example \#2:
    \\

    
      The numbers on gun violence speak for themselves. We should ban guns in the country!
    \\

    
      Explanation: While the numbers on gun violence are alarming, we can't ignore the qualitative benefits of gun ownership. When making a decision, all factors need to be considered, even if they cannot be measured quantitatively.
    \\

    
      Exception: It is possible that certain decisions do not have any qualitative components, or the qualitative components are irrelevant. For example, very often salespeople or high-end stores will attempt to sell us overpriced products or services that we can get elsewhere for 1/2 the price. They might justify their higher prices with "service," where nobody needs "service" when buying toilet paper.
    \\

    
      Tip: Qualitative factors are often measured with a degree of subjectivity, meaning that one might give different moral weight to an idea based on one's core values. Consider this before being too harsh in your judgment of others' political or religious views.
    \\

    References:

    
      
        Fischer, D. H. (1970). {\it Historian’s Fallacies}. Harper Collins.
      
    
  }


Managing the news
    
      (Also known as: News Management, Information Control)
    \\

  
    
      - **Description**: Managing the news refers to the deliberate influencing of the presentation and dissemination of information within the news media. This technique is often employed to control the narrative, minimize the impact of negative news, and maintain a favorable public image.
    \\

    
      
    \\

    
      - **Logical Form**:
    \\

    
        1. Negative information (Bad News) exists.
    \\

    
        2. Release the information at a time or in a manner that minimizes its impact.
    \\

    
        3. Control the narrative to maintain a favorable public image.
    \\

    
      
    \\

    
      - **Example \#1**:
    \\

    
        - **Scenario**: A company discovers a product defect that could harm users. To manage the news, the company releases the information late on a Friday evening.
    \\

    
        - **Explanation**: Releasing bad news late on a Friday gives journalists less time to cover the story before the weekend, reducing the potential for widespread negative coverage and public backlash.
    \\

    
      
    \\

    
      - **Example \#2**:
    \\

    
        - **Scenario**: During a political scandal, a government releases a significant report early to selected officials who are likely to support the administration's narrative.
    \\

    
        - **Explanation**: By giving the report early to supportive officials, the government ensures that the initial interpretations and comments in the media are favorable, helping to shape public perception positively before opposing views can be presented.
    \\

    
      
    \\

    
      - **Variation**:
    \\

    
        - **Staying on Message**: Repeatedly focusing on a specific, favorable narrative while avoiding topics that could lead to negative publicity or difficult questions.
    \\

    
        - **Information Overload**: Releasing a large volume of information to bury the bad news within a sea of less significant details, making it harder for journalists and the public to find and focus on the negative aspects.
    \\

    
      
    \\

    
      - **Tip**: Be critical of the timing and context in which news is released. Consider the possibility of deliberate news management tactics when assessing the significance and truthfulness of information presented.
    \\

    
      
    \\

    
      - **Exception**: In crisis management, releasing information in a controlled manner can be necessary to prevent public panic and allow for an organized response. The intent here is not to deceive but to manage the situation effectively.
    \\

    
      
    \\

    
      - **Fun Fact**: The term "news dump" is colloquially used to describe the practice of releasing bad news at a time when it is less likely to attract attention, such as late on a Friday or during a major event.
    \\

  

Nut-picking

Package-Deal Fallacy
    
      (also known as: false conjunction)
    \\

  
    Description: Assuming things that are often grouped together must always be grouped together, or the assumption that the ungrouping will have significantly more severe effects than anticipated.

    
      Logical Form:
    \\

    
      X and Y usually go together.
    \\

    
      Therefore, X or Y cannot be separated.
    \\

    
      Example \#1:
    \\

    
      Michael is part of the Jackson Five.  Without Tito and company, he will never make it.
    \\

    
      Explanation: Michael Jackson was sure great in the {\it Jackson Five}, but as history proves, he was legendary on his own.  Assuming he would not make it on his own is a judgment call not founded on evidence or reason.
    \\

    
      Example \#2:
    \\

    
      If indoor smoking laws are passed for bars, the bars will go out of business since people who drink, smoke while they drink.
    \\

    
      Explanation: This was a common argument against the banning of indoor smoking for bars and other drinking establishments.  The fear of separating smoking and drinking arose from the fear of going out of business, not from statistical data or any other evidence that would normally be deemed reasonable.  Many years later, it appears that the smoking ban had no significant impact on these kinds of establishments.[1]
    \\

    
      Exception: An exception can be made for personal tastes.
    \\

    
      I can’t even imagine eating just a peanut-butter sandwich without jelly (or Fluff).
    \\

    
      Tip: Never underestimate the human ability to adapt and prosper.
    \\

    References:

    
      
        
      \\

      
        This a logical fallacy frequently used on the Internet. No academic sources could be found.
      \\

      
        [1] Mark Engelen, Matthew Farrelly \& Andrew Hyland: The Health and Economic Impact of New York's Clean Indoor Air Act. July 2006, p. 21
      \\

    
  \section{Weak Analogy
    
      (also known as: bad analogy, false analogy, faulty analogy, questionable analogy, argument from spurious similarity, false metaphor, Analogical Fallacy)
    \\

  
    Description: When an analogy is used to prove or disprove an argument, but the analogy is too dissimilar to be effective, that is, it is unlike the argument more than it is like the argument.

    
      Logical Form:
    \\

    
      X is like Y.
    \\

    
      Y has property P.
    \\

    
      Therefore, X has property P.
    \\

    
      (but X really is not too much like Y)
    \\

    
      Example \#1:
    \\

    
      Not believing in the literal resurrection of Jesus because the Bible has errors and contradictions, is like denying that the Titanic sank because eye-witnesses did not agree if the ship broke in half before or after it sank.
    \\

    
      Explanation: This is an actual analogy used by a Christian debater (one who usually appears to value reason and logic).  There are several problems with this analogy, including:
    \\

    \begin{itemize}
  \item 
        The Titanic sank in recent history
      \item 
        We know for a fact that the testimonies we have are of eye-witnesses
      \item 
        We have physical evidence of the sunken Titanic
      
    \end{itemize}
  
    
      Example \#2:
    \\

    
      Believing in the literal resurrection of Jesus is like believing in the literal existence of zombies.
    \\

    
      Explanation: This is a common analogy used by some atheists who argue against Christianity.  It is a {\it weak analogy} because:
    \\

    \begin{itemize}
  \item 
        Jesus was said to be alive not just undead
      \item 
        If God is assumed, then God had a reason to bring Jesus (himself) back—no such reason exists for zombies
      \item 
        Zombies eat brains, Jesus did not (as far as we know)
      
    \end{itemize}
  
    
      Exception: It is important to note that analogies cannot be “faulty” or “correct”, and even calling them “good” or “bad” is not as accurate as referring to them as either “weak” or “strong”.  The use of an analogy is an argument in itself, the strength of which is very subjective.  What is weak to one person, is strong to another.
    \\

    
      Tip: Analogies are very useful, powerful, and persuasive ways to communicate ideas.  Use them -- just make them strong.
    \\

    References:

    
      
        
      \\

      
        
          Luckhardt, C. G., \& Bechtel, W. (1994). {\it How to Do Things with Logic}. Psychology Press.
        
      
    
  }


appeal to the Moon
    
      (also known as: Argumentum ad lunam)
    \\

  
    Description: Using the argument, “If we can put a man on the moon, we could...” as evidence for the argument. This is a specific form of the {\it weak analogy}.

    
      Logical Form:
    \\

    
      If we can put a man on the moon, we can X.
    \\

    
      Example \#1:
    \\

    
      If we can put a man on the moon, we can cure all forms of cancer.
    \\

    
      Explanation: Putting a man on the moon is seen to be a virtually impossible task, but since we did it, the (faulty) reasoning is we can then do any virtually impossible task.  Remember that mere possibility is not the same as probability.  These kinds of arguments are not suggesting the mere possibility, but probability, based on the fact that we succeeded getting a man on the moon.
    \\

    
      Example \#2:
    \\

    
      If NASA can put a man on the moon, you can certainly sleep with me tonight.
    \\

    
      Explanation: This is an even worse analogy. The accomplishments of NASA are independent of our personal accomplishments.
    \\

    
      Exception: If the argument is for getting a man on the moon again, then this would work.
    \\

    
      If we can put a man on the moon in 1969, we can do it today.
    \\

    
      Tip: Believe in the possible just don’t count on it unless it is probable.
    \\

  

Extended Analogy
    Description: Suggesting that because two things are alike in some way and one of those things is like something else, then both things must be like that "something else".

    
      In essence, the {\it reductio ad hitlerum} is an extended analogy because it is the attempt to associate someone with Hitler’s psychotic behavior by way of a usually much more benign connection.
    \\

    
      Logical Form:
    \\

    
      A is like B in some way.
    \\

    
      C is like B in a different way.
    \\

    
      Therefore, A is like C.
    \\

    
      Example \#1:
    \\

    
      Jennie: Anyone who doesn’t have a problem with slaughtering animals for food, should not, in principle, have a problem with an advanced alien race slaughtering us for food.
    \\

    
      Carl: Fruitarians, the crazy people who won’t eat anything except for fruit that fell from the tree, are also against slaughtering animals for food.  Are you crazy like them?
    \\

    
      Explanation: Although I don’t think I can ever give up delicious chicken, Jennie does make a good point via a valid analogy.  Ignoring Carl’s attempt to{\it  poisoning the well }by using {\it argument by emotive language} he is, by{\it  
}, claiming the “craziness” of the fruitarians must be shared by her, as well, since they both are alike because they share a view on using animals for food.
    \\

    
      Example \#2:
    \\

    
      Science often gets things wrong.  It wasn’t until the early 20th century when particle physics came along that scientists realized that the atom wasn’t the smallest particle in existence.  So perhaps science will soon realize that it is wrong about the age of the universe, the non-existence of a global flood, evolution, and every other science fact that contradicts the Bible when read literally.
    \\

    
      Explanation: To see this fallacy, let’s put it in the logical form, using just the evolution claim:
    \\

    
      P1. Thinking the atom was the smallest particle was a mistake of science.
    \\

    
      P2. Evolution is also a mistake of science.
    \\

    
      C. Therefore, science thinking the atom was the smallest particle is like science thinking evolution is true.
    \\

    
      Premise two (P2) should jump out as a bold assumption, although not fallacious.  Remember, the premises don’t have to be true for the argument to be valid, but if both premises were true, does the conclusion (C) follow?  No, because of the {\it extended fallacy}.  The reason is if evolution were false it would not be for the same reason that science thought the atom was the smallest particle.  Science “was wrong” in that case because it did not have access to the whole truth due to discoveries yet to be made at the time.  If evolution were wrong, then all the discoveries that have been made, the facts that have been established, the foundation of many sciences that have led to countless advances in medicine, would all be dead wrong.  This would be a mistake of unimaginable proportions and consequences that would unravel the very core of scientific understanding and inquiry.
    \\

    
      Exception: If one can show evidence that the connection between all the subjects is the same, it is not fallacious.
    \\

    
      It is crazy to think that carrots have feelings.
    \\

    
      It is crazy to think that cows have feelings.
    \\

    
      Therefore, vegetarians are just as crazy as fruitarians.[1]
    \\

    
      Tip: Don’t call people crazy -- leave that kind of psychological assessment for the licensed professionals.  You can call them, “nutjobs”.
    \\

  

Faulty Comparison
    
      (also known as: bad comparison, false comparison, inconsistent, comparison [form of])
    \\

  
    Description: Comparing one thing to another that is really not related, in order to make one thing look more or less desirable than it really is.

    
      Logical Form:
    \\

    
      {\em X is different from Y in way Z.} \newline
{\em It is unreasonable to compare X to Y in way Z.} \newline
{\em Therefore, X is seen as more/less favorable.}
    \\

    
      Example \#1:
    \\

    
      {\em Broccoli has significantly less fat than the leading candy bar!}
    \\

    
      Explanation: While both broccoli and candy bars can be considered snacks, comparing the two in terms of fat content and ignoring the significant difference in taste, leads to the false comparison.
    \\

    
      Example \#2:
    \\

    
      {\em Religion may have been wrong about a few things, but science has been wrong about many more things!}
    \\

    
      Explanation: We are comparing a {\it method of knowledge}  (science) to a {\it system of belief} (faith), which is not known for revising itself based on new evidence.  Even when it does, the “wrongs” are blamed on human interpretation.  Science is all about improving ideas to get closer to the truth, and, in some cases, completely throwing out theories that have been proven wrong.  Furthermore, the claims of religion are virtually all {\it unfalsifiable}, thus cannot be proven wrong.  Therefore, comparing religion and science on the basis of falsifiability is a faulty comparison.
    \\

    
      Exception: One can argue what exactly is “really not related”.
    \\

    
      Variation: An {\em inconsistent comparison} is when something is compared to multiple things in different ways, giving the impression that what is being compared is far better or worse than it actually is. For example,
    \\

    
      {\em Serial killer Ted Bundy wasn’t so bad. He didn’t kill children like Luis Garavito, he didn’t kill nearly as many people as Hitler, and he is much kinder than Satan.}
    \\

    
      Tip: Comparisons of any kind almost always are flawed.  Think carefully before you accept any kind of comparison as evidence.
    \\

    References:

    
      
        
      \\

      
        
          Dowden, B. H. (1993). {\it Logical Reasoning}. Bradley Dowden.
        
      
    
  

Incomplete comparison
    Description: An incomplete assertion that cannot possibly be refuted. This is popular in advertising.

    
      Logical Form:
    \\

    
      X is said to be superior, but to nothing specifically.
    \\

    
      Example \#1:
    \\

    
      One of my favorite candies, {\it Raisinets}, advertises on their package that the product contains 40\% less fat. In fairness, they do have an asterisk then in much smaller writing, "than the leading candy bar."
    \\

    
      Explanation: The question is, "40\% less fat than what?" The hope is that most people won't read the fine print and make their own assumptions. "Oh, this candy bar has 40\% less fat than this apple!"
    \\

    
      Example \#2:
    \\

    
      Our widgets cost less and last longer!
    \\

    
      Explanation: Cost less than what? Last longer than what? By not specifically saying "our competition" they cannot get in trouble when a competitor shows that their product actually costs less and last longer.
    \\

    
      Exception: The terminology used has to be a comparison word or phrase. For example, saying "Bo Rocks!" is great. Not just because I do rock (not musically), but because "rocks" is not a comparison word. There is a complete assertion. Another exception is when the object of comparison is assumed. For example, "Johnny, you need to better in school." Clearly, the implication here is that Johnny needs to {\it improve}, that is, do better than he did in the past.
    \\

    
      Tip: When possible, read the fine print.
    \\

    References:

    
      
        
      \\

      
        
          Incomplete Comparisons. (n.d.). Retrieved from http://www.mhhe.com/mayfieldpub/tsw/comp-i.htm
        
      
    
  

Appeal to Extremes
    Description: Erroneously attempting to make a reasonable argument into an absurd one, by taking the argument to the extremes. Note that this is not a valid {\it reductio ad absurdum}.

    
      Logical Form:
    \\

    
      If X is true, then Y must also be true (where Y is the extreme of X).
    \\

    
      Example \#1:
    \\

    
      There is no way those Girl Scouts could have sold all those cases of cookies in one hour.  If they did, they would have to make \$500 in one hour, which, based on an 8 hour day is over a million dollars a year.  That is more than most lawyers, doctors, and successful business people make!
    \\

    
      Explanation: The Girl Scouts worked just for one hour -- not 40 per week for a year.  Suggesting the extreme leads to an absurd conclusion; that Girl Scouts are among the highest paid people in the world.   Not to mention, there is a whole troop of them doing the work, not just one girl.
    \\

    
      Example \#2:
    \\

    
      Don’t forget God’s commandment, “thou shall not kill”.  By using mouthwash, you are killing 99.9\% of the germs that cause bad breath.  Prepare for Hell.
    \\

    
      Explanation: It is unlikely that God had mouthwash on his mind when issuing that commandment, but if he did, we’re all screwed (at least those of us with fresh breath).
    \\

    
      Exception: This fallacy is a misuse of one of the greatest techniques in argumentation, {\it reductio ad absurdum}, or reducing the argument to the absurd.  The difference is where the absurdity actually is in the argument or in the reasoning of the one trying to show the argument is absurd.
    \\

    
      Here is an example of an argument that is proven false by reducing to the absurd, legitimately.
    \\

    
      Big Tony: The more you exercise, the stronger you will get!
    \\

    
      Nerdy Ned: Actually, if you just kept exercising and never stopped, you would drop dead. There is a limit to how much exercise you should get. At some point, the exercise becomes excessive and causes more harm than good.
    \\

    
      Tip: People very often say stupid things.  Sometimes it is easy to reduce their arguments to absurdity, but remember, in most cases, your goal should be diplomacy, not making the other person look foolish.  Especially when dealing with your spouse—unless you really like sleeping on the couch.
    \\

  

appeal to coincidence
    
      (also known as: Slothful induction, appeal to luck, appeal to bad luck, appeal to good luck)
    \\

  
    Description: Concluding that a result is due to chance when the evidence strongly suggests otherwise.  The {\it appeal to luck}  variation uses luck in place of coincidence or chance.

    
      Logical Form:
    \\

    
      Evidence suggests that X is the result of Y.
    \\

    
      Yet one insists that X is the result of chance.
    \\

    
      Example \#1: 
    \\

    
      Bill: Steve, I am sorry to say, but you are a horrible driver!
    \\

    
      Steve: Why do you say that?
    \\

    
      Bill: This is your fourteenth accident this year.
    \\

    
      Steve: It’s just been an unlucky year for me.
    \\

    
      Explanation: Based on statistical norms, it is very clear that anyone getting into fourteen accidents in a single year has a safety issue as a driver.  Ignoring this obvious fact and writing it off as “bad luck”, is seen as the {\it appeal to coincidence}.
    \\

    
      Example \#2: 
    \\

    
      Mom: This is the eighth time you have been sent to the principal's office this year.  The principal tells me she has seen you more times in her office than any other student.  Why is this?
    \\

    
      Dwight: A teacher just happens to sneak up on me whenever I am doing something against the rules, which is no more often than any other student.
    \\

    
      Explanation: Dwight is a trouble-maker -- that is quite clear.  Rather than face the facts, he is {\it appealing to} {\it coincidence }by suggesting he just gets caught more often due to bad timing.
    \\

    
      Exception: Coincidences do happen.  When the evidence points in the direction of coincidence, the coincidence might be the best option.
    \\

    
      Tip: Remember that million-to-one thing happened to you? It was probably a coincidence. Given how many things happen to us every day, all of us should experience many million-to-one events throughout our lifetimes.
    \\

  

overwhelming exception
    Description: A generalization that is technically accurate, but has one or more qualifications which eliminate so many cases that the resulting argument is significantly weaker than the arguer implies.  In many cases, the listed exceptions are given in place of evidence or support for the claim, not in addition to evidence or support for the claim.

    
      Logical Form:
    \\

    
      Claim A is made.
    \\

    
      Numerous exceptions to claim A are made.
    \\

    
      Therefore, claim A is true.
    \\

    
      Example \#1:
    \\

    
      Besides charities, comfort, community cohesion, rehabilitation, and helping children learn values, religion poisons everything.
    \\

    
      Explanation: Besides being a {\it self-refuting statement}, the listing of the ways religion does not poison everything, is a clear indicator that the claim is false, or at best, very weak.
    \\

    
      Example \#2:
    \\

    
      Our country is certainly in terrible shape.  Sure, we still have all kinds of freedoms, cultural diversity, emergency rooms and trauma care, agencies like the FDA out to protect us, the entertainment industry, a free market, national parks, we are considered the most powerful nation in the world, have amazing opportunities, and free public education, but still...
    \\

    
      Explanation: We have many reasons supporting the opposite claim -- that this country is in great shape still, or at least that it is not in {\it terrible}  shape.  By the time all the reasons are listed, the original claim of our country being in terrible shape is a lot less agreeable.
    \\

    
      Exception: The fewer exceptions, the less overwhelming, the less likely the fallacy.
    \\

    
      Fun Fact: This fallacy is usually made by people who are well on their way to seeing the reasonable conclusion in the argument, but can’t quite let go of their unreasonable belief or claim.
    \\

    References:

    
      
        
      \\

      
        
          Lindgren, J. (1983). More Blackmail Ink: A Critique of Blackmail, Inc., Epstein’s Theory of Blackmail. {\it Connecticut Law Review}, {\it 16}, 909.
        
      
    
  \section{Hasty generalization
    (also known as: argument from small numbers, statistics of small numbers, insufficient statistics, argument by generalization, faulty generalization, hasty induction, inductive generalization, insufficient sample, lonely fact fallacy, over generality, overgeneralization, unrepresentative sample)
  
    Description: Drawing a conclusion based on a small sample size, rather than looking at statistics that are much more in line with the typical or average situation.

    
      Logical Form:
    \\

    
      Sample S is taken from population P.
    \\

    
      Sample S is a very small part of population P.
    \\

    
      Conclusion C is drawn from sample S and applied to population P.
    \\

    
      Example \#1:
    \\

    
      My father smoked four packs of cigarettes a day since age fourteen and lived until age sixty-nine.  Therefore, smoking really can’t be that bad for you.
    \\

    
      Explanation: It is extremely unreasonable (and dangerous) to draw a universal conclusion about the health risks of smoking by the case study of one man.
    \\

    
      Example \#2:
    \\

    
      Four out of five dentists recommend Happy Glossy Smiley toothpaste brand.  Therefore, it must be great.
    \\

    
      Explanation: It turns out that only five dentists were actually asked.  When a random sampling of 1000 dentists was polled, only 20\% actually recommended the brand.  The four out of five result was not necessarily a {\it biased sample} or a dishonest survey; it just happened to be a statistical anomaly common among small samples.
    \\

    
      Exception: When statistics of a larger population are not available, and a decision must be made or opinion formed if the small sample size is all you have to work with, then it is better than nothing.  For example, if you are strolling in the desert with a friend, and he goes to pet a cute snake, gets bitten, then dies instantly, it would not be fallacious to assume the snake is poisonous.
    \\

    
      Tip: Don’t base decisions on small sample sizes when much more reliable data exists.
    \\

    References:

    
      
        
      \\

      
        
          Hurley, P. J. (2011). {\it A Concise Introduction to Logic}. Cengage Learning.
        
      
    
  }


Spotlight fallacy
    
      (also known as: selection bias)
    \\

  
    Description: Assuming that the media’s coverage of a certain class or category is representative of the class or category in whole.

    
      Logical Form:
    \\

    
      The media have been covering X quite a bit by describing it as Y.
    \\

    
      Therefore, X can be described as Y.
    \\

    
      Example \#1:
    \\

    
      It seems like we are constantly hearing about crimes committed on our streets.  America is a very dangerous place.
    \\

    
      Explanation: The media reports on stories of interest, which include crimes.  It does not report on all the non-crimes.  Assuming from this, “American is a very dangerous place” is fallacious reasoning.
    \\

    
      Example \#2:
    \\

    
      I am seeing more and more miracles being reported on respectable news programs.  The other day there was a story about a guy who had trouble walking, prayed to the recently deceased Pope, now walks just fine!  Miracles are all around us!
    \\

    
      Explanation: People love stories of hope and miracles.  You won’t find stories about how someone prayed to be healed then died.  These are not the kind of stories that attract viewers and sell papers.  As a result, the spotlight fallacy makes us think the rare cases, almost certainly due to normal and necessary statistical fluctuations, seem like the norm.  Believing that they are, is fallacious reasoning.
    \\

    
      Exception: Complete coverage of a small, manageable class, by an unbiased media outlet, may accurately be representative of the entire class.
    \\

    
      Tip: Be very selective of the types of “news” programs you watch.
    \\

    References:

    
      
        
      \\

      
        
          Tanner, K. (2013). Common Nonsense Based on Faulty Appeals. In {\it Common Sense} (pp. 31–43). Apress. https://doi.org/10.1007/978-1-4302-4153-9\_3
        
      
    
  \subsection{Jumping to Conclusions
    
      (also known as: hasty conclusion, hasty decision, leaping to conclusions, specificity)
    \\

  
    Description: Drawing a conclusion without taking the needed time to evaluate the evidence or reason through the argument.

    
      Logical Form:
    \\

    
      {\em Little or no evidence is provided or reviewed.} \newline
{\em Conclusion is made.}
    \\

    
      Example \#1:
    \\

    
      {\em Wife: Should we buy the house? \newline
Husband: The Realtor didn’t say anything about any problems, so I am sure it is fine. Let’s get it!}
    \\

    
      Explanation: The husband is jumping to the conclusion that the house is without problems simply because the person who gets paid to sell the house did not mention any. This is fallacious reasoning.
    \\

    
      Example \#2:
    \\

    
      {\em It’s getting late, and we still have to decide on the school budget. What do you say we just leave it as is and we can call it a night?}
    \\

    
      Explanation: It is not reasonable to assume the conclusion that the budget should be left where it is based on the desire to go home.
    \\

    
      Exception: There are many times when quick decisions are required, and evidence cannot be fully examined, and in such circumstances, we need to come to the best conclusion we can with the resources we have.
    \\

    
      Tip: If anyone gives you an unreasonable timeframe for making a decision, it is almost always an attempt to discourage you from critical thought.  If you cannot have what you feel is a reasonable amount of time to come to a well-reasoned conclusion -- walk away.
    \\

  }


Mind reading

Fortune telling

Labeling
    
      - **Description**: In the context of argumentation and debate, labeling is used as a "red herring" to divert attention from the actual argument. This tactic involves dismissing an argument or debater by associating them with an emotionally charged label, often negatively. It constitutes an informal fallacy by discrediting or accrediting a position without valid reasoning.
    \\

    
      
    \\

    
      - **Logical Form**:
    \\

    
        1. Person A makes claim X.
    \\

    
        2. Person B labels Person A or claim X with a charged term Y.
    \\

    
        3. Therefore, claim X is dismissed or accepted based on term Y rather than its merit.
    \\

    
      
    \\

    
      - **Example \#1**:
    \\

    
        - **Scenario**: "Jones believes that it can be done with the right technology."
    \\

    
        - **Labeling**: "Jones is a deluded fool."
    \\

    
        - **Explanation**: The argument is dismissed by attacking Jones personally, labeling him as deluded, instead of addressing the feasibility of the technology.
    \\

    
      
    \\

    
      - **Example \#2**:
    \\

    
        - **Scenario**: "The witness claims to have seen something that indicates foul play."
    \\

    
        - **Labeling**: "It is paranoid to assume foul play."
    \\

    
        - **Explanation**: The witness's observation is dismissed by labeling the assumption of foul play as paranoid, diverting attention from the potential validity of the claim.
    \\

    
      
    \\

    
      - **Variation**:
    \\

    
        - **Positive Labeling**: Attempting to gain sympathy or credibility by associating a debater or position with positive labels, such as authoritative or noble.
    \\

    
          - Example: "It has been suggested that something in the book could be wrong."
    \\

    
          - Positive Labeling: "It is a very noble book."
    \\

    
          - Explanation: The suggestion of error is dismissed by emphasizing the book's nobility.
    \\

    
      
    \\

    
      - **Tip**: Focus on the actual argument and evidence presented rather than the labels or personal attacks used. Evaluate claims based on their merits and logical coherence.
    \\

    
      
    \\

    
      - **Exception**: In some cases, highlighting a debater's qualifications or expertise can be relevant to assessing their credibility. However, it should not replace the evaluation of the argument itself.
    \\

    
      
    \\

    
      - **Fun Fact**: The term "red herring" originates from the practice of dragging a smoked herring across the ground to distract hunting dogs from the scent they were following, illustrating how labeling can distract from the core argument.
    \\

  

Ex-post-facto Statistics
    
      - Description: The fallacy of ex-post-facto statistics occurs when probability laws or statistical reasoning are applied to past events to suggest that the outcome was influenced by some special or supernatural force. This fallacy involves using the improbability of past events to argue for extraordinary explanations, ignoring that such events are part of a range of possibilities where any one could have occurred.
    \\

    
      
    \\

    
      - Logical Form:
    \\

    
        1. Present a past event as having occurred with low probability.
    \\

    
        2. Conclude that this low probability implies an extraordinary or supernatural explanation.
    \\

    
        3. Ignore the fact that all outcomes in a given probabilistic range have low probabilities and that one must occur.
    \\

    
      
    \\

    
      - Example \#1:
    \\

    
        - Scenario: "I drew the ace of spades. It was only a 1 in 52 chance, but it came up."
    \\

    
        - Explanation: Drawing any card from a deck of 52 cards has a 1 in 52 chance. The low probability of drawing the ace of spades does not imply anything extraordinary; it's simply a matter of probability.
    \\

    
      
    \\

    
      - Example \#2:
    \\

    
        - Scenario: "I met my aunt in Trafalgar Square on Wednesday. Think of the hundreds of thousands going through the square that day, and you’ll realize how unlikely it was that we should meet there. Maybe we are telepathic."
    \\

    
        - Explanation: While the probability of meeting a specific person in a large crowd is low, this is a normal statistical occurrence, not evidence of telepathy. The same low probability applies to many other random encounters.
    \\

    
      
    \\

    
      - Variation:
    \\

    
        - Scenario: "How lucky we are that our planet has just the right temperature range for us, and just the right atmosphere for us to breathe. It has to be more than luck."
    \\

    
        - Explanation: This argument uses the improbability of Earth’s conditions as evidence for extraordinary luck or design. However, if conditions were different, other forms of life would have evolved elsewhere in the universe, with their own claims of improbability.
    \\

    
      
    \\

    
      - Tip: When analyzing past events, recognize that probability laws apply to future possibilities and do not imply special or supernatural influences. Focus on the statistical likelihood within the context of the range of outcomes.
    \\

    
      
    \\

    
      - Exception: The fallacy may not apply if the statistical improbability is used to argue for a known, specific mechanism or cause that is demonstrated to have influenced the outcome. For instance, if a rare event had a known and specific cause, using probability to explain it might not be fallacious.
    \\

    
      
    \\

    
      - Fun Fact: The fallacy of ex-post-facto statistics often appears in speculative discussions about the origins of life and the universe. People might cite incredibly low probabilities of specific conditions or events as evidence for divine intervention or destiny, while ignoring that similar improbabilities would be cited in alternative scenarios.
    \\

  
    
      (Also known as: Retroactive Probability Fallacy, After-the-Fact Statistics)
    \\

  

Self–selection

Double counting (fallacy)

Misleading vividness
    Description: A small number of dramatic and vivid events are taken to outweigh a significant amount of statistical evidence.

    
      Logical Form:
    \\

    
      Dramatic or vivid event X occurs (does not jibe with the majority of the statistical evidence).
    \\

    
      Therefore, events of type X are likely to occur.
    \\

    
      Example \#1:
    \\

    
      In Detroit, there is a 10-year-old living on the street selling drugs to stay alive.  In Los Angeles, a 19-year-old prostitute works the streets.  America’s youth is certainly in serious trouble.
    \\

    
      Explanation: While the stories of the 10-year-old illegal pharmacist and the 19-year-old village bicycle are certainly disturbing, they are just two specific cases out of tens of millions -- a vast majority of youth live pretty regular lives, far from being considered in any “serious trouble”.  This is a form of {\it appeal to emotion} that causes us to hold irrational beliefs about a population due to a few select cases.  The example could have featured two other youths:
    \\

    
      In Detroit, there is a 10-year-old who plays the piano as beautifully as Beethoven.  In Los Angeles, a 19-year-old genius is getting her PhD in nuclear physics.  America’s youth is certainly something of which we can be proud.
    \\

    
      Example \#2:
    \\

    
      It was freezing today as it was yesterday.  My plants are now dead, and my birdbath turned to solid ice...and it is only October!  This global warming thing is a load of crap.
    \\

    
      Explanation: Whether global warming is a “load of crap” or not, concluding that, by a couple of unusually cold days, is fallacious reasoning at its finest.
    \\

    
      Exception: If the cases featured are typical of the population in general, then no fallacy is committed.
    \\

    
      Tip: Don’t let your pessimism or optimism cloud your judgments on reality.
    \\

    References:

    
      
        
      \\

      
        
          Nisbett, R. E., \& Ross, L. (1980). {\it Human inference: strategies and shortcomings of social judgment}. Prentice-Hall.
        
      
    
  

Thought-terminating cliché
    
      (Also known as: Semantic Stop-Sign, Thought-Stopper, Bumper Sticker Logic, Cliché Thinking)
    \\

  
    
      - **Description**: A thought-terminating cliché is a form of loaded language used to end an argument and quell cognitive dissonance by stopping further discussion. Often passing as folk wisdom, these clichés are designed to dismiss dissent or justify fallacious logic without addressing the argument itself.
    \\

    
      
    \\

    
      - **Logical Form**:
    \\

    
        1. Person A makes claim X.
    \\

    
        2. Person B responds with thought-terminating cliché Y.
    \\

    
        3. Therefore, claim X is dismissed or the debate is ended without further discussion.
    \\

    
      
    \\

    
      - **Example \#1**:
    \\

    
        - **Scenario**: "Jones believes that it can be done with the right technology."
    \\

    
        - **Cliché**: "Jones is a deluded fool."
    \\

    
        - **Explanation**: The argument is dismissed by attacking Jones personally rather than addressing the feasibility of the technology.
    \\

    
      
    \\

    
      - **Example \#2**:
    \\

    
        - **Scenario**: "The witness claims to have seen something that indicates foul play."
    \\

    
        - **Cliché**: "It is paranoid to assume foul play."
    \\

    
        - **Explanation**: The witness's observation is dismissed by labeling the assumption of foul play as paranoid, diverting attention from the potential validity of the claim.
    \\

    
      
    \\

    
      - **Variation**:
    \\

    
        - **Positive Labeling**: Attempting to gain sympathy or credibility by associating a debater or position with positive labels.
    \\

    
          - Example: "It has been suggested that something in the book could be wrong."
    \\

    
          - Positive Labeling: "It is a very noble book."
    \\

    
          - Explanation: The suggestion of error is dismissed by emphasizing the book's nobility.
    \\

    
      
    \\

    
      - **Tip**: Recognize thought-terminating clichés by their tendency to end discussion abruptly. Challenge these clichés by asking for evidence or a more detailed explanation.
    \\

    
      
    \\

    
      - **Exception**: Phrases that are backed by evidence or strong claims may not function as thought-terminating clichés.
    \\

    
      
    \\

    
      - **Fun Fact**: The term was popularized by Robert Jay Lifton in his 1961 book "Thought Reform and the Psychology of Totalism," where he described it as "the language of non-thought."
    \\

  

Appeal to probability
    
      (also known as: Murphy's law, Appeal to Possibility)
    \\

  
    Description: When a conclusion is assumed not because it is probably true or it has not been demonstrated to be impossible, but because it is {\it possible}  that it is true, no matter how improbable.

    
      Logical Forms:
    \\

    
      X is possible.
    \\

    
      Therefore, X is true.
    \\

    
       \newline

       \newline

      
    \\

    
      X is possible.
    \\

    
      Therefore, X is probably true.
    \\

    
      Example \#1:
    \\

    
      Brittany: I haven’t applied to any other schools besides Harvard.
    \\

    
      Casey: You think that is a good idea?  After all, you only have a 2.0 GPA, your SAT scores were pretty bad, and frankly, most people think you are not playing with a full deck.
    \\

    
      Brittany: Are you telling me that it is impossible for me to get in?
    \\

    
      Casey: Not *impossible*, but...
    \\

    
      Brittany: Then shut your trap.
    \\

    
      Explanation: Yes, it is possible that Harvard will accept Brittany to fill some sympathy quota, or perhaps someone at admissions will mix Brittany up with “Britney”, the 16-year-old Asian with the 4.0 average who also discovered a vaccine for a rare flu in her spare time, but because Brittany is {\it appealing to possibility}, she is committing this fallacy.
    \\

    
      Example \#2:
    \\

    
      Dave: Did you know that Jesus liked to dress up as a woman and sing show tunes?
    \\

    
      Tim: And why do you say that?
    \\

    
      Dave: You have to admit, it is possible!
    \\

    
      Tim: So is the fact that you are a moron.
    \\

    
      Explanation: We cannot assume Jesus liked to dress like a woman while belting out 2000-year-old show tunes based on the possibility alone. This also includes the {\it argument from ignorance} fallacy -- concluding a possibility based on missing information (an outright statement that Jesus did not do these things).
    \\

    
      Exception:  There are no exceptions. Possibility alone never justifies probability.
    \\

    
      Tip: Catch yourself every time you are about to use the word “impossible”.  Yes, there are many things that are logically and physically impossible, and it is a valid concept and word, but so often we use that word when we really mean  “improbable”.  Confusing the impossible with the improbable or unlikely, could, in many cases, destroy the possibility of great success.
    \\

  

Availability heuristic
    
      (also known as: Deceptive Sharing)
    \\

  
    Description: Sharing an article, post, or meme on social media with the intent to influence public perception to perceive a statistically rare event as a common event. The cognitive bias behind this fallacy is the {\em availability heuristic} that causes us to have a skewed perception of reality based on specific examples that easily come to mind.

    
      Like most fallacies, we have the fallacious tactic by the one who shares an article, post, or meme on social media depicting a statistically rare event with the intent to manipulate public perception of the event as a common event. Someone who views an article, post, or meme on social media depicting a statistically rare event and believes it to be a common event, has been a victim of {\em deceptive sharing} and is using fallacious reasoning. It is also possible for those sharing content to be deceived, themselves.
    \\

    
      There is a strong, implied argument with the share that the event is far more common than it actually is. If the content is shared with an explicit argument claiming that the event is more common than it really is, then there is no fallacy; it is a factual error. What makes this fallacious is the implied argument that remains unsaid but clearly implied.
    \\

    
      Logical Forms:
    \\

    
      {\em [fallacious tactic]} \newline
{\em Event X is statistically rare.} \newline
{\em Person 1 shares article, post, or meme of an instance of event X with the intent to deceive others in thinking event X is far more common than it is.}
    \\

    
      {\em [fallacious reasoning]} \newline
{\em Event X is statistically rare.} \newline
{\em Person 1 shares article, post, or meme of an instance of event X.} \newline
{\em Person 2 believes the event is far more common than it is.}
    \\

    
      {\em Event X is statistically rare.} \newline
{\em Person 1 shares article, post, or meme of an instance of event X, believing that event X is far more common than it is.}
    \\

    
      Example \#1: During the COVID-19 pandemic, there was much debate on reopening schools. Like with many issues, there was a strong partisan split where liberals were mostly against sending kids back to school, and conservatives were mostly for sending kids back to school. Several of my liberal friends commonly shared stories about kids getting sick, including this story posted with no comment nor argument:
    \\

    
      {\em Health officials in Georgia confirmed this week that a 7-year-old boy with no underlying health conditions has died from the coronavirus, the youngest fatality from the pandemic in the state so far.}
    \\

    
      The strategy is to let those who comment make the arguments that the one posting is not willing to make nor defend, and the one who shared the article can remain blameless.
    \\

    
      Children, especially healthy children with no underlying health conditions, dying from COVID-19 is extremely rare. Sharing stories of this happening gives people the impression that it is far more common than it actually is, causing them to make important decisions based on a poor understanding of the data.
    \\

    
      Example \#2: Some people who were adamantly against wearing masks shared the story published by the {\em New York Post} on May 6th, 2020, with the headline “Two Boys Drop Dead in China While Wearing Masks During Gym Class.” Ignoring the problem with causation (i.e., it has not been determined that the masks had anything to do with the deaths), the purpose for sharing the article is very clear: To attempt to vindicate the sharer’s anti-mask-wearing position. The fact is, recorded instances of people dying from wearing masks are exceptionally rare.
    \\

    
      Exception: It may be the case that statistically rare events are also “feel good” events that have no manipulative intent, for example, sharing a story about a local child who saved his family from a gas leak. There is no implication here that kids saving their families from gas leaks is a common occurrence.
    \\

    
      A more questionable exception is the statistically rare, “feel good” event that may be shared with the intent to manipulate. An example is FOX News, with their regular posts of “hero cops” who save babies, help old ladies across the street, or save the world from intergalactic invaders. The manipulative intent is debatable. While the public does love these stories, they also counter the police brutality stories commonly posted by mainstream media. 
    \\

    
      Tip: Remember the {\em principle of charity}. If you are not confident that the sharer is implying that the event they are sharing is far more common than reality dictates, ask, “what is the reason you are sharing this?”
    \\

  

Blind men and an elephant
    
      \#\#\# Blind Men and an Elephant
    \\

    
      
    \\

    
      - **Name**: Blind Men and an Elephant
    \\

    
      
    \\

    
      - **Also known as**: The Blind Men and the Elephant
    \\

    
      
    \\

    
      - **Description**: This is a parable that illustrates how different people can have different perspectives on the same reality, often leading to incomplete or contradictory conclusions. Each person describes only part of the elephant, not realizing it's part of a larger whole.
    \\

    
      
    \\

    
      - **Logical Form**:
    \\

    
        1. Different individuals observe a complex situation.
    \\

    
        2. Each individual provides a limited perspective based on their own experience.
    \\

    
        3. The collective perspectives are incomplete or contradictory but together form a complete understanding.
    \\

    
      
    \\

    
      - **Example \#1**:
    \\

    
        - **Scenario**: Several people are asked to describe a large object in a dark room.
    \\

    
        - **Perspective 1**: One person touches the trunk and says, "It's like a snake."
    \\

    
        - **Perspective 2**: Another person touches the leg and says, "It's like a tree trunk."
    \\

    
        - **Explanation**: Each person describes the elephant based on the part they touch, leading to different and incomplete descriptions.
    \\

    
      
    \\

    
      - **Example \#2**:
    \\

    
        - **Scenario**: Different departments in a company are asked to evaluate a new software.
    \\

    
        - **Perspective 1**: The IT department says, "It's complex and requires a lot of resources."
    \\

    
        - **Perspective 2**: The user department says, "It's user-friendly and increases productivity."
    \\

    
        - **Explanation**: Each department focuses on their specific interaction with the software, leading to differing views that together provide a fuller picture of the software's impact.
    \\

    
      
    \\

    
      - **Variation**:
    \\

    
        - **Cultural Interpretations**: Different cultures may interpret the parable in various ways, emphasizing different morals or lessons.
    \\

    
          - Example: Some may use it to teach the importance of collaboration, while others may focus on the limitations of individual perception.
    \\

    
      
    \\

    
      - **Tip**: When encountering different perspectives, consider that each may hold part of the truth. Strive to integrate these perspectives to gain a comprehensive understanding.
    \\

    
      
    \\

    
      - **Exception**: If all observers are blind to the same crucial aspects or there is significant misinformation, their combined perspectives may still not lead to a correct or complete understanding.
    \\

    
      
    \\

    
      - **Fun Fact**: The story of the blind men and the elephant originated in ancient India and has been used in various philosophical and religious texts, including Jain, Buddhist, and Hindu traditions, to illustrate the limitations of human perception and the importance of holistic understanding.
    \\

  \section{Confirmation bias
    
      (Also known as: Myside Bias)
    \\

  
    
      - **Description**: Confirmation bias is the tendency to search for, interpret, favor, and recall information in a way that confirms one's preexisting beliefs or hypotheses, while giving disproportionately less consideration to alternative possibilities. This cognitive bias can lead to statistical errors and poor decision-making.
    \\

    
      
    \\

    
      - **Logical Form**:
    \\

    
        1. Person holds belief X.
    \\

    
        2. Person encounters evidence or information.
    \\

    
        3. Person favors evidence that supports belief X and dismisses or underweights evidence against belief X.
    \\

    
      
    \\

    
      - **Example \#1**:
    \\

    
        - **Scenario**: A person believes that left-handed people are more creative.
    \\

    
        - **Observation**: They meet two left-handed people who are artists.
    \\

    
        - **Explanation**: They take this as strong evidence supporting their belief while ignoring the many right-handed creative individuals they know.
    \\

    
      
    \\

    
      - **Example \#2**:
    \\

    
        - **Scenario**: A person believes that a particular stock is a good investment.
    \\

    
        - **Observation**: They focus on news articles that predict the stock's rise and ignore reports warning about potential declines.
    \\

    
        - **Explanation**: The person only acknowledges information that aligns with their positive outlook on the stock, leading to a biased assessment of the investment.
    \\

    
      
    \\

    
      - **Variation**:
    \\

    
        - **Selection Bias**: This occurs when individuals only look for evidence in places where they expect to find it, thereby reinforcing their existing beliefs.
    \\

    
        - **Belief Perseverance**: The tendency to maintain beliefs even after the evidence supporting them has been discredited.
    \\

    
      
    \\

    
      - **Tip**: To mitigate confirmation bias, actively seek out and consider evidence that challenges your beliefs. Engage with diverse perspectives and critically evaluate all information.
    \\

    
      
    \\

    
      - **Exception**: Confirmation bias may be less prevalent in situations where individuals have strong incentives to be accurate and objective, such as in scientific research with peer review and replication.
    \\

    
      
    \\

    
      - **Fun Fact**: The term "confirmation bias" was coined by English psychologist Peter Wason in his 1960 paper "On the failure to eliminate hypotheses in a conceptual task." His experiments demonstrated that people tend to seek confirmatory evidence for their hypotheses rather than attempting to falsify them, which is a more robust method of testing validity.
    \\

  }


Motivated reasoning
    
      - **Description**: Motivational reasoning bias is the tendency to use emotional desires and preferences to influence one's reasoning and judgment. This bias causes individuals to rationalize or distort evidence in a way that aligns with their wishes, goals, or motivations, rather than objectively evaluating the facts.
    \\

    
      
    \\

    
      - **Logical Form**:
    \\

    
        1. Person has a strong desire or motivation to believe in a certain outcome or idea.
    \\

    
        2. Person processes information in a way that supports this desired outcome or idea.
    \\

    
        3. Evidence that contradicts the desired outcome is minimized or ignored.
    \\

    
      
    \\

    
      - **Example \#1**:
    \\

    
        - **Scenario**: An investor strongly wants a particular stock to perform well because they have a large investment in it.
    \\

    
        - **Observation**: Despite emerging negative reports about the company, the investor focuses on any positive news and dismisses the negative information.
    \\

    
        - **Explanation**: The investor’s desire for the stock to succeed influences them to selectively interpret information, favoring data that aligns with their hope for a profitable return.
    \\

    
      
    \\

    
      - **Example \#2**:
    \\

    
        - **Scenario**: A person is determined to prove that their preferred political party has the best policies.
    \\

    
        - **Observation**: They highlight and share favorable statistics and success stories related to their party while ignoring or discrediting similar evidence from the opposing party.
    \\

    
        - **Explanation**: Their motivation to support their party leads to biased reasoning, where only information that aligns with their preferred viewpoint is considered credible.
    \\

    
      
    \\

    
      - **Variation**:
    \\

    
        - **Wishful Thinking**: A subset of motivational reasoning where individuals believe in outcomes because they desire them, without regard for actual evidence.
    \\

    
        - **Self-Serving Bias**: When individuals interpret information in a way that benefits their self-esteem or personal interests, which can overlap with motivational reasoning.
    \\

    
      
    \\

    
      - **Tip**: To counteract motivational reasoning bias, strive to adopt a critical thinking approach and seek out information from a range of sources, especially those that challenge your preexisting beliefs or desires.
    \\

    
      
    \\

    
      - **Exception**: Motivational reasoning bias can be reduced in contexts where individuals are trained to use objective criteria and evidence-based decision-making processes, such as in scientific research or high-stakes professional evaluations.
    \\

    
      
    \\

    
      - **Fun Fact**: Motivational reasoning bias is not limited to personal desires; it can also occur in group settings where the collective goals or interests of a group influence the way its members process information. For instance, fans of a sports team might irrationally dismiss evidence of the team's poor performance as they strongly desire the team to win.
    \\

  
    
      (Also known as: Desire-Driven Reasoning)
    \\

  

Counterinduction
    
      - **Description**: Counterinduction is a form of reasoning used to argue against an inductive generalization by suggesting that if the generalization were true, then the opposite or contrary case should also be true, but it is not. Essentially, it challenges the validity of an inductive conclusion by highlighting a counterexample or contradiction that undermines the general pattern being presented.
    \\

    
      
    \\

    
      - **Logical Form**:
    \\

    
        1. An inductive generalization is made based on a pattern observed in several cases.
    \\

    
        2. A counterexample is presented that contradicts the generalization.
    \\

    
        3. The presence of the counterexample suggests that the original generalization may be flawed or not universally applicable.
    \\

    
      
    \\

    
      - **Example \#1**:
    \\

    
        - **Scenario**: An inductive generalization claims that "All swans are white" based on numerous observations of white swans.
    \\

    
        - **Counterinductive Argument**: A black swan is discovered.
    \\

    
        - **Explanation**: The existence of the black swan contradicts the generalization that all swans are white, showing that the original inductive conclusion is not universally true.
    \\

    
      
    \\

    
      - **Example \#2**:
    \\

    
        - **Scenario**: A study suggests that "Eating carrots improves night vision" based on observations of individuals who improved their night vision after consuming carrots.
    \\

    
        - **Counterinductive Argument**: An individual who eats a large amount of carrots but shows no improvement in night vision.
    \\

    
        - **Explanation**: This counterexample challenges the generalization that eating carrots universally improves night vision, indicating that the relationship may not be as straightforward as initially believed.
    \\

    
      
    \\

    
      - **Variation**:
    \\

    
        - **Direct Counterexample**: Providing a specific example that directly contradicts the inductive generalization.
    \\

    
        - **Counterexample-Based Refutation**: Using multiple counterexamples to argue against a broader inductive claim.
    \\

    
      
    \\

    
      - **Tip**: When using counterinduction, ensure that the counterexample is relevant and robust enough to genuinely challenge the generalization. A single, isolated counterexample may not be sufficient to refute a well-supported inductive argument.
    \\

    
      
    \\

    
      - **Exception**: In some cases, a counterinductive argument may not fully disprove an inductive generalization but rather suggests that the generalization may have exceptions or require refinement.
    \\

    
      
    \\

    
      - **Fun Fact**: The term "counterinduction" is not as commonly used in everyday language as "counterexample," but both concepts are integral to critical thinking and evaluating the strength of inductive arguments in various fields, including science and philosophy.
    \\

  
    
      (Also known as: Inductive Counterargument)
    \\

  

Group attribution error
    
      (Also known as: Aggregation Bias, Group Fallacy)
    \\

  
    
      - **Description**: Group attribution error occurs when individuals ascribe the characteristics, behaviors, or traits of a group to all its members. This cognitive bias involves assuming that what is true for the group as a whole must also be true for each individual within that group.
    \\

    
      
    \\

    
      - **Logical Form**:
    \\

    
        1. A group is observed to exhibit a certain trait or behavior.
    \\

    
        2. An individual is then assumed to possess the same trait or exhibit the same behavior based on their membership in that group.
    \\

    
        3. This generalization is made without considering individual differences within the group.
    \\

    
      
    \\

    
      - **Example \#1**:
    \\

    
        - **Scenario**: A company has a reputation for having aggressive sales tactics.
    \\

    
        - **Group Attribution Error**: Assuming that every employee in the company is aggressive in their sales approach.
    \\

    
        - **Explanation**: The behavior of some members of the company does not necessarily reflect the behavior of every individual. Assuming all employees are aggressive because of the company's reputation is an error in attributing the group's trait to each member.
    \\

    
      
    \\

    
      - **Example \#2**:
    \\

    
        - **Scenario**: A particular ethnic group is often portrayed in the media as being highly entrepreneurial.
    \\

    
        - **Group Attribution Error**: Concluding that every person from that ethnic group is likely to be entrepreneurial.
    \\

    
        - **Explanation**: Media portrayals and stereotypes do not account for the diversity of individuals within the group. Not every member of the ethnic group will necessarily fit the entrepreneurial stereotype.
    \\

    
      
    \\

    
      - **Variation**:
    \\

    
        - **Stereotyping**: A broader concept where generalizations about a group’s characteristics are applied to all its members.
    \\

    
        - **Homogenization**: Treating diverse individuals within a group as if they were uniform in their traits and behaviors.
    \\

    
      
    \\

    
      - **Tip**: Be cautious about making generalizations based on group characteristics. Always consider individual differences and avoid assuming that traits or behaviors observed in a group apply universally to all its members.
    \\

    
      
    \\

    
      - **Exception**: The group attribution error might be less pronounced when there is substantial evidence that the group as a whole consistently exhibits certain traits or behaviors. However, even in such cases, it’s important to avoid overgeneralizing to individuals.
    \\

    
      
    \\

    
      - **Fun Fact**: Group attribution error is a common bias in social psychology and can contribute to various forms of prejudice and discrimination. Recognizing this bias helps in fostering more accurate and fair assessments of individuals within any group.
    \\

  \section{Idola}


Idola tribus

Idola specus

Idola theatri

Idola fori

Biased Sample Fallacy
    
      (also known as: biased statistics, loaded sample, prejudiced statistics, prejudiced sample, loaded statistics, biased induction, biased generalization, biased generalizing, unrepresentative sample, unrepresentative generalization)
    \\

  
    Description: Drawing a conclusion about a population based on a sample that is biased, or chosen in order to make it appear the population on average is different than it actually is.

    
      This differs from the {\it hasty generalization} fallacy, where the biased sample is specifically chosen from a select group, and the small sample is just a random sample, but too small to get any accurate information.
    \\

    
      Logical Form:
    \\

    
      Sample S, which is biased, is taken from population P.
    \\

    
      Conclusion C is drawn about population P based on S.
    \\

    
      Example \#1:
    \\

    
      Based on a survey of 1000 American homeowners, 99\% of those surveyed have two or more automobiles worth on average \$100,000 each.  Therefore, Americans are very wealthy.
    \\

    
      Explanation: Where did these homeowners live?  Beverly Hills, CA.  If the same survey was taken in Detroit, the results would be quite different.  It is fallacious to accept the conclusion about the American population in general based on not just the geographical sample, but also the fact that homeowners were only surveyed.
    \\

    
      Example \#2:
    \\

    
      Pastor Pete: People are turning to God everywhere!  9 out of 10 people I interviewed said that they had a personal relationship with Jesus Christ.
    \\

    
      Fred: Where did you find these people you interviewed?
    \\

    
      Pastor Pete: In my church.
    \\

    
      Explanation: Pastor Pete has drawn a conclusion about religious beliefs from people “everywhere” based on people he has interviewed in his church.  That’s like concluding that the world likes to dance naked in front of strangers after interviewing a group of strippers.
    \\

    
      Exception: What exactly is “biased” is subjective, but some biases are very clear.
    \\

    
      Tip: Be very wary of statistics.  Look at the source and details of the studies which produced the statistics.  Very often you will find some kind of bias.
    \\

  

Lying with Statistics
    
      (also known as: statistical fallacy/fallacies, misunderstanding the nature of statistics [form of], fallacy of curve fitting, the fallacy of overfitting)
    \\

  
    Description: This can be seen as an entire class of fallacies that result in presenting statistical data in a very biased way, and of course, interpreting statistics without questioning the methods behind collecting and presenting the data.

    
      The many methods are outside the scope of this book, but if you really want to jump in here, and see how deceptive statistics can be, get the book, {\it How to Lie with Statistics} by Darrell Huff, a 1954 classic that is just as relevant today as it was in his time.
    \\

    
      Logical Form:
    \\

    
      Claim A is made.
    \\

    
      Statistic S is manipulated to support claim A.
    \\

    
      Example \#1:
    \\

    
      Did you see that bar graph in USA Today?  It showed a HUGE spike in the moral decline of our country!
    \\

    
      Explanation: The first question that should immediately come to mind is, how on earth can one measure morality?  With such a loose definition, it is not hard to imagine one collecting and measuring the data that only supports her desired outcome for the “numbers”.  Furthermore, what is a “huge spike?”  Visually, you can play with graphs to make numbers seem much more dramatic by not starting at zero, or by doing that little “chopped section” thing.  For example, let’s accept that last year 20\% of all people were immoral.  This year it is 22\%.  Not a big deal, and if shown on a graph with a vertical axis of 0\% to 100\%, the line connecting the 20\% to the 22\% would be barely inclined.  However, if shown on a graph with a vertical axis of 20\% to 25\%, the line connecting the 20\% to the 22\% would appear to be a huge spike. The same data, a very different presentation.
    \\

    
      Example \#2:
    \\

    
      Looking at that pie chart, there is a very small percentage of people who declare themselves atheist.  Therefore, atheism is not that popular of a belief.
    \\

    
      Explanation: First, atheism is better described as a lack of belief.  Second, many non-believers are not even familiar with the term “atheist,” and often consider themselves Christian, Jewish, or some other religion, based on their culture and family tradition, not necessarily their beliefs.  Statistics don’t account for this.
    \\

    
      Exception: At times, careful and honest explanations of the data and the presentation can help one avoid statistical fallacies, but like virtually all exceptions, this can be debatable.
    \\

    
      Variation: {\it Misunderstanding the nature of statistics} is related to this fallacy, but the fallacy rests on the person interpreting the statistics.  For example, you might be very troubled to find out that your doctor graduated in the bottom half of her class, but that is half of all the doctors in the world, and to be expected.
    \\

    
      Tip: There are many free, online courses on statistics. Take one. It will be worth your time (and money)!
    \\

    References:

    
      
        
      \\

      
        
          Huff, D. (1993). {\it How to Lie with Statistics} (Reissue edition). New York: W. W. Norton \& Company.
        
      
    
  

Damning the Alternatives
    
      - Description: The fallacy of damning the alternatives occurs when one argues that a particular option is the best by highlighting the flaws of other, often unexamined, alternatives. This approach is flawed because it does not genuinely evaluate the merits of the option in question but instead focuses on disparaging other options. It is particularly fallacious when the alternatives are not fixed or known.
    \\

    
      
    \\

    
      - Logical Form:
    \\

    
        1. Assert that a particular option is the best or only valid choice.
    \\

    
        2. Criticize other alternatives to support this claim.
    \\

    
        3. The validity of the preferred option is assumed based on the shortcomings of the others, without direct evaluation.
    \\

    
      
    \\

    
      - Example \#1:
    \\

    
        - Scenario: "Hawkins’ theory has to be the right answer. All the others have been proved hopelessly wrong."
    \\

    
        - Explanation: The argument assumes Hawkins' theory is correct simply because all other theories have been discredited. However, Hawkins’ theory may still be wrong, and the argument fails to independently validate it.
    \\

    
      
    \\

    
      - Example \#2:
    \\

    
        - Scenario: "Chelsea is a really great team. Look at Liverpool and Manchester United; they are both useless."
    \\

    
        - Explanation: By denigrating Liverpool and Manchester United, the speaker does not provide evidence that Chelsea is actually great. The argument might ignore other teams or aspects that could affect the overall evaluation of Chelsea.
    \\

    
      
    \\

    
      - Variation:
    \\

    
        - Scenario: "No design for a new building ever meets with universal approval, but look at the alternatives: a glass-fronted matchbox, something with all the pipes on the outside, or a moulded concrete monstrosity."
    \\

    
        - Explanation: The speaker uses the flaws in various building designs to argue that their preferred design is the best, without addressing its own potential shortcomings.
    \\

    
      
    \\

    
      - Tip: When evaluating proposals or options, focus on the merits and demerits of each option individually rather than relying on the flaws of others. This ensures a fair and comprehensive assessment.
    \\

    
      
    \\

    
      - Exception: Damning the alternatives may not be fallacious if the alternatives are known to be definitively inferior and are the only relevant options available. However, this is rare and requires clear evidence that all other options are indeed flawed.
    \\

    
      
    \\

    
      - Fun Fact: The fallacy of damning the alternatives is often employed in political and marketing rhetoric to sway opinion by focusing attention on the weaknesses of competitors rather than the strengths of one’s own position. It’s a popular tactic in campaigns and debates where appealing to the audience’s negativity can be more effective than promoting positive attributes.
    \\

  
    
      (Also known as: Denigrating the Alternatives, Fallacy of Partisan Comparisons)
    \\

  \chapter{Accident Fallacy
    
      (also known as: a dicto simpliciter ad dictum secundum quid, destroying the exception, dicto secundum quid ad dictum simpliciter, dicto simpliciter, converse accident, reverse accident, fallacy of the general rule, sweeping generalization)
    \\

  
    Description: When an attempt is made to apply a general rule to all situations when clearly there are exceptions to the rule. Simplistic rules or laws rarely take into consideration legitimate exceptions, and to ignore these exceptions is to bypass reason to preserve the illusion of a perfect law.  People like simplicity and would often rather keep simplicity at the cost of rationality.

    
      Logical Form:
    \\

    
      X is a common and accepted rule.
    \\

    
      Therefore, there are no exceptions to X.
    \\

    
      Example \#1:
    \\

    
      I believe one should never deliberately hurt another person, that’s why I can never be a surgeon.
    \\

    
      Explanation: Classifying surgery under “hurting” someone, is to ignore the obvious benefits that go with surgery.  These kinds of extreme views are rarely built on reason.
    \\

    
      Example \#2:
    \\

    
      The Bible clearly says, “Thou shall not bear false witness.” Therefore, as a Christian, you better answer the door and tell our drunk neighbor with the shotgun, that his wife, whom he is looking to kill, is hiding in our basement. Otherwise, you are defying God himself!
    \\

    
      Explanation: To assume any law, even divine, applies to every person, in every time, in every situation, even though not explicitly stated, is an assumption not grounded in evidence, and fallacious reasoning.
    \\

    
      Exception: Stating the general rule when a good argument can be made that the action in question is a violation of the rule, would not be considered fallacious.
    \\

    
      The Bible says, “Thou shall not murder,” therefore, as a Christian, you better put that chainsaw down and untie that little kid.
    \\

    
      Tip: It is your right to question laws you don’t understand or laws with which you don’t agree. 
    \\

  }


Stereotyping fallacy
    
      Stereotyping
    \\

    
      Description: The general beliefs that we use to categorize people, objects, and events while assuming those beliefs are accurate generalizations of the whole group.
    \\

    
      
    \\

    
      Logical Form:
    \\

    
      
    \\

    
      All X’s have the property Y (this being a characterization, not a fact).
    \\

    
      
    \\

    
      Z  is an X.
    \\

    
      
    \\

    
      Therefore, Z has the property Y.
    \\

    
      
    \\

    
      Example \#1:
    \\

    
      
    \\

    
      French people are great at kissing.  Julie is French.  Get me a date!
    \\

    
      
    \\

    
      Explanation: “French people are great at kissing” is a stereotype, and believing this to be so is a fallacy.  While it may be the case that some or even most are great at kissing, we cannot assume this without valid reasons.
    \\

    
      
    \\

    
      Example \#2:
    \\

    
      
    \\

    
      Atheists are morally bankrupt.
    \\

    
      
    \\

    
      Explanation: This isn’t an argument, but just an assertion, one not even based on any kind of facts.  Stereotypes such as these usually arise from prejudice, ignorance, jealousy, or even hatred.
    \\

    
      
    \\

    
      Exception: Statistical data can reveal properties of a group that are more common than in other groups, which can affect the probability of any individual member of the group having that property, but we can never assume that all members of the group have that property.
    \\

    
      
    \\

    
      Tip: Remember that people are individuals above being members of groups or categories.
    \\

    
      
    \\

  \chapter{
    
      {\bf Illicit transference}
    \\

  }


Fallacy of composition
    
      (also known as: composition fallacy, exception fallacy, faulty induction)
    \\

  
    Description: Inferring that something is true of the whole from the fact that it is true of some part of the whole.  This is the opposite of the {\it fallacy of division}.

    
      Logical Form:
    \\

    
      A is part of B.
    \\

    
      A has property X.
    \\

    
      Therefore, B has property X.
    \\

    
      Example \#1:
    \\

    
      Each brick in that building weighs less than a pound.  Therefore, the building weighs less than a pound.
    \\

    
      Example \#2:
    \\

    
      Hydrogen is not wet.  Oxygen is not wet.  Therefore, water (H2O) is not wet.
    \\

    
      Example \#3:
    \\

    
      Your brain is made of molecules.  Molecules are not the source of consciousness.  Therefore, your brain cannot be the source of consciousness.
    \\

    
      Explanation: I included three examples that demonstrate this fallacy from the very obvious to the less obvious, but equally as flawed.  In the first example, it is obvious because weight is cumulative.  In the second example, we know that water is wet, but we only experience the property of wetness when the molecules are combined and in large scale.  This introduces the concept of {\it emergent properties}, which when ignored, tends to promote {\it magical thinking}.  The final example is a common argument made for a supernatural explanation for consciousness.  On the surface, it is difficult to imagine a collection of molecules resulting in something like consciousness because we are focusing on the properties of the parts (molecules) and not the whole system, which incorporates emergence, motion, the use of energy, temperature (vibration), order, and other {\it relational properties}.
    \\

    
      Exception: If the whole is very close to the similarity of the parts, then more assumptions can be made from the parts to the whole.  For example, if we open a small bag of potato chips and discover that the first one is delicious, it is not fallacious to conclude that the whole snack (all the chips, minus the bag) will be just as delicious, but we cannot say the same for one of those giant family size bags because most of us would be hurling after about 10 minutes of our chip-eating frenzy.
    \\

    
      Tip: It is worth a few minutes of your time to research the topic of “emergence.” 
    \\

    References:

    
      
        
      \\

      
        
          Goodman, M. F. (1993). {\it First Logic}. University Press of America.
        
      
    
  

Fallacy of division
    
      (also known as: false division, faulty deduction, division fallacy)
    \\

  
    Description: Inferring that something is true of one or more of the parts from the fact that it is true of the whole.  This is the opposite of the {\it fallacy of composition} 

    
      Logical Form:
    \\

    
      A is part of B.
    \\

    
      B has property X.
    \\

    
      Therefore, A has property X.
    \\

    
      Example \#1:
    \\

    
      His house is about half the size of most houses in the neighborhood. Therefore, his doors must all be about 3 1/2 feet high.
    \\

    
      Explanation: The size of one’s house almost certainly does not mean that the doors will be smaller, especially by the same proportions.  The size of the whole (the house) is not directly related to the size of every part of the house.
    \\

    
      Example \#2:
    \\

    
      I heard that the Catholic Church was involved in a sex scandal cover-up.  Therefore, my 102-year-old Catholic neighbor, who frequently attends Church, is guilty as well!
    \\

    
      Explanation: While it is possible that the 102-year-old granny is guilty for some things, like being way too liberal with her perfume, she would not be guilty in any sex scandals just by her association with the Church alone. Granted, it can be argued that Granny’s financial support of the Church makes her morally complicit, but it is clear her “crimes” are in a different category than those behind the cover-ups.
    \\

    
      Exception: When a part of the whole has a property that, by definition, causes the part to take on that property.
    \\

    
      My 102-year-old neighbor is a card-carrying member of an organization of thugs that requires its members to kick babies.  Therefore, my neighbor is a thug... and she wears way too much perfume.
    \\

    
      Tip: Review the fallacy of composition and see if you understand the difference well-enough to explain it to someone.
    \\

  

Ecological fallacy
    
      (also known as: ecological inference fallacy)
    \\

  
    
      Description: The interpretation of statistical data where inferences about the nature of individuals are deduced from inference for the group to which those individuals belong.
    \\

    
      Logical Form:
    \\

    
      Group X has characteristic Y.
    \\

    
      Person 1 is in group X.
    \\

    
      Therefore, person 1 has characteristic Y.
    \\

    
      Example \#1:
    \\

    
      Men score better on math than women do. Jerry is a man. Therefore, Jerry is better at math than Sylvia, who is a woman.
    \\

    
      Explanation: The fact that men score better on math than women is a group generalization. This does not mean that any individual man will score better than any individual woman on math. Educated guesses could be made if we knew more about the statistics. For example, if we just knew that men scored an average of 8\% higher than women, we could not even say that any given man is likely to be better at math than any given woman. This is because there could be what is referred to as an uneven distribution, that is, there could be a small group of women who are really bad at math or a small group of men who are really good at math that throws off the curve.
    \\

    
      Example \#2:
    \\

    
      A study was done recently showing that church attendance was positively correlated with marriage longevity, that is, those couples who attended church together more often were more likely to stay married. This really should not be a surprise considering the general view of divorce within religion. What this does not mean is that any given couple who does not attend church is more likely to get divorced than any given couple that does attend church.
    \\

    
      Explanation: To make this claim, we would need more information on the raw data used. Perhaps just religious fanatics who go to church daily have a practically non-existent divorce rate of say 2\%. Then it is possible that those who never go to church have a lower divorce rate than those who do go to church every Sunday, but because of the fanatics, the distribution is not evenly distributed.
    \\

    
      Exception: It is not unreasonable to make a probabilistic claim about any given member of a group if the data warrants such a claim. For example, if 999 jellybeans in a jar are red and one is green, we can say that any given jellybean in the jar is  99.9\% likely to be red.
    \\

    
      Fun Fact: Although this is a statistical fallacy, it is commonly extended to everyday situations. If you don't mind getting into statistics a bit, understanding statistical fallacies could improve one's overall reasoning ability.
    \\

    References:

    
      
        
      \\

      
        
          Babbie, E. R. (2016). {\it The Basics of Social Research}. Cengage Learning.
        
      
    
  \chapter{Question-begging fallacies}
\section{Circular reasoning
    
      (also known as: circulus in demonstrando, paradoxical thinking, circular argument, circular cause and consequence, reasoning in a circle, vicious circle, tautology, redundancy)
    \\

  
    Description: A type of reasoning in which the proposition is supported by the premises, which is supported by the proposition, creating a circle in reasoning where no useful information is being shared.  This fallacy is often quite humorous.

    
      Logical Form:
    \\

    
      X is true because of Y.
    \\

    
      Y is true because of X.
    \\

    
      Example \#1:
    \\

    
      Pvt. Joe Bowers: What are these electrolytes? Do you even know?
    \\

    
      Secretary of State: They're... what they use to make Brawndo!
    \\

    
      Pvt. Joe Bowers: But why do they use them to make Brawndo?
    \\

    
      Secretary of Defense: [raises hand after a pause] Because Brawndo's got electrolytes.
    \\

    
      Explanation: This example is from a favorite movie of mine, {\it Idiocracy, }where Pvt. Joe Bowers (played by Luke Wilson) is dealing with a bunch of not-very-smart guys from the future.  Joe is not getting any useful information about electrolytes, no matter how hard he tries.
    \\

    
      Example \#2:
    \\

    
      The Bible is the Word of God because God tells us it is... in the Bible.
    \\

    
      Explanation: This is a very serious circular argument on which many people base their entire lives. The circularity is in the stated or implied claim that the reason they trust in the Bible is because it is the Word of God.  This is like getting an e-mail from a Nigerian prince, offering to give you his billion dollar fortune -- but only after you wire him a “good will” offering of \$50,000.  Of course, you are skeptical until you read the final line in the e-mail that reads {\it “I, prince Nubadola, assure you that this is my message, and it is legitimate.  You can trust this e-mail and any others that come from me.”  }Now you know it is legitimate... because it says so in the e-mail.
    \\

    
      Exception: Some philosophies state that we can never escape circular reasoning because the arguments always come back to axioms or first principles, but in those cases, the circles are very large and do manage to share useful information in determining the truth of the proposition.
    \\

    
      Tip: Do your best to avoid circular arguments, as it will help you reason better because better reasoning is often a result of avoiding circular arguments.
    \\

  }
\subsection{Conflicting Conditions
    
      (also known as: contradictio in adjecto, Friendly fire, Own goal, Shooting oneself in the foot, Smuggled concept, Stolen concept fallacy, Unforced error, Self-refuting idea, a self-contradiction)
    \\

    
      (also known as: contradictio in adjecto, a self-contradiction, self-refuting idea)
    \\

  
    Description: When the argument is self-contradictory and cannot possibly be true.

    
      Logical Form:
    \\

    
      {\em Claim X is made, which is impossible as demonstrated by all or part of claim X.}
    \\

    
      Example \#1:
    \\

    
      {\em The only thing that is certain is uncertainty.}
    \\

    
      Explanation: Uncertainty itself cannot be certain by definition.  It is a self-contradiction.
    \\

    
      Example \#2:
    \\

    
      {\em I don’t care what you believe, as long as your beliefs don’t harm others.}
    \\

    
      Explanation: This is a contradiction.  At first glance, “as long as” appears to be a condition for the assertion, “I don’t care what you believe”, but it’s not; the assertion {\it has to be} false in all cases.  {\it The arguer must always care if the person believes something that will harm others or not.}
    \\

    
      Exception: When the self-contradictory statement is not put forth as an argument, but rather as an ironic statement, perhaps with the intent to convey some kind of deeper truth or meaning, but not necessarily to be taken literally, then this fallacy is not committed.
    \\

    
      Fun Fact: This sentence is false.
    \\

    References:

    
      
        
      \\

      
        
          Cicero: Academic Skepticism | Internet Encyclopedia of Philosophy. (n.d.). Retrieved from http://www.iep.utm.edu/cicero-a/
        
      
    
  }


Kettle logic
    Description: Making (usually) multiple, contradicting arguments, in an attempt to support a single point or idea.

    
      Logical Form:
    \\

    
      Statement 1 is made.
    \\

    
      Statement 2 is made and contradicts statement 1.
    \\

    
      Statement 3 is made and contradicts statement 1 or 2
    \\

    
      ... etc.
    \\

    
      Example \#1:
    \\

    
      In an example used by Sigmund Freud in {\it The Interpretation of Dreams}, a man accused by his neighbor of having returned a kettle in a damaged condition offered three arguments:
    \\

    
      That he had returned the kettle undamaged; \newline
That it was already damaged when he borrowed it; \newline
That he had never borrowed it in the first place.
    \\

    
      Explanation: Each statement contradicts the one before. If it was already damaged, how did he return it undamaged? If he never borrowed it, how was it already damaged when he borrowed it?
    \\

    
      Example \#2:
    \\

    
      I don't believe in God. One reason is that I think he is evil. And by the way, there is no such thing as "evil;" it is just our evaluation of what we believe is wrong.
    \\

    
      Explanation: There are multiple statements here that all contradict, stemmed from the idea that the person does not believe in God. If he or she does not believe in God, how can he or she think God is evil? And if there is no such thing as evil, how can something be evil?
    \\

    
      Exception: If one acknowledges that their previous argument has been defeated (explicitly or implicitly), and they make a new argument, this would be more like moving the goalposts.
    \\

    
      Tip: If you are trying to be funny, kettle logic can be a great tool. In the movie “Philomena,” Steve Coogan’s character, Martin Sixsmith, leaves a church ceremony early. When asked why he says, “I don’t believe in God, and I think he knows.”
    \\

    References:

    
      
        
      \\

      
        
          Freud, S. (1913). {\it The Interpretation of Dreams}. Macmillan.
        
      
    
  

Stolen concept fallacy
    
      
        Description: Requiring the truth of the something that you are simultaneously trying to disprove.
      \\

      
        Logical Form:
      \\

      
        {\em Person 1 is attempting to disprove X.} \newline
{\em X is required to disprove X.}
      \\

      
        Example \#1:
      \\

      
        {\em Reason and logic are not always reliable, so we should not count on it to help us find truth.}
      \\

      
        Explanation: Here we are using reason to disprove the validity of reason, which is unreasonable -- reasonably speaking.
      \\

      
        Example \#2:
      \\

      
        {\em Science cannot be trusted.  It is a big conspiracy to cover up the truth of the Bible and the creation story.  Besides, I saw fossils in the creation museum with humans and dinosaurs together, which proves science is wrong!}
      \\

      
        Explanation: Geology is a branch of science.  Using science (examining fossils through the science of geology) to disprove science is absurd, a contradiction and, therefore, a fallacy in reasoning.
      \\

      
        Exception: Intentional irony.
      \\

      
        Fun Fact: The {\em nofallacies fallacy} is the belief that fallacies don't exist (not really).
      \\

    
    References:

    
      
        
      \\

      
        
          Peikoff, L. (1993). {\it Objectivism: The Philosophy of Ayn Rand}. Penguin.
        
      
    
  

Circular Definition
    Description: A circular definition is defining a term by using the term in the definition.  Ironically, that definition is partly guilty by my use of the term “definition” in the definition.  Okay, I am using definition way too much. Damn!  I just did it again.

    
      Logical Form:
    \\

    
      Term 1 is defined by using term 1 in its definition.
    \\

    
      Example \#1:
    \\

    
       Circular definition: a definition that is circular.
    \\

    
      Explanation: The definition is not at all helpful because we used the same words in the term to define them.
    \\

    
      Example \#2:
    \\

    
      Flippityflu: smarter than a Floppityflip.
    \\

    
      Floppityflip: dumber than a Flippityflu.
    \\

    
      Explanation: Here we have two definitions that result in a slightly larger circle, but a circle nonetheless.
    \\

    
      Exception: Many definitions are circular, but in the process of defining, there might be enough other information to help us understand the term.
    \\

    
      Ethics: moral principles that govern a person's behavior or the conducting of an activity.
    \\

    
      Moral Principles: the principles of right and wrong that are accepted by an individual or a social group.
    \\

    
      Principles of Right and Wrong: moral principles or one's foundation for ethical behavior.
    \\

    
      Tip: Don’t be too quick to call “fallacy” with this one. As demonstrated in the definition, some words in the term can be reused if they are commonly understood.
    \\

    
      {\em Nimbus cloud: A cloud that produces precipitation.}
    \\

    References:

    
      
        
      \\

      
        
          Lavery, J., \& Hughes, W. (2008). {\it Critical Thinking, fifth edition: An Introduction to the Basic Skills}. Broadview Press.
        
      
    
  

Homunculus Fallacy
    
      (also known as: homunculus argument, infinite regress)
    \\

  
    Description: An argument that accounts for a phenomenon in terms of the very phenomenon that it is supposed to explain, which results in an infinite regress.

    
      Logical Form:
    \\

    
      Phenomenon X needs to be explained.
    \\

    
      Reason Y is given.
    \\

    
      Reason Y depends on phenomenon X.
    \\

    
      Example \#1:
    \\

    
      Bert: How do eyes project an image to your brain?
    \\

    
      Ernie: Think of it as a little guy in your brain watching the movie projected by your eyes.
    \\

    
      Bert: Ok, but what is happening in the little guy in your head’s brain?
    \\

    
      Ernie: Well, think of it as a little guy in his brain watching a movie...
    \\

    
      Explanation: This fallacy creates an endless loop that actually explains nothing.  It is fallacious reasoning to accept an explanation that creates this kind of endless loop that lacks any explanatory value.
    \\

    
      Example \#2:
    \\

    
      Dicky: So how do you think life began?
    \\

    
      Ralphie: Simple.  Aliens from another planet seeded this planet with life billions of years ago.
    \\

    
      Dicky: OK, but how did that alien life form begin?
    \\

    
      Ralphie: Simple.  Aliens from another planet seeded that planet with life.
    \\

    
      Explanation: This fallacy can be tricky because maybe it is true that aliens are responsible for spreading life, so the answers might be technically right, but the question implied is how life {\it ultimately}  began, which this form of reasoning will not answer.
    \\

    
      Exception: There might be some exceptions that rely on high-level epistemology having to do with a large enough loop and validating feedback.  The important question to ask is, does the explanation have any value and is the question being answered or deflected?
    \\

    
      Fun Fact: Several fallacies are a result of a lack of explanatory value. For example, the {\em homunculus fallacy} can be seen as a specific form of the {\it failure to elucidate}.
    \\

    References:

    
      
        
      \\

      
        
          Tulving, E. (2000). {\it Memory, Consciousness, and the Brain: The Tallinn Conference}. Psychology Press.
        
      
    
  \section{Begging the Question
    
      (also known as: petitio principii, assuming the initial point, assuming the answer, chicken and the egg argument, circulus in probando)
    \\

  
    Description: Any form of argument where the conclusion is assumed in one of the premises.  Many people use the phrase “begging the question” incorrectly when they use it to mean, “prompts one to ask the question”.  That is NOT the correct usage. {\it Begging the question} is a form of {\it circular reasoning}.

    
      Logical Form:
    \\

    
      Claim X assumes X is true.
    \\

    
      Therefore, claim X is true.
    \\

    
      Example \#1:
    \\

    
      Paranormal activity is real because I have experienced what can only be described as paranormal activity.
    \\

    
      Explanation: The claim, “paranormal activity is real” is supported by the premise, “I have experienced what can only be described as paranormal activity.”  The premise presupposes, or assumes, that the claim, “paranormal activity is real” is already true.
    \\

    
      Example \#2:
    \\

    
      The reason everyone wants the new "Slap Me Silly Elmo" doll is because this is the hottest toy of the season!
    \\

    
      Explanation: Everyone wanting the toy is the same thing as it being "hot," so the reason given is no reason at all—it is simply rewording the claim and trying to pass it off as support for the claim.
    \\

    
      Exception: Some assumptions that are universally accepted could pass as not being fallacious.
    \\

    
      People like to eat because we are biologically influenced to eat.
    \\

    

    
      Tip: Remember that {\em begging the question} doesn’t require a question, but the {\it complex question fallacy} does.
    \\

    References:

    
      
        
      \\

      Walton, D. N., \& Fallacy, A. A. P. (1991). Begging the Question.
    
  }


Loaded label

Complex Question Fallacy
    (also known as: plurium interrogationum, many questions fallacy, fallacy of presupposition, loaded question, trick question, false question, Appeal to Complexity, ill-posed question, fallacy of many questions, surfeit of questions)
  
    Description: A question that has a presupposition built in, which implies something but protects the one asking the question from accusations of false claims.  It is a form of misleading discourse, and it is a fallacy when the audience does not detect the assumed information implicit in the question and accepts it as a fact.

    
      Logical Form:
    \\

    
      {\em Question X is asked that requires implied claim Y to be accepted before question X can be answered.}
    \\

    
      Example \#1:
    \\

    
      {\em How many times per day do you beat your wife?}
    \\

    
      Explanation: Even if the response is an emphatic, “none!” the damage has been done.  If you are hearing this question, you are more likely to accept the possibility that the person who was asked this question is a wife-beater, which is fallacious reasoning on your part.
    \\

    
      Example \#2:
    \\

    
      {\em How many school shootings should we tolerate before we change the gun laws?}
    \\

    
      Explanation: The presupposition is that changing the gun laws will decrease the number of school shootings.  This may be the case, but it is a claim that is implied in the statement and hidden by a more complex question.  Reactively, when one hears a question such as this, one's mind will attempt to search for an answer to the question—which is actually a distraction from rejecting the implicit claim being made.  It is quite brilliant but still fallacious.
    \\

    
      Exception: It is not a fallacy if the implied information in the question is known to be an accepted fact.
    \\

    
      {\em How long can one survive without water?}
    \\

    
      Here, it is presumed that we need water to survive, which very few would deny that fact.
    \\

    
      Tip: Remember that fallacious techniques don’t require the person using them to be lacking in reason; it is the audience or interlocutor that is at risk for falling for the deception.
    \\

    References:

    
      
        
      \\

      
        
          Menssen, S., \& Sullivan, T. D. (2007). {\it The Agnostic Inquirer: Revelation from a Philosophical Standpoint}. Wm. B. Eerdmans Publishing.
        
      
    
  \section{Loaded language
    
      (also known as: Argument by Emotive Language. loaded words, loaded language, euphemisms)
    \\

  
    Description: Substituting facts and evidence with words that stir up emotion, with the attempt to manipulate others into accepting the truth of the argument.

    
      Logical Form:
    \\

    
      Person A claims that X is true.
    \\

    
      Person A uses very powerful and emotive language in the claim.
    \\

    
      Therefore, X is true.
    \\

    
      Example \#1:
    \\

    
      By {\it rejecting} God, you are rejecting goodness, kindness, and love itself.
    \\

    
      Explanation: Instead of just “not believing” in God, we are  “rejecting” God, which is a much stronger term—especially when God is associated with “goodness.”
    \\

    
      Example \#2:
    \\

    
      The Bible is filled with stories of God's {\it magic}.
    \\

    
      Explanation: Instead of using the more accepted term “miracles,” the word “magic” is used that connotes powers associated with fantasy and make-believe in an attempt to make the stories in the Bible seem foolish.
    \\

    
      Example \#3:
    \\

    
      I don’t see what’s wrong with engaging the services of a professional escort.
    \\

    
      Explanation: That’s just a nice way of saying, “soliciting a hooker.”  No matter what you call it unless you live in certain parts of Nevada (or other parts of the world), it is still legally wrong (not necessarily morally wrong).
    \\

    
      Exception: Language is powerful and should be used to draw in emotions, but never at the expense of valid reasoning and evidence.
    \\

    
      Tip: {\em Euphemisms}, when used correctly, reflect good social intelligence. When in a business meeting, say, “Pardon me for a moment, I have to use the restroom,” rather than “Pardon me for a moment, I have to move my bowels.”
    \\

  }


Buzzword
    
      (Also known as: Jargon, Catchphrase, Trendy Term)
    \\

  
    
      - **Description**: A buzzword is a word or phrase that is fashionable and widely used within a particular field or context. It often lacks precise meaning but is used to create an impression of sophistication or expertise.
    \\

    
      
    \\

    
      - **Logical Form**:
    \\

    
        - Term or phrase used to convey a sense of relevance or modernity, often without substantive content.
    \\

    
      
    \\

    
      - **Example \#1**:
    \\

    
        - **Term**: "Synergy"
    \\

    
        - **Explanation**: In business contexts, "synergy" is often used to suggest that combining efforts will result in greater effectiveness. However, it can be vague and used more to impress than to convey specific strategies.
    \\

    
      
    \\

    
      - **Example \#2**:
    \\

    
        - **Term**: "Disruptive innovation"
    \\

    
        - **Explanation**: This term is used to describe innovations that radically change industries. While it sounds impactful, it can be overused to the point of losing clear meaning and becoming a cliché.
    \\

    
      
    \\

    
      - **Variation**:
    \\

    
        - **Jargon**: Specialized terms used by a particular profession or group, often becoming buzzwords when they enter wider usage.
    \\

    
        - **Catchphrase**: A memorable phrase that becomes popular and is frequently used, sometimes as a buzzword.
    \\

    
      
    \\

    
      - **Tip**: Be cautious of buzzwords in communication; they can obscure meaning and reduce clarity. Always consider the specific context and seek precise definitions when encountering buzzwords.
    \\

    
      
    \\

    
      - **Exception**: Not all widely used terms are buzzwords. Terms that have clear, substantive meanings and are used accurately do not fall into the category of buzzwords.
    \\

    
      
    \\

    
      - **Fun Fact**: The term "buzzword" itself can become a buzzword when overused in discussions about trends or jargon, highlighting its own example of the phenomenon it describes.
    \\

  

Dog whistle (politics)
    
      (also known as: Code word (figure of speech))
    \\

  
    
      - **Description**: In politics, a dog whistle is a subtle or coded message designed to appeal to a specific group of people without attracting the attention of others. It uses language that has a specific, often controversial, meaning to certain audiences while appearing innocuous to the general public.
    \\

    
      
    \\

    
      - **Logical Form**:
    \\

    
        - A statement or phrase that conveys one meaning to the general public while carrying a different, often politically charged, connotation to a targeted audience.
    \\

    
      
    \\

    
      - **Example \#1**:
    \\

    
        - **Statement**: "We need to restore traditional values."
    \\

    
        - **Explanation**: To some, this may seem like a call to return to conservative principles, but to others, it may be interpreted as a veiled appeal to more controversial social or racial prejudices.
    \\

    
      
    \\

    
      - **Example \#2**:
    \\

    
        - **Statement**: "Law and order."
    \\

    
        - **Explanation**: While this might be seen as a straightforward call for increased policing, it can also resonate with voters who have concerns about crime in certain communities, and may subtly signal support for harsher policies.
    \\

    
      
    \\

    
      - **Variation**:
    \\

    
        - **Coded Language**: Similar to a dog whistle, but can be used in various contexts beyond politics, such as marketing or social commentary.
    \\

    
        - **Veiled Speech**: General term for language that subtly implies certain ideas without overtly stating them.
    \\

    
      
    \\

    
      - **Tip**: When analyzing political rhetoric, be attentive to language that may carry hidden meanings or appeal to specific demographic groups, particularly when it seems vague or ambiguous.
    \\

    
      
    \\

    
      - **Exception**: Not all ambiguous or vague language is a dog whistle; some statements might be unclear due to poor communication rather than intentional manipulation.
    \\

    
      
    \\

    
      - **Fun Fact**: The term "dog whistle" comes from the idea that a dog whistle emits a sound that is inaudible to humans but can be heard by dogs. Similarly, dog whistle politics involves messages that are clear to a targeted group but not to the general populace.
    \\

  

Glittering generality
    
      (Also known as: Vague Praise, Euphemism)
    \\

  
    
      - **Description**: A glittering generality is a propaganda technique that uses vague, emotionally appealing phrases that sound good but lack substantive content. These terms are designed to evoke positive feelings without providing specific information or evidence.
    \\

    
      
    \\

    
      - **Logical Form**:
    \\

    
        - **P** is presented with positive, emotionally charged terms.
    \\

    
        - **P** is made to appear virtuous or desirable without clear explanation or details.
    \\

    
      
    \\

    
      - **Example \#1**:
    \\

    
        - **Phrase**: "Freedom and democracy for all"
    \\

    
        - **Explanation**: This phrase sounds idealistic and positive but is vague about how freedom and democracy will be achieved or what specific actions will be taken. It aims to elicit support by appealing to broad, desirable values.
    \\

    
      
    \\

    
      - **Example \#2**:
    \\

    
        - **Phrase**: "Join the movement for a brighter future"
    \\

    
        - **Explanation**: This phrase is emotionally appealing and suggests a positive outcome ("brighter future") but does not explain what the movement entails or how it will create this future. It focuses on the attractiveness of the goal rather than the specifics.
    \\

    
      
    \\

    
      - **Variation**:
    \\

    
        - **Euphemism**: A type of glittering generality where a pleasant or neutral term is used to replace a more straightforward or harsh term.
    \\

    
        - **Positive Labeling**: Using attractive labels to make something seem more appealing without detailing its actual attributes.
    \\

    
      
    \\

    
      - **Tip**: When evaluating statements or slogans, look beyond the glittering generalities. Seek specific information and evidence to understand the true meaning and implications of the claims being made.
    \\

    
      
    \\

    
      - **Exception**: Glittering generalities are not inherently deceptive; they become problematic when used to obscure or avoid substantive discussion. A phrase can be both emotionally appealing and specific if it provides clear and actionable information.
    \\

    
      
    \\

    
      - **Fun Fact**: The term "glittering generality" is often used in discussions about advertising and political rhetoric. It highlights how language can be crafted to influence emotions rather than convey factual information.
    \\

  

OPROP!
    
      \#\#\# OPROP!
    \\

    
      
    \\

    
      - **Name**: OPROP!
    \\

    
      
    \\

    
      - **Also known as**: Opraab! (1940-Danish), Opråb! (modern Danish), Proclamation, Exclamation
    \\

    
      
    \\

    
      - **Description**: OPROP! refers to a German airborne propaganda leaflet dropped over Danish cities during the German invasion of Denmark on April 9, 1940. Authored by General Leonhard Kaupisch, the leaflet was written in broken Danish mixed with Norwegian. It aimed to justify the German invasion, portray the action as protective of Danish and Norwegian neutrality against British aggression, and persuade the Danish people to accept the German presence without resistance while negotiations were underway.
    \\

    
      
    \\

    
      - **Logical Form**:
    \\

    
        - **P** is presented as a protective action against a greater threat.
    \\

    
        - **P** is justified as a necessary measure to safeguard neutrality.
    \\

    
        - **P** implies that resistance is futile and would lead to greater harm.
    \\

    
      
    \\

    
      - **Example \#1**:
    \\

    
        - **Text**: "We are here to protect Danish neutrality from British aggression."
    \\

    
        - **Explanation**: This statement frames the German invasion as a protective measure rather than an aggressive act, aiming to legitimize the invasion and reduce resistance.
    \\

    
      
    \\

    
      - **Example \#2**:
    \\

    
        - **Text**: "Churchill is a warmonger; his actions threaten us all."
    \\

    
        - **Explanation**: By labeling Winston Churchill as a warmonger, the leaflet attempts to redirect blame and justify the German invasion, portraying it as a defensive necessity.
    \\

    
      
    \\

    
      - **Variation**:
    \\

    
        - **Propaganda Leaflets**: Similar leaflets were used in other contexts to influence public perception and morale.
    \\

    
        - **Psychological Warfare**: This technique is part of broader psychological strategies used during conflicts to impact civilian and military morale.
    \\

    
      
    \\

    
      - **Tip**: When interpreting historical propaganda, consider the context in which it was produced and the intended impact on its audience. Evaluate the claims made and the language used critically.
    \\

    
      
    \\

    
      - **Exception**: While effective in this case, propaganda leaflets like OPROP! can sometimes backfire if the claims are not credible or if the audience is already skeptical of the propagandist's intentions.
    \\

    
      
    \\

    
      - **Fun Fact**: The broken Danish used in the OPROP! leaflet was actually due to the hasty translation of a similar leaflet intended for Norway. The quick adaptation underscores the urgency and ad-hoc nature of wartime propaganda efforts.
    \\

  

Talking point

Weasel word
    
      (Also known as: Weasel term)
    \\

  
    
      - **Description**: Weasel words are vague or ambiguous terms used to create an impression of meaning or certainty without committing to a specific position. They are often used to mislead or evade a clear statement, allowing the speaker or writer to avoid accountability or precise definition. The term "weasel word" comes from the behavior of weasels, which are said to suck the contents out of eggs while leaving the shell intact.
    \\

    
      
    \\

    
      - **Logical Form**:
    \\

    
        - **Claim**: The statement is made using vague language.
    \\

    
        - **Weasel Word**: The word or phrase that undermines the clarity or specificity of the claim.
    \\

    
        - **Implication**: The statement appears substantive but lacks concrete detail or commitment.
    \\

    
      
    \\

    
      - **Example \#1**:
    \\

    
        - **Text**: "Studies suggest that many people find this product effective."
    \\

    
        - **Explanation**: The term "studies suggest" is a weasel phrase that implies evidence without specifying the nature of the studies or the findings, thus avoiding direct validation.
    \\

    
      
    \\

    
      - **Example \#2**:
    \\

    
        - **Text**: "Our product may help you achieve your goals."
    \\

    
        - **Explanation**: The word "may" is a weasel term that indicates potential without guaranteeing results, thus shielding the speaker from accountability if the product does not meet expectations.
    \\

    
      
    \\

    
      - **Variation**:
    \\

    
        - **Vague Terms**: Phrases like "experts agree," "often," or "some say" serve similar purposes by implying authority or consensus without providing specifics.
    \\

    
        - **Evasive Language**: Words and phrases used to avoid direct answers or commitments, such as "could," "might," or "likely."
    \\

    
      
    \\

    
      - **Tip**: When analyzing statements, identify and scrutinize weasel words to assess the clarity and validity of the claim. Ask for specifics and evidence to verify the substance of the statement.
    \\

    
      
    \\

    
      - **Exception**: Weasel words can sometimes be used legitimately to reflect uncertainty or to allow for flexibility in complex or evolving situations. In such cases, context and intent are crucial for evaluation.
    \\

    
      
    \\

    
      - **Fun Fact**: The term "weasel word" derives from the idea that a weasel can sneakily remove the contents of an egg without breaking the shell. Similarly, weasel words sneakily avoid making a clear or direct statement.
    \\

  

Dysphemism

Leading question
    
      (Also known as: Suggestive Question)
    \\

  
    
      
    \\

    
      - **Description**: A leading question is designed to prompt or guide the respondent towards a specific answer, often reflecting the questioner's bias or desired outcome. It suggests or implies a particular answer within the question itself, which can influence or manipulate the respondent's reply.
    \\

    
      
    \\

    
      - **Logical Form**:
    \\

    
        - **Question**: The question is phrased in a way that guides or suggests a particular response.
    \\

    
        - **Implication**: The structure of the question steers the respondent toward a specific answer, thus revealing the questioner's bias or intention.
    \\

    
      
    \\

    
      - **Example \#1**:
    \\

    
        - **Text**: "Don’t you think that the new policy is unfair to employees?"
    \\

    
        - **Explanation**: This question implies that the policy is unfair and seeks to elicit agreement, rather than allowing the respondent to independently evaluate the fairness of the policy.
    \\

    
      
    \\

    
      - **Example \#2**:
    \\

    
        - **Text**: "How did you feel about the movie, considering how many awards it won?"
    \\

    
        - **Explanation**: The question assumes that winning many awards is a positive aspect of the movie, guiding the respondent to evaluate the movie positively.
    \\

    
      
    \\

    
      - **Variation**:
    \\

    
        - **Loaded Question**: A question that contains a presumption or assumption that the respondent may not agree with, but which is difficult to answer without accepting the assumption.
    \\

    
        - **Push Poll**: A type of survey designed to sway respondents towards a particular viewpoint through the phrasing of questions.
    \\

    
      
    \\

    
      - **Tip**: To avoid or counter leading questions, rephrase the question to be neutral and open-ended, allowing the respondent to provide an unbiased answer.
    \\

    
      
    \\

    
      - **Exception**: Leading questions can be used effectively in certain contexts, such as in legal cross-examinations, where the aim is to reveal inconsistencies or specific details through strategic questioning.
    \\

    
      
    \\

    
      - **Fun Fact**: The concept of leading questions is well-known in legal contexts, where they are often used deliberately to guide witnesses or to reveal particular aspects of a case.
    \\

  

Meaningless Question
    Description: Asking a question that cannot be answered with any sort of rational meaning. This is the textual equivalent of dividing by zero.

    
      Logical Form:
    \\

    
      {\em Question asked.} \newline
{\em Question is meaningless.}
    \\

    
      Example \#1:
    \\

    
      {\em What’s north of the North Pole?}
    \\

    
      Explanation: The North Pole is the most northern point of the space in which we measure direction using the north, south, east, and west coordinates. “North” of the North Pole is meaningless.
    \\

    
      Example \#2:
    \\

    
      {\em What happened before time?}
    \\

    
      Explanation: “Before” is a term related to a place in time. Without time, this concept is meaningless.
    \\

    
      Example \#3:
    \\

    
      {\em How many angels can you fit on a head of a pin?}
    \\

    
      Explanation: Angels are said to be ghost-like in that they don’t take up space. “Fit” is a word that refers to space. The question is meaningless.
    \\

    
      Fun Fact: The answers to our three meaningless questions are 1) Northness, 2) space and energy had sex, and 3) the answer depends on if Heaven has Krispy Kreme doughnut shops or not.
    \\

    
      References:
    \\

    
      
        
      \\

      
        
          Yngve, V., \& Wasik, Z. (2006). {\it Hard-Science Linguistics}. A\&C Black.
        
      
    
  

Hypnotic Bait and Switch
    Description: Stating several uncontroversially true statements in succession, followed by a claim that the arguer wants the audience to accept as true.  This is a propaganda technique, but also a fallacy when the audience lends more credibility to the last claim because it was preceded by true statements.  The negative can also be used in the same way.

    
      This is a classic sales technique often referred to as, “getting the customer used to saying ‘yes’!”
    \\

    
      Logical Forms:
    \\

    
      A succession of uncontroversial true claims is made.
    \\

    
      Therefore, claim X (which is controversial), is true.
    \\

    
       
    \\

    
      A succession of uncontroversial false claims is made.
    \\

    
      Therefore, claim X (which is controversial), is false.
    \\

    
      Example \#1:
    \\

    
      Do you love your country?
    \\

    
      Do you love your family?
    \\

    
      Do you care about their wellbeing?
    \\

    
      Then you would love Eatme ice-cream!
    \\

    
      Example \#2:
    \\

    
      Is it right that such a small percentage of Americans control the vast majority of wealth?
    \\

    
      Is it right that you have to work overtime just to make ends meet?
    \\

    
      Is it right that you can’t even afford to leave the state for vacation?
    \\

    
      Do you really want to vote for Reggie Lipshitz?
    \\

    
      Explanation: As you read through the examples, you can see from where the word “hypnotic” comes.  Your subconscious mind starts to take over, and it seems almost reactionary that you start chanting “yes” or “no” (as in the second example) while not really considering with what you are agreeing or disagreeing.  These kinds of techniques work best in rallies where those doing the rallying count on you to act with emotion at the expense of your reason.
    \\

    
      Exception: It’s an effective persuasion technique, so if you're trying to convince your kids to stay off drugs, then manipulate away.  However, if you are trying to get someone to buy a vacuum cleaner, then take your {\it hypnotic bait and switch} and shove it up your reusable, hypoallergenic, dust bag.
    \\

    
      Tip: Become a human fallacy detector.  Look for these kinds of techniques everywhere you go.  As a result, your reasonable self will become conditioned to resist taking a back seat to emotional propaganda.
    \\

  \chapter{
    
      {\bf Equivocation}
    \\

  
    
      (also known as: doublespeak)
    \\

  
    Description: Using an ambiguous term in more than one sense, thus making an argument misleading.

    
      Logical Form:
    \\

    
      {\em Term X is used to mean Y in the premise.} \newline
{\em Term X is used to mean Z in the conclusion.}
    \\

    
      Example \#1:
    \\

    
      {\em I want to have myself a merry little Christmas, but I refuse to do as the song suggests and make the yuletide gay.  I don't think sexual preference should have anything to do with enjoying the holiday.}
    \\

    
      Explanation: The word, “gay” is meant to be in light spirits, joyful, and merry, not in the homosexual sense.
    \\

    
      Example \#2:
    \\

    
      {\em The priest told me I should have faith. \newline
I have faith that my son will do well in school this year. \newline
Therefore, the priest should be happy with me.}
    \\

    
      Explanation: The term “faith” used by the priest, was in the religious sense of believing in God without sufficient evidence, which is different from having “faith” in your son in which years of good past performance leads to the “faith” you might have in your son.
    \\

    
      Exception: Equivocation works great when deliberate attempts at humor are being made.
    \\

    
      Tip: When you suspect equivocation, substitute the word with the same definition for all uses and see if it makes sense.
    \\

    References:

    
      
        
      \\

      
        
          Parry, W. T., \& Hacker, E. A. (n.d.). {\it Aristotelian Logic}. SUNY Press.
        
      
    
  }


False equivalence
    Description: An argument or claim in which two completely opposing arguments appear to be logically equivalent when in fact they are not. The confusion is often due to one shared characteristic between two or more items of comparison in the argument that is way off in the order of magnitude, oversimplified, or just that important additional factors have been ignored.

    
      Logical Form:
    \\

    
      Thing 1 and thing 2 both share characteristic A.
    \\

    
      Therefore, things 1 and 2 are equal.
    \\

    
      Example \#1:
    \\

    
      President Petutti ordered a military strike that killed many civilians. He is no different than any other mass murder and he belongs in prison!
    \\

    
      Explanation: Both president Petutti and a mass murder share the characteristic that something they did resulted in the death of civilians. However, the circumstances, the level of responsibility, and the intent are significantly different for the president than the typical mass murder and ignoring these factors is unreasonable, thus makes the argument fallacious.
    \\

    
      Example \#2: Using the second amendment as justification to allow civilians to own nuclear submarines.
    \\

    
      Explanation: In this case, the first "thing" is the weapon as understood at the time the second amendment was passed. The second "thing" of comparison is the nuclear submarine, also a weapon, but one of significantly different magnitude. This example also introduces the difference between a legal justification and an argumentative one (see {\it appeal to the law}).
    \\

    
      Exception: Like most fallacies, this is one of degree rather than kind. The order of magnitude can be debated. Some may exaggerate this order of magnitude claiming a fallacy where it would be unreasonable to do so.
    \\

    
      Tip: Listen and read carefully. Often, people will make analogies and others will interpret them as claims of equivalence.
    \\

  \section{False attribution
    Description: Appealing to an irrelevant, unqualified, unidentified, biased, or fabricated source in support of an argument (modern usage). Historical use of this fallacy was in the attribution of "religious" or "spiritual" experiences to outside "higher" sources rather than internal, psychological processes (see {\it fantasy projection}).

    
      Logical Form:
    \\

    
      Claim X is made.
    \\

    
      Source Y, a fake or unverifiable source, is used to verify claim X.
    \\

    
      Therefore, claim X is true.
    \\

    
      Example \#1:
    \\

    
      But professor, I got all these facts from a program I saw on TV once... I don’t remember the name of it though.
    \\

    
      Explanation: Without a credible, verifiable source, the argument or claim being made is very weak.
    \\

    
      Example \#2:
    \\

    
      I had this book that proved that leprechauns are real and have been empirically verified by scientists, but I lost it.  I forgot the name of it as well -- and who the author was.
    \\

    
      Explanation: A story of “this book” hardly can serve as proof of an event as potentially significant as the discovery of leprechauns that have been empirically verified by scientists.  While it might be the case that the person telling this story really does remember reading a convincing argument, it very well could be the case that this person is fabricating this book -- it sure sounds like it.  In either case, it is fallacious to accept the claim that leprechauns are real and have been empirically verified by scientists based on this argument.
    \\

    
      Exception: No Exceptions.
    \\

    
      Tip: Don't falsify facts.  If you get caught lying, you will almost certainly lose the argument, even if you are right.
    \\

  }


Contextomy
    
      (also known as: fallacy of quoting out of context, quoting out of context, Quote mining, contextotomy, contextomy)
    \\

  
    Description: Removing a passage from its surrounding matter in such a way as to distort its intended meaning.

    
      Logical Form:
    \\

    
      Argument X has meaning 1 in context.
    \\

    
      Argument X has meaning 2 when taken out of context.
    \\

    
      Therefore, meaning 2 is said to be correct.
    \\

    
      Example \#1:
    \\

    
      David: Can you believe that the president said, "fat people are losers"?
    \\

    
      Sam: Where did you hear this?
    \\

    
      David: I read it in a headline on BrightBert News.
    \\

    
      Sam: He actually said, "People who say, 'fat people are losers' are not only cruel, but they are also wrong as well as being irrational."
    \\

    
      Explanation: David fell for {\it click bait}—a technique used by the media to get people to click their links or discuss their article. He did not bother to investigate the context from which the quote was taken, and he perpetuated the lie.
    \\

    
      Example \#2:
    \\

    
      Trisha: In an interview, your candidate admitted that he was a thief!
    \\

    
      Derek: He actually said that when he was three years old, he stole a lollipop from a store, and felt so guilty, that he never stole anything again.
    \\

    
      Explanation: Trisha managed to twist the meaning of the candidate's story from one showing the candidate's strong moral character, to one where he is a criminal. Clearly, context is important.
    \\

    
      Exception: People often use "you're taking that out of context" to soften something that would otherwise be hard to swallow, yet they are unable to explain adequately how it makes sense in any other context.
    \\

    
      Tip: A great response for “you’re taking that out of context” is “please do explain it to me in context.” If they can’t or won’t, it is likely that context doesn’t make the argument any more palatable.
    \\

    References:

    
      
        
      \\

      
        
          McGlone, M. S. (2005). Contextomy: the art of quoting out of context. {\it Media, Culture \& Society}, {\it 27}(4), 511–522. https://doi.org/10.1177/0163443705053974
        
      
    
  

Recontextualisation

No True Scotsman
    
      (also known as: appeal to purity [form of], no true Christian, no true crossover fallacy [form of])
    \\

  
    Description: When a universal (“all”, “every”, etc.) claim is refuted, rather than conceding the point or meaningfully revising the claim, the claim is altered by going from universal to specific, and failing to give any objective criteria for the specificity.

    
      Logical Form:
    \\

    
      All X are Y.
    \\

    
      (the claim that all X are Y is clearly refuted)
    \\

    
      Then all true X are Y.
    \\

    
      Example \#1: In 2011, Christian broadcaster, Harold Camping, (once again) predicted the end of the world via Jesus, and managed to get many Christians to join his alarmist campaign.  During this time, and especially after the Armageddon date had passed,  many Christian groups publicly declared that Camping is not a “true Christian”.  On a personal note, I think Camping was and is as much of a Christian as any other self-proclaimed Christian and religious/political/ethical beliefs aside, I give him credit for having the cojones to make a falsifiable claim about his religious beliefs.
    \\

    
      Example \#2:
    \\

    
      John: Members of the UbaTuba White Men's Club are upstanding citizens of the community.
    \\

    
      Marvin: Then why are there so many of these members in jail?
    \\

    
      John: They were never{\it  true} UbaTuba White Men's Club members.
    \\

    
      Marvin: What’s a true UbaTuba White Men's Club member?
    \\

    
      John: Those who don't go to jail.
    \\

    
      Explanation: This is a very common form of this fallacy that has many variations.  Every time one group member denounces another group member for doing or saying something that they don’t approve of, usually by the phrase, “he is not really a {\it true} [insert membership here]”, this fallacy is committed.
    \\

    
      The universal claim here is that no UbaTuba White Men's Club member will ever (universal) go to jail.  Marvin points out how clearly this is counterfactual as there are many UbaTuba White Men's Club members in jail.  Instead of conceding or meaningfully revising the claim, the implication that no "UbaTuba White Men's Club members" is changed to “no true UbaTuba White Men's Club members”, which is not meaningful because John’s definition of a “true UbaTuba White Men's Club member” apparently can only be demonstrated in the negative if an UbaTuba White Men's Club member goes to jail.  This results in the {\it questionable cause} fallacy as it is also an {\it unfalsifiable} claim, and of course, it commits the {\it no true Scotsman} fallacy.
    \\

    
      Exception: A revised claim going from universal to specific that does give an objective standard would not be fallacious.
    \\

    
      Variations: The more generic {\it appeal to purity} can be seen when the claim is that someone "does not have enough of" something, which is why they are not meeting the condition. For example, "If you have the desire for success, you will succeed!" Billy has the desire but is not succeeding. Therefore, Billy's desire is not strong (or pure) enough. The difference between the {\it appeal to purity} and the {\it no true Scotsman} is one of degree versus authenticity.
    \\

    
      Another variation is what I call the {\em no true crossover fallacy}. This fallacy is committed when one denies, for the purpose of protecting one of the groups, that an individual can be a part of two or more non-exclusionary groups. During the protests in the spring of 2020, many businesses have been vandalized, looted, and even burned down. When people blamed this on the protesters, the defense was that “protesters protest and looters loot.” The implication is that the moment a protester loots, they become a “looter,” and no longer are part of the group “protesters.” This is a way of absolving every “true” member of the group “protesters” from any wrongdoing. It is likely that there are a group of looters who are opportunists and have no ideological position and would not qualify in any reasonable way as “protesters.” It is clear, however, that there are protesters who believe looting is an effective tool for protesting. Similarly, attempting to absolve police officers from wrongdoing because “police officers police and criminals commit crimes” is equally as fallacious for all the same reasons.
    \\

    
      Tip: People will sometimes claim outright that if any person who claims to be a member of group X and has Y characteristic, is not a member of group X. Ask them if those who claim to be a member of group X and has Y characteristic would agree.
    \\

    References:

    
      
        
      \\

      
        
          Flew, A. (1984). {\it A Dictionary of Philosophy: Revised Second Edition}. Macmillan.
        
      
    
  

motte-and-bailey fallacy
    
      (Also Known As Motte-and-Bailey Doctrine, Motte-and-Bailey Strategy)
    \\

  
    
      - **Description:** The motte-and-bailey fallacy is an informal logical fallacy in which an arguer conflates two positions: one that is modest and defensible (the "motte") and another that is more controversial and harder to defend (the "bailey"). When faced with criticism, the arguer retreats to the more defensible position (the motte) and asserts that this is the position they were arguing for all along, while the more contentious position (the bailey) remains unchallenged.
    \\

    
      - **Logical Form:**
    \\

    
        - **P1:** An argument is advanced for a controversial and difficult-to-defend position (the bailey).
    \\

    
        - **P2:** When challenged, the arguer retreats to a more modest and easily defensible position (the motte).
    \\

    
        - **C:** The original controversial position is left unaddressed, as the critic is misled into focusing on the less contentious position.
    \\

    
      - **Example \#1:** Claiming "morality is socially constructed" as a broad argument (the bailey) and when challenged, retreating to the claim that "our beliefs about right and wrong are socially constructed" (the motte).
    \\

    
      - **Explanation:** The arguer shifts from the broader, more contentious claim about the non-existence of objective morality to the more defensible position that moral beliefs are shaped by social constructs, thereby avoiding direct criticism of the controversial claim.
    \\

    
      - **Example \#2:** Defending a strong sociological perspective that all knowledge is equally valid (the bailey) while retreating to the less contentious claim that knowledge is what people generally accept (the motte) when faced with criticism.
    \\

    
      - **Explanation:** The arguer deflects attention from the contentious idea that scientific knowledge is no different from other beliefs by focusing on the more defensible claim that knowledge is based on societal acceptance, sidestepping a deeper critique of the original argument.
    \\

    
      - **Variation:** The motte-and-bailey fallacy can manifest in various contexts, including political debates, academic arguments, and everyday discussions, where terms and positions are strategically shifted to evade criticism.
    \\

    
      - **Tip:** To identify and avoid this fallacy, critically examine whether an arguer is shifting between positions and ensure that responses address the original controversial claim directly, rather than being sidetracked by a more defensible but irrelevant position.
    \\

    
      - **Exception:** The motte-and-bailey fallacy is a specific type of strategic argumentation. Not all shifts in argument are fallacious; sometimes, they reflect genuine adjustments in response to new evidence or perspectives.
    \\

    
      - **Fun Fact:** The term "motte-and-bailey" comes from medieval castle architecture. The motte is a fortified tower on a mound, while the bailey is a larger, less defensible area surrounding it. The metaphor captures the tactic of retreating to a well-defended position when the more vulnerable one is under attack.
    \\

  

Accent Fallacy
    (also known as: accentus, emphasis fallacy, fallacy of accent, fallacy of prosody, misleading accent)
  
    Description: When the meaning of a word, sentence, or entire idea is interpreted differently by changing where the accent falls. \newline


    
      Logical Form:
    \\

    
      {\em Claim is made with accent on word X giving claim meaning Y.} \newline
{\em Claim is interpreted with accent on word A giving claim meaning B.}
    \\

    
      Example \#1: In the movie, {\it My Cousin Vinny}, Ralph Maccio's character, Bill, was interrogated for suspected murder. When the police officer asks him, "At what point did you shoot the clerk?" Bill replies in shock, "{\em I} shot the clerk? {\em I} shot the clerk?" Later in the film, the police officer reads Bill's statement as a confession in court, "...and he said, 'I shot the clerk. I shot the clerk.'"
    \\

    
      Explanation: In the movie, it appeared that the police officer did understand Bill's question as a confession. So it did not appear to be a fallacious tactic of the police officer, rather a failure of critical thought perhaps due to a strong confirmation bias (the officer was very confident that Bill was guilty, thus failed to detect the nuance in the question).
    \\

    
      Example \#2: In the hilarious Broadway musical, {\it The Book of Mormon}, there is a musical number where one character is explaining how to bury "bad thoughts" by just "turning them off" (like a light switch). The character doing the explaining (in glorious song) is specifically explaining to the main character how to suppress gay thoughts when the main character's "bad thoughts" have nothing to do with being gay. After the instructions, the main character tries to make this clear by affirming, "{\it I'm} not having gay thoughts," to which the other characters respond "Hurray! It worked!"
    \\

    
      Explanation: The stress on the "I'm" was ignored and confused for "Hey, I'm {\it not} having gay thoughts anymore!" Although this was comedy it portrayed an argument.
    \\

    
      Tip: Our biases can cause us to miss the vocal nuance. Listen actively and critically, and try not to jump to conclusions. And you cannot turn off gay thoughts like a light switch.
    \\

  

Denying the Antecedent
    
      (also known as: inverse error, inverse fallacy)
    \\

  

if-by-whiskey fallacy
    Description: A response to a question that is contingent on the questioner’s opinions and makes use of words with strong connotations.  This fallacy appears to support both sides of an issue -- a tactic common in politics.

    
      Logical Form:
    \\

    
      {\em If you mean X, then (one-sided, loaded-language rant supporting side A).} \newline
{\em If you mean Y, then (one-sided, loaded-language rant supporting side B).}
    \\

    
      Example \#1: This example is actually the origin of the fallacy, which refers to a 1952 speech by Noah S. “Soggy” Sweat, Jr., a young lawmaker from the U.S. state of Mississippi, on the subject of whether Mississippi should continue to prohibit (which it did until 1966) or finally legalize alcoholic beverages.  I think it is hilarious, so I am including it here in its entirety.
    \\

    
      My friends, I had not intended to discuss this controversial subject at this particular time. However, I want you to know that I do not shun controversy. On the contrary, I will take a stand on any issue at any time, regardless of how fraught with controversy it might be. You have asked me how I feel about whiskey. All right, here is how I feel about whiskey:
    \\

    
      If when you say whiskey you mean the devil’s brew, the poison scourge, the bloody monster, that defiles innocence, dethrones reason, destroys the home, creates misery and poverty, yea, literally takes the bread from the mouths of little children; if you mean the evil drink that topples the Christian man and woman from the pinnacle of righteous, gracious living into the bottomless pit of degradation, and despair, and shame and helplessness, and hopelessness, then certainly I am against it.
    \\

    
      But, if when you say whiskey you mean the oil of conversation, the philosophic wine, the ale that is consumed when good fellows get together, that puts a song in their hearts and laughter on their lips, and the warm glow of contentment in their eyes; if you mean Christmas cheer; if you mean the stimulating drink that puts the spring in the old gentleman’s step on a frosty, crispy morning; if you mean the drink which enables a man to magnify his joy, and his happiness, and to forget, if only for a little while, life’s great tragedies, and heartaches, and sorrows; if you mean that drink, the sale of which pours into our treasuries untold millions of dollars, which are used to provide tender care for our little crippled children, our blind, our deaf, our dumb, our pitiful aged and infirm; to build highways and hospitals and schools, then certainly I am for it.
    \\

    
      This is my stand. I will not retreat from it. I will not compromise.
    \\

    
      Explanation: This is an amazing insight into the human mind and the area of rhetoric.  We can see how when both sides of the issue are presented through the same use of emotionally charged words and phrases, the argument is really vacuous and presents very little factual information, nor does it even take a stance on the issue.
    \\

    
      Example \#2: Having evaluated literally thousands of positions on God by people all over the belief spectrum, I thought I would create my own, “If-by-God” version of the argument, showing how carefully placed rhetoric can blur the line between the most perfect being imaginable and the most horrible being imaginable.
    \\

    
      The question is, if God does exist, should we love him and worship him?  My position is clear, and I am not embarrassed to let the world know exactly how I feel.  So here it goes.
    \\

    
      If by God you mean the great dictator in the sky, the almighty smiter, the God who created us with imperfections then holds us responsible for the imperfections, the God who took away paradise and eternal life from us because the first man and woman committed a “wrong” against God before they were capable of knowing right from wrong, the God who commanded his chosen people to utterly destroy every man, woman, and child in dozens of cities, the God who hardened hearts, killed first-borns, demanded blood sacrifices, commanded man to brutally kill other humans for “crimes” such as “not honoring your parents”, the God who destroyed virtually all living creatures on the planet, the God who would demand that his own son be brutally murdered to pay a debt to him, the God who allows children to be born with birth defects, die young, and get cancer, the God who continues to destroy using floods, hurricanes, and other natural disasters, the God who ignores the prayers of billions of his faithful followers, the God who allows a majority of his creation to suffer through unimaginable torture for all eternity in the fiery pits of Hell, then he is certainly not deserving of our love and worship.
    \\

    
      But, if when you say God you mean the defender, the protector, creator of heaven and earth, the father of us all, the being of pure love, kindness, and everything good in the world, the God who led the Israelites from slavery to freedom, the one who looks after us all, the God who heals the sick in his son’s name, the God who gave us his perfect laws for our benefit, the God who loved us so much, that he sacrificed his only son so that we can be saved, the God who allows us to spend a blissful eternity with him and our loved ones, then certainly he is deserving of our love and worship.
    \\

    
      This is my stand. I will not retreat from it. I will not compromise.
    \\

    
      Exception: If you are serving as a moderator and need to remain neutral, plus want to add a little “spice” in the debate, this might be a good technique.
    \\

    
      Fun Fact: Reportedly, the example \#1 speech took Sweat two and a half months to write
    \\

    References:

    
      
        
      \\

      
        
          Brookes, T. (1979). {\it Guitar: an American life}. Grove Press.
        
      
    
  

Fallacy of amphiboly
    
      (Also Known As: Ambiguity Fallacy, Fallacy of Ambiguity, Amphiboly Fallacy)
    \\

  
    
      - **Description:** The fallacy of amphiboly occurs when an argument is based on a statement that is ambiguous due to its grammatical structure. This type of ambiguity arises from the way words or phrases are arranged in a sentence, leading to multiple possible interpretations. The fallacy exploits this grammatical ambiguity to mislead or confuse the audience.
    \\

    
      - **Logical Form:**
    \\

    
        - **P1:** A statement with ambiguous grammatical structure is presented.
    \\

    
        - **P2:** The argument relies on one interpretation of the ambiguous statement.
    \\

    
        - **C:** The argument is misleading or invalid because the statement could be interpreted in multiple ways.
    \\

    
      - **Example \#1:** "The lawyer said he would help me get out of jail, but he was in fact a con artist."
    \\

    
      - **Explanation:** This statement could be interpreted in two ways: either the lawyer was genuinely offering legal help or the lawyer was deceitful. The ambiguity in the lawyer's intentions is exploited to mislead about the nature of their actions.
    \\

    
      - **Example \#2:** "The sign said 'No parking on the grass.' I parked my car on the gravel, which is next to the grass, so I didn't break any rules."
    \\

    
      - **Explanation:** The ambiguity here lies in the interpretation of the term "on the grass." The statement can be read as either strictly prohibiting parking on the grass itself or as implicitly allowing parking on other surfaces close to it. The argument relies on the ambiguous interpretation to avoid breaking the rule.
    \\

    
      - **Variation:** Amphiboly can involve not only ambiguous grammar but also vague or poorly structured statements that lead to multiple possible meanings.
    \\

    
      - **Tip:** To avoid falling for the fallacy of amphiboly, ensure that statements are clearly and unambiguously structured. When interpreting ambiguous statements, consider all possible meanings and clarify the intended interpretation.
    \\

    
      - **Exception:** Not all grammatical ambiguities constitute a fallacy. In some cases, ambiguity is intentional or serves a specific rhetorical purpose without misleading or confusing the audience.
    \\

    
      - **Fun Fact:** The term "amphiboly" derives from the Greek word "amphibolos," meaning "ambiguous" or "having double meaning," reflecting the fallacy's nature of exploiting linguistic ambiguity.
    \\

  \section{Moral equivalence}
\subsection{Fallacy of relative privation
    
      (also known as: appeal to worse problems. not as bad as, it could be worse, it could be better)
    \\

  
    Description: Trying to make a scenario appear better or worse by comparing it to the best or worst case scenario.

    
      Logical Forms:
    \\

    
      Scenario S is presented.
    \\

    
      Scenario B is presented as a best-case.
    \\

    
      Therefore, Scenario S is not that good.
    \\

    
       
    \\

    
      Scenario S is presented.
    \\

    
      Scenario B is presented as a worst-case.
    \\

    
      Therefore, Scenario S is very good.
    \\

    
      Example \#1:
    \\

    
      Be happy with the 1972 Chevy Nova you drive.  There are many people in this country who don’t even have a car.
    \\

    
      Explanation: This person does have a very crappy car by any reasonable standard.  Only comparing his situation with people who have no cars, does his Chevy Nova look like a Rolls Royce.  It is fallacious to make a reasonable judgment based on these extreme cases.
    \\

    
      Example \#2:
    \\

    
      Son: I am so excited!  I got an “A” on my physics exam!
    \\

    
      Dad:  Why not an “A+”?  This means that you answered something incorrectly.  That is not acceptable!
    \\

    
      Explanation: The poor kid is viewing his success from a very reasonable perspective based on norms.  However, the father is using a best case scenario as a comparison, or a very unreasonable perspective.  The conclusion “it is not acceptable,” is unreasonable and, therefore, fallacious.
    \\

    
      Exception: When used intentionally to manipulate emotions (especially with good intentions), not to make an argument on reason, then this might be acceptable.
    \\

    
      I know that you just lost your job, but at least you still have a great education and plenty of experience, which will help you get another job.
    \\

    
      Fun Fact: My first car was a crappy, 1972 Chevy Nova that I bought for \$50 in my sophomore year in high school. This was when I first learned about correlation. Me driving that car was strongly correlated with my lack of female companions.
    \\

  }


First World problems

Fallacy of Every and All
    Description: When an argument contains both universal quantifiers and existential quantifiers (all, some, none, every) with different meanings, and the order of the quantifiers is reversed. This is a specific form of{\it  equivocation. \newline
}

    
      Logical Form:
    \\

    
      {\em Quantifier X then quantifier Y.} \newline
{\em Therefore, quantifier Y then quantifier X.}
    \\

    
      Example \#1:
    \\

    
      {\em Everyone should do something nice for someone. I am someone, so everyone should do something nice for me!}
    \\

    
      Explanation: We have a reversal of the quantifiers. \newline

    \\

    
      {\em Everyone (quantifier X) should do something nice for someone (quantifier Y). I am someone (quantifier Y), so everyone (quantifier X) should do something nice for me!}
    \\

    
      Assuming one accepts the premise that "everyone should do something nice for someone," the word "someone" in that sentence means "some person or another" whereas in the second sentence it means "a specific person." By equivocating the meanings of "someone," we appear to be making a strong argument when in fact we are not.
    \\

    
      Example \#2:
    \\

    
      Everyone loves someone. \newline
I am someone. \newline
Therefore, everyone loves me!
    \\

    
      Explanation: “Someone” (quantifier X) referred to “any given person” in the first premise. In the second premise, “someone” is referring specifically to me. In the conclusion, we have a reversal of the quantifiers as they were presented in the premises.
    \\

    
      Exception: No exceptions.
    \\

    
      Tip: It really is a good idea to do something nice for someone.
    \\

    References:

    
      
        
      \\

      
        
          Salmon, M. H. (2012). {\it Introduction to Logic and Critical Thinking}. Cengage Learning.
        
      
    
  \section{Failure to Elucidate
    (also known as: obscurum per obscurius, Fallacies of definition)
  
    
      Description: When the definition is made more difficult to understand than the word or concept being defined.
    \\

    
      Logical Form:
    \\

    
      Person 1 makes a claim.
    \\

    
      Person 2 asks for clarification of the claim, or a term being used.
    \\

    
      Person 1 restates the claim or term in a more confusing way.
    \\

    
      Example \#1:
    \\

    
      Tracy: I don’t like him because of his aura.
    \\

    
      TJ: What do you mean by that?
    \\

    
      Tracy: I mean that he is projecting a field of subtle, luminous radiation that is negative.
    \\

    
      Explanation: This is such a common fallacy, yet rarely detected as one.  Usually, out of fear of embarrassment, we accept confusing definitions as legitimate elucidations, that is, we pretend the term that was defined is now clear to us.  What exactly is the field?  How is it detected? Are there negative and positive ones? How do we know?
    \\

    
      Example \#2:
    \\

    
      Linda: We live in a spirit-filled world; I am certain of that.
    \\

    
      Rob: What is a “spirit”?
    \\

    
      Linda: A noncorporeal substance.
    \\

    
      Explanation: Many times, we fool ourselves into thinking that because we know other words for the term, we better understand what the term {\it actually represents}.  The above example is an illustration of this.  We can redefine, “spirit” as many times as we like, but our understanding of what a spirit actually is will still be lacking.
    \\

    
      Assuming we did not really understand what was meant by “spirit”, the definition, “noncorporeal substance” might or might not shed any light on what is meant by the term.  In this case, it might be more clear now that Linda is not referring to alcoholic beverages, but conceptually, what is a non-physical substance?  If “substance” is defined as being physical matter or material, does a “non-physical” substance even make sense?
    \\

    
      We fallaciously reason that we now understand what the term represents when, in fact, we don’t.
    \\

    
      Exception: Some may actually just lack the vocabulary needed -- this is not your fault, but you should do your best to attempt to elucidate using words understandable to your audience.
    \\

    
      Tip: Failure to elucidate often results in endless and pointless debates. Take, for example, the common position that guns are not a problem in the USA. We often see this position presented as an unhelpful meme, such as "Guns don't kill people, people kill people." What this needs is a "therefore," followed by a conclusion. For example, "Guns don't kill people, people kill people. Therefore, we should be focusing more on what makes people use guns violently and less on just banning the use of guns." This is very different from, "Guns don't kill people, people kill people. Therefore, anyone should be allowed to carry any kind of gun with no restrictions." Don't waste your time imagining the argument or point the other person is trying to make—ask for clarification.
    \\

    References:

    
      
        
      \\

      
        
          Cederblom, J., \& Paulsen, D. (2011). {\it Critical Reasoning} (7 edition). Boston, MA: Wadsworth Publishing.
        
      
    
  }


Homunculus Fallacy
    
      (also known as: homunculus argument, infinite regress)
    \\

  

Definist fallacy
    
      (also known as: persuasive definition fallacy, redefinition)
    \\

  
    Description: Defining a term in such a way that makes one’s position much easier to defend.

    
      Logical Form:
    \\

    
      A has definition X.
    \\

    
      X is harmful to my argument.
    \\

    
      Therefore, A has definition Y.
    \\

    
      Example \#1:
    \\

    
      Before we argue about the truth of creationism, let’s define creationism as, “The acceptance of a set of beliefs even more ridiculous than those of flat-earthers.”
    \\

    
      Example \#2:
    \\

    
      Before we argue about the truth of creationism, let’s define evolution as, “Faith in a crackpot theory that is impossible to prove with certainty.”
    \\

    
      Explanation: It should be clear by the two examples who is defending what position.  Both arguers are taking the opportunity to define a term as a way to take a cheap shot at the opponent.  In some cases, they might actually hope their definition is accepted, which would make it very easy to defend, compared to the actual definition.
    \\

    
      Exception: When a definition used is really an accurate definition from credible sources, regardless of the damage it might do to a position.
    \\

    
      Tip: Do not accept definitions put forth by the opponent unless you researched your definition on your own, and agree.
    \\

    References:

    
      
        
      \\

      
        
          Bunnin, N., \& Yu, J. (2008). {\it The Blackwell Dictionary of Western Philosophy}. John Wiley \& Sons.
        
      
    
  

Extensional Pruning
    
      - Description: The fallacy of extensional pruning occurs when a person uses a word or phrase in a way that conforms to its commonly accepted meaning but then retreats to a strictly literal or technical definition when challenged. This allows the user to escape criticism or maintain a position that would otherwise be untenable.
    \\

    
      
    \\

    
      - Logical Form:
    \\

    
        1. User presents a term or phrase with a broad, commonly understood meaning.
    \\

    
        2. When challenged, user insists on a strict, literal definition of the term.
    \\

    
        3. This retreat to a narrow definition avoids addressing the actual implications or expectations associated with the term.
    \\

    
      
    \\

    
      - Example \#1:
    \\

    
        - Scenario: "While I said I would accept an inquiry, I at no time said that it would be independent, that it would be a public one, or that its findings would be published."
    \\

    
        - Explanation: The speaker may have agreed to an "inquiry" in a general sense, but by focusing on a strictly literal definition, they evade the broader expectations of what an inquiry typically involves.
    \\

    
      
    \\

    
      - Example \#2:
    \\

    
        - Scenario: "All we said was that we’d install a switchboard. We didn’t say it would work."
    \\

    
        - Explanation: The speaker uses a technical definition of "install" to escape accountability for the switchboard’s functionality. While they may have technically fulfilled the promise, they have avoided the implied responsibility for operational performance.
    \\

    
      
    \\

    
      - Variation:
    \\

    
        - Scenario: "We’ll take your one-year-old car as a trade-in, at whatever you paid for it."
    \\

    
        - Explanation: The promise appears generous, but by using a technical definition that excludes the tax or additional costs, the offer is less favorable than it seems.
    \\

    
      
    \\

    
      - Tip: To avoid falling into extensional pruning, ensure that your promises or statements are clear and comprehensive. When evaluating someone else's claims, consider both the commonly understood meaning and any potential technical definitions they might use to narrow their commitment.
    \\

    
      
    \\

    
      - Exception: Extensional pruning is not inherently fallacious if it is clear from the context that a literal definition is intended. If the terms are explicitly defined at the outset, then shifting to that definition when challenged is not deceptive.
    \\

    
      
    \\

    
      - Fun Fact: The fallacy of extensional pruning often appears in advertising and legal contexts, where the fine print or technical details can significantly alter the perceived value or scope of a promise. This technique is particularly prevalent in industries where ambiguity can be strategically used to limit liability.
    \\

  
    
      (Also known as: Semantic Pruning, Definitional Retreat)
    \\

  

Proof by Intimidation

Deepity

Jingle-jangle fallacies
    
      - **Description:** Jingle-jangle fallacies occur when different concepts are mistakenly considered the same due to shared labels (jingle fallacy) or when the same concept is mistakenly treated as different because of varied labels (jangle fallacy). These fallacies lead to confusion and misinterpretation in research and discussions by conflating or differentiating concepts based on superficial naming conventions.
    \\

    
      - **Logical Form:**
    \\

    
        - **Jingle Fallacy:**
    \\

    
          - **P1:** Different concepts or phenomena are given the same name.
    \\

    
          - **P2:** This leads to the assumption that these concepts are identical or closely related.
    \\

    
          - **C:** The different concepts are wrongly treated as the same due to their shared label.
    \\

    
        - **Jangle Fallacy:**
    \\

    
          - **P1:** The same concept or phenomenon is given different names.
    \\

    
          - **P2:** This leads to the assumption that these different names represent different concepts or phenomena.
    \\

    
          - **C:** The single concept is wrongly treated as multiple distinct concepts due to its varied labels.
    \\

    
      - **Example \#1:** Labeling both "self-esteem" and "self-confidence" as the same construct in psychological research, even though they are distinct constructs.
    \\

    
      - **Explanation:** The jingle fallacy is at work here, as the shared label causes researchers to overlook the distinct nature of self-esteem and self-confidence, leading to confusion in understanding and measuring these constructs.
    \\

    
      - **Example \#2:** Treating "cognitive flexibility" and "adaptive thinking" as different constructs simply because they are referred to by different terms, despite both referring to the same underlying ability to adapt one's thinking to new information.
    \\

    
      - **Explanation:** The jangle fallacy is evident here, as the different labels cause the mistaken belief that these terms represent different concepts, while they actually refer to the same cognitive ability.
    \\

    
      - **Variation:** Jingle-jangle fallacies can manifest in various fields, including psychology, education, and social sciences, where terminology and conceptual definitions are crucial for accurate research and theory development.
    \\

    
      - **Tip:** To avoid jingle-jangle fallacies, carefully define and distinguish concepts at the outset of research and ensure that terminology is consistently applied and understood within the context of the study.
    \\

    
      - **Exception:** Some level of conceptual overlap is sometimes unavoidable, and minor variations in terminology may be used to capture different facets of a broad concept. However, it is essential to clarify these distinctions to avoid confusion.
    \\

    
      - **Fun Fact:** The term "jingle-jangle fallacy" was coined by psychologist Campbell in the 1950s to describe how misleading terminology can impact psychological research and theory.
    \\

  
    
      (Also Known As: Jingle-Jangle Effect, Jingle Fallacy, Jangle Fallacy)
    \\

  \section{Ambiguity fallacy
    
      (also known as: ambiguous assertion, amphiboly, amphibology, semantical ambiguity, vagueness)
    \\

  
    Description: When an unclear phrase with multiple definitions is used within the argument; therefore, does not support the conclusion.  Some will say single words count for the ambiguity fallacy, which is really a specific form of a fallacy known as {\it equivocation}.

    
      Logical Form:
    \\

    
      {\em Claim X is made.}
    \\

    
      {\em Y is concluded based on an ambiguous understanding of X.}
    \\

    
      Example \#1:
    \\

    
      {\em It is said that we have a good understanding of our universe.  Therefore, we know exactly how it began and exactly when.}
    \\

    
      Explanation: The ambiguity here is what exactly “good understanding” means.  The conclusion assumes a much better understanding than is suggested in the premise; therefore, we have the {\it ambiguity fallacy}.
    \\

    
      Example \#2:
    \\

    
      {\em All living beings come from other living beings.  Therefore, the first forms of life must have come from a living being.  That living being is God.}
    \\

    
      Explanation: This argument is guilty of two cases of ambiguity.  First, the first use of the phrase, “come from”, refers to {\it reproduction}, whereas the second use refers to {\it origin}.  The fact that we know quite a bit about reproduction is irrelevant when considering origin.  Second, the first use of, “living being”, refers to an empirically verifiable, biological, living organism.  The second use of, “living being”, refers to a belief in an immaterial god.  As you can see, when a term such as, “living being”, describes a Dodo bird as well as the all-powerful master of the universe, it has very little meaning and certainly is not specific enough to draw logical or reasonable conclusions.
    \\

    
      Example \#3:
    \\

    
      {\em Bernice: Do you support Black Lives Matter?} \newline
{\em Mildred: Of course, I do!} \newline
{\em Bernice: Then you should support me looting that store.} \newline
{\em Mildred: Wait... what?}
    \\

    
      Explanation: “Black Lives Matter” is a political ideology that has evolved from a simple declaration. Like all ideologies, it has many branches ranging from moderate to extreme. Bernice clearly has a different concept of Black Lives Matter than Mildred does. An argument can be had over who is perverting the ideology (we have seen this among different Christians for millennia), but the point remains that what is meant by “supporting Black Lives Matter” has become unclear.
    \\

    
      Exception: Ambiguous phrases are extremely common in the English language and are a necessary part of informal logic and reasoning.  As long as these ambiguous phrases mean the same thing in all uses of phrases in the argument, this fallacy is not committed.
    \\

    
      Tip: Don’t be afraid to ask for clarification, especially if the alternative is to assume your interlocutor is being unreasonable or deceptive.
    \\

  }


Fake precision
    
      (also known as: overprecision, false precision, misplaced precision, spurious accuracy)
    \\

  
    Description: Using implausibly precise statistics to give the appearance of truth and certainty, or using a negligible difference in data to draw incorrect inferences.

    
      Logical Forms:
    \\

    
      {\em Statistic X is unnecessarily precise and has probability A of being true.} \newline
{\em Statistic X is interpreted as having probability A+B as being true.}
    \\

    
      {\em Statistic X represents position A.} \newline
{\em Statistic Y represents position B.} \newline
{\em Statistic Y is insignificantly different from statistic X.} \newline
{\em Position A is seen as significantly different from position B.}
    \\

    
      Example \#1:
    \\

    
      {\em Tour Guide: This fossil right here is 120,000,003 years old. \newline
Guest: How do you know that? \newline
Tour Guide: Because when I started working here three years ago, the experts did some radiometric dating and told me that it was 120,000,000 years old.}
    \\

    
      Explanation: Although more of a comedy skit than anything else, this demonstrates the fallacious reasoning by the tour guide in her assumption that the dates given to her were precise to the year.
    \\

    
      Example \#2: 
    \\

    
      The difference between first and second in many cases is negligible, statistically, yet we give those differences artificial meaning.  Who was the {\it second} person to walk on the moon... just minutes after Neil Armstrong?  Does anyone remember who the {\it second}  fastest man in the world is, even though he might come in .01 seconds after the first place winner?
    \\

    
      Explanation:  We often artificially assign significant meaning to tiny statistical differences.  It is a fallacy when we infer that the first place runner is “much faster” than the second place runner, even though the difference is .01 seconds.
    \\

    
      Exception: In reality, tiny statistical differences can have a significant impact, regardless of our interpretation.  For example, jumping out of the way of a car .01 seconds too late can mean the difference between a close call, and death.
    \\

    
      Tip: Don’t confuse fake precision with real performance.
    \\

  

Moving the goalposts
    
      (also known as: gravity game, raising the bar, argument by demanding impossible perfection [form of], Shifting ground, Changing the Subject)
    \\

  
    
      Description: Demanding from an opponent that he or she address more and more points after the initial counter-argument has been satisfied refusing to concede or accept the opponent’s argument.
    \\

    
      Logical Form:
    \\

    
      Issue A has been raised, and adequately answered.
    \\

    
      Issue B is then raised, and adequately answered.
    \\

    
      .....
    \\

    
      Issue Z is then raised, and adequately answered.
    \\

    
      (despite all issues adequately answered, the opponent refuses to conceded or accept the argument.
    \\

    
      Example \#1:
    \\

    
      Ken: There has to be an objective morality because otherwise terms like “right” and “wrong” would be meaningless since they have no foundation for comparison. 
    \\

    
      Rob: The terms “right” and “wrong” are based on cultural norms, which do have a subjective foundation -- one that changes as the moral sphere of the culture changes.  The term “heavy” does not have an objective standard, yet we have no problem using that term in a meaningful way.  In fact, very few relational terms have any kind of objective foundation.
    \\

    
      Ken: But without an objective morality, we would all be lost morally as a race.
    \\

    
      Rob: Many would say that we are.
    \\

    
      Ken: But how can you say that torturing children for fun is morally acceptable in any situation?
    \\

    
      Rob: Personally, I wouldn’t, but you are implying that anything that is not objective must necessarily be seen in all possible ways. A feather may not be seen as “heavy” to anyone, but that doesn’t mean its “lightness” is still not relative to other objects.
    \\

    
      Ken: But God is the standard of objective morality.  Prove that wrong!
    \\

    
      Rob: That I cannot do.
    \\

    
      Explanation: Ken starts with a statement explaining why he thinks there{\it  has to be} an objective morality -- a statement based on a reasonable argument that can be pursued with reason and logic.  Rob adequately answers that objection, as indicated by Ken’s move away from that objection to a new objection.  This pattern continues until we arrive at an impossible request.  Despite all the objections being adequately answered, at no time does Ken concede any points or abandon the argument.
    \\

    
      Example \#2: Perhaps the most classic example of this fallacy is the argument for the existence of God.  Due to the understanding of nature through science, many of the arguments that used to be used for God (or gods) were abandoned, only to be replaced with new ones, usually involving questions to which science has not definitively answered yet.  The move from creationism to {\it intelligent design} is a prime example.  Currently the origin of life is a popular argument for God (although a classic {\it argument from ignorance}), and an area where we very well may have a scientific answer in the next decade, at which time, the “origin of life” argument will fade away and be replaced by another, thus moving the figurative goalposts farther back as our understanding of the natural world increases.
    \\

    
      Exception: This fallacy should not be confused with an argument or set of arguments, with multiple propositions inherent in the argument.  The reason for the difference between this kind of argument and the {\it moving the goalposts fallacy}, is a subtle one, but indicated by a strong initial claim (“has to be”, “must”, “required for”, etc.) that gets answered and/or what appears to be {\it ad hoc} objections that follow eventually leading to an impossible request for proof.
    \\

    
      Fun Fact: The name “moving the goalposts” comes from the analogy of kicking a perfect field goal in American football, only to have the goalposts be moved on you. This would be very unfair.
    \\

  

Syntactic ambiguity

Wronger than wrong

Type-Token Fallacy
    Description: The type-token fallacy is committed when a word can refer to either a type (representing an abstract descriptive concept) or a token (representing an object that instantiates a concept) and is used in a way that makes it unclear which it refers to. This is a more specific form of the {\it ambiguity fallacy}.

    
      Logical Forms:
    \\

    
      {\em Reference to type is made.} \newline
{\em Response refers to token.}
    \\

    
      {\em Reference to token is made.} \newline
{\em Response refers to type.}
    \\

    
      Example \#1:
    \\

    
      {\em Salesperson: Toyota manufactures like four dozens of cars, so if you don't like this one you can see others. \newline
Prospect: I would have guessed they made closer to millions of cars.}
    \\

    
      Explanation: The salesperson was referring to the different types of cars (models) Toyota makes, not how many instances (or tokens) of each car were manufactured. By not specifically stating "types of cars" or "models," the statement was ambiguous and unnecessarily confusing.
    \\

    
      Example \#2:
    \\

    
      {\em Greg: I have the same suit as George Clooney.} \newline
{\em Tim: Do you guys take turns wearing it?}
    \\

    
      Explanation: Greg means that he had the same type of suit as George Clooney. Tim was probably being a smart-ass with his response, but in case he wasn’t, he confused the type with the token (that unique suit).
    \\

    
      Tip: As always, be as clear in your communication as possible and avoid any unnecessary confusion.
    \\

    References:

    
      
        
      \\

      
        
          Wetzel, L. (2014). Types and tokens. In E. N. Zalta (Ed.), {\it The Stanford Encyclopedia of Philosophy} (Spring 2014). Metaphysics Research Lab, Stanford University. Retrieved from https://plato.stanford.edu/archives/spr2014/entriesypes-tokens/
        
      
    
  \section{Mistaking the map for the territory}


Reification (fallacy)
    
      (also known as: abstraction, concretism, fallacy of misplaced concreteness, hypostatisation, pathetic fallacy [form of], hypostatisation)
    \\

  
    Description: When an abstraction (abstract belief or hypothetical construct) is treated as if it were a concrete, real event or physical entity -- when an idea is treated as if had a real existence.

    
      Logical Form:
    \\

    
      {\em Abstraction X is treated as if it were concrete, a real event, or a physical entity. \newline
Because the abstraction is seen as if it were concrete, a real event, or a physical entity, the conclusion is true.}
    \\

    
      Example \#1:
    \\

    
      {\em Dr. Simmons: I am working on a way to lengthen the human lifespan to about 200 years. \newline
Misty: You are declaring war on Mother Nature, and Mother Nature always wins!}
    \\

    
      Explanation: Here, “Mother Nature” is being portrayed as an autonomous being capable of going to war with humanity. If this were the case, it would seem that messing with Mother Nature is futile. In reality, we are part of nature and can and always have changed nature, sometimes for the worse, but often for the better.
    \\

    
      Example \#2:
    \\

    
      {\em If you are open to it, love will find you.}
    \\

    
      Explanation:  Love is an abstraction, not a little fat flying baby with a bow and arrow that searches for victims.  Cute sayings such as this one can serve as bad advice for people who would otherwise make an effort to find a romantic partner, but choose not to, believing that this "love entity" is busy searching for his or her ideal mate.
    \\

    
      Exception: In most cases, even in the above examples, these are used as rhetorical devices. When the reification is deliberate and harmless, and not used as evidence to support a claim or conclusion, then it is not fallacious.
    \\

    
      {\em It’s time to grab my future by the balls.}
    \\

    
      The future is an abstraction. It does not have testicles. If it did, you probably wouldn’t want to grab them because your future might sue you for sexual misconduct.
    \\

    
      Variation: The {\it pathetic fallacy} is the treatment of inanimate objects as if they had human feelings, thought, or sensations.  Think of cursing at your computer when it does not give you the results you expect.
    \\

    
      Fun Fact: {\em Reification} is similar to {\it anthropomorphism}{\em ,}  except that {\em reification} does not have to deal with human qualities.
    \\

    References:

    
      
        
      \\

      
        
          reification | literature | Britannica.com. (n.d.). Retrieved from https://www.britannica.com/topic/reification
        
      
    
  

Anthropomorphism
    
      (also known as: personification)
    \\

  
    Description: The attributing of human characteristics and purposes to inanimate objects, animals, plants, or other natural phenomena, or to gods. This becomes a logical fallacy when used within the context of an argument.

    
      Logical Form:
    \\

    
      Non-human thing is described with human characteristics.
    \\

    
      Claim X is made that requires the human characteristics of the thing.
    \\

    
      Therefore, claim X is true.
    \\

    
      Example \#1:
    \\

    
      How dare you murder those carrots!
    \\

    
      Explanation: Murder applies to humans, not carrots. By definition, one cannot murder carrots. In this example, the carrots are assumed to have the human characteristics that make murder “murder” and not just “killing” or “eating.”
    \\

    
      Example \#2:
    \\

    
      Akoni: The Polynesian fire goddess, Pele, sacrificed her own daughter in the volcano to bring peace to the islands. This is how I know she loves us.
    \\

    
      Ubon: Aren't the gods and goddesses immortal?
    \\

    
      Akoni: Err, yes.
    \\

    
      Ubon: Then what happened to Pele's daughter after she was thrown in the volcano?
    \\

    
      Akoni: She was reunited with Pele in the heavens.
    \\

    
      Ubon: So why was this a sacrifice?
    \\

    
      Explanation: The goddess, Pele, and her daughter are given the human quality of the frailty of life in order for the concept of “sacrifice” to be meaningful.
    \\

    
      Tip: If you want to make people laugh, give objects and small animals human qualities. Anthropomorphism is a comedy goldmine for stand-up comedians.
    \\

  

Tokenism
    
      
        Description: Interpreting a token gesture as an adequate substitute for the real thing.
      \\

      
        Logical Form:
      \\

      
        {\em Problem X exists.} \newline
{\em Solution Y is offered.} \newline
{\em Solution Y is inadequate to solve problem X but accepted as adequate.}
      \\

      
        Example \#1:
      \\

      
        {\em The presidential nominee has been accused of being racist.  But he recently stated that he really liked the movie, “Roots,” so I guess he isn’t racist.}
      \\

      
        Explanation: Liking one movie that exposes racism and encourages equality, is far from the same as not being a racist.
      \\

      
        Example \#2:
      \\

      
        {\em Mr. McBoss' company consists of 50 executives who are all men, and 50 secretaries who are all women.  To show he is all about equal opportunity, he has agreed to hire a woman executive.}
      \\

      
        Explanation: This "token" gesture does not come close to making up for the disproportionate hiring practices of Mr. McBoss' company.
      \\

      
        Example \#3: In the summer of 2020 when racial tensions were high, many media sources did their part to solve racism by capitalizing the “B” when referring to black people while keeping the “w” lowercase when referring to white people. Some sources that picked up this story saw this as a token gesture to help shield them from public claims of racism.
      \\

      
        Exception: If a token gesture is seen as a token, and not as an adequate substitute, it is not a fallacy.
      \\

      
        {\em I know I have a weight problem, and I am trying.  So far, I have replaced my usual breakfast of doughnuts with a single grapefruit.}
      \\

      
        Fun Fact: Some attempts to solve racism are Stupid with a capital “S.”
      \\

    
    References: Cogan, R. (1998). Critical Thinking: Step by Step. University Press of America.
  
    
      (also known as: Type-token fallacy)
    \\

  

Use-Mention Error
    
      (also known as: UME)
    \\

  
    
      Description:  Confusing the word used to describe a thing, with the thing itself.  To avoid this error, it is customary to put the word used to describe the thing in quotes.
    \\

    
      This fallacy is most common when used as an {\it equivocation}.
    \\

    
      Logical Form:
    \\

    
      “X” is the same as X.
    \\

    
      Example \#1:
    \\

    
      My son is made up of five letters.
    \\

    
      Example \#2:
    \\

    
      Tyrone: I am a sophisticated word genius. \newline
Suzie: Prove it. Define some. \newline
Tyrone: An unspecified amount or number of. Proven!
    \\

    
      Example \#3: Many podcast hosts, journalists, and other public figures who discuss issues of race have got themselves in serious trouble by reporting on people who use the “N-word,” by actually using the word themselves (I won’t use it here because I don’t want to be one of those casualties). This is because a small, but vocal, group view the mention of the word as nefarious as the use of the word, when the two are substantially different.
    \\

    
      Explanation: Suzie meant to “define some words” but Tyrone defined the word “some.” Tyrone thinks he won the exchange but he did not really define any sophisticated words.
    \\

    
      Exception: When this “fallacy” is used in humor and riddles.
    \\

    
      Fun Fact: What part of London is in France? The letter “n.” 
    \\

    
      Explanation: The words (mention), “my son”, are made up of five letters.  My son (use) is made up of molecules.
    \\

    
      References:
    \\

    
      
        
      \\

      
        
          Azzouni, J. (2010). {\it Talking About Nothing: Numbers, Hallucinations and Fictions}. Oxford University Press.
        
      
    
  \chapter{
    
      {\bf Correlative-based fallacies}
    \\

  }
\section{False dilemma
    
      (also known as: all-or-nothing fallacy, false dichotomy [form of], the either-or fallacy, either-or reasoning, fallacy of false choice, fallacy of false alternatives, black-and-white thinking, the fallacy of exhaustive hypotheses, bifurcation, excluded middle, no middle ground, polarization, the bogus dilemma)
    \\

  
    
      Description: When only two choices are presented yet more exist, or a spectrum of possible choices exists between two extremes.  False dilemmas are usually characterized by “either this or that” language, but can also be characterized by omissions of choices.  Another variety is the false trilemma, which is when three choices are presented when more exist.
    \\

    
      Logical Forms:
    \\

    
      Either X or Y is true.
    \\

    
       
    \\

    
      Either X, Y, or Z is true.
    \\

    
      Example (two choices):
    \\

    
      You are either with God or against him.
    \\

    
      Explanation: As Obi-Wan Kenobi so eloquently puts it in {\it Star Wars episode III}, “Only a Sith deals in absolutes!”  There are also those who simply don’t believe there is a God to be either with or against.
    \\

    
      Example (omission):
    \\

    
      I thought you were a good person, but you weren’t at church today.
    \\

    
      Explanation: The assumption here is that if one doesn't attend chuch, one {\it must} be bad.  Of course, good people exist who don’t go to church, and good church-going people could have had a really good reason not to be in church -- like a hangover from the swingers' gathering the night before.
    \\

    
      Exception: There may be cases when the number of options really is limited.  For example, if an ice cream man just has chocolate and vanilla left, it would be a waste of time insisting he has mint chocolate chip. 
    \\

    
      It is also not a fallacy if other options exist, but you are not offering other options as a possibility.  For example:
    \\

    
      Mom: Billy, it’s time for bed.
    \\

    
      Billy: Can I stay up and watch a movie?
    \\

    
      Mom: You can either go to bed or stay up for another 30 minutes and read.
    \\

    
      Billy: That is a false dilemma!
    \\

    
      Mom: No, it’s not.  Here, read Bo’s book and you will see why.
    \\

    
      Billy: This is freaky, our exact conversation is used as an example in this book!
    \\

    
      Tip: Be conscious of how many times you are presented with false dilemmas, and how many times you present yourself with false dilemmas.
    \\

    
      Variation: Staying true to the definitions, the {\it false dilemma}  is different from the {\it false dichotomy} in that a dilemma implies two equally unattractive options whereas a dichotomy generally comprises two opposites. This is a fine point, however, and is generally ignored in common usage.
    \\

    References:

    
      
        
      \\

      
        
          Moore, B. N., \& Parker, R. (1989). {\it Critical thinking: evaluating claims and arguments in everyday life}. Mayfield Pub. Co.
        
      
    
  }
\subsection{Nirvana fallacy
    
      (also known as: perfect solution fallacy, perfectionist fallacy)
    \\

  
    Description: Comparing a realistic solution with an idealized one, and discounting or even dismissing the realistic solution as a result of comparing to a “perfect world” or impossible standard, ignoring the fact that improvements are often good enough reason.

    
      Logical Form:
    \\

    
      X is what we have.
    \\

    
      Y is the perfect situation.
    \\

    
      Therefore, X is not good enough.
    \\

    
      Example \#1:
    \\

    
      What’s the point of making drinking illegal under the age of 21?  Kids still manage to get alcohol.
    \\

    
      Explanation: The goal in setting a minimum age for drinking is to deter underage drinking, not abolish it completely.  Suggesting the law is fruitless based on its failure to abolish underage drinking completely, is fallacious.
    \\

    
      Example \#2:
    \\

    
      What’s the point of living?  We’re all going to die anyway.
    \\

    
      Explanation: There is an implication that the goal of life is not dying.  While that is certainly a worthwhile goal, many would argue that it is a bit empty on its own, creating this fallacy where one does not really exist.
    \\

    
      Exception: Striving for perfection is not the same as the {\it nirvana fallacy}.  Having a goal of perfection or near perfection, and working towards that goal, is admirable.  However, giving up on the goal because perfection is not attained, despite major improvements being attained, is fallacious.
    \\

    
      Tip: Sometimes good enough is really good enough.
    \\

    References:

    
      
        
      \\

      
        
          {\it George Mason University law review}. (1991).
        
      
    
  }


You are either with us, or against us

Unobtainable perfection fallacy
    
      - Description: This fallacy occurs when an argument is rejected solely because it is not perfect, even though no alternative is perfect either. It involves criticizing an option for its imperfections rather than comparing it to the available alternatives, which may also have flaws.
    \\

    
      
    \\

    
      - Logical Form:
    \\

    
        1. Option A is not perfect.
    \\

    
        2. No other available options are perfect either.
    \\

    
        3. Therefore, Option A should be rejected.
    \\

    
      
    \\

    
      - Example \#1:
    \\

    
        - Scenario: "We should ban the generation of nuclear power because it can never be made completely safe."
    \\

    
        - Explanation: This argument rejects nuclear power due to its imperfections (e.g., safety issues) without considering that other energy sources like coal or oil also have significant risks. The decision should be based on a comparative assessment of all options, not just the flaws of one.
    \\

    
      
    \\

    
      - Example \#2:
    \\

    
        - Scenario: "I'm against going to the Greek islands because we cannot guarantee we would enjoy ourselves there."
    \\

    
        - Explanation: This objection assumes that if perfect enjoyment cannot be guaranteed, the option should be discarded. It fails to acknowledge that no travel destination can guarantee complete enjoyment, and the comparison should be made based on overall potential benefits versus risks.
    \\

    
      
    \\

    
      - Variation:
    \\

    
        - Scenario: "We must ban the new heart drug because it has been occasionally associated with neurological disorders."
    \\

    
        - Explanation: The argument highlights the drug's imperfections without considering the potential benefits of saving lives from heart disease. The fallacy lies in focusing on one option's flaws without comparing it to the status quo.
    \\

    
      
    \\

    
      - Tip: To avoid the unobtainable perfection fallacy, evaluate options based on their relative merits and practical benefits rather than insisting on an unattainable ideal. Consider how each option compares to the current situation and other available alternatives.
    \\

    
      
    \\

    
      - Exception: If a proposed option is flawed to such an extent that it would cause more harm than benefit or is fundamentally unfeasible, then it may justifiably be rejected. However, this should be based on realistic assessments and not an unattainable standard of perfection.
    \\

    
      
    \\

    
      - Fun Fact: The unobtainable perfection fallacy is often used in political and public debate to undermine proposed changes by focusing on their imperfections. This technique can be effective in stalling progress and maintaining the status quo, even if the status quo itself is imperfect.
    \\

  
    
      (Also known as: Perfect Solution Fallacy, Ideal Solution Fallacy)
    \\

  

Oversimplified Cause Fallacy
    Description: When a contributing factor is assumed to be the cause, or when a complex array of causal factors is reduced to a single cause. It is a form of simplistic thinking that implies something is either a cause, or it is not. It overlooks the important fact that, especially when referring to human behavior, causes are very complex and multi-dimensional.

    
      Logical Form:
    \\

    
      X is a contributing factor to Y. \newline
X and Y are present. \newline
Therefore, to remove Y, remove X.
    \\

    
      Example \#1:
    \\

    
      P1. Lead poisoning can contribute to violent behavior.
    \\

    
      P2. Many inner city children have dangerous levels of lead in their blood. \newline
C. Therefore, violent crime in the inner city can be solved  by curing the lead problem.
    \\

    
      Explanation: We already established that lead poisoning {\it can contribute} to violent behavior (note the probabilistic language). This means that there is some unspecified chance. We are taking an unreasonable leap in suggesting that violent crime can be {\it solved}  (binary language) by curing the lead problem. And, in case you missed it, there is {\it question begging} here in assuming that violent {\it behavior} leads to violent {\it crime}.
    \\

    
      Example \#2:
    \\

    
      P1. A sedentary lifestyle contributes to obesity. \newline
P2. People have become more sedentary in the last few decades. \newline
C. Therefore, the rise in obesity can be fixed by people getting more exercise.
    \\

    
      Explanation: We made the leap from "contributes" to "can be fixed." At best, we can conclude that the problem of obesity can be {\it mitigated}  by people getting more exercise.
    \\

    
      Example \#3:
    \\

    
      P1. Smoking has been empirically proven to cause lung cancer. \newline
C. Therefore, if we eradicate smoking, we will eradicate lung cancer.
    \\

    
      Explanation: Even though it is reasonable to consider smoking a "cause" of lung cancer versus a "contributing factor" to lung cancer, assuming it is the only cause is fallacious.
    \\

    
      Tip: Establishing causality is extremely tricky. Unless you are stating an established fact, start using more probabilistic language such as "contributes to," "leads to," "has been known to reduce the effects of," or similar language.
    \\

    References: Hurley, P. J. (2011). A Concise Introduction to Logic. Cengage Learning.
  
    
      {\bf denying the correlative}
    \\

  
    
      (also known as: denying the correlative conjunction)
    \\

  
    Description: Introducing alternatives when, in fact, there are none.  This could happen when you have two mutually exclusive statements ({\it correlative conjunction}) presented as choices, and instead of picking one or the other, introduce a third -- usually as  a distraction from having to choose between the two alternatives presented. 

    
      Logical Form:
    \\

    
      Either X or not X.
    \\

    
      Therefore, Y.
    \\

    
      Example \#1:
    \\

    
      Rocco:  Do ya have the five grand you owe me or not?
    \\

    
      Paulie: I can get it.
    \\

    
      Rocco: That means you don’t have it?
    \\

    
      Paulie: I know someone who does.
    \\

    
      Rocco: Read my lips: do you have my money or not?
    \\

    
      Paulie: No.
    \\

    
      (sound of a baseball bat breaking kneecaps)
    \\

    
      Explanation: Rocco was asking a simple question, and out of personal safety, Paulie was committing the fallacy of {\it denying the correlative} by offering up another option to a choice that only had two.  If Paulie were smarter, he could not have committed the fallacy and saved his kneecaps, by honesty and a little negotiation:
    \\

    
      Rocco:  Do ya have the five grand you owe me or not?
    \\

    
      Paulie: No. I realize I did not hold up my end of the deal, so I will compensate you for that.
    \\

    
      Rocco: What are you sayin’?
    \\

    
      Paulie: I can have your \$5000 by this time tomorrow, plus an extra \$500 for making you have to wait an extra day.
    \\

    
      Rocco: Deal.  I’ll be back this time tomorrow.
    \\

    
      (sound of heart dropping from throat)
    \\

    
      Example \#2:
    \\

    
      Judge: So did you kill your landlord or not?
    \\

    
      Kirk: I fought with him.
    \\

    
      Explanation: Here is a classic case where a “yes” or “no” answer is expected, and the only acceptable answer to such a question, yet Kirk is deflecting the question by providing a third answer option, that leaves the original question unanswered.
    \\

    
      Exception: When non-mutually exclusive choices are presented as mutually exclusive choices, the fallacy lies with the one presenting the choices ({\it false dilemma} ).
    \\

    
      Tip: Don’t borrow money from anyone named "Rocco."
    \\

  

Suppressed correlative
    
      (also known as: fallacy of lost contrast, fallacy of the suppressed relative)
    \\

  
    Description: The attempt to redefine a {\it correlative} (one of two mutually exclusive options) so that one alternative encompasses the other, i.e. making one alternative impossible. The redefinition, therefore, makes the word it is redefining essentially meaningless.

    
      Logical Form:
    \\

    
      Person 1 claims that all things are either X or not X (the correlatives: X–not X).
    \\

    
      Person 2 defines X such that all things that you claim are not X are included in X (the suppressed correlative: not X).
    \\

    
      Example \#1:
    \\

    
      Rick: I need to know if we should stop for lunch or not.  You are either hungry or not hungry, which is it?
    \\

    
      Tina: If being hungry is being able to eat, I am always hungry.
    \\

    
      Explanation: If we redefine hungry as, “being able to eat” then except for the few occasions where people are medically incapable, everyone is always hungry, and it has lost all meaning.
    \\

    
      Example \#2:
    \\

    
      Kent: My new car is really fast.
    \\

    
      Cal: I doubt that it is as fast as a jet fighter so, therefore, it is not fast.
    \\

    
      Explanation: In Kent’s statement, there is an implied correlative, that is, his car is either fast or not fast.  Now if what Cal says is true, then no cars would ever be considered “fast,” and speed would lose all meaning for cars.
    \\

    
      Exception: Refusing to give into a {\it false dilemma} is not the same as committing the {\it suppressed correlative} fallacy.  In example \#1, while one cannot be both hungry and not hungry, one can be a little bit hungry.
    \\

    
      Rick: I need to know if we should stop for lunch or not.  You are either hungry or not hungry, which is it?
    \\

    
      Tina: I am a little bit hungry, so go ahead and stop if you are hungry otherwise I can wait.
    \\

    
      Note that this fallacy is not committed because Tina did not attempt to redefine “hungry” so “not hungry” is essentially impossible.
    \\

    
      Tip: Whenever presented with just two options, take a moment to consider if those are actually your only two options.
    \\

    References:

    
      
        
      \\

      
        
          Shafer-Landau, R. (2007). {\it Ethical Theory: An Anthology}. John Wiley \& Sons.
        
      
    
  \chapter{Formal fallacies}
\section{
    
      {\bf Quantification fallacies}
    \\

  }


Existential fallacy
    (also known as: existential instantiation, the Fallacy of Existential Assumption)
  
    Description: A formal logical fallacy, which is committed when a categorical syllogism employs two universal premises (“all”) to arrive at a particular (“some”) conclusion.

    
      In a valid categorical syllogism, if the two premises are universal, then the conclusion {\it must} be universal, as well.
    \\

    
      The reasoning behind this fallacy becomes clear when you use classes without any members, and the conclusion states that there are members of this class -- which is wrong.
    \\

    
      Logical Form:
    \\

    
      All X are Y.
    \\

    
      All Z are X.
    \\

    
      Therefore, some Z are Y.
    \\

    
      Example \#1: 
    \\

    
      All babysitters have pimples.
    \\

    
      All babysitter club members are babysitters.
    \\

    
      Therefore, some babysitter club members have pimples.
    \\

    
      Example \#2: 
    \\

    
      All forest creatures live in the woods.
    \\

    
      All leprechauns are forest creatures.
    \\

    
      Therefore, some leprechauns live in the woods.
    \\

    
      Explanation: In both examples, the fallacy is committed because we have two universal premises and a particular conclusion, but our example \#1 conclusion makes sense, no?  Just because the conclusion {\it might}  be true, does not mean the logic used to produce it, was valid.  This is how tests like SAT’s and GRE’s screw us over and, technically, in the above example, {\it all} babysitter club members have pimples, not just{\it   some}.
    \\

    
      Now, look at the second example.  Same form, but when we use classes that obviously (to most people) have no members (leprechauns), we can see that it results in a conclusion that is false. 
    \\

    
      Exception: There actually {\it is} an exception to this formal fallacy -- if we are strictly using Aristotelian logic, then it is permissible because apparently, Aristotle did not see a problem with presupposing that classes have members even when we are not explicitly told that they do.
    \\

    
      Tip: When making a claim, be as precise as possible in the scope of the claim. Don’t just say “men are bastards,” say “some men are bastards,” or better yet, “12.62\% of men are bastards.”
    \\

    References:

    
      
        
      \\

      
        
          Goodman, M. F. (1993). {\it First Logic}. University Press of America.
        
      
    
  \subsection{Half-Concealed Quantification
    
      - Description: This fallacy occurs when a statement includes a qualification that limits its scope, but the qualification is downplayed or glossed over. The main assertion is emphasized, making it seem more general than it actually is. This can mislead the audience into accepting a broader claim than what is explicitly stated.
    \\

    
      
    \\

    
      - Logical Form:
    \\

    
        1. A general statement (A) is made with a qualification (B).
    \\

    
        2. The qualification (B) is downplayed or half-concealed.
    \\

    
        3. The audience perceives the statement as more universally applicable than it is.
    \\

    
      
    \\

    
      - Example \#1:
    \\

    
        - Scenario: "Practically every single case of monetary expansion is followed within months by an attendant general price rise of the same proportions."
    \\

    
        - Explanation: The word "practically" qualifies the statement, indicating exceptions, but the emphasis on "every single case" misleads the audience into thinking there are no exceptions.
    \\

    
      
    \\

    
      - Example \#2:
    \\

    
        - Scenario: "The link between poltergeist phenomena and psychological troubles is now clearly established. In almost every case of unexplained breakages and moving objects, there is a disturbed youngster in the household."
    \\

    
        - Explanation: The word "almost" qualifies the claim, but the strong assertion of a clear link misleads the audience into overlooking the exceptions.
    \\

    
      
    \\

    
      - Variation:
    \\

    
        - Scenario: "Palm trees don’t normally grow in England, so it must be something else."
    \\

    
        - Explanation: The word "normally" acknowledges exceptions, but the statement implies a near absolute rule, diverting attention from any exceptions.
    \\

    
      
    \\

    
      - Tip: When you hear a statement that sounds absolute, look for any qualifying words. Consider how the qualification affects the claim and whether it significantly changes the argument's strength.
    \\

    
      
    \\

    
      - Exception: If the qualification is clearly stated and its implications are fully explored, then the fallacy is avoided. Transparent and honest communication about the limits of a claim prevents misleading the audience.
    \\

    
      
    \\

    
      - Fun Fact: The phrase "up to 50\% off" in advertisements is a common example of half-concealed qualification. While it suggests a significant discount, the "up to" means that some items may have much smaller discounts, or none at all.
    \\

  
    
      (Also known as: Weasel Words, Limited Claim Fallacy)
    \\

  }


Concealed Quantification
    
      - Description: This fallacy occurs when a statement about a class of things is vague about the quantity being referred to, leading to ambiguity. The statement might seem to imply that it applies to all members of the class, some members, or it might be unclear. This ambiguity can cause misunderstandings about whether the statement applies universally or partially.
    \\

    
      
    \\

    
      - Logical Form: 
    \\

    
        1. A statement is made about a class
    \\

    
        2. The quantity referred to is ambiguous or concealed.
    \\

    
        3. The ambiguity allows the statement to be interpreted in multiple ways, leading to potential misinterpretation.
    \\

    
      
    \\

    
      - Example \#1:
    \\

    
        - Scenario: "Garage mechanics are crooks."
    \\

    
        - Explanation: The statement does not specify whether it means all garage mechanics or just some. The ambiguity can lead people to believe it refers to all mechanics, which might not be intended.
    \\

    
      
    \\

    
      - Example \#2:
    \\

    
        - Scenario: "It is well known that members of the Campaign for Nuclear Disarmament are communists."
    \\

    
        - Explanation: The statement implies a broad generalization, but it is not clear whether it means all members or just some. This can lead to a misunderstanding of the group's composition.
    \\

    
      
    \\

    
      - Variation:
    \\

    
        - Scenario: "Subversives teach at the Open University."
    \\

    
        - Explanation: The statement can be interpreted as implying that either all subversives are teachers there, or that all teachers there are subversives. The true extent of the statement's claim is concealed.
    \\

    
      
    \\

    
      - Tip: When evaluating statements, be cautious of ambiguous quantifiers like "many," "most," or "some," as they can hide the true scope of the claim. Clarify whether the statement refers to all members of a group or just a subset.
    \\

    
      
    \\

    
      - Exception: If the statement explicitly clarifies the extent of its claim (e.g., "Some garage mechanics are crooks"), the fallacy is avoided. Clear and precise language prevents misinterpretation.
    \\

    
      
    \\

    
      - Fun Fact: The fallacy of concealed quantification is often used in political and social arguments to make broad generalizations about groups or individuals based on limited or ambiguous evidence. This technique can manipulate perceptions by obscuring the true scope of the claim.
    \\

  
    
      (Also known as: Ambiguous Quantification, Hidden Scope Fallacy)
    \\

  \section{
    
      {\bf Formal syllogistic fallacies}
    \\

  }


Scope fallacy
    
      (also known as: Quantifier Shift Fallacy, illicit quantifier shift, modal scope fallacy)
    \\

  
    Description: A fallacy of reversing the order of two quantifiers.

    
      Logical Form:
    \\

    
      Every X has a related Y.
    \\

    
      Therefore, there is some Y related to every X.
    \\

    
      Example \#1:
    \\

    
      Everybody has a mother.
    \\

    
      Therefore, there is some woman out there who is the mother of us all.
    \\

    
      Explanation: While it is true that everyone has (or had) a mother, the term “mother” is not a singular term that is shared -- it is implied that it is a category in which many mothers reside.  The conclusion is asserting the opposite of the meaning -- that there is actually just one mother shared by everyone.  This form of reasoning is invalid; therefore, fallacious.
    \\

    
      Example \#2:
    \\

    
      Everybody has a brain.
    \\

    
      Therefore, there is a single brain we all share.
    \\

    
      Explanation: Everybody has {\it his or her own} brain, not one we all share.  Although I have met many people who seem not to have their own brain.  This form of reasoning is invalid; therefore, fallacious.
    \\

    
      Tip: Remember that a quantifier is an expression (e.g. all, some) that indicates the scope of a term to which it is attached.
    \\

    References:

    
      
        
      \\

      
        
          Cook, R. T. (2009). {\it A Dictionary of Philosophical Logic}. Edinburgh University Press.
        
      
    
  

Illicit major
    
      (also known as: illicit process of the major term)
    \\

  
    Description: Any form of a {\it categorical syllogism} in which the major term is distributed in the conclusion, but not in the major premise.

    
      Logical Form:
    \\

    
      All A are B.
    \\

    
      No C are A.
    \\

    
      Therefore, no C are B.
    \\

    
      Example \#1:
    \\

    
      All hotdogs are fast food.
    \\

    
      No hamburgers are hotdogs.
    \\

    
      Therefore, no hamburgers are fast food.
    \\

    
      Explanation: In our example, the major term is “fast food”, because it is the term that appears in the major premise (first premise) as the predicate and in the conclusion.  As such, in this position, it is “undistributed”.
    \\

    
      Example \#2:
    \\

    
      All Jim Carrey movies are hilarious.
    \\

    
      No horror movies are Jim Carrey movies.
    \\

    
      Therefore, no horror movies are hilarious.
    \\

    
      Explanation: In our example, the major term is “hilarious”, because it is the term that appears in the major premise (first premise) as the predicate and in the conclusion.  As such, in this position, it is “undistributed”.
    \\

    
      Exception: None.
    \\

    
      Tip: If you are not sure if a syllogism is fallacious or not, keep substituting the items in the syllogism for other items. If a syllogism is consistent with the form but clearly doesn’t make sense, you will know that it is fallacious.
    \\

    References:

    
      
        
      \\

      
        
          Neil, S. (1853). {\it The Art of Reasoning: A Popular Exposition of the Principles of Logic}. Walton \& Maberly.
        
      
    
  

Illicit minor
    
      (also known as: illicit process of the minor term)
    \\

  
    Description: Any form of a {\it categorical syllogism} in which the minor term is distributed in the conclusion, but not in the minor premise.

    
      Logical Form:
    \\

    
      All A are B.
    \\

    
      All B are C.
    \\

    
      Therefore, all C are A.
    \\

    
      Example \#1:
    \\

    
      All Catholics are Christian.
    \\

    
      All Christians are Jesus lovers.
    \\

    
      Therefore, all Jesus lovers are Catholic.
    \\

    
      Explanation: In our example, the minor term is “Jesus lovers” because it is the term that appears in the minor premise (second premise) as the predicate and in the conclusion.  As such, in this position, it is “undistributed”.
    \\

    
      Example \#2:
    \\

    
      All Paul Newman movies are great.
    \\

    
      All great movies are Oscar winners.
    \\

    
      Therefore, all Oscar winners are Paul Newman movies.
    \\

    
      Explanation: In our example, the minor term is “Oscar winners” because it is the term that appears in the minor premise (second premise) as the predicate and in the conclusion.  As such, in this position, it is “undistributed”.
    \\

    
      Exception: None.
    \\

    
      Fun Fact: The Catholic bible has 73 books, the Protestant bible only has 66.
    \\

    References:

    
      
        
      \\

      
        
          Neil, S. (1853). {\it The Art of Reasoning: A Popular Exposition of the Principles of Logic}. Walton \& Maberly.
        
      
    
  

Modal (Scope) Fallacy
    
      (also known as: fallacy of modal logic, misconditionalization, fallacy of neccessity)
    \\

  
    Description: {\it Modal logic} studies ways in which propositions can be true or false, the most common being {\it necessity} and {\it possibility}.  Some propositions are necessarily true/false, and others are possibly true/false.  In short, a modal fallacy involves making a formal argument invalid by confusing the {\it scope} of what is actually necessary or possible.

    
      Logical Form:
    \\

    
      {\em A conditional claim is made using a necessary truth.} \newline
{\em Therefore, conclusion is reached that a possible truth is necessary with no conditional statement.}
    \\

    
      Example \#1:
    \\

    
      {\em If Debbie and TJ have two sons and two daughters, then they must have at least one son.} \newline
{\em Debbie and TJ have two sons and two daughters.} \newline
{\em Therefore, Debbie and TJ must have at least one son.}
    \\

    
      Explanation: We are told that Debbie and TJ have two sons and two daughters, so logically, by necessity, they must {\it have at} least one son.  However, to say that Debbie and TJ {\it must} have at least one son, is to {\it confuse the scope of the modal}, or in this case, to take the {\it contingent fact} that applies to the specific case that is conditional upon Debbie and TJ having the two sons and two daughters, to the general hypothetical case where they don’t have to have any children.  Therefore, if they don’t {\it have to} have any children, then they certainly don’t {\it have to} ({\it necessary fact}) have at least one son.
    \\

    
      Example \#2:
    \\

    
      {\em If Barak is President, then he must be 35 years-old or older.}
    \\

    
      Explanation: Technically this is fallacious.  There is no condition in which someone {\it necessarily} is a certain age.  More accurately, we would say:
    \\

    
      {\em It must be the case that if Barak is President, then he is 35 or older.}
    \\

    
      The “must” in this second statement covers the whole condition, not just the age of the president.
    \\

    
      Exception: This is one of those fallacies that would make you look foolish for calling someone out on, unless you are among all academic philosophers. In casual argumentation many people do mean "possibility" and not "necessity." Remember to offer others a charitable interpretation of their argument.
    \\

    
      Fun Fact: According to Article II of the U.S. Constitution, the president of the United States must be a natural-born citizen of the United States; be at least 35 years old; have been a resident of the United States for 14 years; and be able to identify a snake, an elephant, and an alligator.
    \\

    References:

    
      
        
      \\

      
        
          Curtis, G. N. (1993). {\it The Concept of Logical Form}. Indiana University.
        
      
    
  

Fallacy of (the) Undistributed Middle
    
      (also known as: maldistributed middle, undistributed middle term, non distributio medii)
    \\

  
    Description: A formal fallacy in a categorical syllogism where the {\it middle term}, or the term that does not appear in the conclusion, is not distributed to the other two terms.

    
      Logical Form:
    \\

    
      All A's are C's.
    \\

    
      All B's are C's.
    \\

    
      Therefore, all A’s are B’s.
    \\

    
      Example \#1:
    \\

    
      All lions are animals.
    \\

    
      All cats are animals.
    \\

    
      Therefore, all lions are cats.
    \\

    
      Explanation: We are tricked because the conclusion makes sense, so out of laziness we accept the argument, but the argument is invalid, and by plugging in new terms, like in the next example, we can see why.
    \\

    
      Example \#2:
    \\

    
      All ghosts are imaginary.
    \\

    
      All unicorns are imaginary.
    \\

    
      Therefore, all ghosts are unicorns.
    \\

    
      Explanation: While there may be ghosts that are unicorns, it does not follow from the premises: the only thing the premises tell us about ghosts and unicorns is that they are both imaginary -- we have no information how they are related to each other.
    \\

    
      Fun Fact: I have it on good authority that unicorns do poop rainbows.
    \\

    References:

    
      
        
      \\

      
        
          Goodman, M. F. (1993). {\it First Logic}. University Press of America.
        
      
    
  

Illicit Contraposition
    New Terminology:

    
      Illicit: Forbidden by the rules, or in our cases, by the laws of logic.
    \\

    
      Contraposition: Switching the subject and predicate terms of a categorical proposition, and negating each.
    \\

    
      Description: A formal fallacy where switching the subject and predicate terms of a categorical proposition, then negating each, results in an invalid argument form.  The examples below make this more clear.  This is a fallacy only for type “E” and type “I” forms, or forms using the words “no” and “some”, respectively.
    \\

    
      Logical Forms:
    \\

    
      No S are P.
    \\

    
      Therefore, no non-P are non-S.
    \\

    
       
    \\

    
      Some S are P.
    \\

    
      Therefore, some non-P are non-S.
    \\

    
      Example \#1:
    \\

    
      No Catholics are Jews.
    \\

    
      Therefore, no non-Jews are non-Catholics. (contraposition)
    \\

    
      Explanation: By definition, no Catholics are Jews (using type “E” form here) -- clear enough.  Now let’s take the contraposition of that proposition by switching the placement of “Catholics” and “Jews”, and negating each, and we can see we have a false proposition.  “No non-Jews are non-Catholics” clearly does not mean the same thing as “No Catholics are Jews”.  In this example, the premise is true but the conclusion is false (I am a non-Jew and a non-Catholic, and statistically speaking, you probably are too.)
    \\

    
      Example \#2:
    \\

    
      Some dogs bark.
    \\

    
      Therefore, some non-barking things are non-dogs. (contraposition)
    \\

    
      Explanation: We now see the type “I” form in action, stating, “Some” dogs bark.  This is true, but that really does not matter in determining what form of an argument is valid or not.  The conclusion, “some non-barking things are non-dogs” is also a true statement (my toothbrush, which is a non-dog, does not bark), but this does not matter either.  What does matter, is that it does not logically follow.  Don’t be misled by truth!  Focus on the form of the argument.  If we substitute other terms we can see the fallacy more clearly:
    \\

    
      Some humans are mortal.
    \\

    
      Therefore, some immortals are non-human. (contraposition)
    \\

    
      By using the word, “some”, we are not asserting that there are definitely some {\it that are not}.  Above, just by saying that some humans are mortal, we automatically are saying that there are others who are not mortal.  Therefore, our conclusion is supposing a group that does not exist, thus fallacious.
    \\

    
      Exception: None, but remember that the following type “A” and type “O” forms are valid:
    \\

    
      All P are Q.
    \\

    
      Therefore, all non-Q are non-P.
    \\

    
       
    \\

    
      Some P are not Q.
    \\

    
      Therefore, some non-Q are not non-P.
    \\

    
      Using the type “A” form, let’s say that all humans are mortals.  The contraposition: all immortals are non-human.  Not only does this make sense in terms of truth, but it follows necessarily from the premise; therefore, it is valid (and not a fallacy).
    \\

    
      Tip: Don’t give up on formal fallacies!  Once you get it, it actually will help you in everyday reasoning.
    \\

    References:

    
      
        
      \\

      
        
          Welton, J. (1896). {\it A Manual of Logic}. W. B. Clive.
        
      
    
  

Negating Antecedent and Consequent
    
      (also known as: improper transposition)
    \\

  
    New Terminology:

    
      Transposition (contraposition): In a syllogism, taking the antecedent and consequent in the first premise, then “transposing” them in the second premise, and negating each term.
    \\

    
      Description: A formal fallacy where in the valid transpositional form of an argument, we fail to switch the antecedent and consequent.  The valid form of this argument is as follows:
    \\

    
      If P then Q.
    \\

    
      Therefore, if not-Q then not-P.
    \\

    
      Notice we switch (transpose) the P and the Q, then negate them both.  We commit the fallacy when we fail to transpose (switch) them.
    \\

    
      Logical Forms:
    \\

    
      If P then Q.
    \\

    
      Therefore, if not-P then not-Q.
    \\

    
       
    \\

    
      If not-P then not-Q.
    \\

    
      Therefore, if P then Q.
    \\

    
      Example \#1:
    \\

    
      If Barry Manilow sings love songs, then he is gay.
    \\

    
      Therefore, if Barry Manilow does not sing love songs, then he is not gay.
    \\

    
      Explanation: Besides the wildly incorrect premise that if Barry sings love songs he is gay, the conclusion fails to switch the antecedent (Barry Manilow sings love songs) with the consequent (he is gay); therefore, it is fallacious.  However, if we did transpose the antecedent and the consequent in the conclusion, it would be a perfectly valid formal argument, even though the premise might not be a reasonable assumption.  Remember, a valid, non-fallacious formal argument does not have to have a true conclusion, it just needs to be truth-preserving -- in the case that the premises are all true.
    \\

    
      If Barry Manilow sings love songs, then he is gay.
    \\

    
      Therefore, if Barry Manilow is not gay, then he does not sing love songs.
    \\

    
      Example \#2:
    \\

    
      If Tom thinks that all people who sing love songs are gay, then he is an idiot.
    \\

    
      Therefore, if Tom doesn’t think that all people who sing love songs are gay, then he is not an idiot.
    \\

    
      Explanation: We have the same problem with the failure to transpose the antecedent (Tom thinks that all people who sing love songs are gay) with the consequent (he is an idiot) in the conclusion, although we did negate them both.  I hope you can see that just because Tom does not think all people who sing love songs are gay, does not mean that Tom is not an idiot for some other reason.  This argument is invalid, thus fallacious.
    \\

    
      Fun Fact: Barry Manilow was born “Barry Alan Pincus.”
    \\

    References:

    
      
        
      \\

      
        
          Carlsen-Jones, M. T. (1983). {\it Introduction to Logic}. McGraw-Hill.
        
      
    
  

Negative Conclusion from Affirmative Premises
    
      (also known as: illicit affirmative, Positive conclusion from negative premise)
    \\

  
    Description: Suggesting that because two things are alike in some way and one of those things is like something else, then both things must be like that "something else".

    
      In essence, the {\it reductio ad hitlerum} is an extended analogy because it is the attempt to associate someone with Hitler’s psychotic behavior by way of a usually much more benign connection.
    \\

    
      Logical Form:
    \\

    
      A is like B in some way.
    \\

    
      C is like B in a different way.
    \\

    
      Therefore, A is like C.
    \\

    
      Example \#1:
    \\

    
      Jennie: Anyone who doesn’t have a problem with slaughtering animals for food, should not, in principle, have a problem with an advanced alien race slaughtering us for food.
    \\

    
      Carl: Fruitarians, the crazy people who won’t eat anything except for fruit that fell from the tree, are also against slaughtering animals for food.  Are you crazy like them?
    \\

    
      Explanation: Although I don’t think I can ever give up delicious chicken, Jennie does make a good point via a valid analogy.  Ignoring Carl’s attempt to{\it  poisoning the well }by using {\it argument by emotive language} he is, by{\it  
}, claiming the “craziness” of the fruitarians must be shared by her, as well, since they both are alike because they share a view on using animals for food.
    \\

    
      Example \#2:
    \\

    
      Science often gets things wrong.  It wasn’t until the early 20th century when particle physics came along that scientists realized that the atom wasn’t the smallest particle in existence.  So perhaps science will soon realize that it is wrong about the age of the universe, the non-existence of a global flood, evolution, and every other science fact that contradicts the Bible when read literally.
    \\

    
      Explanation: To see this fallacy, let’s put it in the logical form, using just the evolution claim:
    \\

    
      P1. Thinking the atom was the smallest particle was a mistake of science.
    \\

    
      P2. Evolution is also a mistake of science.
    \\

    
      C. Therefore, science thinking the atom was the smallest particle is like science thinking evolution is true.
    \\

    
      Premise two (P2) should jump out as a bold assumption, although not fallacious.  Remember, the premises don’t have to be true for the argument to be valid, but if both premises were true, does the conclusion (C) follow?  No, because of the {\it extended fallacy}.  The reason is if evolution were false it would not be for the same reason that science thought the atom was the smallest particle.  Science “was wrong” in that case because it did not have access to the whole truth due to discoveries yet to be made at the time.  If evolution were wrong, then all the discoveries that have been made, the facts that have been established, the foundation of many sciences that have led to countless advances in medicine, would all be dead wrong.  This would be a mistake of unimaginable proportions and consequences that would unravel the very core of scientific understanding and inquiry.
    \\

    
      Exception: If one can show evidence that the connection between all the subjects is the same, it is not fallacious.
    \\

    
      It is crazy to think that carrots have feelings.
    \\

    
      It is crazy to think that cows have feelings.
    \\

    
      Therefore, vegetarians are just as crazy as fruitarians.[1]
    \\

    
      Tip: Don’t call people crazy -- leave that kind of psychological assessment for the licensed professionals.  You can call them, “nutjobs”.
    \\

    \chapter{
      Negative Conclusion from Affirmative Premises
    }
  
    

    
      
        (also known as: illicit affirmative)
      \\

      
        Description: The conclusion of a standard form categorical syllogism is negative, but both of the premises are positive.  Any valid forms of categorical syllogisms that assert a negative conclusion must have at least one negative premise.
      \\

      
        Logical Form:
      \\

      
        If A is a subset of B, and B is a subset of C, then A is not a subset of C.
      \\

      
        Example \#1:
      \\

      
        All cats are animals.
      \\

      
        Some pets are cats.
      \\

      
        Therefore, some pets are not animals.
      \\

      
        Explanation: The conclusion might be true -- I had a pet rock growing up, but the argument still does not logically support that.  Think of sets and subsets.  All cats are animals: we have a set of animals and a subset of cats.  “Some” pets are cats: so all we know is that there is a part of our set, “pets” that intersects with the subset, “cats”, but we don’t have the information we need to conclude logically that some pets are not animals.  This argument is invalid, thus as a formal argument, it is fallacious.
      \\

      
        Example \#2:
      \\

      
        All boys are sports fans.
      \\

      
        Some bakers are boys.
      \\

      
        Therefore, some bakers are not sports fans.
      \\

      
        Explanation: The conclusion might be true -- but the argument still does not logically support that for the same reasons in the first example.  This argument is invalid, thus as a formal argument, it is fallacious.
      \\

      
        Fun Fact: I taught my pet rock how to play dead. It was its only trick.
      \\

    
    References:

    
      
        
      \\

      
        
          Goodman, M. F. (1993). {\it First Logic}. University Press of America.
        
      
    
  

contradictory premises
    
      
        - Description: This fallacy occurs when an argument is based on premises that cannot both be true simultaneously. Since the premises contradict each other, at least one must be false, which means the argument cannot be sound regardless of how valid the logical structure is.
      \\

      
        
      \\

      
        - Logical Form:
      \\

      
          1. Premise 1: Statement A.
      \\

      
          2. Premise 2: Statement B.
      \\

      
          3. Statement A and Statement B are contradictory.
      \\

      
          4. Conclusion drawn from the contradictory premises.
      \\

      
        
      \\

      
        - Example \#1:
      \\

      
          - Scenario: "Everything is mortal, and God is not mortal, so God is not everything."
      \\

      
          - Explanation: The premises "Everything is mortal" and "God is not mortal" contradict each other. Since one of these must be false, any conclusion based on these premises is unreliable.
      \\

      
        
      \\

      
        - Example \#2:
      \\

      
          - Scenario: "Milk is white, and milk is not white. Therefore, the moon is made of green cheese."
      \\

      
          - Explanation: The premises "Milk is white" and "Milk is not white" are contradictory. The conclusion, while logically valid, is based on false premises and thus cannot be sound.
      \\

      
        
      \\

      
        - Variation:
      \\

      
          - Scenario: "He’s a real professional but a bit of an amateur at times."
      \\

      
          - Explanation: This statement contains a contradiction between being a professional and being an amateur, which can lead to absurd conclusions if used in a formal argument.
      \\

      
        
      \\

      
        - Tip: To avoid the fallacy of contradictory premises, ensure that all premises of your argument are consistent with each other. Check for internal contradictions before drawing conclusions.
      \\

      
        
      \\

      
        - Exception: If you identify contradictory premises, acknowledge the inconsistency and correct it before proceeding. Valid arguments require premises that are mutually compatible.
      \\

      
        
      \\

      
        - Fun Fact: The fallacy of contradictory premises can be used to construct absurd arguments, like proving the moon is made of green cheese. This highlights the importance of sound premises in logical reasoning.
      \\

    
  
    
      (Also known as: Self-Contradiction, Inconsistent Premises)
    \\

  

Conjunction fallacy
    
      (also known as: conjunction effect)
    \\

  
    Description: Similar to the {\it disjunction fallacy} , the {\em conjunction fallacy} occurs when one estimates a conjunctive statement (this {\em and}  that) to be more probable than at least one of its component statements. It is the assumption that more specific conditions are more probable than general ones.  This fallacy usually stems from thinking the choices are alternatives, rather than members of the same set.  The fallacy is further exacerbated by priming the audience with information leading them to choose the subset as the more probable option.

    
      Logical Form:
    \\

    
      {\em X is a subset of Y. \newline
Therefore, X is more probable than Y.}
    \\

    
      {\em Conjunction X and Y (both taken together) is more likely than a constituent X.}
    \\

    
      Example \#1:
    \\

    
      {\em While jogging around the neighborhood, you are more likely to get bitten by someone’s pet dog, than by any member of the canine species.}
    \\

    
      Explanation: Actually, that is not the case.  “Someone’s pet dog”, assuming a real dog and not some robot dog, would also be a member of the canine species.  Therefore, the canine species includes wolves,  coyotes, as well as your neighbor’s Shih Tzu, who is likely to bite you just because he’s pissed for being so small.
    \\

    
      Example \#2: Karen is a thirty-something-year-old female who drives a mini-van, lives in the suburbs, and wears mom jeans. Is Karen more likely to be a woman or a mom?
    \\

    
      Explanation: It would be fallacious to say that Karen is more likely to be a mom—even if we found out that Karen spent an hour each day at the playground with other moms. There is a 100\% chance the Karen is a woman (we know she is female), and a smaller chance that she is a mom.
    \\

    
      Exception: When contradicting conditions are implied, but incorrectly stated.
    \\

    
      In the example above, the way the question reads, we now know that there is a 100\% chance Karen is a woman and a smaller chance that she is a mom. However, if the questioner meant to imply, “not a mom” or “mom” as the choices, then it could be more of a poorly phrased question than a fallacy.
    \\

    
      Tip: Read {\em Thinking, Fast and Slow}, by Daniel Kahneman and Amos Tversky for a deep dive on cognitive errors.
    \\

    References:

    
      
        
      \\

      
        
          Kahneman, D. (2013). {\it Thinking, Fast and Slow} (1st edition). New York: Farrar, Straus and Giroux.
        
      
    
  

Disjunction Fallacy
    Description: Similar to the{\em  {\it conjunction fallacy}}, the {\em disjunction fallacy} occurs when one estimates a disjunctive statement (this {\em or} that) to be less probable than at least one of its component statements.

    
      Logical Forms:
    \\

    
      Disjunction X or Y (both taken together) is less likely than a constituent Y.
    \\

    
      Example \#1:
    \\

    
      Mr. Pius goes to church every Sunday.  He gets most of his information about religion from church and does not really read the Bible too much.  Mr. Pius has a figurine of St. Mary at home.  Last year, when he went to Rome, he toured the Vatican.  From this information, Mr. Pius is more likely to be Catholic than a Catholic or a Muslim.
    \\

    
      Explanation: This is incorrect.  While it is very likely that Mr. Pius is Catholic based on the information, it is more likely that he is Catholic or Muslim.
    \\

    
      Example \#2:
    \\

    
      Bill is 6’11” tall, thin, but muscular.  We know he either is a pro basketball player or a jockey.  We conclude that it is more probable that he is a pro basketball player than a pro basketball player or a jockey.
    \\

    
      Explanation: This is incorrect.  While it is very likely that Bill plays the B-ball, given that he would probably crush a horse, it is statistically more likely that he is either a pro basketball player or a jockey since that option includes the option of him being just a pro basketball player.  Don’t let what seems like common sense fool you.
    \\

    
      Exception: No exceptions due to basic probability.
    \\

    
      Tip: Go back and read the entry for the {\it conjunction fallacy}  again and make sure you know the difference.
    \\

    
      References:
    \\

    
      
        
      \\

      
        
          Gilovich, T., Griffin, D., \& Kahneman, D. (2002). {\it Heuristics and Biases: The Psychology of Intuitive Judgment}. Cambridge University Press.
        
      
    
  \section{
    
      {\bf  Propositional fallacies}
    \\

  }


Affirming a disjunct
    
      (also known as: the fallacy of the alternative disjunct, false exclusionary disjunct, affirming one disjunct, the fallacy of the alternative syllogism, asserting an alternative, improper disjunctive syllogism, fallacy of the disjunctive syllogism)
    \\

  
    New Terminology:

    
      Disjunction: A proposition of the "either/or" form, which is true if one or both of its propositional components is true; otherwise, it is false.
    \\

    
      Disjunct: One of the propositional components of a disjunction.
    \\

    
      Description: Making the false assumption that when presented with an either/or possibility, that if one of the options is true that the other one must be false.  This is when the “or” is not explicitly defined as being {\it exclusive}.
    \\

    
      This fallacy is similar to the {\it unwarranted contrast} fallacy.
    \\

    
      Logical Forms:
    \\

    
      P or Q.
    \\

    
      P.
    \\

    
      Therefore, not Q.
    \\

    
       
    \\

    
      P or Q.
    \\

    
      Q.
    \\

    
      Therefore, not P.
    \\

    
      Example \#1:
    \\

    
      I can’t stop eating these chocolates.  I really love chocolate, or I seriously lack willpower.  I know I really love chocolate; therefore, I cannot lack willpower.
    \\

    
      Explanation: Ignoring the possible {\it false dilemma}, the fact that one really loves chocolate does not automatically exclude the other possibility of lacking willpower.
    \\

    
      Example \#2:
    \\

    
      I am going to bed or watching TV.  I am exhausted, so I will go to bed; therefore, I cannot watch TV.
    \\

    
      Explanation: It is logically and physically possible to go to bed and watch TV at the same time, I know that for a fact as I do it just about every night.  The “or” does not logically exclude the option that is not chosen.
    \\

    
      Exception: If the choices {\it are} mutually exclusive (either by necessity or indicated by the word "either"), then it can be deduced that the other choice must be false.  Again, we are working under the assumption that one of the choices we are given represents the truth.
    \\

    
      Today is either Monday or Sunday.  It is Monday.  Therefore, it is not Sunday.
    \\

    
      In formal logic, the above is referred to as a {\it valid disjunctive syllogism}.
    \\

    
      If you are thinking, “But it can be both Monday and Sunday if we are talking about two different time zones,” then give your self three points for being clever, then subtract four points for missing the whole point of the fallacy.
    \\

  

Affirming the consequent
    
      (also known as: converse error, fallacy of the consequent, asserting the consequent, affirmation of the consequent)
    \\

  
    New Terminology:

    
      Consequent: the propositional component of a conditional proposition whose truth is conditional; or simply put, what comes after the “then” in an “if/then” statement.
    \\

    
      Antecedent: the propositional component of a conditional proposition whose truth is the condition for the truth of the consequent; or simply put, what comes after the “if” in an “if/then” statement.
    \\

    
      Description: An error in formal logic where if the consequent is said to be true, the antecedent is said to be true, as a result.
    \\

    
      Logical Form:
    \\

    
      If P then Q.
    \\

    
      Q.
    \\

    
      Therefore, P.
    \\

    
      Example \#1:
    \\

    
      If taxes are lowered, I will have more money to spend.
    \\

    
      I have more money to spend.
    \\

    
      Therefore, taxes must have been lowered.
    \\

    
      Explanation: I could have had more money to spend simply because I gave up crack-cocaine, prostitute solicitation, and baby-seal-clubbing expeditions.
    \\

    
      Example \#2:
    \\

    
      If it’s brown, flush it down.
    \\

    
      I flushed it down.
    \\

    
      Therefore, it was brown.
    \\

    
      Explanation: No!  I did not have to follow the, “if it’s yellow, let it mellow” rule -- in fact, if I did follow that rule I would probably still be single.  The stated rule is simply, “if it’s brown” (the {\it antecedent}), then (implied), “flush it down” (the {\it consequent}).  From this, we cannot imply that we can ONLY flush it down if it is brown.  That is a mistake -- a logical fallacy.
    \\

    
      Exception: None.
    \\

    
      Tip: If it’s yellow, flush it down too. Especially, if you are married and want to stay that way.
    \\

  

Politician's syllogism
    
      (also known as: politician's fallacy, politician's logic)
    \\

  \section{Non sequitur
    
      (also known as: derailment, “that does not follow”, irrelevant reason, invalid inference, non-support, argument by scenario [form of], false premise [form of], questionable premise [form of])
    \\

  
    Description: When the conclusion does not follow from the premises.  In more informal reasoning, it can be when what is presented as evidence or reason is irrelevant or adds very little support to the conclusion.

    
      Logical Form:
    \\

    
      Claim A is made.
    \\

    
      Evidence is presented for claim A.
    \\

    
      Therefore, claim C is true.
    \\

    
      Example \#1:
    \\

    
      People generally like to walk on the beach.  Beaches have sand.  Therefore, having sand floors in homes would be a great idea!
    \\

    
      Explanation: As cool as the idea of sand floors might sound, the conclusion does not follow from the premises.  The fact that people generally like to walk on sand does not mean that they want sand in their homes, just like because people generally like to swim, they shouldn’t flood their houses.
    \\

    
      Example \#2:
    \\

    
      Buddy Burger has the greatest food in town.  Buddy Burger was voted \#1 by the local paper.  Therefore, Phil, the owner of Buddy Burger, should run for president of the United States.
    \\

    
      Explanation: I bet Phil makes one heck of a burger, but it does not follow that he should be president.
    \\

    
      Exception: There really are no exceptions to this rule. Any good argument must have a conclusion that follows from the premises.
    \\

    
      Tip: One of the best ways to expose non sequiturs is by constructing a valid analogy that exposes the absurdity in the argument.
    \\

    
      Variations: There are many forms of non sequiturs including {\it argument by scenario}, where an irrelevant scenario is given in an attempt to support the conclusion.  Other forms use different rhetorical devices that are irrelevant to the conclusion.
    \\

    
      {\it False} or {\it questionable premises} could be seen as errors in facts, but they can also lead to the conclusion not following, so just keep that in mind, as well.
    \\

    References:

    
      
        
      \\

      
        
          Hyslop, J. H. (1892). {\it The Elements of Logic, Theoretical and Practical}. C. Scribner’s Sons.
        
      
    
  }


Anangeon
    
      (also known as: dicaeologia)
    \\

  
    
      - **Description:** Anangeon is a fallacious argument based on the concept of inevitability or necessity. It argues that certain outcomes or actions were unavoidable, thus justifying or excusing the failure to meet a standard or obligation. This method often serves to mitigate or shift the blame away from the individual’s responsibility.
    \\

    
      - **Logical Form:**
    \\

    
        - **P1:** An action or outcome is presented as necessary or inevitable.
    \\

    
        - **P2:** The argument uses this inevitability to justify or excuse a failure or fault.
    \\

    
        - **C:** The individual’s failure or fault is diminished or negated due to the perceived inevitability.
    \\

    
      - **Example \#1:** "Yes, I missed school today, but I was sick and wouldn't have learned anything anyway."
    \\

    
      - **Explanation:** This argument excuses the absence from school by suggesting that the outcome (not learning) was inevitable due to the illness, thus undermining the importance of attending school despite the illness.
    \\

    
      - **Example \#2:** "I didn’t finish the project on time, but it was going to be delayed regardless because of unforeseen circumstances."
    \\

    
      - **Explanation:** Here, the argument claims that the delay was unavoidable due to circumstances, thereby minimizing the responsibility for not completing the project on time.
    \\

    
      - **Variation:** The term is sometimes used interchangeably with "dicaeologia," which specifically refers to a defense plea, although "anangeon" is broader in scope.
    \\

    
      - **Tip:** Be cautious of arguments that rely on inevitability to excuse failures or responsibilities; assess whether the necessity claimed genuinely absolves the fault or merely distracts from addressing the issue.
    \\

    
      - **Exception:** Anangeon may be relevant in contexts where unavoidable circumstances genuinely prevent certain actions or outcomes. In such cases, it is important to distinguish between genuine necessity and an excuse to evade responsibility.
    \\

    
      - **Fun Fact:** The concept of anangeon reflects ancient Greek rhetorical techniques, showcasing how historical methods of argumentation still influence contemporary discussions on responsibility and fault.
    \\

  

Argument by Gibberish
    
      (also known as: bafflement, argument by [prestigious] jargon)
    \\

  
    Description: When incomprehensible jargon or plain incoherent gibberish is used to give the appearance of a strong argument, in place of evidence or valid reasons to accept the argument.

    
      The more common form of this argument is when the person making the argument defaults to highly technical jargon or details not directly related to the argument, then restates the conclusion.
    \\

    
      Logical Form:
    \\

    
      Person 1 claims that X is true.
    \\

    
      Person 1 backs up this claim by gibberish.
    \\

    
      Therefore, X is true.
    \\

    
      Example \#1:
    \\

    
      Fortifying the dextrose coherence leads to applicable inherent of explicable tolerance; therefore, we should not accept this proposal.
    \\

    
      Explanation: I have no idea what I just wrote, and the audience will have no idea either -- but the audience (depending on who the audience is) will most likely make the assumption that I do know what I am talking about, believe that they are incapable of understanding the argument and therefore, agree with my conclusion since they think I do understand it.  This is fallacious reasoning.
    \\

    
      Example \#2: (The following was taken from the movie Spies Like Us where Emmett Fitz-Hume, played by Chevy Chase, was addressing the press.)
    \\

    
      Well, of course, their requests for subsidies was not Paraguayan in and of it is as it were the United States government would never have if the president, our president, had not and as far as I know that's the way it will always be. Is that clear?
    \\

    
      Explanation: Emmett Fitz-Hume was clearly avoiding having to answer the question, and substituted gibberish for an answer. While the press was metaphorically scratching their heads to figure what what was just said, Emmett moved on without answering the question.
    \\

    
      Exception: Some arguments require some jargon or technical explanations.
    \\

    
      Tip: Remember that good communication is not about confusing people; it’s about mutual understanding.  Don’t try to impress people with fancy words and jargon, when simpler words will do just fine.
    \\

  

Mathematical fallacy\section{Improper premise}


Affirmative conclusion from a negative premise
    
      (also known as: illicit negative, drawing an affirmative conclusion from negative premises, fallacy of negative premises)
    \\

  
    
      This is our first fallacy in {\it formal logic} out of about a dozen presented in this book.  Formal fallacies can be confusing and complex and are not as common in everyday situations, so please don’t feel lost when reading through the formal fallacies—do your best to understand them as I do my best to make them understandable.
    \\

    
      New Terminology:
    \\

    
      Syllogism: an argument typically consisting of three parts: a major premise, a minor premise, and a conclusion.
    \\

    
      Categorical Term: usually expressed grammatically as a noun or noun phrase, each categorical term designates a class of things.
    \\

    
      Categorical Proposition: joins exactly two categorical terms and asserts that some relationship holds between the classes they designate.
    \\

    
      Categorical Syllogism: an argument consisting of exactly three categorical propositions: a major premise, a minor premise, and a conclusion, in which there appears a total of exactly three categorical terms, each of which is used exactly twice.
    \\

    
      Description: The conclusion of a standard form categorical syllogism is affirmative, but at least one of the premises is negative. Any valid forms of categorical syllogisms that assert a negative premise must have a negative conclusion.
    \\

    
      Logical Form:
    \\

    
      Any form of categorical syllogism with an affirmative conclusion and at least one negative premise.
    \\

    
      Example \#1:
    \\

    
      No people under the age of 66 are senior citizens.
    \\

    
      No senior citizens are children.
    \\

    
      Therefore, all people under the age of 66 are children.
    \\

    
      Explanation: In this case, the conclusion is obviously counterfactual although both premises are true.  Why?  Because this is a categorical syllogism where we have one or more negative premises (i.e., “no people...” and “no senior citizens...”), and we are attempting to draw a positive (affirmative) conclusion (i.e., “all people...”). 
    \\

    
      Example \#2:
    \\

    
      No donkeys are fish.
    \\

    
      Some asses are donkeys.
    \\

    
      Therefore, some asses are fish.
    \\

    
      Explanation: This is a categorical syllogism where we have a single negative premise (i.e., “no donkeys”), and we are attempting to draw a positive (affirmative) conclusion (i.e., “some asses”).
    \\

    
      On a somewhat related note, some lawyers are asses.
    \\

    
      Exception: No exceptions as this is a formal fallacy.
    \\

    
      Tip: Syllogisms and identifying formal fallacies (at least by form) are common on intelligence tests. Know this and be more intelligent (at least on paper).
    \\

  

Exclusive Premises
    
      (also known as: fallacy of exclusive premises)
    \\

  
    Description: A standard form categorical syllogism that has two negative premises either in the form of  “no X are Y” or “some X are not Y”.

    
      Logical Forms:
    \\

    
      No X are Y.
    \\

    
      Some Y are not Z.
    \\

    
      Therefore, some Z are not X.
    \\

    
       
    \\

    
      No X are Y.
    \\

    
      No Y are Z.
    \\

    
      Therefore, no Z are X.
    \\

    
      Example \#1:
    \\

    
      No kangaroos are MMA fighters.
    \\

    
      Some MMA fighters are not Mormons.
    \\

    
      Therefore, some Mormons are not kangaroos.
    \\

    
      Example \#2:
    \\

    
      No animals are insects.
    \\

    
      Some insects are not dogs.
    \\

    
      Therefore, some dogs are not animals.
    \\

    
      Example \#3:
    \\

    
      No animals are insects.
    \\

    
      No insects are dogs.
    \\

    
      Therefore, no dogs are animals.
    \\

    
      Explanation: Remember why fallacies are so dangerous: because they appear to be good reasoning.  The conclusion in example \#1 makes sense, but it does not follow logically -- it is an invalid argument.  Based on the first two premises, there is no way logically to deduce that conclusion.  Now, look at examples \#2 and \#3.  We use the same logical form of the argument, committing the same fallacy, but by changing the terms it is much more clear that something went wrong somewhere, and it did.  This kind of argument, the {\it categorical syllogism}, cannot have two negative premises and still be valid.
    \\

    
      Just because the conclusion appears true, it does not mean the argument is valid (or strong, in the case of an informal argument).
    \\

    
      Exception: No exceptions.
    \\

    
      Tip: Learn to recognize the forms of formal fallacies, and you will easily spot invalid formal arguments.
    \\

    References:

    
      
        
      \\

      
        
          Goodman, M. F. (1993). {\it First Logic}. University Press of America.
        
      
    
  

Vacuous truth
    
      - **Description:** A vacuous truth is a statement that is considered true because its antecedent (the condition or premise) is false. This makes the entire conditional statement or universal statement true regardless of the truth value of the consequent (the result or conclusion). It often appears as a result of the logical structure of material conditionals or universal quantifications where the antecedent cannot be satisfied.
    \\

    
      - **Logical Form:**
    \\

    
        - **P1:** A conditional statement is in the form "If P, then Q."
    \\

    
        - **P2:** The antecedent (P) is false.
    \\

    
        - **C:** The conditional statement is true regardless of the truth value of the consequent (Q).
    \\

    
      - **Example \#1:** "If Tokyo is in Spain, then the Eiffel Tower is in Bolivia."
    \\

    
      - **Explanation:** The antecedent "Tokyo is in Spain" is false. Therefore, the entire conditional statement is vacuously true regardless of whether "the Eiffel Tower is in Bolivia" is true or false.
    \\

    
      - **Example \#2:** "All cell phones in the room are turned off." (when no cell phones are present in the room)
    \\

    
      - **Explanation:** The antecedent "There are cell phones in the room" is false because there are no cell phones. Therefore, the statement is vacuously true because the condition (presence of cell phones) is not met.
    \\

    
      - **Variation:** The concept also applies to universal statements. For example, "All elements of an empty set satisfy a given property" is vacuously true because there are no elements to contradict the property.
    \\

    
      - **Tip:** When encountering statements that seem to be true due to the impossibility of the antecedent, check if they are vacuous truths. This can help clarify whether the statement provides meaningful information or is merely a logical artifact.
    \\

    
      - **Exception:** Vacuous truths are not meaningful in themselves but are valid within the framework of classical logic. They are particularly useful in mathematical proofs, such as base cases in induction, where they help establish the validity of broader statements.
    \\

    
      - **Fun Fact:** Vacuous truths often arise in everyday language as idiomatic expressions indicating impossibility, such as "when pigs fly," where the antecedent ("pigs flying") is understood as impossible, making any consequent true by default.
    \\

  \section{Phantom distinction
    
      (also known as: distinction without a difference, difference without a distinction, phantom difference, sham difference, sham distinction)
    \\

  
    Description: The assertion that a position is different from another position based on the language when, in fact, both positions are the same -- at least in practice or practical terms.

    
      Logical Form:
    \\

    
      {\em Claim X is made where the truth of the claim requires a distinct difference between A and B.} \newline
{\em There is no distinct difference between A and B.} \newline
{\em Therefore, claim X is true.}
    \\

    
      Example \#1:
    \\

    
      {\em Sergio: There is no way I would ever even consider taking dancing lessons. \newline
Kitty: How about I ask my friend from work to teach you? \newline
Sergio: If you know someone who is willing to teach me how to dance, then I am willing to learn, sure.}
    \\

    
      Explanation: Perhaps it is the stigma of “dancing lessons” that is causing Sergio to hold this view, but the fact is, someone teaching him how to dance is the same thing.  Sergio has been duped by language.
    \\

    
      Example \#2:
    \\

    
      {\em We must judge this issue by what the Bible says, not by what we think it says or by what some scholar or theologian thinks it says.}
    \\

    
      Explanation: Before you say, “Amen!”, realize that this is a clear case of {\it distinction without a difference}.  There is absolutely no difference here because the only possible way to read the Bible is through interpretation, in other words, what we think it says.  What is being implied here is that one's own interpretation (what he or she thinks the Bible says) is what it really says, and everyone else who has a different interpretation is not really reading the Bible for what it says.
    \\

    
      Exception: It is possible that some difference can be very minute, exist in principle only, or made for emphasis, in which case the fallacy could be debatable.
    \\

    
      {\em Coach:  I don’t want you to try to get the ball; I want you to GET the ball!}
    \\

    
      In practical usage, this means the same thing, but the effect could be motivating, especially in a non-argumentative context.
    \\

    
      Tip: Replace the phrase, “I’ll try” in your vocabulary with, “I’ll do my best”.  While the same idea in practice, perceptually it means so much more.
    \\

    References:

    
      
        
      \\

      
        
          Smart, B. H. (1829). {\it Practical Logic,: Or Hints to Theme-writers: to which are Now Added Some Prefatory Remarks on Aristotelian Logic, with Particular Reference to a Late Work of Dr. Whatley’s}. Whittaker, Treacher, \& Company.
        
      
    
  }


Masked-man fallacy
    
      (also known as: Epistemic fallacy, Hooded man fallacy, Illicit substitution (of identicals, intensional fallacy, illicit substitution of identicals)
    \\

  
    Description: A formal fallacy due to confusing the knowing of a thing ({\it extension}) with the knowing of it under all its various names or descriptions ({\it intension}).

    
      We need to define two terms here to understand this fallacy fully: {\it intensional}  and {\it extensional}.  In logic and mathematics, an {\it intensional}  definition gives the meaning of a term by specifying all the properties required to come to that definition, that is, the necessary and sufficient conditions for belonging to the set being defined.  In contrast, an {\it extensional} definition is defined by its listing everything that falls under that definition.  Confused?  You should be, but relax because I am not done.
    \\

    
      Imagine Superman, who is also Clark Kent, flew to Italy for a slice of pizza.  If we said, “Clark Kent flew to Italy for pizza” we would be right, because of the {\it extensional context} of that statement.  Conversely, if we said, “Lois Lane thinks Superman flew to Italy for pizza”, we would still be making a true claim, although the context is now {\it intensional} as indicated by the term, “thinks”.  Now if we said, “Lois Lane thinks Clark Kent flew to Italy for pizza”, we would be wrong and would have committed this fallacy because Lois{\it  does not believe that}, even though {\it extensionally} it is the case (this is after the kiss that wiped her memory of Clark being Superman).
    \\

    
      Logical Forms:
    \\

    
      {\em X is Y.} \newline
{\em Person 1 thinks X does Z.} \newline
{\em Therefore, person 1 thinks Y did Z.}
    \\

    
      {\em X is Y.} \newline
{\em Person 1 thinks Y does Z.} \newline
{\em Therefore, person 1 thinks X did Z.}
    \\

    
      Example \#1:
    \\

    
      The lady in the pink dress is Julia Roberts.
    \\

    
      The reporter thinks the lady in the pink dress drives a Prius.
    \\

    
      Therefore, the reporter thinks Julia Roberts drives a Prius.
    \\

    
      Example \#2:
    \\

    
      The lady in the pink dress is Julia Roberts.
    \\

    
      The reporter thinks Julia Roberts drives a Prius.
    \\

    
      Therefore, the reporter thinks the lady in the pink dress drives a Prius.
    \\

    
      Explanation: The examples used are just two different logical forms of the same fallacy.  Because the reporter, “thinks” the statement is made in an intensional context, we cannot switch the terms.  However, if we were to keep the premises in an extensional context, we could get away with switching the terms.  This would be a valid logical argument form known as{\it  Leibniz’ Law}.
    \\

    
      Exception: Technically, none, but here is the above example \#1 using {\it Leibniz’ Law, }with no fallacy:
    \\

    
      The lady in the pink dress is Julia Roberts.
    \\

    
      The lady in the pink dress drives a Prius.
    \\

    
      Therefore, Julia Roberts drives a Prius.
    \\

    
      Fun Fact: Julia Roberts really does own a Prius (at the time of this writing).
    \\

    References:

    
      
        
      \\

      
        
          Neil, S. (1853). {\it The Art of Reasoning: A Popular Exposition of the Principles of Logic}. Walton \& Maberly.
        
      
    
  

Denying a Conjunct
    Description: A formal fallacy in which the first premise states that at least one of the two {\it conjuncts} (antecedent and consequent) is false and concludes that the other conjunct must be true.

    
      Logical Forms:
    \\

    
      Not both P and Q.
    \\

    
      Not P.
    \\

    
      Therefore, Q.
    \\

    
       
    \\

    
      Not both P and Q.
    \\

    
      Not Q.
    \\

    
      Therefore, P.
    \\

    
      Example \#1:
    \\

    
      I am not both a moron and an idiot.
    \\

    
      I am not a moron.
    \\

    
      Therefore, I am an idiot.
    \\

    
      Explanation:  I might be an idiot, but the truth of both premises does not guarantee that I am; therefore, this argument is invalid -- technically, the form of this formal argument is invalid.  Being “not both” a moron and an idiot, only means that if I am not one of the two, I am simply not one of the two -- we cannot logically conclude that I am the other.
    \\

    
      Example \#2:
    \\

    
      I am not both a Christian and a Satanist.
    \\

    
      I am not a Satanist.
    \\

    
      Therefore, I am a Christian.
    \\

    
      Explanation:  The truth of both premises does not guarantee that I am a Christian; therefore, this argument is invalid -- the form of this formal argument is invalid.  Being “not both” a Satanist and a Christian, only means that if I am not one of the two, I am simply not one of the two -- we cannot logically conclude that I am the other.
    \\

    
      Exception: No exceptions.
    \\

    
      Fun Fact: Atheists don’t eat babies.
    \\

    References:

    
      
        
      \\

      
        
          Kiersky, J. H., \& Caste, N. J. (1995). {\it Thinking Critically: Techniques for Logical Reasoning}. West Publishing Company.
        
      
    
  

Unwarranted Contrast
    
      (also known as: some are/some are not)
    \\

  
    Description: Assuming that {\it implicature} means {\it implication}, when it logically does not.  {\it Implicature} is a relation between the fact that someone makes a statement and a proposition.  {\it Implication}  is a relation between propositions, that is, the meanings of statements.

    
      Logical Forms:
    \\

    
      Some S are P.
    \\

    
      Therefore, some S are not P.      
    \\

    
       
    \\

    
      Some S are not P.
    \\

    
      Therefore, some S are P.
    \\

    
      Example \#1:
    \\

    
      Some atheists are human.
    \\

    
      Therefore, some atheists are not human.
    \\

    
      Explanation: This might be the case, but we cannot logically {\it imply}  that this is the case because the use of “some” does not logically imply that it does not mean “all”.  In everyday use, “some” does {\it implicate}  “not all”, that is why this is fallacious and could be used to deceive without technically lying.
    \\

    
      Example \#2:
    \\

    
      Some Christians are not Jews.
    \\

    
      Therefore, some Christians are Jews.
    \\

    
      Explanation: Just because we stated that some Christians are not  Jews, does not mean we can logically conclude that some Christians are Jews.  While we may {\it implicate} it, the statement does not {\it imply}  it.
    \\

    
      Fun Fact: This is another of those odd fallacies that are included for completeness sake. People don’t typically use “some” in place of “all” or “none.”
    \\

    References: Hansen, H. V., \& Pinto, R. C. (1995). Fallacies: Classical and Contemporary Readings. Penn State Press.
  

quantificational fallacy
    \begin{itemize}
  \item 
        (also known as: Fallacy of Predicate Logic, Fallacy of Quantificational Logic)
      
    \end{itemize}
  
  

Fallacy of Four Terms
    
      (also known as: quaternio terminorum, ambiguous middle term, fallacy of four terms, Four-term fallacy)
    \\

  
    Description: This fallacy occurs in a categorical syllogism when the syllogism has four terms rather than the requisite three (in a sense, it cannot be a categorical syllogism to begin with!)  If it takes on this form, it is invalid.  The {\it equivocation} fallacy can also fit this fallacy because the same term is used in two different ways, making four distinct terms, although only appearing to be three.

    
      Logical Form: There are many possible forms, this is one example:
    \\

    
      All X are Y.
    \\

    
      All A are B.
    \\

    
      Therefore, all X are B.
    \\

    
      Example \#1:
    \\

    
      All cats are felines.
    \\

    
      All dogs are canines.
    \\

    
      Therefore, all cats are canines.
    \\

    
      Explanation: When you add in a fourth term to a categorical syllogism that can only have three terms to be logically valid, we get nonsense -- or at least an invalid argument. 
    \\

    
      Example \#2:
    \\

    
      All Greek gods are mythical.
    \\

    
      All modern day gods are real.
    \\

    
      Therefore, all Greek gods are real.
    \\

    
      Explanation: Again, nonsense.  If we take away one of the terms, we end up with a valid syllogism:
    \\

    
      All Greek gods are mythical.
    \\

    
      All mythical gods don’t really exist.
    \\

    
      Therefore, all Greek gods don’t really exist.
    \\

    
      Exceptions: No exceptions. \newline

    \\

    
      Fun Fact: Greek gods may not exist, but Greek yogurt does.
    \\

  \section{Base rate fallacy
    
      (also known as: neglecting base rates, base rate neglect, prosecutor's fallacy [form of])
    \\

  
    Description: Ignoring statistical information in favor of using irrelevant information, that one incorrectly believes to be relevant, to make a judgment.  This usually stems from the irrational belief that statistics don’t apply to a situation, for one reason or another when, in fact, they do.

    
      Example \#1:
    \\

    
      Only 6\% of applicants make it into this school, but my son is brilliant!  They are certainly going to accept him!
    \\

    
      Explanation: Statistically speaking, the son may still have a low chance of acceptance. The school is for brilliant kids (and everyone knows this), so the vast majority of kids who apply are brilliant. Of the whole population of brilliant kids who apply, only about 6\% get accepted. So even if the son is brilliant, he still has a low chance of being accepted (about 6\%).
    \\

    
      Example \#2: Faith healing "works," but not all the time, especially when one’s faith is not strong enough (as generally indicated by the size of one’s financial offering).  Unbiased, empirical tests, demonstrate that a small but noticeable percentage of people are cured of “incurable” diseases such as cancer.
    \\

    
      Explanation: This is true.  However, what is not mentioned in the above is the number of cases of cancer that just go away without any kind of faith healing, in other words, the {\it base rate} of cancer remission.  It is a statistical certainty that among those with cancer, there will be a percentage with spontaneous remission.  If that percentage is the same as the faith-healing group, then that is what is to be expected, and no magic or divine healing is taking place.  The following is from the American Cancer Society:
    \\

    
      Available scientific evidence does not support claims that faith healing can cure cancer or any other disease. Some scientists suggest that the number of people who attribute their cure to faith healing is lower than the number predicted by calculations based on the historical percentage of spontaneous remissions seen among people with cancer. However, faith healing may promote peace of mind, reduce stress, relieve pain and anxiety, and strengthen the will to live.[1]
    \\

    
      Exception: If there are factors that increase one’s odds and alter the known statistical probabilities, it would be a reasonable assumption, as long as the variations from the statistical norm are in line with the factors that cause the variation.  In other words, perhaps the mother in our first example knows that her son is gifted musically, that counts for something, then it is not unreasonable to expect a better than 6\% probability -- but assuming a 50\%, 80\%, or 100\% probability, is still committing the fallacy.
    \\

    
      Variation: The {\em prosecutor's fallacy} is a fallacy of statistical reasoning best demonstrated by a prosecutor when exaggerating the likelihood of a defendant's guilt. In mathematical terms, it is the claim that the probability of A given B is equal to the probability of B given A. For example,
    \\

    
      {\em The probability that you have a cervix given that you are pregnant is the same as the probability that you are pregnant given that you have a cervix.}
    \\

    
      Clearly, this is wrong. The probability that you have a cervix if you are pregnant is close to 100\% (leaving room for weird science and magic). The probability that you are pregnant if you have a cervix is dependent on many other factors, but let’s just say it is a lot less than 100\%. In legal cases, a prosecutor may abuse this fallacy to convince the jury that the chance of the defendant being innocent is very small, when it fact, if the whole population were considered (as it should be), the chance of the defendant being guilty (based on that statistic alone) is very small.
    \\

    
      Tip: Take some time in your life to read a book or take a course on probability.  Probability affects our lives in so many ways that having a good understanding of it will continually pay off.
    \\

  }


Defense attorney's Fallacy
    
      - **Description:** The Defense Attorney's Fallacy is a logical fallacy that occurs when a defense attorney argues that evidence implicating their client is not sufficient because it could apply to a large segment of the population. Essentially, it denies the significance of the evidence by arguing that since many others could fit the description or exhibit the characteristic, it cannot be used to single out the defendant.
    \\

    
      - **Logical Form:**
    \\

    
        - **P1:** A piece of evidence is presented that links a characteristic (e.g., blood type) to a crime.
    \\

    
        - **P2:** A significant portion of the population has this characteristic.
    \\

    
        - **C:** Therefore, the evidence cannot implicate the defendant, as it could apply to anyone with that characteristic.
    \\

    
      - **Example \#1:** "The defendant has B positive blood, but 10 percent of the population has B positive blood. Therefore, this blood type does not implicate the defendant because many others could also be the perpetrator."
    \\

    
      - **Explanation:** This argument is fallacious because it overlooks the fact that the evidence (B positive blood) is relevant in combination with other evidence or circumstances. The presence of a characteristic in many people does not negate its significance in the context of the case.
    \\

    
      - **Example \#2:** "The witness identified someone with a red hat at the crime scene, but many people own red hats. The fact that the defendant owns a red hat does not prove they are the perpetrator because anyone with a red hat could have been at the scene."
    \\

    
      - **Explanation:** This argument fails to consider that while many people may own red hats, the combination of other evidence (such as the defendant's presence or behavior) is necessary to determine guilt. The sheer number of people with similar characteristics does not invalidate the evidence against the defendant.
    \\

    
      - **Variation:** Sometimes known as the "Many People Fallacy," it focuses on the statistical probability of characteristics or behaviors being common among a large number of individuals.
    \\

    
      - **Tip:** To counter this fallacy, emphasize the cumulative weight of evidence and its relevance to the specific case, rather than focusing on how common a characteristic is in the general population.
    \\

    
      - **Exception:** The fallacy does not apply if the evidence is indeed irrelevant or insufficient when considering the full context of the case. However, it’s important to demonstrate how the evidence fits into the broader narrative and investigation.
    \\

    
      - **Fun Fact:** This fallacy can sometimes be seen in both criminal defense and other contexts where individuals or groups attempt to discredit evidence by pointing out its general applicability, rather than addressing its relevance to the specific situation at hand.
    \\

  

Denying the antecedent
    
      (also known as: conclusion which denies premises, inverse error, inverse fallacy)
    \\

  
    Description: It is a fallacy in formal logic where in a standard if/then premise, the antecedent (what comes after the “if”) is made not true, then it is concluded that the consequent (what comes after the “then”) is not true.

    
      Logical Form:
    \\

    
      If P, then Q.
    \\

    
      Not P.
    \\

    
      Therefore, not Q.
    \\

    
      Example \#1:
    \\

    
      If it barks, it is a dog.
    \\

    
      It doesn’t bark.
    \\

    
      Therefore, it’s not a dog.
    \\

    
      Explanation: It is not that clear that a fallacy is being committed, but because this is a formal argument following a strict form, even if the conclusion seems to be true, the argument is still invalid.  This is why fallacies can be very tricky and deceptive.  Since it doesn’t bark, we cannot conclude with certainty that it isn’t a dog -- it could be a dog who just can’t bark.
    \\

    
      The arguer has committed a formal fallacy, and the argument is invalid because the truth of the premises does not guarantee the truth of the conclusion.
    \\

    
      Example \#2:
    \\

    
      If I have cable, then I have seen a naked lady.
    \\

    
      I don’t have cable.
    \\

    
      Therefore, I have never seen a naked lady.
    \\

    
      Explanation: The fallacy is more obvious here than in the first example. Denying the antecedent (saying that I don’t have cable) does not mean we must deny the consequent (that I have seen a naked lady...I have, by the way, in case you were wondering).
    \\

    
      The arguer has committed a formal fallacy, and the argument is invalid because the truth of the premises does not guarantee the truth of the conclusion.
    \\

    
      Exception: No exceptions.
    \\

    
      Tip: If you ever get confused with formal logic, replace the words with letters, like we do in the logical form, then replace the letters with different phrases and see if it makes sense or not.
    \\

    References:

    
      
        
      \\

      
        
          Kiersky, J. H., \& Caste, N. J. (1995). {\it Thinking Critically: Techniques for Logical Reasoning}. West Publishing Company.
        
      
    
  

False Conversion
    
      New Terminology:
    \\

    
      Type “A” Logical Forms: A proposition or premise that uses the word, “all” or “every” (e.g., All P are Q)
    \\

    
      Type “E” Logical Forms: A proposition or premise that uses the word, “none” or “no” (e.g., No P are Q)
    \\

    
      Type “I” Logical Forms: A proposition or premise that uses the word, “some” (e.g., Some P are Q)
    \\

    
      Type “O” Logical Forms: A proposition or premise that uses the terms, “some/not” (e.g., Some P are not Q)
    \\

    
      Description: The formal fallacy where the subject and the predicate terms of the proposition are switched ({\it conversion}) in the conclusion, in a proposition that uses “all” in its premise (type “A” forms), or “some/not” (type “O” forms).
    \\

    
      Logical Form:
    \\

    
      All P are Q.
    \\

    
      Therefore, all Q are P.
    \\

    
       
    \\

    
      Some P are not Q.
    \\

    
      Therefore, some Q are not P.
    \\

    
      Example \#1:
    \\

    
      All Hollywood Squares contestants are bad actors.
    \\

    
      Therefore, all bad actors are Hollywood Squares contestants.
    \\

    
      Example \#2:
    \\

    
      Some people in the film industry do not win Oscars.
    \\

    
      Therefore, some Oscar winners are not people in the film industry.
    \\

    
      Explanation: It does not follow logically that just because all {\it Hollywood Square}s contestants are bad actors that all bad actors actually make it on {\it Hollywood Squares}.  Same form problem with the second example -- but we used “some” and “are not”.
    \\

    
      Exception: None, but remember that type “E” and type “I” forms can use conversion and remain valid.
    \\

    
      No teachers are psychos.
    \\

    
      Therefore, no psychos are teachers.
    \\

    
      Tip: Remember that formal fallacies are often obscured by unstructured rants. Creating a formal argument from such rants is both an art and a science.
    \\

  \chapter{Ad hoc rescue
    
      (also known as: making stuff up, MSU fallacy)
    \\

  
    Description: Very often we desperately want to be right and hold on to certain beliefs, despite any evidence presented to the contrary.  As a result, we begin to make up excuses as to why our belief could still be true, and is still true, despite the fact that we have no real evidence for what we are making up.

    
      Logical Form:
    \\

    
      Claim X is true because of evidence Y.
    \\

    
      Evidence Y is demonstrated not to be acceptable evidence.
    \\

    
      Therefore, it must be guess Z then, even though there is no evidence for guess Z.
    \\

    
      Example \#1:
    \\

    
      Frieda: I just know that Raymond is just waiting to ask me out.
    \\

    
      Edna: He has been seeing Rose for three months now.
    \\

    
      Frieda: He is just seeing her to make me jealous.
    \\

    
      Edna: They’re engaged.
    \\

    
      Frieda: Well, that’s just his way of making sure I know about it.
    \\

    
      Explanation: Besides being a bit deluded, poor Frieda refuses to accept the evidence that leads to a truth she is not ready to accept.  As a result, she creates an {\it ad hoc} reason in an attempt to rescue her initial claim.
    \\

    
      Example \#2:
    \\

    
      Mark: The president of the USA is the worst president ever because unemployment has never been so bad before!
    \\

    
      Sam: Actually, it was worse in 1982 and far worse in the 1930s.  Besides, the president might only be partly responsible for the economy during his term.
    \\

    
      Mark: Well... the president kicks animals when nobody is looking.
    \\

    
      Explanation: Out of desperation, Mark makes a claim about the president's private treatment of animals after his original claim has been refuted.
    \\

    
      Exception: Proposing possible solutions is perfectly acceptable when an argument is suggesting only a possible solution -- especially in a hypothetical situation. For example, “If there is no God, then life is meaningless.”  If there is no God who dictates meaning to our lives, perhaps we are truly free to find our own meaning.
    \\

    
      Tip: When you suspect people are just making stuff up, rather than providing evidence to support their claim, simply ask them, “What evidence do you have to support that?”
    \\

  }


Escape hatch
    
      - **Description:** An "escape hatch" is a rhetorical device used to evade the burden of proof or sidestep criticism in an argument. It typically involves making unfalsifiable, circular, or absurd claims that are designed to be unanswerable. The intention is to declare victory in the argument by presenting a point that is immune to counter-argument or scrutiny, often leaving the debater with a sense of smugness or relief.
    \\

    
      - **Logical Form:**
    \\

    
        - **P1:** A claim is made that is unfalsifiable, circular, or otherwise unanswerable.
    \\

    
        - **P2:** The claim is presented as a decisive argument or rebuttal.
    \\

    
        - **C:** The argument is declared won, and the discussion is terminated.
    \\

    
      - **Example \#1:** "Science doesn’t know everything, so my belief in astrology must be correct."
    \\

    
      - **Explanation:** The argument uses the unknowns in science to assert the validity of astrology, avoiding the need to provide concrete evidence for astrology itself.
    \\

    
      - **Example \#2:** "If you don’t succeed in moving mountains through faith, it’s because you didn’t have enough faith."
    \\

    
      - **Explanation:** The failure to achieve a desired outcome is attributed to the individual's lack of faith rather than a flaw in the belief system, making the argument unchallengeable.
    \\

    
      - **Variation:** The escape hatch can appear in various forms, such as "science isn’t complete," "you can’t prove anything," or claiming that certain practices or beliefs are "beyond the scope of understanding."
    \\

    
      - **Tip:** To avoid falling into an escape hatch argument, focus on asking for concrete evidence and clarification rather than accepting vague or untestable claims.
    \\

    
      - **Exception:** Escape hatch arguments can sometimes serve as a signal that the discussion may need to shift from debating specifics to addressing underlying assumptions or methodologies.
    \\

    
      - **Fun Fact:** The concept of the escape hatch is frequently employed in debates involving pseudosciences and fringe theories, where proponents often resort to unfalsifiable claims to protect their beliefs from scrutiny.
    \\

  

God of the gaps
    
      - **Description:** "God of the gaps" is a theological concept where gaps in scientific knowledge are attributed to divine intervention or action. Essentially, it is the idea that if something is not yet understood by science, it must be the work of God. This argument is often used to assert the existence of a deity based on the limitations of current scientific explanations.
    \\

    
      - **Logical Form:**
    \\

    
        - **P1:** There is a gap in scientific knowledge or explanation.
    \\

    
        - **P2:** The gap cannot be currently explained by scientific means.
    \\

    
        - **C:** The gap must be filled by divine action or intervention.
    \\

    
      - **Example \#1:** "Science cannot explain how life began, therefore God must have created life."
    \\

    
      - **Explanation:** The lack of a scientific explanation for the origin of life is used to argue that God must be the cause. This is an instance of attributing the unknown to divine action rather than waiting for scientific advancement.
    \\

    
      - **Example \#2:** "We don’t yet understand how consciousness arises from brain activity, so it must be a result of a divine soul."
    \\

    
      - **Explanation:** The current inability to fully explain consciousness through neuroscience is used to argue for the existence of a divine element. The gap in understanding is filled with a theological explanation rather than a scientific one.
    \\

    
      - **Variation:** The concept can also be applied to other areas where science is lacking, such as explanations of natural disasters or specific phenomena, where divine action is posited as the cause.
    \\

    
      - **Tip:** Be cautious of relying on "God of the gaps" arguments as they often depend on the current limitations of science rather than a robust argument for the existence of a deity. As scientific knowledge progresses, these gaps may be filled, making the argument less compelling.
    \\

    
      - **Exception:** The "God of the gaps" argument is often critiqued for shifting as scientific knowledge advances. What was once considered a gap may become understood through future discoveries, potentially rendering the argument obsolete.
    \\

    
      - **Fun Fact:** The term "God of the gaps" was popularized by theologian and philosopher Karl Barth, who critiqued the use of divine intervention to fill gaps in human understanding.
    \\

  

Handwave
    
      - **Description:** Hand-waving is a term used pejoratively to describe arguments or actions that seem effective or impressive but lack substantive evidence or detail. It often involves glossing over important aspects, using fallacies, or employing misdirection to avoid rigorous scrutiny. In various contexts, it can indicate an attempt to appear productive or convincing without actually providing meaningful contributions.
    \\

    
      - **Logical Form:**
    \\

    
        - **P1:** A claim or argument is presented that is vague, incomplete, or unsupported.
    \\

    
        - **P2:** The proponent tries to divert attention away from the lack of substance or detail.
    \\

    
        - **C:** The argument or action is perceived as valid or effective, despite its lack of rigor.
    \\

    
      - **Example \#1:** "Clearly, the new policy will improve efficiency, but we don’t need to discuss the specifics right now."
    \\

    
      - **Explanation:** The speaker uses the term "clearly" to assert that the policy will improve efficiency, while avoiding any detailed discussion or evidence that might support or refute the claim.
    \\

    
      - **Example \#2:** "The product will revolutionize the industry—just trust us on this; the technical details are too complex to explain here."
    \\

    
      - **Explanation:** The claim is made without providing concrete details or evidence, with the expectation that the audience will accept the assertion at face value.
    \\

    
      - **Variation:** Hand-waving can vary in context, such as in debates where it involves logical fallacies and misdirection, in business where it may involve vague claims of productivity, or in academia where it might involve unproven assertions or skipped steps in proofs.
    \\

    
      - **Tip:** To counter hand-waving, ask for specific evidence and detailed explanations. Challenge any claims that are made with insufficient support or that rely on vague assertions.
    \\

    
      - **Exception:** In some contexts, such as initial presentations or high-level overviews, a degree of hand-waving might be acceptable if detailed information is provided later. However, persistent or deliberate hand-waving without follow-up can be problematic.
    \\

    
      - **Fun Fact:** The term "hand-waving" metaphorically relates to the gesture of waving one’s hands to distract or divert attention, much like how stage magicians use hand movements to misdirect their audience from the real tricks being performed. The concept has also been popularized by its use in the Star Wars franchise, where Jedi use hand-waving as a form of mind control.
    \\

  

PIDOOMA
    
      (also known as:  Pulled It Directly Out Of My Ass, argumentum ex culo, Colonic autoextraction methodology, Toilet fishing, Proctolojustification)
    \\

  
    
      - **Description:** PIDOOMA is a derogatory term for arguments or claims made with little to no evidence, often involving blatant and intentional lies. It represents the act of making up arguments on the spot in a desperate attempt to support a failing position, often characterized by ad hoc reasoning and dubious statistics.
    \\

    
      - **Logical Form:**
    \\

    
        - **P1:** A claim or argument is made with minimal or no evidence.
    \\

    
        - **P2:** The claim often involves blatant, fabricated information or lies.
    \\

    
        - **C:** The argument is presented as valid despite its lack of evidence and reliance on falsehoods.
    \\

    
      - **Example \#1:** A proponent of a conspiracy theory claims, “Statistics show that 90\% of all historical events are manipulated by shadowy elites,” without providing any reliable data or sources.
    \\

    
      - **Explanation:** The statistic is fabricated to support the conspiracy theory, relying on spurious and unverifiable information to lend credibility to the unfounded claim.
    \\

    
      - **Example \#2:** Andrew Schlafly’s claim that "Wikipedia is 6 times more liberal than the American public" is an example of PIDOOMA, as the statistic was made up to discredit Wikipedia.
    \\

    
      - **Explanation:** The statistic is created without basis to serve a specific agenda, showing how PIDOOMA involves generating false claims to support a particular argument or viewpoint.
    \\

    
      - **Variation:** PIDOOMA can manifest in various forms, including bogus statistics, exaggerated claims, or ad hoc justifications that lack substantiation.
    \\

    
      - **Tip:** To identify PIDOOMA, look for claims that lack credible evidence, rely on dubious statistics, or seem designed to evade scrutiny. Cross-check information with reliable sources and be cautious of arguments that seem to have been pulled out of nowhere.
    \\

    
      - **Exception:** In some cases, a claim that initially seems like PIDOOMA might be corrected or substantiated with further evidence. It's important to differentiate between initial fabrications and evolving arguments with emerging evidence.
    \\

    
      - **Fun Fact:** The term "PIDOOMA" humorously references the idea of arguments being "pulled directly out of one's ass," emphasizing the absurdity and lack of credibility of such claims.
    \\

  

Game theory fallacies
\end{document}
